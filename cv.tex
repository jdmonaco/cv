%! TEX program = xelatex
%
% Curriculum vitae for Joseph D. Monaco
%
\documentclass[10pt]{article}

\usepackage{fullpage}
\usepackage[usenames,dvipsnames]{color}
\usepackage[hidelinks,xetex]{hyperref}
\usepackage{multirow}
\usepackage{sectsty}
\usepackage{tabu}
\usepackage{longtable}
\usepackage{soul}
\usepackage{fontspec}

% Page geometry formatting
\raggedbottom
\raggedright
\textheight=9in
\setlength{\tabcolsep}{0in}
\addtolength{\footskip}{-.2in}
\addtolength{\voffset}{-.4in}
\addtolength{\textheight}{.9in}
\addtolength{\hoffset}{-.25in}
\addtolength{\textwidth}{.2in}

% Define colors
\definecolor{hopkinsblue}{RGB}{0,48,130}
\definecolor{lightblue}{rgb}{.88,.92,.97}
\definecolor{lightgold}{rgb}{1,.80,0}
\definecolor{lightgray}{rgb}{.80,.80,.80}
\definecolor{dimgray}{rgb}{.50,.50,.50}
\definecolor{darkgray}{rgb}{.30,.30,.30}

% Set up highlighting and underlining
\sethlcolor{lightblue}
\setul{0.13ex}{}

% Choose font package
\setmainfont[Mapping=tex-text,Ligatures=TeX]{Helvetica Neue LT Std} 

% Section section* formatting
\sectionfont{\large\color{hopkinsblue}\bfseries}
\subsectionfont{\vspace{-1ex}\hspace{1ex}\color{hopkinsblue}\normalsize\bfseries}

% Formatting macros
\newcommand{\itemtitle}[1]{{\color{hopkinsblue}\ul{#1}}}
\newcommand{\unpubtitle}[1]{{\color{hopkinsblue} #1}}
\newcommand{\itemnote}[1]{
  \begin{description}
    \item[$\rightarrow$] \hspace{.09in}{\color{darkgray}\it #1}
  \end{description}
}
\newcommand{\joehl}[1]{\hl{\textbf{#1}}}
\newcommand{\doi}[1]{{\color{darkgray}doi:}~{\color{dimgray}\texttt{#1}}}
\newcommand{\abs}[1]{{\color{darkgray}abs:}~{\color{dimgray}\texttt{#1}}}
\newcommand{\aurl}[1]{{\color{dimgray}\texttt{#1}}}
\newcommand{\researchnote}[1]{
  \begin{description}
    \item[] {\hspace{2.2ex}\color{darkgray} #1}
  \end{description}
}
\newcommand{\researchactivity}[4]{
  \begin{minipage}[t]{\textwidth}
    \begin{tabular}{@{\hspace{2ex}}l>{\raggedright\arraybackslash}p{.8\textwidth}}
      \makebox[1.2in][l]{#1} & \textbf{#2:}
      ``\unpubtitle{#3}.'' 
    \end{tabular}
  \researchnote{\hspace{1ex} #4}
  \end{minipage}
  \medbreak
}
\newcommand{\lefttabline}[3]{\hspace{2ex}\makebox[#1][l]{#2} #3\\}

% PDF document info and setup
\hypersetup{
  baseurl=http://jdmonaco.com/cv-monaco.pdf,
  pdftitle=Curriculum Vitae for Joseph D. Monaco,
  pdfauthor=Joseph D. Monaco,
  pdfdisplaydoctitle=true,
  pdfpagemode=UseThumbs,
  pdfstartview=FitV,
  pdfpagelayout=TwoColumnLeft,
  pdftoolbar=false,
  pdfwindowui=false,
  pdfcenterwindow=true,
}

\begin{document}
%! TEX program = xelatex
{\fontspec{Verdana}\small
\begin{tabular*}{6.675in}{c@{\extracolsep{\fill}}rl}
  \hline\\[0.02in]
  \textbf{\LARGE\color{hopkinsblue} Joseph Daniel Monaco, Ph.D.} 
                                                       & \textsc{Email}          & \href{mailto:jmonaco@jhu.edu}{\texttt{jmonaco@jhu.edu}} \\
  \multirow{2}{*}{\large }                             & \textsc{Web}            & \href{http://jdmonaco.com/}{\texttt{jdmonaco.com}} \\
  {\small Johns Hopkins University School of Medicine} & \textsc{ORCID}          & \href{http://jdmonaco.com/orcid}{\texttt{0000-0003-0792-8322}} \\
  {\small 720 Rutland Avenue, 407 Traylor}             & \textsc{GitHub}         & \href{https://github.com/jdmonaco?tab=repositories}{\texttt{github.com/jdmonaco}} \\
  {\small Baltimore, MD, 21205, USA}                   & \textsc{Google Scholar} & \href{http://jdmonaco.com/google-scholar}{\texttt{gceOLZEAAAAJ}} \\[0.1in]
  \hline
\end{tabular*}
}\\[0.1in]

\pagestyle{empty}

\section*{Education}

\begin{itemize}
  \item
    \begin{tabular*}{6.3in}{l@{\extracolsep{\fill}}r}
      \textbf{Columbia University} & New York, NY \\
      Department of Neurobiology \& Behavior & 2005--2009 \\
      Center for Theoretical Neuroscience\\
      Degrees: Ph.D. (2009); M.Phil. (2008); M.A. (2006) \\
      Advisor: Laurence~F.~Abbott\\
      %Collaborators: Isabel Muzzio, Eric~R.~Kandel \\
    \end{tabular*}
  \item
    \begin{tabular*}{6.3in}{l@{\extracolsep{\fill}}r}
      \textbf{Brandeis University} & Waltham, MA \\
      Department of Biology & 2003--2005\\
      Volen Center for Complex Systems\\
      Graduate Program in Neuroscience, \textit{\ul{Continued at Columbia University}} \\
      Advisor: Laurence~F.~Abbott\\
      %Collaborator: Michael~J.~Kahana \\
    \end{tabular*}
  \item
    \begin{tabular*}{6.3in}{l@{\extracolsep{\fill}}r}
      \textbf{University of Virginia} & Charlottesville, VA \\
      Laboratory of Computational Neurodynamics & 1999--2003\\
      Degrees: B.A.~Mathematics; B.A.~Cognitive Science \\
      Advisor: William `Chip' Levy\\
      Echols Scholar \\
    \end{tabular*}
\end{itemize}

\section*{Positions}

\begin{itemize}
  \item
    \begin{tabular*}{6.3in}{l@{\extracolsep{\fill}}r}
      \textbf{Johns Hopkins University School of Medicine} & Baltimore, MD\\
      Research Associate (Faculty) & 2019--present\\
      Department of Biomedical Engineering\\
      Collaborators: Kechen Zhang; Kathleen E.~Cullen; JHU/Applied Physics \\
      \qquad Laboratory (JHU/APL)\\
    \end{tabular*}

  \item
    \begin{tabular*}{6.3in}{l@{\extracolsep{\fill}}r}
      \textbf{Johns Hopkins University School of Medicine} & Baltimore, MD\\
      Postdoctoral Fellow & 2013--2019\\
      Department of Biomedical Engineering\\
      PI: Kechen Zhang\\
      Collaborators: Hugh T.~Blair (UCLA); David J.~Foster (UC Berkeley); Mark N.~Wu (JHU); \\
      \qquad Kathleen E.~Cullen (JHU); JHU/APL\\
    \end{tabular*}

  \item
    \begin{tabular*}{6.3in}{l@{\extracolsep{\fill}}r}
      \textbf{Johns Hopkins University} & Baltimore, MD\\
      Postdoctoral Fellow & 2009--2013\\
      Zanvyl Krieger Mind/Brain Institute\\
      PI: James J. Knierim \\
      Collaborator: Kechen Zhang \\
    \end{tabular*}
\end{itemize}

\section*{Publications}

\begin{description}
  \item \joehl{Monaco JD}, \ldots, Hwang GM. (\emph{In preparation}). \itemtitle{How
    AI can learn from complex brain dynamics}. \emph{Nature Machine
    Intelligence}. {\bfseries Perspective article invited by the editors based on presubmission enquiry.}
  \item \href{https://doi.org/10.1007/s00422-020-00823-z}
    {\joehl{Monaco JD}, Hwang GM, Schultz KM, and Zhang K. (2020).
    \itemtitle{Cognitive swarming in complex environments with attractor
      dynamics and oscillatory computing}. \emph{Biological Cybernetics}, 114,
    269--284. \doi{10.1007/s00422-020-00823-z}}.
  \item \begin{samepage}\href{https://doi.org/10.1016/j.cub.2020.01.083} 
      {Wang CH, \joehl{Monaco JD}, and Knierim JJ. (2020). \itemtitle{Hippocampal
        place cells encode local surface texture boundaries}. \emph{Current Biology},
      30, 1--13. \doi{10.1016/j.cub.2020.01.083}}.
  \itemnote{I mentored the first author in data analysis of rat behavior and
      single-unit recordings, developed the software toolchain used to conduct the
    analyses, and provided intellectual guidance.}\end{samepage}
  \item \href{https://doi.org/10.1117/12.2518966}
    {\joehl{Monaco JD}, Hwang GM, Schultz KM, and Zhang K. (2019).
    \itemtitle{Cognitive swarming: An approach from the theoretical neuroscience
        of hippocampal function}. \emph{Proceedings of SPIE (International society
      for optics and photonics) Defense \& Commercial Sensing}. Micro- and
      Nanotechnology Sensors, Systems, and Applications XI, 109822D, 1--10.
    \doi{10.1117/12.2518966}}.
  \item \href{https://doi.org/10.1371/journal.pcbi.1006741}
    {\joehl{Monaco JD}, De Guzman RM, Blair HT, and Zhang K. (2019).
    \itemtitle{Spatial synchronization codes from coupled rate-phase
      neurons}. \emph{PLOS Computational Biology}, 15(1), e1006741.
    \doi{10.1371/journal.pcbi.1006741}}.
  \item \href{https://www.cell.com/cell/fulltext/S0092-8674(18)31228-5}
    {Tabuchi M, \joehl{Monaco JD}, Duan G, Bell BJ, Liu S, Zhang K, and
      Wu MN. (2018). \itemtitle{Clock-generated temporal codes determine
      synaptic plasticity to control sleep}. \emph{Cell}, 175(5), 1213--27.
    \doi{10.1016/j.cell.2018.09.016}}.
  \itemnote{I developed two modeling strategies for the Wu lab’s circadian clock
      neuron experiments in \emph{Drosophila}. My generative statistical model was
      integrated into stimulation protocols as a timing control for behavioral
      results, and my mechanistic molecular/neuronal model explained observed trends
      and made predictions corroborated by the data. My results or contributions are
    featured in 3/7 main figures and 3/6 supplementary figures.}
  \item \href{http://doi.org/10.1038/nn.3687}
    {\joehl{Monaco JD}, Rao G, Roth ED, and Knierim JJ. (2014).
    \itemtitle{Attentive scanning behavior drives one-trial potentiation of
      hippocampal place fields}. \emph{Nature Neuroscience}, 17(5), 725--731.
    \doi{10.1038/nn.3687}}.
  \item \href{http://doi.org/10.3389/fncom.2011.00039}
    {\joehl{Monaco JD}, Knierim JJ, and Zhang K. (2011). \itemtitle{Sensory
        feedback, error correction, and remapping in a multiple oscillator model of
      place cell activity}. \emph{Frontiers in Computational Neuroscience}, 5:39.
    \doi{10.3389/fncom.2011.00039}}.
  \item \href{http://doi.org/10.1523/JNEUROSCI.1433-11.2011}
    {\joehl{Monaco JD} and Abbott LF. (2011). \itemtitle{Modular
        realignment of entorhinal grid cell activity as a basis for hippocampal
      remapping}. \emph{Journal of Neuroscience}, 31(25), 9414--25.
    \doi{10.1523/jneurosci.1433-11.2011}}.
  \item \href{http://doi.org/10.1371/journal.pbio.1000140}
    {Muzzio IA, Levita L, Kulkarni J, \joehl{Monaco J}, Kentros CG, Stead
      M, Abbott LF, and Kandel ER. (2009). \itemtitle{Attention enhances the
        retrieval and stability of visuospatial and olfactory representations
      in the dorsal hippocampus}. \emph{PLOS Biology}, 7(6), e1000140.
    \doi{10.1371/journal.pbio.1000140}}.
  \itemnote{I contributed oscillatory power analyses and group-level statistical
      analyses of spiking and bursting for odor vs.\ visuospatial tasks in
    single-unit hippocampal recordings from freely-moving mice.}
  \item \href{http://doi.org/10.1101/lm.363207}
    {\joehl{Monaco JD}, Abbott LF, and Kahana MJ. (2007).
    \itemtitle{Lexico-semantic structure and the recognition
      word-frequency effect}. \emph{Learning \& Memory}, 14(3), 204--213.
    \doi{10.1101/lm.363207}}.
  \item \href{http://doi.org/10.1109/IJCNN.2003.1223655}
    {\joehl{Monaco JD} and Levy WB. (2003). \itemtitle{T-maze training
        of a recurrent CA3 model reveals the necessity of novelty-based
        modulation of LTP in hippocampal region CA3}. \emph{Proceedings of
      International Joint Conference on Neural Networks}, 1655--1660.
    \doi{10.1109/IJCNN.2003.1223655}}.
\end{description}

\section*{Preprints}

\begin{description}
  \item \href{https://arxiv.org/abs/2003.13825}
    {Levenstein D, Alvarez VA, Amarasingham A, Azab H, Gerkin RC, Hasenstaub
      A, Iyer R, Jolivet RB, Marzen~S, \joehl{Monaco JD}, Prinz AA, Quraishi
      S, Santamaria F, Shivkumar S, Singh MF, Stockton DB, Traub R, Rotstein
      HG, Nadim F, and Redish AD. (2020). \itemtitle{On the role of theory and
    modeling in neuroscience}. \emph{arXiv}. \abs{2003.13825}}.
  \item \href{https://arxiv.org/abs/1909.06711}
    {\joehl{Monaco JD}, Hwang GM, Schultz KM, and Zhang K. (2019).
    \itemtitle{Cognitive swarming in complex environments with attractor
    dynamics and oscillatory computing}. \emph{arXiv}. \abs{1909.06711}}.
  \item \href{http://doi.org/10.1101/764282}
    {Wang CH, \joehl{Monaco JD}, and Knierim JJ. (2019). \itemtitle{Hippocampal
      place cells encode local surface texture boundaries}. \emph{bioRxiv}.
    \doi{10.1101/764282}}.
  \item \href{http://dx.doi.org/10.1101/211458}
    {\joehl{Monaco JD}, Blair HT, and Zhang K. (2017). \itemtitle{Spatial theta
        cells in competitive burst synchronization networks: Reference frames from
    phase codes}. \emph{bioRxiv}. \doi{10.1101/211458}}.
\end{description}

\section*{Thesis}

\begin{description}
  \item \href{http://search.proquest.com/docview/304862872/abstract}
    {\joehl{Monaco JD}. (2009). \itemtitle{Models and mechanisms for integrating
      cortical feature spaces}. Doctoral Dissertation, Columbia University, New
    York. \emph{ProQuest Publication No. AAT 3393609}}.
\end{description}

\section*{White Papers} \label{sec:whitepapers}

\begin{description}
  \item \joehl{Monaco J}, Zhang K, and Schultz K. (August 15, 2019).
    \unpubtitle{SW2Mem: Graph Spectral Decoding of Hippocampal-Cortical Loops for
      Artificial Consolidation and Dreaming}. \emph{Response to Office of Naval
    Research (ONR) Special Notice N00014-19-S-SN08, Topic 5.1}.
  \item Schultz K, Agarwala S, Zhang K, \joehl{Monaco J}. (August 15, 2019).
    \unpubtitle{Brain-like Distributed Surveillance using Heterogeneous Agents for
      integRated Perception, and Planning (BD-SHARPP)}. \emph{Response to ONR Special
    Notice N00014-18-R-SN05, Topic 3}.
  \item Zhang K, \joehl{Monaco JD}, Hwang GM, Schultz KM, Kobilarov M, Foster
    DJ, Jacobs J, Itti L. (June 29, 2018). \unpubtitle{An Integrative Theoretical
      Framework of the Neural Self-Organization of Active Perception for Autonomous
      Spatial Navigation}. \emph{Response to ONR MURI Announcement N00014-18-S-F006,
    Topic 6}.
  \item Schultz K, Zhang K, \joehl{Monaco J}. (March 22, 2018).
    \unpubtitle{BrainSWARRMM: Brain-like Sharp-Waves for Autonomous Replanning
      \& Reconnaissance on Matrix Manifolds}. \emph{Response to ONR Special Notice
    N00014-18-R-SN05, Topic 3}.
\end{description}

\section*{Media Releases and Interviews}

\begin{description}
  \item \href{http://kavlijhu.org/news/32} {``\itemtitle{Can
        robotic swarms navigate using learning rules devised for brain
      dynamics?}'' JHU/Kavli Neuroscience Discovery Insitute. May 3, 2020.
    \aurl{http://kavlijhu.org/news/32}}
  \item \href{https://www.youtube.com/watch?v=ic4zEgVMSsA}
    {``\itemtitle{Swarmalators}.'' JHU/APL Press Office. May 9, 2019.
    \aurl{https://www.youtube.com/watch?v=ic4zEgVMSsA}}
  \item \href{https://hub.jhu.edu/2018/10/02/brain-robot-swarms-study/}
    {``\itemtitle{What do animal brains have in common with swarms of robots?
      Maybe more than you think}.'' Geoff Brown/JHU Office of Communications. Oct 2,
    2018. \aurl{https://hub.jhu.edu/2018/10/02/brain-robot-swarms-study/}}
  \item \href{https://www.jhuapl.edu/PressRelease/181001}
    {``\itemtitle{Do Robot Swarms Work Like Brains?}'' JHU/APL Press Office. October 1, 2018.
    \aurl{https://www.jhuapl.edu/PressRelease/181001}}
  \item \href{https://hub.jhu.edu/2014/04/14/memory-brain-place-cells/}
    {``\itemtitle{Where does a memory begin? Johns Hopkins neuroscientists think they
      know}.'' Latarsha Gatlin/JHU Office of Communications. April 14, 2014.
    \aurl{https://hub.jhu.edu/2014/04/14/memory-brain-place-cells/}}
  \item \href{https://www.youtube.com/watch?v=Jm8OiLJqKJQ}
    {``\itemtitle{Johns Hopkins Researchers Probe Mysteries of
      the Brain}.'' JHU Office of Communications. April 14, 2014.
    \aurl{https://www.youtube.com/watch?v=Jm8OiLJqKJQ}}
\end{description}

\section*{Websites and Social Media}

\begin{description}
  \item \href{http://jdmonaco.com/}
    {``\itemtitle{Briefly Balanced: Research in the computational neuroscience
    of spatial memory}.” Website. \aurl{http://jdmonaco.com/}}
  \item \href{https://www.ncbi.nlm.nih.gov/pubmed/?term=monaco_jd+OR+(monaco_j+AND+muzzio_ia)}
    {\itemtitle{PubMed Listing}. Website. \aurl{https://www.ncbi.nlm.nih.gov/pubmed/?term=monaco\_jd+OR+ (monaco\_j+AND+muzzio\_ia)}}
  \item \href{http://jdmonaco.com/google-scholar}
    {\itemtitle{Google Scholar}. Website. \aurl{https://scholar.google.com/citations?hl=en\& user=gceOLZEAAAAJ\&view\_op=list\_works\&sortby=pubdate}}
  \item \href{https://github.com/jdmonaco}
    {\itemtitle{GitHub Overview}. Website. \aurl{https://github.com/jdmonaco}}
  \item \href{https://twitter.com/j_d_monaco}
    {\itemtitle{Twitter Feed}. Social Media. \aurl{https://twitter.com/j\_d\_monaco}}
\end{description}

\section*{Media and Community Coverage of My Work}

\subsection*{News \& Views and Post-Publication Reviews}

\begin{itemize}
  \item \href{https://doi.org/10.1016/j.cub.2020.02.085}
    {Place R, Nitz DA. (2020). \itemtitle{Cognitive Maps: Distortions of the Hippocampal 
      Space Map Define Neighborhoods}. \emph{Current Biology}, 30(8): R340--R342.}
  \item \href{https://doi.org/10.1016/j.cell.2018.10.047}
    {Colwell CS, Donlea J. (2018). \itemtitle{Temporal coding of sleep}. \emph{Cell}, 175(5): 1177--9.}
  \item \href{http://f1000prime.com/718333676#eval793494783}
    {Moser E, Rowland D. (May 12, 2014). ``\itemtitle{This exciting study finds
        an unexpected relationship between exploratory head scanning behavior and the
        development of new place fields in the rat hippocampus...}” \emph{Faculty of
    1000}.}
  \item \href{https://doi.org/10.1038/nn.3700}
    {Dupret D, Csicsvari J. (2014). \itemtitle{Turning heads to remember
    places}. \emph{Nature Neuroscience}, 17(5): 643--44.}
  \item \href{http://f1000prime.com/718333676#eval793493493}
    {Maler L. (April 10, 2014). ``\itemtitle{This elegant and original study has
        demonstrated a strong link between the neural activity of hippocampal pyramidal
        neurons (PNs) during head scanning behavior and their subsequent acquisition of
    a new place field...}'' \emph{Faculty of 1000}.}
  \item \href{http://f1000.com/11553956}
    {Giocomo L, Moser E. (June 29, 2011) ``\itemtitle{This paper presents an
        interesting computational model which utilizes grid-cell modularity to generate
    robust remapping...}'' \emph{Faculty of 1000}.}
\end{itemize}

\subsection*{Websites and Blogs}

\begin{itemize}
  \item \href{https://www.fastcompany.com/90529833/best-workplaces-for-innovators-2020 -johns-hopkins-university-apl} 
    {``\itemtitle{Johns Hopkins University APL is one of Fast Company’s Best Workplaces
      for Innovators}.'' (July 29, 2020). \emph{Fast Company}.
      \aurl{https://www.fastcompany.com/90529833/best-workplaces-for-innovators- 2020-johns-hopkins-university-apl}}
    \itemnote{My NSF project (see Award No.~1835279 below) was the basis for \#3 ranking of JHU/APL.}
  \item \href{https://blogs.plos.org/biologue/2019/03/20/better-use-of-mouse-models-skin-infection-dynamics-and-phaser-cells-in-navigation/}
    {``\itemtitle{Better Use of Mouse Models, Skin Infection
      Dynamics, and Phaser Cells in Navigation}.'' (March 20,
      2019). \emph{PLOS Computational Biology: Biologue}.
      \aurl{https://blogs.plos.org/biologue/2019/03/20/ 
    better-use-of-mouse-models-skin-infection-dynamics-and-phaser-cells-in-navigation/}}
  \itemnote{Editor-in-Chief's selection of papers.}
  \item \href{https://nationalsciencefoundation.tumblr.com/post/183448836933/brain-awareness-week-2019-rats-and-robots}
    {``\itemtitle{Brain Awareness Week 2019—Rats and Robots: NSF-funded researchers
        take a lesson from rat navigation instincts to improve algorithm[s] for
      robots}.'' (March 14, 2019). \emph{National Science Foundation/Tumblr}.
    \aurl{https://nationalsciencefoundation.tumblr.com/post/ 183448836933/brain-awareness-week-2019-rats-and-robots}}
  \item \href{https://www.medicaldaily.com/cognitive-map-can-show-real-time-when-memories-form-thanks-place-cells-brain-276790}
    {``\itemtitle{Cognitive Map Can Show In Real-Time When Memories Form, Thanks
      To Place Cells In The Brain}.'' (April 15, 2014). Chris Weller/Medical Daily.
    \aurl{https://www.medicaldaily.com/cognitive-map-can- show-real-time-when-memories-form-thanks-place-cells-brain-276790}}
\end{itemize}

\section*{Funding}

\subsection*{Current External Support} \label{sec:cursupport}

\begin{itemize}
  \item \href{https://www.nsf.gov/awardsearch/showAward?AWD_ID=1835279&HistoricalAwards=false}
    {\itemtitle{NCS-FO: Spatial intelligence for swarms based on hippocampal
    dynamics}}\hspace{\stretch{1}}2018--2021
    \begin{itemize}
      \item NSF\slash NCS FOUNDATIONS (BRAIN Initiative) Award No.~1835279: \$862K
      \item \textbf{Lead PI:} Kechen Zhang
      \item \textbf{Co-PIs, JHU/APL}: Grace Hwang, Robert W. Chalmers, Kevin
        Schultz, and M. Dwight Carr
      \item \textbf{Research Associate (FY19)/Co-PI (FY20--FY21): \joehl{Joseph D. Monaco}}
    \end{itemize}
  \itemnote{I co-developed this project and co-wrote the proposal
      with a JHU/APL colleague (see \emph{\nameref{sec:resnsf}} on
      p.\pageref{sec:resnsf}). As a Research Associate faculty at JHU as of
    FY20, my project role was promoted to co-PI.}

\pagebreak 

  \item \href{https://projectreporter.nih.gov/project_info_description.cfm?aid=9652210&icde=42555668&ddparam=&ddvalue=&ddsub=&cr=2&csb=default&cs=ASC&pball=}
    {\itemtitle{Spiking network models of sharp-wave ripple sequences with\\
    gamma-locked attractor dynamics}}\hspace{\stretch{1}}2018--2020
    \begin{itemize}
      \item NIH/NINDS R03 Award No.~NS109923: \$50K
      \item \textbf{PI:} Kechen Zhang
      \item \textbf{Research Associate:} \joehl{Joseph D. Monaco}
    \end{itemize}
  \itemnote{I conceived this project, generated preliminary data, and wrote the
      proposal (see \emph{\nameref{sec:resnih}} on p.\pageref{sec:resnih}).
    As a Postdoctoral Fellow, JHU policy precluded a PI role.}
\end{itemize}

\subsection*{Previous Internal Support} \label{sec:prevsupport}

\begin{itemize}
  \item \href{http://scienceoflearning.jhu.edu/research/learning-to-explore-paths-through-space/}
    {\itemtitle{Learning to explore paths through space}}\hspace{\stretch{1}}2016--2018
    \begin{itemize}
      \item JHU/Science of Learning Institute (SLI) Award: \$150K
      \item \textbf{PI:} Kechen Zhang
      \item \textbf{Co-PI:} David J.~Foster (now at UC Berkeley)
      \item \textbf{Research Associate:} \joehl{Joseph D.~Monaco}
    \end{itemize}
  \itemnote{I conceived this project, initiated the collaboration
      between the Zhang and Foster labs, and wrote the proposal (see
      \emph{\nameref{sec:ressli}} on p.\pageref{sec:ressli}). As a
    Postdoctoral Fellow, JHU policy precluded a PI role.}
\end{itemize}

\section*{Educational Activity}

\subsection*{Classroom Instruction}

\begin{tabular}{@{\hspace{0.2in}}l>{\raggedright\arraybackslash}p{.82\textwidth}}
  Fall 2004 & Teaching Assistant for undergraduate ``Introduction to
  Neuroscience'' course, Brandeis University; I assisted Prof. Eve Marder by
  supervising classes, grading examinations, and giving review lectures.\\
  \tabularnewline
  Spring 2005 \hspace{.1in} & Teaching Assistant for undergraduate ``Biology
  Laboratory'' course, Brandeis University \\
\end{tabular}
\medskip

\subsection*{Workshops/Seminars -- Regional}

\begin{longtable}{@{\hspace{0.2in}}l>{\raggedright\arraybackslash}p{.82\textwidth}}
  10/2/2019 \hspace{0.1in} & ``\unpubtitle{Oscillations, attractors, and
  sequences: Extending hippocampal computations to artificial systems}.''
  \emph{Invited Lecture}. Kavli Neuroscience Discovery Institute, Johns Hopkins
  University, Baltimore, MD\\
  \tabularnewline
  9/25/2019 & ``\unpubtitle{Decoding septohippocampal theta cells during
  exploration reveals unbiased environmental cues in firing phase}.''
  \emph{Poster Session}. Kavli Neuroscience Discovery Institute, Johns Hopkins
  University, Baltimore, MD\\
  \tabularnewline
  12/7/2016 & ``\unpubtitle{Spatial rate/phase correlations in theta cells
  can stabilize randomly drifting path integrators}.'' \emph{Poster Session}.
  Greater Baltimore SfN Meeting, Baltimore, MD\\
  \tabularnewline
  1/22/2016 & ``\unpubtitle{Hippocampal circuits for space, memory, and
  navigation: From minimal models to biologically inferred networks}.''
  \emph{Invited Lecture}. Department of Pharmacology, University of Maryland,
  Baltimore, MD\\
  \tabularnewline
  9/6/2014 & ``\unpubtitle{Stopping to look: How attentive scanning behavior
  reveals the formation of new memories}.'' \emph{Department Retreat Seminar}.
  Department of Neuroscience, Johns Hopkins University, Baltimore, MD\\
  \tabularnewline
  4/21/2014 & ``\unpubtitle{Landmark influence: How attention to sensory cues
  stabilizes and updates the hippocampal cognitive representation of space}.''
  \emph{Advanced Researcher Seminar}. Zanvyl Krieger Mind/Brain Institute, Johns
  Hopkins University, Baltimore, MD\\
  \tabularnewline
  4/1/2014 & ``\unpubtitle{Hippocampus and declarative memory: Head scanning}.''
  \emph{Department `Lab Lunch' Seminar}. Department of Neuroscience, Johns Hopkins
  University, Baltimore, MD\\
  \tabularnewline
\end{longtable}

\subsection*{Workshops/Seminars -- National}

\begin{longtable}{@{\hspace{0.2in}}l>{\raggedright\arraybackslash}p{.82\textwidth}}
  6/1/2020 \hspace{0.3in} & ``\unpubtitle{Can Transitory Neurodynamics Unify
  Learning Theories for Brains and Machines?}'' \emph{Invited Lecture \&
  Panel Discussion}. Symposium on ``How Can Dynamical Systems Neuroscience
  Reciprocally Advance Machine Learning?'', 6th Annual BRAIN Initiative
  Investigators Meeting, NIH, Online\\
  \tabularnewline
  5/18/2020 \hspace{0.3in} & ``\unpubtitle{Computational Approaches to the
  Neural Dynamics of Time, Memory, and Behavior}.'' \emph{Invited Lecture}.
  Department of Neuroscience, Medical Discovery Team for Optical Imaging,
  University of Minnesota, Online\\
  \tabularnewline
  2/24/2020 \hspace{0.3in} & ``\unpubtitle{Computational Mechanisms of Memory:
  Linking Behavior, Space, \& Time}.'' \emph{Invited Lecture}. Department of
  Psychology, University of Nevada, Las Vegas, NV\\
  \tabularnewline
  1/31/2020 \hspace{0.3in} & ``\unpubtitle{Attractors, memory, and oscillations:
  Computational motifs of spatial learning}.'' \emph{Invited Lecture}.
  Department of Biological Sciences, University of Texas at El Paso, El Paso, TX\\
  \tabularnewline
  4/17/2019 \hspace{0.3in} & ``\unpubtitle{Emergent dynamics of hippocampal
  circuitry as a basis for robust self-organized planning in mobile swarms}.''
  \emph{Invited Lecture}. SPIE (International society for optics and photonics)
  Defense \& Commercial Sensing 2019 conference, Baltimore, MD\\
  \tabularnewline
  4/10/2019 & NSF/Neural \& Cognitive Systems (NCS) PI
  Workshop. \emph{Participant}. Marriott Wardman Park Hotel, Washington, D.C.\\
  \tabularnewline
  2/3--2/7/2019 & NSF/BRAIN Initiative Workshop: Present and Future Frameworks
  of Theoretical Neuroscience. \emph{Invited Participant}. University of Texas,
  San Antonio, TX.\\
  \tabularnewline
  1/3/2014 & ``\unpubtitle{Head scans drive the formation and potentiation
  of place fields during exploration}.'' \emph{Data Blitz}. 38th Annual Winter
  Conference on Neurobiology of Learning \& Memory, Park City, UT.\\
  \tabularnewline
  4/10/2009 & ``\unpubtitle{Rapid spatial map formation and remapping by
  competing over grid cell inputs}.'' \emph{Thesis Seminar}. Department of
  Neurobiology \& Behavior, Columbia University Medical Center, New York, NY.\\
\end{longtable}

\subsection*{Workshops/Seminars -- International}

\begin{tabular}{@{\hspace{0.2in}}l>{\raggedright\arraybackslash}p{.82\textwidth}}
  10/29/2020 \hspace{0.2in} &
  \href{https://youtu.be/ZSpg-nhXjw0?t=24400}{``\unpubtitle{Spatial theta-phase
      coding in the lateral septum: a theory of allocentric feedback during
    navigation}.'' \itemtitle{Click for YouTube}. \emph{Contributed Talk}.
  Neuromatch 3.0 Conference, Online. }\\
  \tabularnewline
  10/7/2020 \hspace{0.2in} & ``\unpubtitle{Computing path integration with
  oscillatory phase codes in biological and artificial systems}.'' \emph{Data
  Blitz}. iNAV Symposium 2020, Online.\\
  \tabularnewline
  7/1/2010 \hspace{0.2in} & ``\unpubtitle{Medial versus lateral modes for
  reconfiguring hippocampal representations}.'' \emph{Invited Lecture}. Grid
  Cell Meeting, Gatsby Computational Neuroscience Unit, University College
  London, UK.\\
\end{tabular}

\smallskip
\subsection*{Educational Program Building}

\begin{tabular}{@{\hspace{0.2in}}l>{\raggedright\arraybackslash}p{.82\textwidth}}
  2018--present \hspace{0.1in} & The NSF project (see
  \emph{\nameref{sec:cursupport}} on p.\pageref{sec:cursupport}) was
  successfully funded with a substantial STEM component for high-school students
  involving the development of both a 12-week course and an intense 2-day
  seminar called ``Swarming Powered by Neuroscience.'' I am working with our
  STEM education collaborators at JHU/APL to develop computational resources
  required for the curricula. Additionally, I have begun mentoring a high-school
  student intern, Marc Burlina, at JHU/APL so that he can help develop
  simulation components for this educational program.
\end{tabular}

\subsection*{Mentoring}

\begin{tabular}{@{\hspace{0.2in}}l>{\raggedright\arraybackslash}p{.82\textwidth}}
  2020 \hspace{0.1in} & Armin Hadzic, machine learning engineer at JHU/APL; I
  am supervising Armin in translating computational neuroscience models into
  the domain of reinforcement learning and Bayesian optimization to investigate
  autonomous swarming with neural control. \\
  \tabularnewline
  2019--2020 \hspace{0.1in} & Sreelakshmi Rajendrakumar, masters student in
  JHU/Biomedical Engineering (BME); I mentored Sreelakshmi in hippocampal
  physiology and single-unit data analysis.\\
  \tabularnewline
  2014 & Manning Zhang, M.S., graduate student in JHU/BME; I mentored Manning
  through an exchange program with Shanghai Jiao Tong University and
  subsequently submitted a letter of recommendation as part of her (successful)
  application to the JHU/BME graduate program.\\
  \tabularnewline
  2013--2015 & Chia-Hsuan Wang, Ph.D., graduate student at the JHU/Zanvyl
  Krieger Mind/Brain Institute (MBI); I worked extensively with Chia-Hsuan to
  take over my previous studies of behavior and place cells in the Knierim lab,
  leading to a Society for Neuroscience conference poster in 2014. I supported
  her subsequent thesis research based on my analytics and informatics software,
  resulting in a paper in Current Biology.\\
\end{tabular}

\section*{Research Activity}

\subsection*{Research Program Building/Leadership}
\label{sec:resprogram}

\researchactivity
{Mar. 2016--2018}
{Grant Award (JHU/SLI)}
{Learning to explore paths through space}
{This internal JHU award (2016--2018; see \emph{\nameref{sec:prevsupport}} on
  p.\pageref{sec:prevsupport}) resulted from a collaboration with David J. Foster
  (now at UC Berkeley) that I initiated to conduct modeling studies informed
  by his lab’s hippocampal reactivation data. By integrating Prof.~Zhang’s
  mathematical theories of spatial cognitive maps, I wrote and submitted a
  proposal for a \$200K/2-year project to the JHU Science of Learning Institute.
  The proposal was awarded at the \$150K level and research outcomes included (1)
  novel theories of temporal synchronization coding that inspired the 2017 NSF
  proposal effort, and (2) preliminary dynamical models of sharp-wave reactivation
that provided the foundation for the 2018 NIH R03 award.}
\label{sec:ressli}

\researchactivity
{April--June 2016}
{Grant Proposal (DARPA/BTO)}
{Noninvasive Gastrovagal Stimulation for Enhanced Neuroplasticity of Cortical
and Hippocampal Networks during Cognitive Training (GEN-C)}
{In response to DARPA announcement BAA-16-24 of the “Targeted Neuroplasticity
  Training (TNT)” program, I worked with colleagues from JHU/APL and JHU/SoM
  Center for Neurogastroenterology to develop a collaborative program involving 3
  PIs and 5 co-Is (8 labs) across divisions, departments, and fields. I recruited
  experimental labs from JHU/MBI and coordinated proposed contributions to
  optimize responsiveness to DARPA’s mission and maximize scientific impact with
  a total project cost of \$9.8M over 5 years. I was the main coordinator of
  the 40-page research narrative, including editing and integrating each lab’s
  contributions (and writing several), and worked with the primary JHU/SoM PI
  (Prof.~Pankaj Pasricha) and ORA to submit the proposal. While not successfully
  funded in total, DARPA/BTO PM Doug Weber funded select components, leading to
  JHU/APL Work Agreement No.~145563 “BCI (Brain Computer Interface) Technologies”
in 2018 for \$24,604 to the lab of Prof.~Pasricha.}

\researchactivity
{Nov. 2017-pres.}
{Grant Award (NSF/NCS)}
{NCS-FO: Spatial intelligence for swarms based on hippocampal dynamics}
{This NSF-awarded project (2018--2020; see \emph{\nameref{sec:cursupport}}
  on p.\pageref{sec:cursupport}) was the result of 6 months of collaboration,
  brain-storming, and team-building between the Zhang lab at JHU/SoM and a group
  of JHU/APL engineers, mathematicians, and scientists. The project was initially
  inspired by results that I presented at my Society for Neuroscience 2017 meeting
  poster. I wrote Aim 1 and integrated the full research narrative with inputs
  from our collaborators for the proposal of this \$997K/2-year project to develop
  those initial ideas into technological applications (e.g., robotics, autonomous
  control, AI) that reciprocally inform neuroscience. The project has so far
  produced three posters, a conference talk \& proceedings publication, three
  patent applications, a preprint, a research article in Biological Cybernetics, a
NIH BRAIN Investigators Meeting symposium talk, and a substantial STEM program.}
\label{sec:resnsf}

\researchactivity
{Jan. 2018-pres.}
{Grant Award (NIH/NINDS)}
{Spiking network models of sharp-wave ripple sequences with gamma-locked
attractor dynamics}
{To continue with the collaboration that I initiated with David J. Foster
  (UC Berkeley) on the basis of the internal SLI award (see above), I wrote
  a small modeling proposal that integrated preliminary results from the SLI
  project and recent research developments in the memory reactivation field.
  This proposal was awarded (2018--2020; see \emph{\nameref{sec:cursupport}}
  on p.\pageref{sec:cursupport}) through the NIH/NINDS R03 mechanism and I am
  currently utilizing this support to build a foundation for future efforts along
this research track.}
\label{sec:resnih}

\researchactivity
{Feb.--Mar. 2018}
{White Paper (ONR)}
{BrainSWARRMM: Brain-like Sharp-Waves for Autonomous Replanning \&
Reconnaissance on Matrix Manifolds}
{In response to ONR Special Notice N00014-18-R-SN05, I helped organize a
  series of collaborative meetings to design a \$2M/4-year project between
  JHU/APL and JHU/SoM. I co-authored the resulting white paper that was
  submitted for consideration to ONR (see \emph{\nameref{sec:whitepapers}} on
p.\pageref{sec:whitepapers}).}

\researchactivity
{May--June 2018}
{White Paper (ONR)}
{An Integrative Theoretical Framework of the Neural Self-Organization of Active
Perception for Autonomous Spatial Navigation}
{In response to ONR MURI Announcement N00014-18-S-F006 and with the help of
  JHU/APL, I coordinated a series of meetings with 5 PIs across 4 universities
  (Columbia, UC Berkeley, USC, JHU) to design an innovative research program that
  targeted reciprocal advances in experimental \& theoretical neuroscience and
  robotics \& AI across species and scales. The resulting \$7.5M/5-year project
  that I outlined in the white paper (see \emph{\nameref{sec:whitepapers}} on
  p.\pageref{sec:whitepapers}) was not invited for a full submission. We debriefed
  with the sponsor, ONR PM Marc Steinberg, who revealed that ONR was impressed
  with the project but that they were seeking a different balance of elements with
respect to neuroscience and AI.}

\researchactivity
{August 2019}
{White Paper (ONR)}
{SW2Mem: Graph Spectral Decoding of Hippocampal-Cortical Loops for Artificial
Consolidation and Dreaming}
{In response to ONR Special Notice N00014-19-S-SN08, Topic 5.1 I conceived this
  project, created the preliminary model and datasets, guided the preliminary
  analyses with JHU/APL collaborators, and wrote \& submitted the white paper
  (see \emph{\nameref{sec:whitepapers}} on p.\pageref{sec:whitepapers}) to ONR
  outlining a potential \$1.05M/3-year project. ONR declined to invite us to
submit a full proposal.}

\researchactivity
{August 2019}
{White Paper (ONR)}
{Brain-like Distributed Surveillance using Heterogeneous Agents for integRated
Perception, and Planning (BD-SHARPP)}
{In response to ONR Special Notice N00014-18-R-SN05, we submitted a revised
  version of the March 2018 white paper (see \emph{\nameref{sec:whitepapers}}
  on p.\pageref{sec:whitepapers}) that was specifically invited by ONR PM Tom
McKenna.}

\researchactivity
{Sept. 11, 2019}
{NSF Project Review}
{Annual advisory board review symposium}
{I delivered a seminar on Aim 1 progress at a JHU/APL-hosted symposium for our
  project’s yearly review, attended by DARPA/I2O PM Hava Siegelmann and other
outside experts.}

\researchactivity
{Feb. 26, 2020}
{Grant Proposal (NSF/NCS) }
{NCS-FO: Neuroeconomics as a biomimetic control theory for mobile robotic
decision making}
{This FY21 proposal was submitted to the NSF/NCS FOUNDATIONS program; while
  it was discussed and received high scores, the application was declined. I
  co-developed this project in collaboration with colleagues at the University of
  Pittsburgh Medical Center (UPMC), JHU Whiting School of Engineering (JHU/WSE),
  and JHU/APL. Our interdisciplinary project brought together multiscale human
  electrophysiological recordings (UPMC), latent state-space models (JHU/WSE),
  control- and game-theoretic analysis (JHU/APL), and mechanistic neural models
  (JHU/BME, for which I would have been co-PI). We proposed to investigate and
  characterize the neural bases of metacognitive brain states that influence
  decision-making during social \& economic games. As a high-risk/high-reward
  element, we proposed to algorithmicize our results to advance human-robot
interaction.}

\subsection*{Inventions \& Patents}

\lefttabline{0.8in}{1/3/2020}{Inventor, Autonomous Navigation Technology, patent
application 16734294}
\lefttabline{0.8in}{5/10/2019}{Inventor, Neuroinspired Algorithms for Swarming
Applications, provisional patent 62/845,957}
\lefttabline{0.8in}{1/3/2019}{Inventor, Neuroinspired Algorithms for Swarming
Applications, provisional patent 62/787,891}

\subsection*{Conference Abstracts}

\begin{description}
  \item[\quad]
    \href{http://www.cvent.com/events/6th-annual-brain-initiative-investigators-meeting/custom-116-4e2aadaa6cd549a3a4b53113cd172ad2.aspx}
    {\joehl{Monaco JD}, Hwang GM, Schultz K, Zhang K. (2020).
    \itemtitle{Cognitive swarming in complex environments with attractor
        dynamics and oscillatory computing}. \emph{6th Annual BRAIN Initiative
    Investigators Meeting}. Online, with audio narration. June~2020.}
  \item[\quad]
    \href{https://www.fens.org/Meetings/The-Brain-Conferences/Dynamics-of-the-brain/}
    {\joehl{Monaco JD}, Hwang GM, De Guzman RM, Blair HT, Zhang K. (2019).
    \itemtitle{Spatial rate-phase coding in lateral septal ‘phaser cells’:
        single-unit data and theta-bursting models}. \emph{FENS (Federation of
        European Neuroscience Societies) Dynamics of the brain: Temporal aspects of
    computation}. North Copenhagen, Denmark. June~2019.}
  \item[\quad]
    \href{http://www.cvent.com/events/5th-annual-brain-initiative-investigators-meeting/event-summary-de9c0d8f934b46eb8d80b55bcfbfe96a.aspx}
    {\joehl{Monaco JD}, Hwang GM, Schultz K, Zhang K. (2019).
    \itemtitle{Self-organized swarm control using neural principles of spatial
      phase coding}. \emph{5th Annual BRAIN Initiative Investigators Meeting}.
    Washington, D.C. April~2019.} 
  \item[\quad]
    \href{https://abstractsonline.com/pp8/#!/4649/presentation/10884}
    {Hwang GM, Schultz K, \joehl{Monaco JD}, Chalmers RW, Lau SW, Yeh BY,
      Zhang K. (2018). \itemtitle{Self-organized swarm control using neural
      principles of spatial phase coding}. \emph{Society for Neuroscience}.
    San Diego, CA. November~2018.}
  \item[\quad]
    \href{http://www.abstractsonline.com/pp8/#!/4376/presentation/6085}
    {\joehl{Monaco J}, Blair HT, Zhang K. (2017). \itemtitle{Decoding
        septohippocampal theta cells during exploration reveals unbiased
      environmental cues in firing phase}. \emph{Society for Neuroscience}.
    Washington, D.C. November~2017.}
  \item[\quad]
    \href{http://jdmonaco.com/files/monaco-paper-cosyne15.pdf}
    {\joehl{Monaco JD}, Blair HT, Zhang K. (2015). \itemtitle{Spatial
        rate/phase correlations in theta cells can stabilize randomly drifting path
    integrators}. \emph{Cosyne}. Salt Lake City, UT. March~2015.}
  \item[\quad]
    \href{http://www.abstractsonline.com/Plan/ViewAbstract.aspx?sKey=973d2662-ba7a-4ad2-aff9-fe0d4b77c262&cKey=9917ffaf-9e31-4213-acb9-4aab498ab4cd&mKey=54c85d94-6d69-4b09-afaa-502c0e680ca7}
    {\joehl{Monaco J}, Blair HT, Zhang K. (2014). \itemtitle{Spatial rate/phase
        codes provide landmark-based error correction in a temporal model of theta
    cells}. \emph{Society for Neuroscience}. Washington, D.C.  November~2014.}
  \item[\quad]
    \href{http://www.abstractsonline.com/Plan/ViewAbstract.aspx?sKey=bfb59866-8deb-44a6-9515-a7aab630507b&cKey=d201b3aa-7725-452e-b0dd-c41d204b5b54&mKey=54c85d94-6d69-4b09-afaa-502c0e680ca7}
    {Wang CH, Rao G, \joehl{Monaco JD}, Deshmukh SS, Knierim JJ. (2014).
    \itemtitle{Potentiation of place fields along the CA1 transverse axis by
      investigatory head-scanning behavior}. \emph{Society for Neuroscience}. 
    Washington, D.C. November~2014.}
  \item[\quad]
    \href{http://www.abstractsonline.com/Plan/ViewAbstract.aspx?sKey=32eccac1-4e1d-4e81-bf5c-f39bcb605757&cKey=4710dece-cc8e-4b48-8764-49ea174b91ef&mKey=8d2a5bec-4825-4cd6-9439-b42bb151d1cf}
    {\joehl{Monaco J}, Rao G, Knierim JJ. (2013). \itemtitle{Scanning behavior
        in novel environments promotes \emph{de novo} formation of hippocampal place
    fields in rats}. \emph{Society for Neuroscience}. San Diego, CA. November~2013.}
  \item[\quad]
    \href{http://www.abstractsonline.com/Plan/ViewAbstract.aspx?sKey=f5b9fa94-7d15-48c7-9d67-b89cd2883025&cKey=a53349ca-41b1-4664-b022-85d0d1fe59b8&mKey=70007181-01C9-4DE9-A0A2-EEBFA14CD9F1}
    {\joehl{Monaco J}, Rao G, Knierim JJ. (2012). \itemtitle{Hippocampal LFP
      during rodent head-scanning behavior: Theta and sharp-wave ripples}.
    \emph{Society for Neuroscience}. New Orleans, LA. October~2012.}
  \item[\quad]
    \href{http://www.abstractsonline.com/Plan/ViewAbstract.aspx?sKey=c48e9f5f-1274-4486-85bf-38ee591629e1&cKey=190bd951-c183-428d-a4c5-01eb61556d79&mKey=8334BE29-8911-4991-8C31-32B32DD5E6C8}
    {\joehl{Monaco J}, Rao G, Knierim JJ. (2011). \itemtitle{Hippocampal place
        cell firing during head-scanning movements is associated with the formation
    of new place fields}. \emph{Society for Neuroscience}. Washington, D.C. November~2011.}
  \item[\quad]
    \href{http://www.abstractsonline.com/Plan/ViewAbstract.aspx?sKey=c48e9f5f-1274-4486-85bf-38ee591629e1&cKey=3ec26e6f-8c59-4be2-bad3-e1572d75e07e&mKey=8334BE29-8911-4991-8C31-32B32DD5E6C8}
    {Rao G, \joehl{Monaco J}, Knierim JJ. (2011). \itemtitle{Environmental
        novelty promotes rodent head-scanning behavior linked to enhanced entorhinal
      activity}. \emph{Society for Neuroscience}. Washington,
    D.C. November~2011.}
  \item[\quad]
    \href{http://www.frontiersin.org/10.3389/conf.fnins.2010.03.00192/event_abstract}
    {\joehl{Monaco JD}, Zhang K, Blair HT, Knierim JJ. (2010).
    \itemtitle{Cue-based feedback enables remapping in a multiple oscillator
      model of place cell activity}. \emph{Cosyne}. Salt Lake City, UT.
    February~2010.}
  \item[\quad] \joehl{Monaco JD}, Abbott LF. (2009). \unpubtitle{Dynamic
      hippocampal remapping using recurrent inhibition on realigning grid cell
    inputs}. \emph{Cosyne}. Salt Lake City, UT. February~2009.
  \item[\quad] \joehl{Monaco JD}, Muzzio IA, Levita L, Abbott LF. (2006).
    \unpubtitle{Entorhinal input and global remapping of hippocampal place
    fields}. \emph{CNS}. Edinburgh, UK. July~2006.
  \item[\quad] \joehl{Monaco JD}, Abbott LF. (2006). \unpubtitle{Entorhinal
    input and the remapping of hippocampal place fields}. \emph{Cosyne}. Salt Lake
    City, UT. March~2006.
  \item[\quad] \joehl{Monaco JD}, Levy WB. (2003). \unpubtitle{T-maze training
      of a recurrent CA3 model reveals the necessity of novelty-based modulation of
    LTP in hippocampal region CA3}. \emph{IJCNN}. Portland, OR. July~2003.
  \item[\quad] \joehl{Monaco JD}, Perlstein RP. (1997). \unpubtitle{Monte-Carlo
      analysis of deoxyhypusine synthase inhibitor ligand conformations}. \emph{NIH
    Poster Day}. Bethesda, MD. August~1997.
\end{description}

\section*{Professional Activity}

\subsection*{Journal Peer Review}

\lefttabline{0.8in}{2020}{Neuroscience and Biobehavioral Reviews}
\lefttabline{0.8in}{2020}{Scientific Reports}
\lefttabline{0.8in}{2019}{eLife}
\lefttabline{0.8in}{2019}{Hippocampus}
\lefttabline{0.8in}{2018--2019}{Neuron}
\lefttabline{0.8in}{2018}{Neural Computation (including as `Communicator')}
\lefttabline{0.8in}{2018}{PLOS ONE}
\lefttabline{0.8in}{2017}{PeerJ}
\lefttabline{0.8in}{2015}{IEEE Transactions in Biomedical Engineering}
\lefttabline{0.8in}{2012--2020}{IEEE Neural Networks}
\lefttabline{0.8in}{2012}{Biological Cybernetics}
\lefttabline{0.8in}{2012}{Neurocomputing}
\lefttabline{0.8in}{2012}{Neuroscience}

\subsection*{Scientific Societies}

\lefttabline{0.8in}{2019}{Society for Neuroscience, Regular Member}
\lefttabline{0.8in}{2011--2018}{Society for Neuroscience, Postdoc Member}

\subsection*{Conference Organization}

\lefttabline{0.8in}{2020-2021}{Cosyne, Review committee member}
\lefttabline{0.8in}{2016}{Cosyne, Review committee member}

\section*{Recognition}

\subsection*{Awards and Honors}

\lefttabline{0.8in}{2003}{IJCNN Student Paper Competition, First Place}
\lefttabline{0.8in}{2002}{U.Va. John A. Harrison III Undergraduate Research Award}
\lefttabline{0.8in}{1999--2003}{U.Va. Echols Scholar}
\lefttabline{0.8in}{1999}{State of Maryland Merit Scholastic Award}
\lefttabline{0.8in}{1999}{AP Scholar with Distinction}
\lefttabline{0.8in}{1999}{National Merit Scholarship Commended Student}
\lefttabline{0.8in}{1999}{Johns Hopkins Mathematics Competition (2nd Place, Individual Calculus)}
\lefttabline{0.8in}{1999}{Maryland Distinguished Scholar}

\end{document}
