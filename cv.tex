%
% Curriculum vitae for Joseph D. Monaco
%
% jdmonaco-cv.tex
%
% For the initial template h/t Matthew M. Boedicker <mboedick@mboedick.org> http://mboedick.org
% $Id: resume.tex,v 1.7 2004/02/02 14:21:06 mboedick Exp $
%

\documentclass[10pt]{article}
\usepackage{fullpage}
\usepackage{palatino}
\usepackage{hyperref}
\usepackage{multirow}
\usepackage{tabu}
\textheight=9.0in
\pagestyle{empty}
\raggedbottom
\raggedright
\setlength{\tabcolsep}{0in}
\addtolength{\voffset}{-.4in}
\addtolength{\textheight}{.9in}
\addtolength{\hoffset}{-.25in}
\addtolength{\textwidth}{.2in}

\begin{document}

\begin{tabular*}{6.675in}{c@{\extracolsep{\fill}}rl}
\hline\\[0.02in]
\textsc{\textbf{\Large Joseph Daniel Monaco, Ph.D.}}    & \textsc{Email}      & \href{mailto:jmonaco@jhu.edu}{\texttt{jmonaco@jhu.edu}} \\
\multirow{2}{*}{\large }                                & \textsc{Web}      & \href{http://jdmonaco.com/}{\texttt{jdmonaco.com}} \\
{\small Johns Hopkins University School of Medicine}    & \textsc{ORCID}        & \href{http://jdmonaco.com/orcid}{\texttt{0000-0003-0792-8322}} \\
{\small 720 Rutland Avenue, 407 Traylor}                & \textsc{GitHub}      & \href{https://github.com/jdmonaco?tab=repositories}{\texttt{github.com/jdmonaco}} \\  
{\small Baltimore, MD, 21205, USA}                      & \textsc{Google Scholar}   & \href{http://jdmonaco.com/google-scholar}{\texttt{gceOLZEAAAAJ}} \\[0.1in]
\hline
\end{tabular*}

% {\large \textbf{Recent Publications}}

% \begin{description}
% \item \textbf{Monaco JD}, Rao G, Roth ED, and Knierim JJ. (2014). \href{http://dx.doi.org/10.1038/nn.3687}{\emph{Attentive scanning behavior drives one-trial potentiation of hippocampal place fields}. Nature Neuroscience, 17(5), 725--731. doi:~10.1038/nn.3687}.
% \begin{itemize}
%   \item \underline{\bf News \& Views}: Dupret D and Csicsvari J. (2014). \href{http://dx.doi.org/10.1038/nn.3700}{\emph{Turning heads to remember places}. Nature Neuroscience, 17(5), 643--644. doi:~10.1038/nn.3700}.
%   \item \underline{\bf F1000}: Moser E and Rowland D. \href{http://f1000prime.com/718333676#eval793494783}{\emph{Faculty of 1000}, May 12, 2014. f1000.com/718333676}.
%   \item \underline{\bf F1000}: Maler L. \href{http://f1000prime.com/718333676#eval793493493}{\emph{Faculty of 1000}, April 10, 2014. f1000.com/718333676}.
% \end{itemize}
% \item \textbf{Monaco JD}, Knierim JJ, and Zhang K. (2011). \href{http://dx.doi.org/10.3389/fncom.2011.00039}{\emph{Sensory feedback, error correction, and remapping in a multiple oscillator model of place cell activity}. Frontiers in Computational Neuroscience, 5:39. doi:~10.3389/fncom.2011.00039}.
% \item \textbf{Monaco JD} and Abbott LF. (2011). \href{http://dx.doi.org/10.1523/JNEUROSCI.1433-11.2011}{\emph{Modular realignment of entorhinal grid cell activity as a basis for hippocampal remapping}. Journal of Neuroscience, 31(25), 9414--25. doi:~10.1523/jneurosci.1433-11.2011}.
% \begin{itemize}
%   \item \underline{\bf F1000}: Giocomo L and Moser E. \href{http://f1000.com/11553956}{\emph{Faculty of 1000}, June 29, 2011. f1000.com/11553956}.
% \end{itemize}
% \end{description}

\vspace{0.25in}
{\large \textbf{Education}}
\begin{itemize}
  \item 
  \begin{tabular*}{6.3in}{l@{\extracolsep{\fill}}r}
    \textbf{Columbia University} & New York, NY \\
    Center for Theoretical Neuroscience & 2005--9 \\
    Ph.D., Neurobiology \& Behavior\\
    Degrees: M.A. (2006), M.Phil. (2008), Ph.D. (2009) & \\
    Advisor: L.~F. Abbott, Ph.D.\\
    Collaborators: Isabel Muzzio, Ph.D., Eric R. Kandel, Ph.D. (2006--8) \\
    % \textit{Transferred from Brandeis University} & \\
  \end{tabular*}
  \item 
  \begin{tabular*}{6.3in}{l@{\extracolsep{\fill}}r}
    \textbf{Brandeis University} & Waltham, MA \\
    Volen Center for Complex Systems & 2003--5\\
    Graduate Program in Neuroscience, \textit{Continued at Columbia University}  \\
    Advisor: L.~F. Abbott, Ph.D.\\
    Collaborator: Michael J. Kahana, Ph.D. (2004--6) \\
  \end{tabular*}
  \item
  \begin{tabular*}{6.3in}{l@{\extracolsep{\fill}}r}
    \textbf{University of Virginia} & Charlottesville, VA \\
    Laboratory of Computational Neurodynamics & 1999--2003\\
    Degrees: B.A. Cognitive Science; B.A. Mathematics\\
    Advisor: W.~B Levy, Ph.D.\\
    Echols Scholar & \\
  \end{tabular*}
\end{itemize}

\vspace{\stretch{1}}
{\large \textbf{Work}}
\begin{itemize}

\item
  \begin{tabular*}{6.3in}{l@{\extracolsep{\fill}}r}
    \textbf{Johns Hopkins University} & 2009--present\\
    Postdoctoral Fellow\\
    Biomedical Engineering Department, School of Medicine (2013--present)\\
    Zanvyl Krieger Mind/Brain Institute (2009--13)\\
    PIs: James J. Knierim, Ph.D., Kechen Zhang, Ph.D. \\
    Collaborator: H. Tad Blair, Ph.D. \\
  \end{tabular*}

  % \begin{itemize}
  %   \item data-driven modeling efforts to provide explanatory and predictive insights into cortico-hippocampal responses in double cue-rotation experiments; and
  %   \item quantitative data analysis of electrophysiological recordings in rat hippocampus for insight into region-specific dynamics of spatial code formation in altered environments.
  % \end{itemize}

% \item
%   \begin{tabular*}{6in}{l@{\extracolsep{\fill}}r}
%     \textbf{Graduate Research Assistant} & 2005--9\\
%     Center for Theoretical Neuroscience, Columbia University, New York, NY\\
%     Advisor: L.~F. Abbott, Ph.D.\\
%     Collaborators: Isabel Muzzio, Ph.D., Eric R. Kandel, Ph.D. (2006--8) \\
%   \end{tabular*}

  % \begin{itemize}
  %   \item developed Python-based framework for analyzing an extensive dataset of experimental mouse brain recordings, including both spike-timing-based and frequency-domain statistical analyses such as theta-band phaselocking; and
  %   \item added functionality to a previous C codebase for acquisition and analysis to allow flexible cross-platform access to the dataset.
  % \end{itemize}

% \item
%   \begin{tabular*}{6in}{l@{\extracolsep{\fill}}r}
%     \textbf{Graduate Research Assistant} & 2003--5\\
%     Volen Center for Complex Systems, Brandeis University, Waltham, MA \\
%     Advisor: L.~F. Abbott, Ph.D.\\
%     Collaborator: Michael J. Kahana, Ph.D. (2004--6) \\
%   \end{tabular*}
%
% \item
%   \begin{tabular*}{6in}{l@{\extracolsep{\fill}}r}
%     \textbf{Undergraduate Research Assistant} & 2000--3\\
%     Lab. of Computational Neurodynamics, University of Virginia, Charlottesville, VA \\
%     Advisor: W.~B Levy, Ph.D.
%   \end{tabular*}

  % \begin{itemize}
  %   \item maintained and updated C/C++ codebase for highly extensible model of rat CA3 hippocampus;
  %   \item employed CA3 model to research a variety of context-dependent sequence learning paradigms: cued completion, sequence disambiguation, goal finding, and T-maze decision making; and
  %   \item performed Linux system administration for a laboratory network of workstations. 
  % \end{itemize}

% \item 
  % \begin{tabular*}{6in}{l@{\extracolsep{\fill}}r}
  %   \textbf{Technical Assistant} & 2001--2\\
  %   Dept. of Economics, University of Virginia, Charlottesville, VA \\
  %   Supervisor: Charles Holt, Ph.D.
  % \end{tabular*}
  
  % \begin{itemize}
  %   \item maintained and provided basic administration for a database-driven web server used in experimental game theory courses; and
  %   \item setup and maintained a large wireless network of handheld computers for use in classroom-based game theory experiments.
  % \end{itemize}

%\item
  %\begin{tabular*}{6in}{l@{\extracolsep{\fill}}r}
    %\textbf{Research Intern} & Summers 1997/8 \\
    %Center for Molecular Modeling, LSB/CIT, NIH, Bethesda, MD\\
    %PI: R.~Perlstein, Ph.D., P.~Steinbeck, Ph.D.
  %\end{tabular*}

  % \begin{itemize}
  %   \item performed inhibitor ligand analysis for deoxyhypusine synthase using clustering and Monte-Carlo techniques to search for functional conformations ('97); and
  %   \item used LoBoS beowulf cluster to perform molecular dynamics simulations of various hyperthermophilic proteins to discern temperature-dependent structural differences from their mesothermic counterparts ('98).
  %   \end{itemize}
\end{itemize}

\vspace{\stretch{1}}
{\large \textbf{Publications}}

\begin{description}
\item \textbf{Monaco JD}, Rao G, Roth ED, and Knierim JJ. (2014). \href{http://dx.doi.org/10.1038/nn.3687}{\emph{Attentive scanning behavior drives one-trial potentiation of hippocampal place fields}. Nature Neuroscience, 17(5), 725--731. doi:~10.1038/nn.3687}.
\item \textbf{Monaco JD}, Knierim JJ, and Zhang K. (2011). \href{http://dx.doi.org/10.3389/fncom.2011.00039}{\emph{Sensory feedback, error correction, and remapping in a multiple oscillator model of place cell activity}. Frontiers in Computational Neuroscience, 5:39. doi:~10.3389/fncom.2011.00039}.
\item \textbf{Monaco JD} and Abbott LF. (2011). \href{http://dx.doi.org/10.1523/JNEUROSCI.1433-11.2011}{\emph{Modular realignment of entorhinal grid cell activity as a basis for hippocampal remapping}. Journal of Neuroscience, 31(25), 9414--25. doi:~10.1523/jneurosci.1433-11.2011}.
% \item \textbf{Monaco JD}. (2009). \href{http://search.proquest.com/docview/304862872/abstract}{\emph{Models and mechanisms for integrating cortical feature spaces}. Ph.D. Dissertation, Columbia University. ProQuest Publication No. AAT 3393609}.
\item Muzzio IA, Levita L, Kulkarni J, \textbf{Monaco J}, Kentros CG, Stead M, Abbott LF, and Kandel ER. (2009). \href{http://dx.doi.org/10.1371/journal.pbio.1000140}{\emph{Attention enhances the retrieval and stability of visuospatial and olfactory representations in the dorsal hippocampus}. PLoS Biology, 7(6), e1000140. doi:~10.1371/journal.pbio.1000140}.
\item \textbf{Monaco JD}, Abbott LF, and Kahana MJ. (2007). \href{http://dx.doi.org/10.1101/lm.363207}{\emph{Lexico-semantic structure and the recognition word-frequency effect}. Learning \& Memory, 14(3), 204--213. doi:~10.1101/lm.363207}.
\item \textbf{Monaco JD} and Levy WB. (2003). \href{http://dx.doi.org/10.1109/IJCNN.2003.1223655}{\emph{T-maze training of a recurrent CA3 model reveals the necessity of novelty-based modulation of LTP in hippocampal region CA3}. Proceedings International Joint Conference on Neural Networks, 1655--1660. doi:~10.1109/IJCNN.2003.1223655}.
% \begin{itemize}
%   \item IJCNN 2003 Best Student Paper, First Place
% \end{itemize}
\end{description}
\vspace{\stretch{1}}

\pagebreak
\vspace{\stretch{1}}
{\large \textbf{Thesis}}

\begin{description}
\item \textbf{Monaco JD}. (2009). \href{http://search.proquest.com/docview/304862872/abstract}{\emph{Models and mechanisms for integrating cortical feature spaces}. Ph.D. Dissertation, Columbia University. ProQuest Publication No. AAT 3393609}.
\end{description}

{\large \textbf{Presentations}}

\begin{description}
\item \textbf{Talks}
\item[\quad] \textbf{Joseph Monaco}, ``Stopping to look: How attentive scanning behavior reveals the formation of new memories.'' \emph{Retreat Research Talk}. Neuroscience Department, Johns Hopkins University, St. Michaels, MD. September 6, 2014.
\item[\quad] \textbf{Joseph Monaco}, ``Landmark influence: How attention to sensory cues stabilizes and updates the hippocampal cognitive representation of space.'' \emph{Advanced Researcher Seminar}. Krieger Mind/Brain Institute, Johns Hopkins University, Baltimore, MD. April 21, 2014.
\item[\quad] \textbf{Joseph Monaco}, ``Head scans drive the formation and potentiation of place fields during exploration.'' \emph{Data Blitz}. Winter Conference on Learning \& Memory, Park City, UT. January 3, 2014.
\item[\quad] \textbf{Joseph Monaco}, ``Medial versus lateral modes for reconfiguring hippocampal representations.'' \emph{Grid Cells: Formation and Function}. Gatsby Computational Neuroscience Unit, UCL, UK. July 1, 2010.
\item[\quad] \textbf{Joseph Monaco}, ``Rapid spatial map formation and remapping by competing over grid cell inputs.'' \emph{Thesis Seminar}. Columbia University Medical Center, New York, NY. April 10, 2009.
\end{description}

\begin{description}
\item \textbf{Posters}
\item[\quad] \textbf{Monaco JD}, Blair HT, and Zhang K. (2015). Spatial rate/phase correlations in theta cells can stabilize randomly drifting path integrators. COSYNE 2015. Salt Lake City, UT. March~2015.
\item[\quad] \textbf{Monaco J}, Blair HT, and Zhang K. (2014). Spatial rate/phase codes provide landmark-based error correction in a temporal model of theta cells. Society for Neuroscience 2014, 752.07/UU25. Washington, D.C. November~2014.
\item[\quad] Wang CH, Rao G, \textbf{Monaco JD}, Deshmukh SS, and Knierim JJ. (2014). Potentiation of place fields along the CA1 transverse axis by investigatory head-scanning behavior. Society for Neuroscience 2014, 848.15/UU30. Washington, D.C. November~2014.
\item[\quad] \textbf{Monaco J}, Rao G, and Knierim JJ. (2013). Scanning behavior in novel environments promotes \emph{de novo} formation of hippocampal place fields in rats. Society for Neuroscience 2013, 670.07/JJJ44. San Diego, CA. November~2013.
\item[\quad] \textbf{Monaco J}, Rao G, and Knierim JJ. (2012). Hippocampal LFP during rodent head-scanning behavior: Theta and sharp-wave ripples. Society for Neuroscience 2012, 812.14/FFF24. New Orleans, LA. October 2012.
\item[\quad] \textbf{Monaco J}, Rao G, and Knierim JJ. (2011). Hippocampal place cell firing during head-scanning movements is associated with the formation of new place fields. Society for Neuroscience 2011, 97.13/WW28. Washington, D.C. November~2011.
\item[\quad] Rao G, \textbf{Monaco J}, and Knierim JJ. (2011). Environmental novelty promotes rodent head-scanning behavior linked to enhanced entorhinal activity. Society for Neuroscience 2011, 97.12/WW27. Washington, D.C. November~2011.
\item[\quad] \textbf{Monaco JD}, Zhang K, Blair HT and Knierim JJ. (2010). Cue-based feedback enables remapping in a multiple oscillator model of place cell activity. COSYNE 2010. Salt Lake City, UT. February~2010.
\item[\quad] \textbf{Monaco JD} and Abbott LF. (2009). Dynamic hippocampal remapping using recurrent inhibition on realigning grid cell inputs. COSYNE 2009. Salt Lake City, UT. February~2009.
\item[\quad] \textbf{Monaco JD}, Muzzio IA, Levita L and Abbott LF. (2006). Entorhinal input and global remapping of hippocampal place fields. CNS*2006. Edinburgh, UK. July~2006.
\item[\quad] \textbf{Monaco JD} and Abbott LF. (2006). Entorhinal input and the remapping of hippocampal place fields. COSYNE 2006. Salt Lake City, UT. March~2006.
\item[\quad] \textbf{Monaco JD} and Levy WB. (2003). T-maze training of a recurrent CA3 model reveals the necessity of novelty-based modulation of LTP in hippocampal region CA3. IJCNN 2003. Portland, OR. July~2003.
\item[\quad] \textbf{Monaco JD} and Perlstein RP. (1997). Monte-Carlo analysis of deoxyhypusine synthase inhibitor ligand conformations. NIH Poster Day. Bethesda, MD. August~1997. 
\end{description}

{\large \textbf{Professional}}
\begin{itemize}
  % \item \emph{Ad-hoc Reviewer:} Neuroscience, IEEE/INNS Neural Networks, Neurocomputing, Biological Cybernetics
  \item COSYNE 2016, \emph{Review committee}
  \item IEEE Transactions in Biomedical Engineering, \emph{Ad-hoc reviewer}
  \item IEEE/INNS Neural Networks, \emph{Ad-hoc reviewer}
  \item Neuroscience, \emph{Ad-hoc reviewer}
  \item Neurocomputing, \emph{Ad-hoc reviewer}
  \item Biological Cybernetics, \emph{Ad-hoc reviewer}
  \item Society for Neuroscience, \emph{Postdoc Member (2011--present)}
\end{itemize}
  
{\large \textbf{Mentoring/Teaching}}
\begin{itemize}
  \item Chia-Hsuan Wang, \emph{Graduate student}, \emph{Johns Hopkins University (2013--present)}
  \item Manning Zhang, \emph{Undergraduate student}, \emph{Shanghai Jiao Tong University (Summer 2014)}
  \item Teaching Assistant, \emph{Biology Laboratory}, \emph{Brandeis University (Spring 2005)}
  \item Teaching Assistant (for Eve Marder), \emph{Introduction to Neuroscience}, \emph{Brandeis University (Fall 2004)}
\end{itemize}

% {\large \textbf{Grantwriting}}
% \begin{itemize}
%   \item NIH NRSA (F-32) Application, NINDS/NIMH: \emph{Behavioral Coordination of Entorhinal-Hippocampal Activity for Real-Time Sensory Updating of Spatial Memory}
%   \begin{itemize}
%     \item Initial submission (2010): 21st percentile, unfunded
%     \item Amended resubmission (2011): unfunded
%   \end{itemize}
% \end{itemize}

{\large \textbf{Awards}}
\begin{itemize}
  \item IJCNN Student Paper Competition, First Place (2003)
  \item University of Virginia John A. Harrison III Undergraduate Research Award (2002)
  \item Echols Scholar at the University of Virginia (1999--2003) 
  \item Pre-university: Maryland Distinguished Scholar (1999), Johns Hopkins Mathematics Competition (2nd Place Individual Calculus, 1999), National Merit Scholarship Commended Student, AP Scholar with Distinction, State of Maryland Merit Scholastic Award
\end{itemize}

% {\large \textbf{Technical Proficiency}}
% \begin{description}
%   \item[\quad Platforms] OSX, GNU/Linux, Windows, Matlab
%   \item[\quad Languages] Python, C, C++, \LaTeX
% \end{description}
\end{document}

