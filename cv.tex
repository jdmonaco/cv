%! TEX program = xelatex
%
% Curriculum vitae for Joseph D. Monaco
%
\documentclass[10pt]{article}

\usepackage{fullpage}
\usepackage[usenames,dvipsnames]{color}
\usepackage[hidelinks,xetex]{hyperref}
\usepackage{multirow}
\usepackage{sectsty}
\usepackage{tabu}
\usepackage{longtable}
\usepackage{soul}
\usepackage{fontspec}

% Page geometry formatting
\raggedbottom
\raggedright
\textheight=9in
\setlength{\tabcolsep}{0in}
\addtolength{\footskip}{-.15in}
\addtolength{\voffset}{-.6in}
\addtolength{\textheight}{1in}
\addtolength{\hoffset}{-.25in}
\addtolength{\textwidth}{.25in}

% Define colors
\definecolor{hopkinsblue}{RGB}{0,48,130}
\definecolor{lightblue}{rgb}{.88,.92,.97}
\definecolor{lightgold}{rgb}{1,.80,0}
\definecolor{lightred}{rgb}{1,.3,.2}
\definecolor{lightgray}{rgb}{.80,.80,.80}
\definecolor{dimgray}{rgb}{.50,.50,.50}
\definecolor{darkgray}{rgb}{.30,.30,.30}

% Set up highlighting and underlining
\sethlcolor{lightblue}
\setul{0.13ex}{}

% Choose font package
\setmainfont[Ligatures=TeX]{Helvetica Neue LT Std} 

% Section section* formatting
\sectionfont{\large\color{hopkinsblue}\bfseries}
\subsectionfont{\vspace{-1.5ex}\color{hopkinsblue}\normalsize\bfseries}

% Formatting macros
\newcommand{\itemtitle}[1]{{\color{hopkinsblue}\ul{#1}}}
\newcommand{\unpubtitle}[1]{{\color{hopkinsblue} #1}}
\newcommand{\itemnote}[1]{
  \begin{description}
    \item[$\rightarrow$] \hspace{.09in}{\color{darkgray}\it #1}
  \end{description}
}
\newcommand{\joehl}[1]{\hl{\textbf{#1}}}
\newcommand{\doi}[1]{{\color{darkgray}doi:}~{\color{dimgray}\texttt{#1}}}
\newcommand{\arxiv}[1]{\emph{ArXiv Preprint}.
  {\color{darkgray}arxiv:}~{\color{dimgray}\texttt{#1}}}
\newcommand{\arxivlink}[1]{\href{https://arxiv.org/abs/#1}
  {{\color{darkgray}arxiv:}~{\color{dimgray}\texttt{#1}}}}
\newcommand{\biorxivlink}[1]{\href{https://dx.doi.org/#1}
  {{\color{darkgray}biorxiv:}~{\color{dimgray}\texttt{#1}}}}
\newcommand{\aurl}[1]{{\color{dimgray}\texttt{#1}}}
\newcommand{\researchnote}[1]{
  \begin{description}
    \item[] {\hspace{2.2ex}\color{darkgray} #1}
  \end{description}
}
\newcommand{\researchactivity}[4]{
  \begin{minipage}[t]{\textwidth}
    \begin{tabular}{@{\hspace{2ex}}l>{\raggedright\arraybackslash}p{.8\textwidth}}
      \makebox[1.2in][l]{#1} & \textbf{#2:}
      ``\unpubtitle{#3}'' 
    \end{tabular}
  \researchnote{\hspace{1ex} #4}
  \end{minipage}
  \medbreak
}
\newcommand{\whitepaper}[4]{
  \begin{minipage}[t]{\textwidth}
    \begin{tabular}{@{\hspace{2ex}}l>{\raggedright\arraybackslash}p{.8\textwidth}}
      \makebox[1.2in][l]{#1} & \textbf{White Paper:} #2.
      ``\unpubtitle{#3}'' 
    \end{tabular}
  \researchnote{\hspace{1ex} #4}
  \end{minipage}
  \medbreak
}
\newcommand{\lefttabline}[3]{\hspace{2ex}\makebox[#1][l]{#2} #3\\}

% PDF document info and setup
\hypersetup{
  baseurl=https://jdmonaco.com/cv-monaco.pdf,
  pdftitle=Curriculum Vitae for Joseph D. Monaco,
  pdfauthor=Joseph D. Monaco,
  pdfdisplaydoctitle=true,
  pdfpagemode=UseThumbs,
  pdfstartview=FitV,
  pdfpagelayout=TwoColumnLeft,
  pdftoolbar=false,
  pdfwindowui=false,
  pdfcenterwindow=true,
}

% "See section ..." macro
\newcommand{\pdot}[1]{[\textcolor{hopkinsblue}{p.\pageref{sec:#1}}]}
\newcommand{\nameonpdot}[1]{\textcolor{hopkinsblue}{\emph{\nameref{sec:#1}} on p.\pageref{sec:#1}}}
\newcommand{\nameonpdottwo}[2]{\textcolor{hopkinsblue}{\emph{\nameref{sec:#1}} on
  p.\pageref{sec:#1}} and \textcolor{hopkinsblue}{\emph{\nameref{sec:#2}} on p.\pageref{sec:#2}}}

% Typesetting
\setlength{\parskip}{0em}
\newcommand{\newsection}[2]{%
  \section*{#1}
  \vspace{-.125in}
  \hrule
  \vspace{.22in}
  \label{sec:#2}
}

\begin{document}

%\pagestyle{empty}

%! TEX program = xelatex
{\fontspec{Verdana}\small
\begin{tabular*}{6.675in}{c@{\extracolsep{\fill}}rl}
  \hline\\[0.02in]
  \textbf{\LARGE\color{hopkinsblue} Joseph Daniel Monaco, Ph.D.} 
                                                       & \textsc{Email}          & \href{mailto:jmonaco@jhu.edu}{\texttt{jmonaco@jhu.edu}} \\
  \multirow{2}{*}{\large }                             & \textsc{Web}            & \href{http://jdmonaco.com/}{\texttt{jdmonaco.com}} \\
  {\small Johns Hopkins University School of Medicine} & \textsc{ORCID}          & \href{http://jdmonaco.com/orcid}{\texttt{0000-0003-0792-8322}} \\
  {\small 720 Rutland Avenue, 407 Traylor}             & \textsc{GitHub}         & \href{https://github.com/jdmonaco?tab=repositories}{\texttt{github.com/jdmonaco}} \\
  {\small Baltimore, MD, 21205, USA}                   & \textsc{Google Scholar} & \href{http://jdmonaco.com/google-scholar}{\texttt{gceOLZEAAAAJ}} \\[0.1in]
  \hline
\end{tabular*}
}\\[0.1in]


\newsection{Education}{edu}

\begin{itemize}
  \item
    \begin{tabular*}{6.3in}{l@{\extracolsep{\fill}}r}
      \textbf{Columbia University} & New York, NY \\
      Department of Neurobiology \& Behavior & 2005--2009 \\
      Center for Theoretical Neuroscience\\
      Degrees: Ph.D. (2009); M.Phil. (2008); M.A. (2006) \\
      Advisor: Larry~Abbott (\textcolor{hopkinsblue}{\href{mailto:lfa2103@columbia.edu}{lfa2103@columbia.edu}})\\
    \end{tabular*}
  \item
    \begin{tabular*}{6.3in}{l@{\extracolsep{\fill}}r}
      \textbf{Brandeis University} & Waltham, MA \\
      Department of Biology & 2003--2005\\
      Volen Center for Complex Systems\\
      Graduate Program in Neuroscience, \textit{\ul{Continued at Columbia University}} \\
      Advisor: Michael Kahana (\textcolor{hopkinsblue}{\href{mailto:kahana@psych.upenn.edu}{kahana@psych.upenn.edu}}) \\ 
      %Advisors: Michael Kahana (\textcolor{hopkinsblue}{\href{mailto:kahana@psych.upenn.edu}{kahana@psych.upenn.edu}}); \\ 
      %\hspace{0.6in} Larry Abbott (\textcolor{hopkinsblue}{\href{mailto:lfa2103@columbia.edu}{lfa2103@columbia.edu}}) \\
    \end{tabular*}
  \item
    \begin{tabular*}{6.3in}{l@{\extracolsep{\fill}}r}
      \textbf{University of Virginia} & Charlottesville, VA \\
      Laboratory of Computational Neurodynamics & 1999--2003\\
      Degrees: B.A.~Mathematics; B.A.~Cognitive Science; Minor, Philosophy \\
      Advisor: William (`Chip') Levy (\textcolor{hopkinsblue}{\href{mailto:wbl@virginia.edu}{wbl@virginia.edu}})\\
      %Echols Scholar \\
    \end{tabular*}
\end{itemize}

\newsection{Positions}{work}

\begin{itemize}
  \item
    \begin{tabular*}{6.3in}{l@{\extracolsep{\fill}}r}
      \textbf{National Institutes of Health} & Rockville, MD\\
      National Institute of Neurological Disorders and Stroke & 2023--present \\
      Office of the BRAIN Director \\
      Scientific Program Manager \\
    \end{tabular*}

  \item
    \begin{tabular*}{6.3in}{l@{\extracolsep{\fill}}r}
      \textbf{SelfMotion Labs} (\emph{stealth mode}) & Baltimore, MD\\
      Founder \& Chief Scientist & 2022--present\\
    \end{tabular*}

  \item
    \begin{tabular*}{6.3in}{l@{\extracolsep{\fill}}r}
      \textbf{Johns Hopkins University School of Medicine} & Baltimore, MD\\
      Research Associate (Faculty) & 2019--2022\\
      Postdoctoral Fellow & 2013--2019\\
      Department of Biomedical Engineering\\
      Sponsor: Kechen Zhang (\textcolor{hopkinsblue}{\href{mailto:kzhang4@jhmi.edu}{kzhang4@jhmi.edu}})\\
    \end{tabular*}

  \item
    \begin{tabular*}{6.3in}{l@{\extracolsep{\fill}}r}
      \textbf{Johns Hopkins University} & Baltimore, MD\\
      Postdoctoral Fellow & 2009--2013\\
      Zanvyl Krieger Mind/Brain Institute\\
      PI: James J. Knierim (\textcolor{hopkinsblue}{\href{mailto:jknierim@jhu.edu}{jknierim@jhu.edu}})\\
    \end{tabular*}
\end{itemize}

% Insert the text for all the publication lists:
\newsection{Publications}{pubs}

\renewcommand{\itemnote}[1]{}
\section*{Journal Publications}

\begin{description}
  \item Buckley E, \joehl{Monaco JD}, Schultz KM, Chalmers R, Hadzic A,
    Zhang K, Hwang GM, and Carr MD. (\emph{\color{lightred}Under review}). \itemtitle{An
      interdisciplinary approach to high school curriculum development: Swarming
    Powered by Neuroscience}. \emph{Frontiers in Education}.
  \item \joehl{Monaco JD}, Rajan K, and Hwang GM. (\emph{\color{lightred}In revision}).
      \itemtitle{A brain basis of dynamical intelligence for AI and computational
    neuroscience}. \emph{Nature Machine Intelligence}.
  \item \href{https://doi.org/10.1007/s00422-020-00823-z}
    {\joehl{Monaco JD}, Hwang GM, Schultz KM, and Zhang K. (2020).
    \itemtitle{Cognitive swarming in complex environments with attractor
      dynamics and oscillatory computing}. \emph{Biological Cybernetics}, 114,
    269--284. \doi{10.1007/s00422-020-00823-z}}.
  \item \begin{samepage}\href{https://doi.org/10.1016/j.cub.2020.01.083} 
      {Wang CH, \joehl{Monaco JD}, and Knierim JJ. (2020). \itemtitle{Hippocampal
        place cells encode local surface texture boundaries}. \emph{Current Biology},
      30, 1--13. \doi{10.1016/j.cub.2020.01.083}}.
  \itemnote{I mentored the first author in data analysis of rat behavior and
      single-unit recordings, developed the software toolchain used to conduct the
    analyses, and provided intellectual guidance.}\end{samepage}
  \item \href{https://doi.org/10.1371/journal.pcbi.1006741}
    {\joehl{Monaco JD}, De Guzman RM, Blair HT, and Zhang K. (2019).
    \itemtitle{Spatial synchronization codes from coupled rate-phase
      neurons}. \emph{PLOS Computational Biology}, 15(1), e1006741.
    \doi{10.1371/journal.pcbi.1006741}}.
  \item \href{https://www.cell.com/cell/fulltext/S0092-8674(18)31228-5}
    {Tabuchi M, \joehl{Monaco JD}, Duan G, Bell BJ, Liu S, Zhang K, and
      Wu MN. (2018). \itemtitle{Clock-generated temporal codes determine
      synaptic plasticity to control sleep}. \emph{Cell}, 175(5), 1213--27.
    \doi{10.1016/j.cell.2018.09.016}}.
  \itemnote{I developed two modeling strategies for the Wu lab’s circadian clock
      neuron experiments in \emph{Drosophila}. My generative statistical model was
      integrated into stimulation protocols as a timing control for behavioral
      results, and my mechanistic molecular/neuronal model explained observed trends
      and made predictions corroborated by the data. My results or contributions are
    featured in 3/7 main figures and 3/6 supplementary figures.}
  \item \href{http://doi.org/10.1038/nn.3687}
    {\joehl{Monaco JD}, Rao G, Roth ED, and Knierim JJ. (2014).
    \itemtitle{Attentive scanning behavior drives one-trial potentiation of
      hippocampal place fields}. \emph{Nature Neuroscience}, 17(5), 725--731.
    \doi{10.1038/nn.3687}}.
  \item \href{http://doi.org/10.3389/fncom.2011.00039}
    {\joehl{Monaco JD}, Knierim JJ, and Zhang K. (2011). \itemtitle{Sensory
        feedback, error correction, and remapping in a multiple oscillator model of
      place cell activity}. \emph{Frontiers in Computational Neuroscience}, 5:39.
    \doi{10.3389/fncom.2011.00039}}.
  \item \href{http://doi.org/10.1523/JNEUROSCI.1433-11.2011}
    {\joehl{Monaco JD} and Abbott LF. (2011). \itemtitle{Modular
        realignment of entorhinal grid cell activity as a basis for hippocampal
      remapping}. \emph{Journal of Neuroscience}, 31(25), 9414--25.
    \doi{10.1523/jneurosci.1433-11.2011}}.
  \item \href{http://doi.org/10.1371/journal.pbio.1000140}
    {Muzzio IA, Levita L, Kulkarni J, \joehl{Monaco J}, Kentros CG, Stead
      M, Abbott LF, and Kandel ER. (2009). \itemtitle{Attention enhances the
        retrieval and stability of visuospatial and olfactory representations
      in the dorsal hippocampus}. \emph{PLOS Biology}, 7(6), e1000140.
    \doi{10.1371/journal.pbio.1000140}}.
  \itemnote{I contributed oscillatory power analyses and group-level statistical
      analyses of spiking and bursting for odor vs.\ visuospatial tasks in
    single-unit hippocampal recordings from freely-moving mice.}
  \item \href{http://doi.org/10.1101/lm.363207}
    {\joehl{Monaco JD}, Abbott LF, and Kahana MJ. (2007).
    \itemtitle{Lexico-semantic structure and the recognition
      word-frequency effect}. \emph{Learning \& Memory}, 14(3), 204--213.
    \doi{10.1101/lm.363207}}.
\end{description}

\section*{Conference Papers}

\begin{description}
  \item \href{https://www.jhuapl.edu/Content/techdigest/pdf/V35-N04/35-04-Hwang.pdf}
    {Hwang GM, Schultz KM, \joehl{Monaco JD}, and Zhang K. (2021).
    \itemtitle{Neuro-Inspired Dynamic Replanning in Swarms—Theoretical
        Neuroscience Extends Swarming in Complex Environments}. \emph{Johns Hopkins
    APL Technical Digest}, 35, 443--447.}
  \item \href{https://doi.org/10.1117/12.2518966}
    {\joehl{Monaco JD}, Hwang GM, Schultz KM, and Zhang K. (2019).
    \itemtitle{Cognitive swarming: An approach from the theoretical neuroscience
        of hippocampal function}. \emph{Proceedings of SPIE (International society
      for optics and photonics) Defense \& Commercial Sensing}. Micro- and
      Nanotechnology Sensors, Systems, and Applications XI, 109822D, 1--10.
    \doi{10.1117/12.2518966}}.
  \item \href{http://doi.org/10.1109/IJCNN.2003.1223655}
    {\joehl{Monaco JD} and Levy WB. (2003). \itemtitle{T-maze training
        of a recurrent CA3 model reveals the necessity of novelty-based
        modulation of LTP in hippocampal region CA3}. \emph{Proceedings of
      International Joint Conference on Neural Networks}, 1655--1660.
    \doi{10.1109/IJCNN.2003.1223655}}.
  \itemnote{This paper received First Place in the IJCNN Student Paper
    Competition.}
\end{description}

\section*{Preprints}

\begin{description}
  \item \href{https://arxiv.org/abs/2109.05545}{Buckley E, \joehl{Monaco JD},
      Schultz KM, Chalmers R, Hadzic A, Zhang K, Hwang GM, and Carr MD. (2021).
    \itemtitle{An interdisciplinary approach to high school curriculum development:
    Swarming Powered by Neuroscience}. \arxiv{2109.05545}}.
  \item \href{https://arxiv.org/abs/2105.07284}
    {\joehl{Monaco JD}, Rajan K, and Hwang GM. (2021). \itemtitle{A brain
      basis of dynamical intelligence for AI and computational neuroscience}.
    \arxiv{2105.07284}}.
  \item \href{https://arxiv.org/abs/2003.13825}
    {Levenstein D, Alvarez VA, Amarasingham A, Azab H, Gerkin RC, Hasenstaub
      A, Iyer R, Jolivet RB, Marzen~S, \joehl{Monaco JD}, Prinz AA, Quraishi
      S, Santamaria F, Shivkumar S, Singh MF, Stockton DB, Traub R, Rotstein
      HG, Nadim F, and Redish AD. (2020). \itemtitle{On the role of theory and
    modeling in neuroscience}. \arxiv{2003.13825}}.
  \item \href{https://arxiv.org/abs/1909.06711}
    {\joehl{Monaco JD}, Hwang GM, Schultz KM, and Zhang K. (2019).
    \itemtitle{Cognitive swarming in complex environments with attractor
    dynamics and oscillatory computing}. \arxiv{1909.06711}}.
  \item \href{http://doi.org/10.1101/764282}
    {Wang CH, \joehl{Monaco JD}, and Knierim JJ. (2019). \itemtitle{Hippocampal
      place cells encode local surface texture boundaries}. \emph{bioRxiv}.
    \doi{10.1101/764282}}.
  \item \href{http://dx.doi.org/10.1101/211458}
    {\joehl{Monaco JD}, Blair HT, and Zhang K. (2017). \itemtitle{Spatial theta
        cells in competitive burst synchronization networks: Reference frames from
    phase codes}. \emph{bioRxiv}. \doi{10.1101/211458}}.
\end{description}

\section*{Thesis}

\begin{description}
  \item \href{http://search.proquest.com/docview/304862872/abstract}
    {\joehl{Monaco JD}. (2009). \itemtitle{Models and mechanisms for integrating
      cortical feature spaces}. Doctoral Dissertation, Columbia University, New
    York. \emph{ProQuest Publication No. AAT 3393609}}.
  \itemnote{\href{https://jdmonaco.com/files/monaco-phdthesis-2009.pdf}{Click
    here for the original version with high-quality color figures.}}
\end{description}


%\renewcommand{\itemnote}[1]{
%\begin{description}
%\item[$\rightarrow$] \hspace{.09in}{\color{darkgray}\it #1}
%\end{description}
%}


\smallskip
\newsection{Websites}{web}

\begin{description}
  \item \href{https://www.ninds.nih.gov/about-ninds/who-we-are/staff-directory/joseph-monaco}
    {``\itemtitle{Joseph Monaco, Ph.D. -- Scientific
      Program Manager, NIH BRAIN Initiative}.'' Website.
    \aurl{https://www.ninds.nih.gov/about-ninds/who-we-are/staff-directory/joseph-monaco}}
  \item \href{https://jdmonaco.com/}
    {``\itemtitle{Briefly Balanced: Theoretical neuroscience of behavior in
    space and time}.'' Website. \aurl{https://jdmonaco.com/}}
  \item \href{https://www.ncbi.nlm.nih.gov/pubmed/?term=monaco_jd+OR+(monaco_j+AND+muzzio_ia)}
    {\itemtitle{PubMed Listing}. Website.
    \aurl{https://www.ncbi.nlm.nih.gov/pubmed/?term=monaco\_jd+OR+(monaco\_j+AND+muzzio\_ia)}}
  \item \href{https://jdmonaco.com/google-scholar}
    {\itemtitle{Google Scholar}. Website. \aurl{https://scholar.google.com/citations?hl=en\& user=gceOLZEAAAAJ\&view\_op=list\_works\&sortby=pubdate}}
  \item \href{https://github.com/jdmonaco}
    {\itemtitle{GitHub Overview}. Website. \aurl{https://github.com/jdmonaco}}
  \item \href{https://twitter.com/j_d_monaco}
    {\itemtitle{Twitter Feed}. Social Media. \aurl{https://twitter.com/j\_d\_monaco}}
\end{description}


\renewcommand{\itemnote}[1]{
\begin{description}
\item[$\rightarrow$] \hspace{.09in}{\color{darkgray}\it #1}
\end{description}
}

\smallskip
\newsection{Funding Awards}{grants}

\begin{itemize}\label{sec:nsfaward}
  \item \href{https://www.nsf.gov/awardsearch/showAward?AWD_ID=1835279&HistoricalAwards=false}
    {\itemtitle{NCS-FO: Spatial intelligence for swarms based on hippocampal
    dynamics}}\hspace{\stretch{1}}2018--2021
    \begin{itemize}
      \item NSF\slash NCS FOUNDATIONS (BRAIN Initiative) Award No.~1835279: \$862K/\$997K (Direct/Total)
      \item \textbf{Lead PI:} Kechen Zhang
      \item \textbf{Co-PIs, JHUAPL}: Grace Hwang, Robert W. Chalmers, Kevin
        Schultz, and M. Dwight Carr
      \item \textbf{Research Associate (FY19)/Co-PI (FY20--FY21): \joehl{Joseph D. Monaco}}
    \end{itemize}
    \itemnote{I co-developed this project and co-wrote the proposal with a
      JHUAPL colleague (see \nameonpdot{nsfgrant}). As a Research Associate
    faculty at JHU as of FY20, my project role was promoted to co-PI.}
    \itemnote{This project was the basis for JHUAPL's 2020 ranking
      as a top workplace for innovation: \\
      \href{https://www.fastcompany.com/90529833/best-workplaces-for-innovators-2020-johns-hopkins-university-apl}
      {``\itemtitle{Johns Hopkins University APL is one of Fast Company’s Best Workplaces
      for Innovators}.'' (July 29, 2020). \emph{Fast Company}.}}
\end{itemize}

\begin{itemize}\label{sec:nihaward}
  \item \href{https://projectreporter.nih.gov/project_info_description.cfm?aid=9652210&icde=42555668&ddparam=&ddvalue=&ddsub=&cr=2&csb=default&cs=ASC&pball=}
    {\itemtitle{Spiking network models of sharp-wave ripple sequences with\\
    gamma-locked attractor dynamics}}\hspace{\stretch{1}}2018--2020
    \begin{itemize}
      \item NIH/NINDS R03 Award No.~NS109923: \$50K/\$82K (Direct/Total)
      \item \textbf{PI:} Kechen Zhang
      \item \textbf{Research Associate:} \joehl{Joseph D. Monaco}
    \end{itemize}
    \itemnote{I conceived this project, generated preliminary data, and wrote
      the proposal (see \nameonpdot{nihgrant}). As a Postdoctoral Fellow, JHU
    policy precluded a PI role.}
\end{itemize}

\begin{itemize}\label{sec:sliaward}
  \item \href{https://jdmonaco.com/files/ScienceOfLearning-2016-award-brief.pdf}
    {\itemtitle{Learning to explore paths through space}}\hspace{\stretch{1}}2016--2018
    \begin{itemize}
      \item JHU/Science of Learning Institute (SLI) Award: \$150K
      \item \textbf{PI:} Kechen Zhang
      \item \textbf{Co-PI:} David J.~Foster (now at UC Berkeley)
      \item \textbf{Research Associate:} \joehl{Joseph D.~Monaco}
    \end{itemize}
    \itemnote{I conceived this project, initiated the collaboration between the
    Zhang and Foster labs, and wrote the proposal (see \nameonpdot{sligrant}).
    As a Postdoctoral Fellow, JHU policy precluded a PI role.}
\end{itemize}

\renewcommand{\itemnote}[1]{}


\smallskip
\newsection{Community Coverage of My Work}{press}

\subsection*{News \& Views}

\begin{itemize}
  \item \href{https://dx.doi.org/10.1016/j.cub.2020.02.085}
    {Place R, Nitz DA. (2020). \itemtitle{Cognitive Maps: Distortions of the Hippocampal 
    Space Map Define Neighborhoods}. \emph{Current Biology}, 30(8): R340--R342.}
  \item \href{https://dx.doi.org/10.1016/j.cell.2018.10.047}
    {Colwell CS, Donlea J. (2018). \itemtitle{Temporal coding of sleep}. \emph{Cell}, 175(5): 1177--9.}
  \item \href{https://dx.doi.org/10.1038/nn.3700}
    {Dupret D, Csicsvari J. (2014). \itemtitle{Turning heads to remember
    places}. \emph{Nature Neuroscience}, 17(5): 643--44.}
\end{itemize}

\subsection*{Post-Publication Reviews}

\begin{itemize}
  \item \href{https://facultyopinions.com/prime/718333676#eval793494783}
    {Moser E, Rowland D. (May 12, 2014). ``\itemtitle{This exciting study finds
        an unexpected relationship between exploratory head scanning behavior
      and the development of new place fields in the rat hippocampus...}”
    \emph{F1000/Faculty Opinions}.}
  \item \href{https://facultyopinions.com/prime/718333676#eval793493493}
    {Maler L. (April 10, 2014). ``\itemtitle{This elegant and original study has
        demonstrated a strong link between the neural activity of hippocampal pyramidal
        neurons (PNs) during head scanning behavior and their subsequent acquisition of
    a new place field...}'' \emph{F1000/Faculty Opinions}.}
  \item \href{https://facultyopinions.com/prime/11553956}
    {Giocomo L, Moser E. (June 29, 2011) ``\itemtitle{This paper presents an
        interesting computational model which utilizes grid-cell modularity to generate
    robust remapping...}'' \emph{F1000/Faculty Opinions}.}
\end{itemize}


\smallskip
\newsection{Media Releases}{pr}

\begin{description}
  \item \href{https://www.jhuapl.edu/NewsStory/220919-stem-teaching-tool-recognized-ieee-isec-2022}{
      ``\itemtitle{Novel Teaching Tool Earns Hopkins Collaborators
      International Conference Honors}.'' JHUAPL Press Office. Sept 19, 2022.
      \aurl{https://www.jhuapl.edu/NewsStory/220919-stem-teaching-tool-recognized- ieee-isec-2022}}
  \item \href{https://kavlijhu.org/news/32} {``\itemtitle{Can
        robotic swarms navigate using learning rules devised for brain
      dynamics?}'' JHU/Kavli Neuroscience Discovery Insitute. May 3, 2020.
    \aurl{https://kavlijhu.org/news/32}}
  \item \href{https://www.youtube.com/watch?v=ic4zEgVMSsA}
    {``\itemtitle{Swarmalators}.'' JHUAPL Press Office. May 9, 2019.
    \aurl{https://www.youtube.com/watch?v=ic4zEgVMSsA}}
  \item \href{https://hub.jhu.edu/2018/10/02/brain-robot-swarms-study/}
    {``\itemtitle{What do animal brains have in common with swarms of robots?
      Maybe more than you think}.'' Geoff Brown/JHU Office of Communications. Oct 2,
    2018. \aurl{https://hub.jhu.edu/2018/10/02/brain-robot-swarms-study/}}
  \item \href{https://www.jhuapl.edu/PressRelease/181001}
    {``\itemtitle{Do Robot Swarms Work Like Brains?}'' JHUAPL Press Office. October 1, 2018.
    \aurl{https://www.jhuapl.edu/PressRelease/181001}}
  \item \href{https://hub.jhu.edu/2014/04/14/memory-brain-place-cells/}
    {``\itemtitle{Where does a memory begin? Johns Hopkins neuroscientists think they
      know}.'' Latarsha Gatlin/JHU Office of Communications. April 14, 2014.
    \aurl{https://hub.jhu.edu/2014/04/14/memory-brain-place-cells/}}
  \item \href{https://www.youtube.com/watch?v=Jm8OiLJqKJQ}
    {``\itemtitle{Johns Hopkins Researchers Probe Mysteries of
      the Brain}.'' JHU Office of Communications. April 14, 2014.
    \aurl{https://www.youtube.com/watch?v=Jm8OiLJqKJQ}}
\end{description}


\smallskip
\newsection{Recognition}{awards}

\lefttabline{0.8in}{2022}{IEEE/ISEC Best Paper Award, First Place}
\lefttabline{0.8in}{2003}{International Joint Conference on Neural Networks (IJCNN) Student Paper Award, First Place}
\lefttabline{0.8in}{2002}{U.Va. John A. Harrison III Undergraduate Research Award}
\lefttabline{0.8in}{1999--2003}{U.Va. Echols Scholar}
\lefttabline{0.8in}{1999}{State of Maryland Merit Scholastic Award}
\lefttabline{0.8in}{1999}{AP Scholar with Distinction}
\lefttabline{0.8in}{1999}{National Merit Scholarship Commended Student}
\lefttabline{0.8in}{1999}{Johns Hopkins Mathematics Competition (2nd Place, Individual Calculus)}
\lefttabline{0.8in}{1999}{Maryland Distinguished Scholar}


\smallskip
\newsection{Professional Service \& Scientific Review}{service}

\subsection*{Journals --- Editorial Boards}\label{sec:edboards}

\lefttabline{1.1in}{2023--present}{Frontiers in Computational Neuroscience}
\lefttabline{1.1in}{2023--present}{Frontiers in Neural Circuits}

\subsection*{Journals --- Peer Review}\label{sec:review}

\lefttabline{0.8in}{2023}{Nature Communications}
\lefttabline{0.8in}{2021}{PLOS Computational Biology}
\lefttabline{0.8in}{2021}{Nature Machine Intelligence}
\lefttabline{0.8in}{2020}{Neuroscience and Biobehavioral Reviews}
\lefttabline{0.8in}{2020}{Scientific Reports}
\lefttabline{0.8in}{2019}{eLife}
\lefttabline{0.8in}{2019}{Hippocampus}
\lefttabline{0.8in}{2018--2019}{Neuron}
\lefttabline{0.8in}{2018}{Neural Computation (including as `Communicator')}
\lefttabline{0.8in}{2018}{PLOS ONE}
\lefttabline{0.8in}{2017}{PeerJ}
\lefttabline{0.8in}{2015}{IEEE Transactions in Biomedical Engineering}
\lefttabline{0.8in}{2012--2020}{IEEE Neural Networks}
\lefttabline{0.8in}{2012}{Biological Cybernetics}
\lefttabline{0.8in}{2012}{Neurocomputing}
\lefttabline{0.8in}{2012}{Neuroscience}

\subsection*{Funding Agencies}\label{sec:programsvc}

\lefttabline{0.8in}{2022}{NSF CAREER Ad-Hoc Reviewer}
\lefttabline{0.8in}{2022}{NSF Emerging Frontiers in Research and Innovation (EFRI) Preliminary Review (3 Panels)}
\lefttabline{0.8in}{2022--2023}{NSF EFRI Final-Round Review (1 Panel, 1 Ad Hoc)}
\lefttabline{0.8in}{2020--2022}{NSF EFRI Program Development, Extramural Contributor}
\lefttabline{0.8in}{2014}{IARPA Program Development, Extramural Contributor}

\subsection*{Conferences}\label{sec:confsvc}

\lefttabline{0.8in}{2020--2021}{Cosyne, Review Committee}
\lefttabline{0.8in}{2016}{Cosyne, Review Committee}

\subsection*{Societies}\label{sec:societies}

\lefttabline{0.8in}{2022--2024}{American Physical Society, Regular Member}
\lefttabline{0.8in}{2011--2022}{Society for Neuroscience, Postdoc/Regular Member}


\smallskip
\newsection{Workshops, Seminars, and Talks}{talks}

\smallskip
\subsection*{International}
\label{sec:talksintl}

\begin{longtable}{@{\hspace{0.1in}}l>{\raggedright\arraybackslash}p{.82\textwidth}}
  10/13/2023 & \href{https://jdmonaco.com/files/monaco-IAPCT-2023-slides.pdf}
    {``\itemtitle{Cognitive-narrative dynamics of self-perspective control
    across the lifespan}.''} \emph{Invited Talk}. 33rd Annual International
    Association for Perceptual Control Theory (IAPCT) Conference, 
  \href{https://www.iapct.org/uncategorized/utc-4-boston-time-zone/}
  {Session 7 on \unpubtitle{\emph{Consciousness and the Self}}}, Virtual
  [\href{https://jdmonaco.com/files/monaco-IAPCT-2023-slides.pdf}{\unpubtitle{pdf}}] \\
  \tabularnewline
  10/12/2023 & \href{https://odin.mit.edu/schedule.html}
    {``\itemtitle{Beyond ‘FAIR’: What does sustainable protocolization of
    open data in neuroscience look like?}'' \emph{Invited Panelist \& Keynote
    Speaker}. Open Data in Neuroscience (ODIN) Symposium, Massachusetts Institute
  of Technology, Boston, MA} \\
  \tabularnewline
  3/8/2023 & \href{https://meetings.aps.org/Meeting/MAR23/Session/M01.13}
  {``\itemtitle{Neurodynamical computing at the information boundaries of
    intelligent systems}.'' \emph{Contributed Talk}. American Physical Society (APS)
  March Meeting, Las Vegas, NV}
  [\href{https://jdmonaco.com/files/monaco-APS-March-Meeting-2023-slides.pdf}{\unpubtitle{pdf}}] \\
  \tabularnewline
  2/1/2022 \hspace{0.2in} & ``\unpubtitle{Theory-Driven Data Science to
  Understand the Neural Dynamics of Memory and Behavior}.'' \emph{Invited Talk}.
  Department of Cell \& Systems Biology, University of Toronto, Canada, Virtual \\
  \tabularnewline
  12/1/2021 \hspace{0.2in} & ``\unpubtitle{Learning as swarming: Cognitive
  flexibility from the neural dynamics of phase-coupled attractor maps}.''
  \emph{Contributed Talk}. Neuromatch 4.0 Conference, Virtual \\
  \tabularnewline
  %\href{https://youtu.be/3mKkLksOyfk}{``\unpubtitle{Learning as swarming:
      %Cognitive flexibility from the neural dynamics of phase-coupled attractor
    %maps}.'' \emph{Contributed Talk}. Neuromatch 4.0 Conference, Virtual
  %\itemtitle{[YouTube]}} \\
  %\tabularnewline
  10/29/2020 \hspace{0.2in} & ``\unpubtitle{Spatial theta-phase coding in
  the lateral septum: A theory of allocentric feedback during navigation}.''
  \emph{Contributed Talk}. Neuromatch 3.0 Conference, Virtual \\
  \tabularnewline
  %\href{https://www.youtube.com/watch?v=WwYDMpD7j4Q}{``\unpubtitle{Spatial
      %theta-phase coding in the lateral septum: A theory of allocentric feedback
    %during navigation}.'' \emph{Contributed Talk}. Neuromatch 3.0 Conference,
  %Virtual \itemtitle{[YouTube]}} \\
  %\tabularnewline
  10/7/2020 \hspace{0.2in} & ``\unpubtitle{Computing path integration with
  oscillatory phase codes in biological and artificial systems}.'' \emph{Data
  Blitz}. iNAV Symposium 2020, Virtual \\
  \tabularnewline
  7/1/2010 \hspace{0.2in} & ``\unpubtitle{Medial versus lateral modes for
  reconfiguring hippocampal representations}.'' \emph{Invited Talk}. Grid
  Cell Meeting, Gatsby Computational Neuroscience Unit, UCL, UK \\
\end{longtable}


\smallskip
\subsection*{National}
\label{talks:natl}

\begin{longtable}{@{\hspace{0.2in}}l>{\raggedright\arraybackslash}p{.82\textwidth}}
  9/26--28/2023 & BRAIN Initiative Cell Atlas Network (BICAN)
  Knowledge Base Workshop. \emph{BRAIN Liaison \& Invited Participant}. Allen
  Institute for Brain Sciences, Seattle, WA \\
  \tabularnewline
  7/17--18/2023 \hspace{0.2in} &
  \href{https://event.roseliassociates.com/brain-newg-ws-july-2023/}
  {\itemtitle{Workshop on Ethics of Sharing Individual Level Human Brain
    Data Collected in Biomedical Research}. \emph{Co-Organizer, Breakout
    Moderator/Reporter}. BRAIN Initiative Neuroethics Working Group (NEWG), NIH,
  Bethesda, MD \& Hybrid} \\
  \tabularnewline
  5/9/2023 \hspace{0.2in} & 
  \href{https://jdmonaco.com/files/monaco_TheoryOfTheory_slides.pdf}
  {``\itemtitle{Theory of theory: On the role of
  theory and modeling in neuroscience}.'' \emph{Invited Extramural Seminar}.
  NIH, Virtual [\unpubtitle{pdf}]} \\
  \tabularnewline
  4/28/2023 \hspace{0.3in} & 
  \href{https://jdmonaco.com/files/monaco-2023-afrl-quest-slides.pdf}
  {``\itemtitle{Neurodynamical Articulation: Decoupling Intelligence from the
  Experiencing Self}.'' \emph{Invited Public Seminar}. QuEST, Air Force Research
  Lab/Autonomous Capabilities Team 3 (AFRL/ACT3), Virtual [\unpubtitle{pdf}]} \\
  \tabularnewline
  12/21/2022 \hspace{0.3in} & ``\unpubtitle{Finding Causal Paths Across Scales:
  Embodied Control, Ethological Interaction, and Theory-Driven Neural Data
  Science}.'' \emph{Invited Talk}. Division of Neuroscience and Behavior,
  NIH/NIDA, Virtual \\
  \tabularnewline
  11/17/2022 \hspace{0.3in} & ``\unpubtitle{Finding Causal Paths Across Scales:
  Embodied Control, Ethological Interaction, and Theory-Driven Neural Data
  Science}.'' \emph{Invited Talk}. Division of Neuroscience and Basic Behavioral
  Science, NIH/NIMH, Virtual \\
  \tabularnewline
  8/26/2022 \hspace{0.3in} &
  \href{https://jdmonaco.com/files/monaco-2022-afrl-quest-slides.pdf}
  {``\itemtitle{Brain oscillations: From cortical computing to the existential
  nonduality of conscious agents}.'' \emph{Invited Public Seminar}. Qualia
  Exploitation for Sensor Technology (QuEST), Air Force Research Lab/Autonomous
  Capabilities Team 3 (AFRL/ACT3), Virtual [\unpubtitle{pdf}]}\\
  \tabularnewline
  6/1/2020
  \hspace{0.3in} &
  \href{https://youtu.be/2jy1ENYHRAw?t=902}
  {``\itemtitle{Can Transitory Neurodynamics Unify Learning Theories for Brains
  and Machines?}'' \emph{Invited Talk \& Panel Discussion}.}
  6th Annual BRAIN Initiative Investigators Meeting,
  \href{https://www.labroots.com/webinar/symposium-1-dynamical-systems-neuroscience-reciprocally-advance-machine-learning}
  {Symposium 1 on \unpubtitle{\emph{How Can Dynamical Systems Neuroscience
  Reciprocally Advance Machine Learning?}}}, NIH, Virtual 
  [\href{https://youtu.be/2jy1ENYHRAw?t=902}{{\itemtitle{\bf YouTube}}}] \\
  \tabularnewline
  5/18/2020 \hspace{0.3in} & ``\unpubtitle{Computational Approaches to the
  Neural Dynamics of Time, Memory, and Behavior}.'' \emph{Invited Talk}.
  Department of Neuroscience, Medical Discovery Team for Optical Imaging,
  University of Minnesota, Virtual \\
  \tabularnewline
  2/24/2020 \hspace{0.3in} & ``\unpubtitle{Computational Mechanisms of Memory:
  Linking Behavior, Space, \& Time}.'' \emph{Invited Talk}. Department of
  Psychology, University of Nevada, Las Vegas, NV \\
  \tabularnewline
  1/31/2020 \hspace{0.3in} & ``\unpubtitle{Attractors, memory, and oscillations:
  Computational motifs of spatial learning}.'' \emph{Invited Talk}.
  Department of Biological Sciences, University of Texas at El Paso, El Paso, TX \\
  \tabularnewline
  4/17/2019 \hspace{0.3in} & ``\unpubtitle{Emergent dynamics of hippocampal
  circuitry as a basis for robust self-organized planning in mobile swarms}.''
  \emph{Invited Talk}. International Society for Optics and Photonics (SPIE)
  Defense \& Commercial Sensing 2019 Conference, Baltimore, MD \\
  \tabularnewline
  4/10/2019 & NSF/Neural \& Cognitive Systems (NCS) PI
  Workshop. \emph{Invited Participant}. Marriott Wardman Park Hotel, Washington, D.C. \\
  \tabularnewline
  2/3--7/2019 & NSF/BRAIN Initiative Workshop: Present and Future Frameworks
  of Theoretical Neuroscience. \emph{Invited Participant}. University of Texas,
  San Antonio, TX \\
  \tabularnewline
  1/3/2014 & \href{https://jdmonaco.com/files/ScanningSlide.pdf}
  {``\itemtitle{Head scans drive the formation and potentiation of place
  fields during exploration}.'' \emph{Data Blitz}. 38th Winter Conference on
  Neurobiology of Learning \& Memory, Park City, UT} \\
  \tabularnewline
  4/10/2009 & ``\unpubtitle{Rapid spatial map formation and remapping by
  competing over grid cell inputs}.'' \emph{Thesis Seminar}. Department of
  Neurobiology \& Behavior, Columbia University, New York, NY 
  [\href{https://jdmonaco.com/files/monaco-2009-thesis-seminar-Keynote.mp4}
  {\unpubtitle{Keynote Movie Export (mp4)}}] \\
\end{longtable}

\smallskip
\subsection*{Regional}

\begin{longtable}{@{\hspace{0.2in}}l>{\raggedright\arraybackslash}p{.82\textwidth}}
  10/2/2019 \hspace{0.1in} & ``\unpubtitle{Oscillations, attractors, and
  sequences: Extending hippocampal computations to artificial systems}.''
  \emph{Invited Talk}. Kavli Neuroscience Discovery Institute, Johns Hopkins
  University, Baltimore, MD\\
  \tabularnewline
  1/22/2016 & ``\unpubtitle{Hippocampal circuits for space, memory, and
  navigation: From minimal models to biologically inferred networks}.''
  \emph{Invited Talk}. Department of Pharmacology, University of Maryland,
  Baltimore, MD\\
  \tabularnewline
  9/6/2014 & ``\unpubtitle{Stopping to look: How attentive scanning behavior
  reveals the formation of new memories}.'' \emph{Department Retreat Seminar}.
  Department of Neuroscience, Johns Hopkins University, Baltimore, MD\\
  \tabularnewline
  4/21/2014 & ``\unpubtitle{Landmark influence: How attention to sensory cues
  stabilizes and updates the hippocampal cognitive representation of space}.''
  \emph{Advanced Researcher Seminar}. Zanvyl Krieger Mind/Brain Institute, Johns
  Hopkins University, Baltimore, MD\\
  \tabularnewline
  4/1/2014 & ``\unpubtitle{Hippocampus and declarative memory:
  Head scanning}.'' \emph{Department `Lab Lunch' Seminar}. Department of
  Neuroscience, Johns Hopkins University, Baltimore, MD\\
\end{longtable}


\smallskip
\newsection{Scientific Conference Presentations}{posters}

\begin{description}
  \item[\quad]
    \href{https://www.cvent.com/events/6th-annual-brain-initiative-investigators-meeting/custom-116-4e2aadaa6cd549a3a4b53113cd172ad2.aspx}
    {\joehl{Monaco JD}, Hwang GM, Schultz K, Zhang K. (2020).
      \itemtitle{Cognitive swarming in complex environments with attractor
      dynamics and oscillatory computing}. \emph{6th Annual BRAIN Initiative
    Investigators Meeting}. Virtual, with audio narration. June~2020.}
  \item[\quad]
    \href{https://www.fens.org/Meetings/The-Brain-Conferences/Dynamics-of-the-brain/}
    {\joehl{Monaco JD}, Hwang GM, De Guzman RM, Blair HT, Zhang K. (2019).
      \itemtitle{Spatial rate-phase coding in lateral septal ‘phaser cells’:
      single-unit data and theta-bursting models}. \emph{FENS (Federation of
        European Neuroscience Societies) Dynamics of the brain: Temporal aspects of
    computation}. North Copenhagen, Denmark. June~2019.}
  \item[\quad]
    \joehl{Monaco JD}. (2019). \unpubtitle{Decoding septohippocampal theta cells
    during exploration reveals unbiased environmental cues in firing phase}.
    Kavli Neuroscience Discovery Institute, Baltimore, MD. October~2019.
  \item[\quad]
    \href{https://www.cvent.com/events/5th-annual-brain-initiative-investigators-meeting/event-summary-de9c0d8f934b46eb8d80b55bcfbfe96a.aspx}
    {\joehl{Monaco JD}, Hwang GM, Schultz K, Zhang K. (2019).
      \itemtitle{Self-organized swarm control using neural principles of spatial
      phase coding}. \emph{5th Annual BRAIN Initiative Investigators Meeting}.
    Washington, D.C. April~2019.} 
  \item[\quad]
    \href{https://abstractsonline.com/pp8/#!/4649/presentation/10884}
    {Hwang GM, Schultz K, \joehl{Monaco JD}, Chalmers RW, Lau SW, Yeh BY,
      Zhang K. (2018). \itemtitle{Self-organized swarm control using neural
      principles of spatial phase coding}. \emph{Society for Neuroscience}.
    San Diego, CA. November~2018.}
  \item[\quad]
    \href{https://www.abstractsonline.com/pp8/#!/4376/presentation/6085}
    {\joehl{Monaco J}, Blair HT, Zhang K. (2017). \itemtitle{Decoding
        septohippocampal theta cells during exploration reveals unbiased
      environmental cues in firing phase}. \emph{Society for Neuroscience}.
    Washington, D.C. November~2017.}
  \item[\quad]
    \joehl{Monaco JD}. (2016). \unpubtitle{Spatial rate/phase correlations in
    theta cells can stabilize randomly drifting path integrators}. \emph{Greater
    Baltimore SfN Meeting}, Baltimore, MD. October~2016
  \item[\quad]
    \href{https://jdmonaco.com/files/monaco-paper-cosyne15.pdf}
    {\joehl{Monaco JD}, Blair HT, Zhang K. (2015). \itemtitle{Spatial
        rate/phase correlations in theta cells can stabilize randomly drifting path
    integrators}. \emph{Cosyne}. Salt Lake City, UT. March~2015.}
  \item[\quad]
    \href{https://www.abstractsonline.com/Plan/ViewAbstract.aspx?sKey=973d2662-ba7a-4ad2-aff9-fe0d4b77c262&cKey=9917ffaf-9e31-4213-acb9-4aab498ab4cd&mKey=54c85d94-6d69-4b09-afaa-502c0e680ca7}
    {\joehl{Monaco J}, Blair HT, Zhang K. (2014). \itemtitle{Spatial rate/phase
        codes provide landmark-based error correction in a temporal model of theta
    cells}. \emph{Society for Neuroscience}. Washington, D.C.  November~2014.}
  \item[\quad]
    \href{https://www.abstractsonline.com/Plan/ViewAbstract.aspx?sKey=bfb59866-8deb-44a6-9515-a7aab630507b&cKey=d201b3aa-7725-452e-b0dd-c41d204b5b54&mKey=54c85d94-6d69-4b09-afaa-502c0e680ca7}
    {Wang CH, Rao G, \joehl{Monaco JD}, Deshmukh SS, Knierim JJ. (2014).
      \itemtitle{Potentiation of place fields along the CA1 transverse axis by
      investigatory head-scanning behavior}. \emph{Society for Neuroscience}. 
    Washington, D.C. November~2014.}
  \item[\quad]
    \href{https://www.abstractsonline.com/Plan/ViewAbstract.aspx?sKey=32eccac1-4e1d-4e81-bf5c-f39bcb605757&cKey=4710dece-cc8e-4b48-8764-49ea174b91ef&mKey=8d2a5bec-4825-4cd6-9439-b42bb151d1cf}
    {\joehl{Monaco J}, Rao G, Knierim JJ. (2013). \itemtitle{Scanning behavior
        in novel environments promotes \emph{de novo} formation of hippocampal place
    fields in rats}. \emph{Society for Neuroscience}. San Diego, CA. November~2013.}
  \item[\quad]
    \href{https://www.abstractsonline.com/Plan/ViewAbstract.aspx?sKey=f5b9fa94-7d15-48c7-9d67-b89cd2883025&cKey=a53349ca-41b1-4664-b022-85d0d1fe59b8&mKey=70007181-01C9-4DE9-A0A2-EEBFA14CD9F1}
    {\joehl{Monaco J}, Rao G, Knierim JJ. (2012). \itemtitle{Hippocampal LFP
      during rodent head-scanning behavior: Theta and sharp-wave ripples}.
    \emph{Society for Neuroscience}. New Orleans, LA. October~2012.}
  \item[\quad]
    \href{https://www.abstractsonline.com/Plan/ViewAbstract.aspx?sKey=c48e9f5f-1274-4486-85bf-38ee591629e1&cKey=190bd951-c183-428d-a4c5-01eb61556d79&mKey=8334BE29-8911-4991-8C31-32B32DD5E6C8}
    {\joehl{Monaco J}, Rao G, Knierim JJ. (2011). \itemtitle{Hippocampal place
        cell firing during head-scanning movements is associated with the formation
    of new place fields}. \emph{Society for Neuroscience}. Washington, D.C. November~2011.}
  \item[\quad]
    \href{https://www.abstractsonline.com/Plan/ViewAbstract.aspx?sKey=c48e9f5f-1274-4486-85bf-38ee591629e1&cKey=3ec26e6f-8c59-4be2-bad3-e1572d75e07e&mKey=8334BE29-8911-4991-8C31-32B32DD5E6C8}
    {Rao G, \joehl{Monaco J}, Knierim JJ. (2011). \itemtitle{Environmental
        novelty promotes rodent head-scanning behavior linked to enhanced entorhinal
      activity}. \emph{Society for Neuroscience}. Washington,
    D.C. November~2011.}
  \item[\quad]
    \href{https://www.frontiersin.org/10.3389/conf.fnins.2010.03.00192/event_abstract}
    {\joehl{Monaco JD}, Zhang K, Blair HT, Knierim JJ. (2010).
      \itemtitle{Cue-based feedback enables remapping in a multiple oscillator
      model of place cell activity}. \emph{Cosyne}. Salt Lake City, UT.
    February~2010.}
  \item[\quad] \joehl{Monaco JD}, Abbott LF. (2009). \unpubtitle{Dynamic
      hippocampal remapping using recurrent inhibition on realigning grid cell
    inputs}. \emph{Cosyne}. Salt Lake City, UT. February~2009.
  \item[\quad] \joehl{Monaco JD}, Muzzio IA, Levita L, Abbott LF. (2006).
    \unpubtitle{Entorhinal input and global remapping of hippocampal place
    fields}. \emph{CNS}. Edinburgh, UK. July~2006.
  \item[\quad] \joehl{Monaco JD}, Abbott LF. (2006). \unpubtitle{Entorhinal
    input and the remapping of hippocampal place fields}. \emph{Cosyne}. Salt Lake
    City, UT. March~2006.
  \item[\quad] \joehl{Monaco JD}, Levy WB. (2003). \unpubtitle{T-maze training
      of a recurrent CA3 model reveals the necessity of novelty-based modulation of
    LTP in hippocampal region CA3}. \emph{IJCNN}. Portland, OR. July~2003.
  \item[\quad] \joehl{Monaco JD}, Perlstein RP. (1997). \unpubtitle{Monte-Carlo
    analysis of deoxyhypusine synthase inhibitor ligand conformations}. \emph{NIH
    Poster Day}. Bethesda, MD. August~1997.
\end{description}


\smallskip
\newsection{Teaching \& Mentoring}{teaching}

\subsection*{Educational Programming}

\begin{tabular}{@{\hspace{0.2in}}l>{\raggedright\arraybackslash}p{.82\textwidth}}
  2018--2021 \hspace{0.1in} & My NSF project with JHUAPL (see
  \nameonpdottwo{nsfaward}{nsfgrant}) was successfully funded with a substantial
  STEM component for high-school students involving the development of both
  a 12-week course and an intense 2-day seminar called ``Swarming Powered by
  Neuroscience.'' I worked with our STEM education collaborators at JHUAPL to
  develop computational resources required for the curricula. Additionally, I
  participated in and delivered two zoom lectures about our research for the
  virtual 4-day STEM workshop (developed due to Covid requirements) with 40+
  students that was held Jan 2021.
\end{tabular}

\pagebreak
\subsection*{Mentoring \& Supervision}

\begin{longtable}{@{\hspace{0.2in}}l>{\raggedright\arraybackslash}p{.82\textwidth}}
  Spring 2021 \hspace{0.05in} & Darius Carr, STEM high school student; I mentored
  Darius as part of a local high school program that facilitates research
  internships for underrepresented students. I developed a computational
  research project with him that deepened his current interests in neuroscience,
  python programming, and scientific research. \\
  \tabularnewline
  2020--2022 \hspace{0.1in} & Armin Hadzic, junior machine learning engineer
  at JHUAPL; I supervised Armin in translating computational neuroscience
  models into the domain of reinforcement learning and Bayesian optimization
  to investigate autonomous swarming with neural control. Our project led to
  a first author peer-reviewed research publication for Armin in Array (see
  \nameonpdot{hadzicpub}). In 2022, I provided letters of recommendation in
  support of Armin's applications to Ph.D. programs in computer science. \\
  \tabularnewline
  2019--2023 & Sreelakshmi Rajendrakumar, master's student in JHU/Biomedical
  Engineering (BME); I mentored Sreelakshmi in hippocampal physiology and
  single-unit data analysis. In 2023, I provided letters of recommendation in
  support of Sree's applications to Ph.D. programs in operations research and
  causal inference. \\
  \tabularnewline
  2014 & Manning Zhang, M.S., graduate student in JHU/BME; I mentored Manning
  through an exchange program with Shanghai Jiao Tong University and submitted
  a letter of recommendation supporting her admission to the JHU/BME master's
  program. \\
  \tabularnewline
  2013--2015 & Chia-Hsuan Wang, Ph.D., graduate student at the JHU/MBI; I worked
  extensively with Chia-Hsuan to take over my previous studies of behavior
  and place cells in the Knierim lab, leading to a Society for Neuroscience
  conference poster in 2014. I supported her subsequent thesis research based on
  my analytics and informatics software, resulting in a paper in Current Biology
  (see \nameonpdot{wangpub}). \\
\end{longtable}


\subsection*{Classroom Instruction}

\begin{tabular}{@{\hspace{0.2in}}l>{\raggedright\arraybackslash}p{.82\textwidth}}
  Fall 2004 & Teaching Assistant for undergraduate ``Introduction to
  Neuroscience'' course, Brandeis University; I assisted Prof. Eve Marder by
  supervising classes, grading examinations, and giving review lectures.\\
  \tabularnewline
  Spring 2005 \hspace{.1in} & Teaching Assistant for undergraduate ``Biology
  Laboratory'' course, Brandeis University \\
\end{tabular}


\smallskip
\newsection{Research Program Development}{research}

\subsection*{Leadership \& Team Coordination}
\label{sec:res}

\researchactivity
{April 2010/2011}
{Fellowship Proposal (NIH/NINDS F32 NRSA)}
{Behavioral Coordination of Entorhinal-Hippocampal Activity for Real-Time
Sensory Updating of Spatial Memory}
{In collaboration with my posdoctoral sponsor Jim Knierim, I conceived and
  developed a postdoctoral fellowship training proposal as a NIH F32 NRSA
  application. The proposal integrated computational modeling with spatial
  navigation experiments based on behavioral data from position-tracking sensors
  and neural data from multiregional hippocampal--entorhinal single-unit ensemble
  recordings. The application received a 21st percentile rank; I followed up the
  2010 application with a 2011 resubmission following discussions with NINDS PO
Jim Gnadt.}
\label{sec:nrsa}

\researchactivity
{Mar. 2016--2018}
{Grant Award (JHU/SLI)}
{Learning to explore paths through space}
{This internal JHU award (2016--2018; see \nameonpdot{sliaward})
  resulted from a collaboration with David J. Foster (now at UC Berkeley) that
  I initiated to conduct modeling studies informed by his lab’s hippocampal
  reactivation data. By integrating Prof.~Zhang’s mathematical theories of
  spatial cognitive maps, I wrote and submitted a proposal for a \$200K/2-year
  project to the JHU Science of Learning Institute. The proposal was awarded at
  the \$150K level and research outcomes included (1) novel theories of temporal
  synchronization coding that inspired the 2017 NSF proposal effort, and (2)
  preliminary dynamical models of sharp-wave reactivation that provided the
foundation for the 2018 NIH R03 award.}
\label{sec:sligrant}

\researchactivity
{April--June 2016}
{Grant Proposal (DARPA/BTO)}
{Noninvasive Gastrovagal Stimulation for Enhanced Neuroplasticity of Cortical
and Hippocampal Networks during Cognitive Training (GEN-C)}
{In response to DARPA announcement BAA-16-24 of the “Targeted Neuroplasticity
  Training (TNT)” program, I worked with colleagues from JHUAPL and JHU/SoM
  Center for Neurogastroenterology to develop a collaborative program involving 3
  PIs and 5 co-Is (8 labs) across divisions, departments, and fields. I recruited
  experimental labs from JHU/MBI and coordinated proposed contributions to
  maximize scientific impact with a budget of \$9.8M/5 years. I coordinated the
  40-page research narrative, including writing, editing, and/or integrating
  each lab’s contributions and worked with ORA to submit the proposal. While
  not funded in total, DARPA/BTO PM Doug Weber funded select components, leading
  to JHUAPL Work Agreement No.~145563 “BCI (Brain Computer Interface)
Technologies” in 2018.}
%Technologies” in 2018 for \$24,604 to the lab of Prof.~Pasricha.}

\researchactivity
{Nov. 2017--2021}
{Grant Award (NSF/NCS)}
{NCS-FO: Spatial intelligence for swarms based on hippocampal dynamics}
{This NSF-awarded project (2018--2021; see \nameonpdot{nsfaward}) was the result
  of 6 months of collaboration, brain-storming, and team-building between the
  Zhang lab at JHU/SOM and a group of JHUAPL engineers, mathematicians, and
  scientists. The project was initially inspired by results that I presented at
  my Society for Neuroscience 2017 meeting poster. I wrote Aim 1 and integrated
  the full research narrative with inputs from our collaborators for the proposal
  of this \$997K/2-year project to develop those initial ideas into technological
  applications (e.g., robotics, autonomous control, AI) that reciprocally inform
  neuroscience. The project has so far produced three posters, a conference talk
  \& proceedings publication, three patent applications, a preprint, a research
  article in Biological Cybernetics, a NIH BRAIN Investigators Meeting symposium
  talk, and a substantial STEM program. We received a no-cost extension through
FY21 to complete the final phase of the project.}
\label{sec:nsfgrant}

\researchactivity
{Jan. 2018--2020}
{Grant Award (NIH/NINDS)}
{Spiking network models of sharp-wave ripple sequences with gamma-locked
attractor dynamics}
{To continue with the collaboration that I initiated with David J. Foster (UC
  Berkeley) on the basis of the internal SLI award (see above), I wrote a small
  modeling proposal that integrated preliminary results from the SLI project and
  recent research developments in the memory reactivation field. This proposal
  was awarded (2018--2020; see \nameonpdot{nihaward}) through the NIH/NINDS R03
  mechanism and I am currently utilizing this support to build a foundation for
future efforts along this research track.}
\label{sec:nihgrant}

\whitepaper
{Feb.--Mar. 2018}
{Schultz K, Zhang K, and \joehl{Monaco J}}
{BrainSWARRMM: Brain-like Sharp-Waves for Autonomous Replanning \&
Reconnaissance on Matrix Manifolds}
{In response to the Office of Naval Research (ONR) Special Notice
  N00014-18-R-SN05, Topic 3, I helped organize a series of collaborative meetings
  to design a \$2M/4-year project between JHUAPL and JHU/SoM. I co-authored the
resulting white paper that was submitted for consideration to ONR.}

\whitepaper
{May--June 2018}
{Zhang K, \joehl{Monaco JD}, Hwang GM, Schultz KM, Kobilarov M, Foster DJ,
Jacobs J, and Itti L}
{An Integrative Theoretical Framework of the Neural Self-Organization of Active
Perception for Autonomous Spatial Navigation}
{In response to ONR MURI Announcement N00014-18-S-F006 and with the help of
  JHUAPL, I coordinated a series of meetings with 5 PIs across 4 universities
  (Columbia, UC Berkeley, USC, JHU) to design an innovative research program that
  targeted reciprocal advances in experimental \& theoretical neuroscience and
  robotics \& AI across species and scales. The resulting \$7.5M/5-year project
  that I outlined in the white paper was not invited for a full submission. We
  debriefed with the sponsor, ONR PM Marc Steinberg, who revealed that ONR was
  impressed with the project but that they were seeking a different balance of
elements with respect to neuroscience and AI.}

\whitepaper
{August 2019}
{\joehl{Monaco J}, Zhang K, and Schultz K.}
{SW2Mem: Graph Spectral Decoding of Hippocampal-Cortical Loops for Artificial
Consolidation and Dreaming}
{In response to ONR Special Notice N00014-19-S-SN08, Topic 5.1 I conceived this
  project, created the preliminary model and datasets, guided the preliminary
  analyses with JHUAPL collaborators, and wrote \& submitted the white paper to
  ONR outlining a potential \$1.05M/3-year project. ONR declined to invite us to
submit a full proposal.}

\whitepaper
{August 2019}
{Schultz K, Agarwala S, Zhang K, and \joehl{Monaco J}}
{Brain-like Distributed Surveillance using Heterogeneous Agents for integRated
Perception, and Planning (BD-SHARPP)}
{In response to ONR Special Notice N00014-18-R-SN05, Topic 3, we submitted a
  revised version of the March 2018 white paper that was specifically invited by
ONR PM Tom McKenna.}

\researchactivity
{Sept. 11, 2019}
{NSF Project Review}
{Annual advisory board review symposium}
{I delivered a seminar on Aim 1 progress at a JHUAPL-hosted symposium for our
  project’s yearly review, attended by DARPA/I2O PM Hava Siegelmann and other
outside experts.}

\researchactivity
{Feb. 26, 2020}
{Grant Proposal (NSF/NCS) }
{NCS-FO: Neuroeconomics as a biomimetic control theory for mobile robotic
decision making}
{This FY21 proposal was submitted to the NSF/NCS FOUNDATIONS program; while
  it was discussed and received high scores, the application was declined. I
  co-developed this project in collaboration with colleagues at the University of
  Pittsburgh Medical Center (UPMC), JHU Whiting School of Engineering (JHU/WSE),
  and JHUAPL. Our interdisciplinary project brought together multiscale human
  electrophysiological recordings (UPMC), latent state-space models (JHU/WSE),
  control- and game-theoretic analysis (JHUAPL), and mechanistic neural models
  (JHU/BME, for which I would have been co-PI). We proposed to investigate and
  characterize the neural bases of metacognitive brain states that influence
  decision-making during social \& economic games. As a high-risk/high-reward
  element, we proposed to algorithmicize our results to advance human-robot
interaction.}

\researchactivity
{Jan. 14, 2022}
{Grant Proposal (JHU/Discovery Award) }
{Algorithms of flexible navigation in mice and robots}
{This intramural FY23 proposal for a JHU Discovery Award resulted from a new
  collaboration with Patricia Janak (PI; JHU/PBS) and Céline Drieu (postdoctoral
  fellow), in which we seek to integrate advanced large-scale neural recording
  technologies with my theoretical modeling of neural systems as a distributed
  control problem. Fundamental questions of neural systems communication will
  be addressed using convergent data-driven and theory-driven approaches
  to understanding the cognitive dynamics that enable mice to perform spatial
goal-directed memory tasks.}


\smallskip
\subsection*{Inventions \& Patents}
\label{sec:patents}

\lefttabline{0.8in}{7/5/2022}{Inventor, 
  \href{https://www.freepatentsonline.com/11378975.html}{\itemtitle{Autonomous
Navigation Technology, US patent issued, 11,378,975}}}
\lefttabline{0.8in}{1/3/2020}{Inventor, Autonomous Navigation Technology, US
patent application, 16,734,294}
\lefttabline{0.8in}{5/10/2019}{Inventor, Neuroinspired Algorithms for Swarming
Applications, provisional patent, 62/845,957}
\lefttabline{0.8in}{1/3/2019}{Inventor, Neuroinspired Algorithms for Swarming
Applications, provisional patent, 62/787,891}


\end{document}
