%! TEX program = xelatex
%
% Curriculum vitae for Joseph D. Monaco
%
\documentclass[10pt]{article}

\usepackage{fullpage}
\usepackage[usenames,dvipsnames]{color}
\usepackage[hidelinks,xetex]{hyperref}
\usepackage{multirow}
\usepackage{sectsty}
\usepackage{enumitem}
\usepackage{tabu}
\usepackage{longtable}
\usepackage{soul}
\usepackage{fontspec}

% Page geometry formatting
\raggedbottom
\raggedright
\textheight=9in
\setlength{\tabcolsep}{0in}
\addtolength{\footskip}{.3in}
\addtolength{\voffset}{-.4in}
\addtolength{\textheight}{.4in}
\addtolength{\hoffset}{-.25in}
\addtolength{\textwidth}{.2in}

% Define colors
\definecolor{hopkinsblue}{RGB}{0,48,130}
\definecolor{lightblue}{rgb}{.88,.92,.97}
\definecolor{lightgold}{rgb}{1,.80,0}
\definecolor{lightred}{rgb}{1,.3,.2}
\definecolor{lightgray}{rgb}{.80,.80,.80}
\definecolor{dimgray}{rgb}{.50,.50,.50}
\definecolor{darkgray}{rgb}{.30,.30,.30}

% Set up highlighting and underlining
\sethlcolor{lightblue}
\setul{0.13ex}{}

% Choose font package
\setmainfont[Ligatures=TeX]{Helvetica Neue LT Std} 

% Section section* formatting
\sectionfont{\large\color{hopkinsblue}\bfseries}
\subsectionfont{\vspace{-1ex}\color{hopkinsblue}\normalsize\bfseries}

% Formatting macros
\newcommand{\itemtitle}[1]{{\color{hopkinsblue}\ul{#1}}}
\newcommand{\unpubtitle}[1]{{\color{hopkinsblue} #1}}
\newcommand{\itemnote}[1]{
  \begin{description}
    \item[$\rightarrow$] \hspace{.09in}{\color{darkgray}\it #1}
  \end{description}
}
\newcommand{\joehl}[1]{\hl{\textbf{#1}}}
\newcommand{\doi}[1]{{\color{darkgray}doi:}~{\color{dimgray}\texttt{#1}}}
\newcommand{\arxiv}[1]{\emph{ArXiv Preprint}.
  {\color{darkgray}arxiv:}{\color{dimgray}\texttt{#1}}}
\newcommand{\arxivlink}[1]{\href{https://arxiv.org/abs/#1}
  {{\color{darkgray}arxiv:}~{\color{dimgray}\texttt{#1}}}}
\newcommand{\biorxivlink}[1]{\href{https://dx.doi.org/#1}
  {{\color{darkgray}biorxiv:}~{\color{dimgray}\texttt{#1}}}}
\newcommand{\aurl}[1]{{\color{dimgray}\texttt{#1}}}
\newcommand{\researchnote}[1]{
  \begin{description}
    \item[] {\hspace{2.2ex}\color{darkgray} #1}
  \end{description}
}
\newcommand{\researchactivity}[4]{
  \begin{minipage}[t]{\textwidth}
    \begin{tabular}{@{\hspace{2ex}}l>{\raggedright\arraybackslash}p{.8\textwidth}}
      \makebox[1.2in][l]{#1} & \textbf{#2:}
      ``\unpubtitle{#3}'' 
    \end{tabular}
  \researchnote{\hspace{1ex} #4}
  \end{minipage}
  \medbreak
}
\newcommand{\whitepaper}[4]{
  \begin{minipage}[t]{\textwidth}
    \begin{tabular}{@{\hspace{2ex}}l>{\raggedright\arraybackslash}p{.8\textwidth}}
      \makebox[1.2in][l]{#1} & \textbf{White Paper:} #2.
      ``\unpubtitle{#3}'' 
    \end{tabular}
  \researchnote{\hspace{1ex} #4}
  \end{minipage}
  \medbreak
}
\newcommand{\lefttabline}[3]{\hspace{2ex}\makebox[#1][l]{#2} #3\\}

% PDF document info and setup
\hypersetup{
  baseurl=https://jdmonaco.com/cv-monaco.pdf,
  pdftitle=Curriculum Vitae for Joseph D. Monaco,
  pdfauthor=Joseph D. Monaco,
  pdfdisplaydoctitle=true,
  pdfpagemode=UseThumbs,
  pdfstartview=FitV,
  pdfpagelayout=TwoColumnLeft,
  pdftoolbar=false,
  pdfwindowui=false,
  pdfcenterwindow=true,
}

% "See section ..." macro
\newcommand{\see}[1]{[\textcolor{hopkinsblue}{p.\pageref{sec:#1}}]}
\newcommand{\cfonly}[1]{\textcolor{hopkinsblue}{\emph{\nameref{sec:#1}} on p.\pageref{sec:#1}}}
\newcommand{\cf}[1]{\textcolor{hopkinsblue}{See \emph{\nameref{sec:#1}} on p.\pageref{sec:#1}}}
\newcommand{\cfcf}[2]{\textcolor{hopkinsblue}{See \emph{\nameref{sec:#1}} on
  p.\pageref{sec:#1} and \emph{\nameref{sec:#2}} on p.\pageref{sec:#2}}}

% Typesetting
\setlength{\parskip}{0.75em}
\setlist{nosep}

% Section-heading macros
\newcommand{\newsection}[2]{%
  \section*{#1}
  \vspace{-.1in}
  \hrule
  \vspace{.2in}
  \label{sec:#2}
}
\newcommand{\newsubsection}[2]{%
  \subsection*{#1}
  \vspace{-.08in}
  \hrule
  \vspace{.1in}
  \label{sec:#2}
}

\begin{document}

\begin{center}
  \textbf{\LARGE\color{hopkinsblue} Joseph D. Monaco, Ph.D.} \\[0.1in]
  Baltimore, Maryland, United States \\
  \href{mailto:joseph.monaco@nih.gov}{\color{hopkinsblue}\texttt{joseph.monaco@nih.gov}} •
  \href{tel:16674067398}{\color{hopkinsblue}\texttt{667.406.7398}} \\
  \vspace{.1in}
\end{center}
%\thispagestyle{empty}

\vspace{-0.1in}
\newsection{Health Scientist Administrator (HSA): Duties, Requirements, \& Qualifications}{duties}
\vspace{-0.1in}

\begin{itemize}
  \color{hopkinsblue}
  \item \emph{Manage program workloads and ensure timely completion of project
milestones.}
\end{itemize}

Since January 2023 (2 years), in my current role supporting the NIH BRAIN
Initiative, I have managed multiple complex workstreams including policy
implementation, workshop organization, strategic planning, and cross-team
coordination. \cf{jobobd}. A key example is my leadership of the BRAIN
NeuroAI Workshop planning process in 2024 (1 year), where I successfully
coordinated internal NIH and external scientific planning committees, secured
authorizations, and executed a major two-day event despite budget constraints
and logistical obstacles, resulting in participation from nearly 2,000 attendees
(200 in-person) from 48 countries.

\begin{itemize}
  \color{hopkinsblue}
  \item \emph{Direct research activities to maintain focus on programmatic
goals.}
\end{itemize}

Since January 2023 (2 years), I have provided strategic direction for multiple
BRAIN research programs~\see{jobobd}, ensuring alignment with BRAIN Initiative
priorities in data science, theory, and computational modeling. For example, I
led development of new data sharing workflows that improved efficiency while
maintaining rigorous standards. In addition, I am currently working to guide the
post-workshop NeuroAI strategic planning porcess across BRAIN Teams and working
groups to establish research priorities advancing BRAIN's mission over the next
decade.

\begin{itemize}
  \color{hopkinsblue}
  \item \emph{Coordinate with leadership to obtain and allocate necessary
program resources.}
\end{itemize}

Since January 2023 (2 years), in my current role~\see{jobobd}, I regularly
advise the BRAIN Director on scientific and technological priorities, supporting
decision-making through analysis and recommendations that guide decisions around
program resource allocations. For example, my implementation strategy for
BRAIN's data management and sharing policy avoided unnecessary expenditures for
up-front development costs which may have otherwise exceeded \$500,000, while
still achieving policy objectives. In 2024, I successfully secured resources for
the BRAIN NeuroAI Workshop held in November despite FY24 budget constraints.

\begin{itemize}
  \color{hopkinsblue}
  \item \emph{Review grant applications and analyze proposals from program
perspective.}
\end{itemize}

Since January 2023 (2 years), in my current role~\see{jobobd}, I have served as
a subject matter expert (SME) for multiple BRAIN Teams, providing scientific
and programmatic review of grant applications for funding opportunities related
to computational neuroscience, theoretical neuroscience, systems neuroscience,
neuromorphic computing, software tools, and data science. I developed reference
language for new funding opportunities (NOFOs) developed by BRAIN Teams and
established improved workflows for program staff to evaluate research proposals
and data management and sharing (DMS) plans. In 2024, I also contributed ad-hoc
reviews of funding applications for the Air Force Office of Sponsored Research
(AFOSR). \cf{service}.

In 2021 and 2022 (2 years), before joining the NIH, I contributed to the
development of a competitive National Science Foundation (NSF) EFRI program
topic, served on several review panels for the successfully adopted EFRI topic,
and provided ad-hoc reviews for the NSF CAREER program. \cf{programsvc}.

Several times from 2016 to 2024 (4 years total), I provided review service
for scientific conferences. In 2024, I contributed ad-hoc reviews of paper
submissions for the Neuro-Inspired Computational Elements (NICE) Conference.
From 2016 to 2021, I served three times on the Review Committee for the
Computational Systems Neuroscience (Cosyne) conference. \cf{confsvc}.

\begin{itemize}
  \color{hopkinsblue}
  \item \emph{Guide applicants and grantees on program requirements, policies,
and opportunities.}
\end{itemize}

Since January 2023 (2 years), in my current role~\see{jobobd}, I have
provided guidance to investigators on grant applications, data sharing policy
expectations and requirements, and program priorities across multiple BRAIN
portfolios. Additionally, I created and implemented new standard operating
procedures that streamlined communications between program staff and applicants,
reducing overall burden while maintaining high standards for maximizing the
scientific impact of NIH and BRAIN data sharing.

From 2018 to 2022 (5 years), before joining the NIH, I met with various
program directors from the Navy (Office of Naval Research) and the NSF (ENG,
CISE) to discuss how to maximize program relevance of applications that I was
preparing. From this experience, I wrote or co-wrote and submitted a number of
whitepapers and grant applications at various levels of project size and budget.
\cf{res}. Through these discussions and the subsequent development and
programmatic review of my submitted applications, I gained substantial and
varied knowledge of the role of program officers in guiding investigators as
they develop applications to align with mission criteria and program priorities.


\begin{itemize}
  \color{hopkinsblue}
  \item \emph{Develop and manage grants and cooperative agreements supporting
program mission and objectives.}
\end{itemize}

Since January 2023 (2 years), in my current role~\see{jobobd}, I have worked
closely with multiple BRAIN Teams to support development and coordination of
cooperative agreements and research awards, providing SME programmatic reviews
during pre- and post-award phases. I have served as a resource for program
teams to ensure alignment of milestones with the DMS policy expectations of
NIH and the BRAIN Initiative. In addition, my contributions include providing
program staff on data sharing requirements for new awards, participating in
BRAIN consortium network meetings like BICAN and CONNECTS, and helping evaluate
milestone progress for complex cooperative agreements to verify continued
alignment with BRAIN objectives.

\begin{itemize}
  \color{hopkinsblue}
  \item \emph{Organize scientific workshops and conferences to advance program goals.}
\end{itemize}

Since January 2023 (2 years), in my current role in the Office of the BRAIN
Director (OBD)~\see{jobobd}, I have demonstrated extensive experience planning
and executing scientific meetings at multiple scales, from co-organizing
focused sessions to leading major workshops. I co-organized the ``Embodied
NeuroAI'' specialty session at the recent BRAIN Initiative Conference (June
2024)~\see{natltalks} and a panel discussion on the future sustainability of the
BRAIN data ecosystem for the INCF Assembly (September 2024)~\see{intltalks}. My
leadership of the BRAIN NeuroAI Workshop planning efforts showcased my ability
to manage complex engagements, including coordinating with logistics contractors
for attendee registration, 508c compliance, venue arrangements, managing hybrid
participation platforms, and organizing an early-career poster competition to
ensure a successful workshop.

From 2019 to 2020 (1 year), prior to joining the NIH, I co-organized a
scientific symposium that took place (virtually) at the 2020 BRAIN Initiative
Investigators Meeting~\see{symposium}. This symposium brought together a panel
of 5 leading researchers, including me, to explore key relationships between
machine learning and dynamical systems neuroscience that are critical to
advancing BRAIN's program objectives relating to theory, modeling, and data
analysis.

\begin{itemize}
  \color{hopkinsblue}
  \item \emph{Visit universities and attend conferences to conduct program
outreach to diverse audiences.}
\end{itemize}

Since January 2023 (2 years), in my current role~\see{jobobd}, I have
regularly represented the NIH BRAIN Initiative at major scientific
conferences and workshops, including recent international presentations at
NeurIPS, SfN, INCF Assembly, IEEE/ACM ICONS, and the Telluride Neuromorphic
Workshop~\see{intltalks}. My outreach activities have built crucial
relationships with academic, commercial, and government partners, exemplified by
my coordination with the Allen Institute, MIT, and other major BRAIN-supported
research centers.

\begin{itemize}
  \color{hopkinsblue}
  \item \emph{Communicate complex scientific topics effectively to a variety of
national audiences.}
\end{itemize}

Since 2010 (14 years), across my career, I have had extensive experience
communicating scientific concepts through diverse channels at national
and international venues, ranging from delivering research talks at
major conferences to participating in policy-focused panel discussions.
\cfcf{talks}{comms}. Recent examples include recording an interview about
theoretical neuroscience and the BRAIN NeuroAI Workshop for the popular
Brain-Inspired Podcast~\see{media}, drafting blog posts for the BRAIN Blog,
developing internal communications and reports about BRAIN Initiative efforts
and priorities, and leading technical and policy-oriented discussions
across scientific domains including neuroscience, AI, informatics, data
science, and neuromorphic computing. 

Since 2010 (14 years), I delivered numerous scientific seminars and invited
lectures~\see{talks} at top research universities---e.g., the University
of Minnesota in 2020 and the University of Toronto in 2022---and public
organizations---e.g., the Neuromatch Conference series from 2020--2021 and
the Air Force Research Laboratory (AFRL) in 2022 and 2023---to communicate
scientific findings and perspectives from my research.

From 2018 to 2021 (4 years), prior to joining the NIH, I collaborated
extensively with the Johns Hopkins University Applied Physics Laboratory
(JHUAPL) on efforts to develop federal funding applications, manage the
resulting awards, and execute the projects by leading a cross-divisional team
to conduct research, publish findings, disseminate outreach materials, and
contribute to educational components. \cfcf{res}{eduprogram}.

My ability to effectively communicate complex topics to varied audiences is
demonstrated through a long history of invited presentations and successful
collaborations at universities, research centers, workshops, conferences,
panels, podcasts, and seminars.

\begin{itemize}
  \color{hopkinsblue}
  \item \emph{Develop policies advancing scientific programs of national
importance related to health.}
\end{itemize}

Since January 2023 (2 years), in my current role~\see{jobobd}, I have led
development and implementation of policies that advance BRAIN's mission,
including data management and sharing requirements that promote open science.
I currently guide strategic planning efforts to establish BRAIN as a leader in
NeuroAI research, working to develop partnerships with other agencies to shape
priorities and policies that will guide the future of neuroscience research and
neurotechnologies critical to brain health.

\begin{itemize}
  \color{hopkinsblue}
  \item \emph{Support program leadership by serving as liaison to teams of
extramural program officers and staff.}
\end{itemize}

Since January 2023 (2 years), in my current role~\see{jobobd}, I have served
as Office of the BRAIN Director liaison to multiple BRAIN Teams and working
groups, including the BRAIN NeuroAI Working Group which I formed in late
2023 and for which I have since served as co-lead. As a team liaison, I have
coordinated activities between program staff, project scientists, intramural
consultants, and leadership, including division directors and branch chiefs
across multiple BRAIN-participating ICs. My accomplishments in developing
DMS policy implementations across BRAIN programs and portfolios demonstrate
my ability to orchestrate complex workflows across diverse roles, resulting
in improved efficiency and reduced burden. 

Since January 2023 (2 years), in addition, I have served as a BRAIN liaison
or representative on trans-NIH working groups (for example, consciousness
research), special interest groups (for example, AI), program (for example,
the NIH Common Fund, including the recently launched Complement-ARIE program),
and the NSF--NIH Smart and Connected Health (SCH) program's cofunding
office team. My contributions to these trans-NIH initiatives include SME
review, implementation planning, concept development, and serving as the
scientific/research contact of record on published NSF and NIH funding
solicitations.

\begin{itemize}
  \color{hopkinsblue}
  \item \emph{Build working relationships with leading scientists by engaging
through conferences and outreach.}
\end{itemize}

In my current role~\see{jobobd}, I have maintained extensive engagement with
the scientific community through conference participation, site visits, and
other investigator meetings. My leadership of the NeuroAI workshop planning
committee exemplifies my ability to build productive relationships with leading
researchers, resulting in successful coordination across academic, government,
and private sector partners.


%%% Requirements & Qualifications %%%

\newsection{Requirements \& Qualifications}{basicreqs}

\subsection*{Conditions of Employment}

\begin{itemize}
  \item \emph{I am a U.S. citizen.}
  \item \emph{I will submit all required documents.}
  \item \emph{All qualifications are already met and will be met by the closing date.}
  \item \emph{I registered with the Selective Service at the appropriate age.}
  \item \emph{Transcripts are attached in fulfillment of the education requirement.}
  \item \emph{I am able to perform the essential duties of the position.}
\end{itemize}

\subsection*{Basic Education Requirements}

I hold a Ph.D. (2009), M.Phil. (2008), and M.A. (2006) in Neurobiology \&
Behavior from Columbia University, where I trained at the Center for Theoretical
Neuroscience. I completed my initial graduate coursework in the Neuroscience
Program at Brandeis University's Volen Center for Complex Systems before
continuing my doctoral research at Columbia. Prior to Brandeis, I earned dual
B.A. degrees in Mathematics and Cognitive Science with a Minor in Philosophy
from the University of Virginia (2003), where I was an Echols Scholar. My
graduate training focused on computational and theoretical approaches to
understanding brain function, directly aligning with the position's requirement
for advanced study in health sciences and allied fields. \cf{education} for
additional details.

\subsection*{Requirements for Applicants at the GS-15 Grade Level}
\vspace{-.1in}
\hrule
\vspace{.1in}
\subsubsection*{A. Direct/Independent Research Qualifying Experience}

\begin{itemize}
  \color{hopkinsblue}
  \item \emph{Managed diverse scientific health-related research programs with
national and international impact.}
\end{itemize}

Since January 2023 (2 years), in my current role supporting the NIH BRAIN
Director, I have supported and contributed to multiple high-impact research
programs focused on whole-brain cell atlasing, circuits/systems neuroscience,
artificial intelligence, ccomputational modeling, theory, open data standards,
and reproducible analytics workflows. \cf{jobobd}. I worked across BRAIN Teams
and led initiatives spanning multiple NIH institutes, including managing the
launch of a new team focused on data sharing and informatics programs, creating
and leading a new working group focused on new opportunities at the intersection
of neuroscience and AI (``NeuroAI''), and leading the planning efforts for the
BRAIN NeuroAI Workshop that took place on November 12 and 13, 2024. The BRAIN
NeuroAI Workshop was a 2-day hybrid event that was attended in-person (200)
or virtually (1,749) by almost 2,000 participants from 48 countries. Through
these efforts, I built collaborative networks across key academic researchers,
commercial partners from startups to the top tech companies, and government
agencies (NIH, ARPA-H, NSF, DOE, DOD), demonstrating my ability to manage
diverse research programs, coordinate crucial outreach, and build the key
relationships needed to enable transformative national and international impact.

From 2019 to 2022 (3 years), as Research Associate faculty in the Johns
Hopkins University School of Medicine (JHU/SOM)~\see{job1}, I served as
principal investigator (or co-PI) of multiple federal grant-awarded projects
that I initiated independently, including a NIH/NINDS R03 award and a NSF/NCS
FOUNDATIONS award, part of the NSF's BRAIN Initiative portfolio. In these
projects, I build computational tools and co-developed cross-disciplinary
mathematical frameworks to address physiological mechanisms underlying
cognitive mapping and episodic memory formation, theoretical advances in
the understanding of neural oscillations as a basis of self-organizing
spatiotemporal activity patterns in brains, and established approaches to
translating theoretical advances to early-stage technological applications,
including multi-agent robotic metacontroller models. \cf{funding}. During that
time, I also independently initiated new collaborations and developed several
other research projects, leading to an inter-disciplinary NSF application in 2020
and an internal JHU application in 2022. \cf{res}.

To manage the above projects (2019--2022, 3 years), I developed protocols
including reference computational models, conducted data analyses, supervised
and mentored graduate students and junior staff~\see{mentoring}, and
administered the grants by coordinating with JHU administrators, integrating and
writing progress reports for funding agencies, applying for no-cost extensions,
and depositing published papers in agency-hosted archives such as the NSF Public
Access Repository. At least six (6) peer-reviewed publications resulted from
these collaborative, cross-divisional projects. \cf{hadzicpub}.
%
%\begin{quote}
  %• Hadzic, Hwang, Zhang, Schultz, Monaco (2022) \emph{Array}. \\
  %• Buckley, Monaco, Schultz, Chalmers, Hadzic, Zhang (2022) In: IEEE
  %Integrated STEM Education Conference (ISEC'22). \\
  %• Schultz, Villafañe-Delgado, Reilly, Saksena, Hwang (2022) In: Emergent
  %Behavior in System of Systems Engineering. \\
  %• Hwang, Schultz, Monaco, Zhang (2021) \emph{JHUAPL Technical Digest}. \\ 
  %• Monaco, Hwang, Schultz, Zhang (2020) \emph{Biological Cybernetics}. \\ 
  %• Monaco, Hwang, Schultz, Zhang (2019) In: International Society for
  %Photonics (SPIE'19). \\
%\end{quote}
%%
%The first author (Hadzic) of the first paper above was a JHUAPL junior staff
%member whom I mentored~\see{mentoring}. \cf{pubs} for complete publication
%details.

\begin{itemize}
  \color{hopkinsblue}
  \item \emph{Developed strategic plans for resource allocation across competing
research programs.}
\end{itemize}

Since January 2023 (2 years), in my current role~\see{jobobd}, I have developed
new priority areas for strategic planning initiatives within the BRAIN
Initiative, including a new vision for a sustainable BRAIN data ecosystem.
At the same time, I have implemented and refined the recently established
data management and sharing (DMS) policies that mandate research data sharing
across the NIH and BRAIN. My DMS workflows have improved program efficiency
while reducing costs. I advised the BRAIN Director on resource allocation
decisions across multiple research programs and helped establish criteria for
evaluating competing proposals. My recommendations on implementation strategies
avoided unnecessary expenditures exceeding \$500,000 while maintaining program
effectiveness.

Since February 2024 (1 year), as founder and co-lead of the BRAIN NeuroAI
Working Group~\see{jobobd}, I organized a substantial strategic planning effort
to identify innovative NeuroAI opportunities for the NIH BRAIN Initiative.
This effort was highly integrated with planning and pre-coordination of the
BRAIN NeuroAI Workshop that I co-organized with my working group in November
2024. The NeuroAI thrust is one of a small number of strategic priority areas
competitively selected by the BRAIN Director in Summer 2024 for continued
scoping and implementation. Since the workshop, I have led the integration,
synthesis, and development of the NeuroAI concept in collaboration with BRAIN
programs and teams, communicated updates to leadership, and developed critical
planning documents to harmonize workshop inputs reflecting community interest
with BRAIN mission-relevance and portfolio synergies.

From 2009 to 2022 (13 years), I was a Research Associate faculty
(3~years) developing and administering an independent cross-divisional
and interdisciplinary research program~\see{job1}; a JHU/SOM Postdoctoral
Fellow (5~years) performing preclinical health-related neuroscience
research~\see{job2}; and a JHU Postdoctoral Fellow (5~years) conducting
behavioral neuroscience and theory-driven computational modeling
research~\see{job3}. In these academic roles, I gained broad experience
developing skills for strategic planning, resource allocation, and project
management in funding-constrained, grant-driven research environments and varied
scientific and administrative contexts.

\begin{itemize}
  \color{hopkinsblue}
  \item \emph{Achieved scientific distinction through research accomplishments
    and professional society membership.}
\end{itemize}

Since obtaining my Ph.D. in 2009 (15 years), I have continually worked to
establish scientific distinction through research contributions in theoretical
and computational neuroscience and adjacent fields, culminating in publications
in prestigious journals including Nature Neuroscience, Cell, PLOS Computational
Biology, Biological Cybernetics, and the Journal of Neuroscience. My work
on spatial cognition, neural computation, and theoretical neuroscience has
been recognized through invited presentations at major institutions and
conferences. Recent invitations to speak at high-profile conferences and venues
including NeurIPS, INCF Assembly, Society for Neuroscience, the BRAIN Initiative
Neuroethics Working Group (NEWG), and the Open Data in Neuroscience (ODIN)
symposium reflect my growing scientific impact. \cfcf{journalpubs}{talks}.

From 2016 to 2018 (3 years), as a JHU/SOM Postdoctoral Fellow~\see{job2}, I
independently initiated collaborations and research projects resulting in an
internal JHU Science of Learning Institute award~\see{funding}
and the development and submission of two external (NIH and NSF) applications
that were awarded starting FY19. In that time, I additionally conceived and
coordinated a large collaborative project proposal among 8~labs across JHU
that was selectively funded by DARPA/BTO and targeted noninvasive stimulation
of neuroplasticity for health-related interventions and augmentation.
\cfcf{funding}{res}

From 2009 to 2014 (5 years), as a JHU Mind/Brain Institute (JHU/MBI) Postdoctoral
Fellow~\see{job3}, I conducted a statistical modeling study of neurobehavioral
datasets and developed protocols for behavioral quantification of rats
performing spatial navigation experiments. My study provided the first evidence
for mammalian long-term memory formation tied to individual behaviors (i.e.,
lateral head-scanning movements). In collaboration with my postdoctoral lab, I
interpreted and published my findings in high-impact journals, including Nature
Neuroscience and Current Biology. \cf{journalpubs}
%
%\begin{quote}
  %• Monaco, Rao, Roth, Knierim (2014) \emph{Nature Neuroscience}. \\ 
  %• Wang, Monaco, Knierim (2020) \emph{Current Biology}. \\ 
%\end{quote}
%
%The first author (Wang) of the second paper above was a graduate student whom
%I mentored~\see{mentoring}. My head-scanning study provided the basis for my
%2010 application and 2011 resubmission of a NIH NRSA Postdoctoral Fellowship
%(F32)~\see{nrsa}, as guided by discussions that I initiated with NINDS PO Jim
%Gnadt. \cf{pubs} for complete details.

\begin{itemize}
  \color{hopkinsblue}
  \item \emph{Represented organizations on committees evaluating scientific
research priorities.}
\end{itemize}

Since January 2023 (2 years), in my current role~\see{jobobd}, I have served as
an extramural SME for programmatic review on multiple BRAIN Teams evaluating
research funding applications across BRAIN programs and portfolios. I have led
working groups focused on artificial intelligence, neuromorphic engineering,
and theoretical neuroscience, culminating in organizing a major workshop that
brought together diverse research communities and stakeholders across multiple
sectors of science, technology, and health to shape future research priorities
for NeuroAI. In addition, I served on the BRAIN Initiative Conference Program
Committee in 2024 and I regularly participate in trans-NIH working groups and
represent BRAIN on program/policy implementation working groups, including for
the NIH Common Fund Complement-ARIE program and for NSF--NIH Smart Health,
for which I am the scientific/research contact of record on the published
solicitations.

\begin{itemize}
  \color{hopkinsblue}
  \item \emph{Published research findings in high-impact scientific journals.}
\end{itemize}

From 2007 to 2024, I have published 14 peer-reviewed papers in leading
journals spanning computational neuroscience, cognitive science, biology, and
artificial intelligence. My work includes seminal contributions in Nature
Neuroscience, Cell, Current Biology, and PLOS Computational Biology. Recent
publications focus on theoretical advances in neurodynamical computing, Bayesian
optimization for neural control, and perspectives on computational modeling in
neurorehabilitation, demonstrating continued scientific productivity and impact.
\cf{journalpubs}.

%From 2014 to 2016, I conducted a computational modeling and neural data analysis
%study in which I developed neural coding protocols, documented my findings, and
%interpreted the results in a substantial original research article integrating
%multiple convergent methodological approaches that I carried from preprint to
%publication in a high-impact journal:
%%
%\begin{quote}
  %• Monaco, De Guzman, Blair, Zhang (2019) \emph{PLOS Computational Biology}. \\ 
  %• Monaco, Blair, Zhang (2017) \emph{bioRxiv}. \\ 
%\end{quote}
%%
%\cf{pubs} for complete publication details. 

\subsubsection*{B. Research Administration/Extramural Qualifying Experience}

\begin{itemize}
  \color{hopkinsblue}
  \item \emph{Established evaluation standards for planning and reviewing
scientific research projects.}
\end{itemize}

Since January 2023 (2 years), I have developed and implemented standard
operating procedures for evaluating data management and sharing (DMS) plans
across the BRAIN Initiative. I created reference language for new funding
opportunity announcements and established workflows for program staff to review
DMS plans submitted with research proposals. My improvements to DMS evaluation
and review processes have streamlined communications between Office of the
BRAIN Director, Program Staff, applicants, and Grants Management, reducing
administrative burden while enhancing the value of rigorous DMS evaluation
standards. \cf{jobobd}.

\begin{itemize}
  \color{hopkinsblue}
  \item \emph{Assessed emerging research trends to advance organizational
missions and program objectives.}
\end{itemize}

Since January 2023 (2 years), I have led efforts to evaluate and shape the
emerging field of NeuroAI within the BRAIN Initiative. Through organizing a
major workshop attended by nearly 2000 participants, I facilitated critical
discussions about integrating artificial intelligence approaches with
neuroscience research. I advised BRAIN leadership on strategic opportunities
in computational modeling, neuromorphic computing, and data-driven research,
helping establish programmatic priorities that advance NIH's mission to improve
human health through scientific discovery. \cf{jobobd}.

Prior to joining the NIH, my 2019 publication (Monaco et al., 2019, PLOS
Comput Biol)~\see{pubs} laid the theoretical and experimental groundwork
for establishing my collaboration with JHUAPL, the resultant NSF-awarded
project~\see{funding}, my co-organization of and participation in a
well-attended NIH BRAIN Investigators symposium in 2020~\see{symposium}, and
my interactions with NSF program directors that led to my initial extramural
contributions to the programmatic development and review of a competitively
funded NSF EFRI program topic (\href{https://www.nsf.gov/awardsearch/advancedSearchResult?PIId=&PIFirstName=&PILastName=&PIOrganization=&PIState=&PIZip=&PICountry=&ProgOrganization=07040000&ProgEleCode=7633&BooleanElement=All&ProgRefCode=&BooleanRef=All&Program=&ProgOfficer=&Keyword=BRAID&AwardNumberOperator=&AwardAmount=&AwardInstrument=&ActiveAwards=true&OriginalAwardDateOperator=&StartDateOperator=&ExpDateOperator=}{\unpubtitle{BRAID}}) 
with \$30M total awarded from 2020 to 2022. \cf{programsvc}.

\begin{itemize}
  \color{hopkinsblue}
  \item \emph{Coordinated multi-agency initiatives to identify and address
research program needs.}
\end{itemize}

Since January 2023 (2 years), in my current role~\see{jobobd}, I have built
collaborative relationships across multiple federal agencies including NSF,
DOE, ARPA-H, and DOD to advance shared research priorities. I initiated the
co-funding participation of the NIH BRAIN Initiative in the joint NSF--NIH Smart
and Connected Health (SCH) program. Moreover, I contributed to internal program
discussions regarding BRAIN co-funding priorities and implementation of the
NSF--NIH CRCNS (Collaborative Research in Computational Neuroscience) Program.

I leveraged this experience during planning for the 2024 BRAIN NeuroAI Workshop
to both initiate and build upon potential cross-agency and cross-sector (private
foundations, investors, and tech companies) partnerships. I am currently leading
post-workshop coordination efforts to coaelesce NeuroAI initiatives across
government agencies and other stakeholders, developing shared interest to lay
the groundwork for for potential joint research investments. \cf{jobobd}.

%\subsection*{Requirements for the GS-13 Level}

%\begin{itemize}
  %\item \emph{Managed significant independent research projects}.
    %\itemnote{\cfcf{basicreqs}{res}.}
  %\item \emph{Supervised graduate researchers and junior technical staff}.
    %\itemnote{\cf{mentoring}.}
  %\item \emph{Published in refereed journals}. \itemnote{\cf{pubs}.}
  %\item \emph{Presented published work to scientific organizations}.
    %\itemnote{\cfcf{talks}{presentations}.}
  %\item \emph{Served as a reviewer on peer-review panels and journals}.
    %\itemnote{\cfcf{service}{programsvc}.}
  %\item \emph{Held the position of Assistant or Associate Professor or
      %equivalent}. \itemnote{\cf{cv} and the above-linked sections to assess the
        %depth and breadth of my academic credentials. Note that, as of 2019, the
        %JHU/SOM Provost for Postdoctoral Affairs evaluated my C.V. as being equivalent
        %to a mid-tenure-track Assistant Professor, i.e., approximately 3 years from
        %tenure and promotion to Associate Professor rank. Given my experience accrued
        %in the three years since that evaluation, I would estimate that my position as
        %of 2022 should be fairly judged as equivalent to a recently tenured Associate
      %Professor.}
  %\end{itemize}

%My professional experience in the academic environment of the JHU/SOM Department
%of Biomedical Engineering has involved each of the listed requirements
%above, particularly with respect to the Research Associate faculty position
%that I held from 2019--2022~\see{job1}. The summaries above starting with
%\cfonly{basicreqs} provide additional context, and the linked sections indicated
%next to the arrows under each requirement provide further supporting details.


%\subsection*{Requirements for the GS-14 Level}

%\begin{itemize}
  %\item \emph{At least one year of professional experience that demonstrates
      %extensive scientific expertise incorporating research experience with
      %varied responsibilities for providing leadership in a scientific area, and
      %functioning as a leader for a variety of efforts, such as directing research
      %and coordinating committee and teaching activities, and organizing and
    %chairing sessions at national scientific meetings.}
  %\item \emph{Served as an appointed member of a scientific peer-review panel
    %or editorial board.} \itemnote{\cf{programsvc}.}
  %\item \emph{Held the position of Associate Professor, Professor or
      %equivalent.} \itemnote{\cf{cv} and also my note under the last listed
      %requirement for the GS-13 level above.}
%\end{itemize}

%I have three years of professional experience as a Research Associate~\see{job1}
%and five years of experience as a Postdoctoral Fellow~\see{job2} in which
%I employed my extensive research experience-based scientific expertise
%to undertake responsibilities for my independently developed portfolio
%of leadership efforts in computational and theoretical neuroscience and
%related subfields of control theory, autonomous systems, embodied cognition,
%and artificial intelligence. While these leadership efforts have been
%described above in response to other duties and requirements, they are
%also detailed in \cfonly{res}, \cfonly{service}, \cfonly{talks},
%\cfonly{presentations}, \cfonly{eduprogram}, and \cfonly{mentoring}. Recognition
%of these efforts---including \emph{Fast Company} magazine ranking JHUAPL as the
%\#3 ``Best Workplace for Innovators'' based on my NSF project---is detailed
%under \cfonly{recognition}.

%Selected examples of my research-based scientific leadership efforts include:
%%
%\begin{itemize}
  %\item Directed research conducted by my student mentees~\see{mentoring}
    %and key personnel or co-investigators on my funded research
    %projects~\see{res}.
  %\item Delivered invited research-based talks at national and international
    %scientific meetings and organizations~\see{natltalks}, e.g., AFRL (Sept 2022),
    %University of Toronto (Feb 2022), NIH BRAIN Investigators Meeting (June 2020),
    %JHU Kavli Neuroscience Discovery Institute (Oct 2019).
  %\item Contributed talks presenting novel theoretical integration of scientific
    %findings to international scientific/educational meetings~\see{intltalks}, e.g.,
    %Neuromatch Conference 3.0 (Oct 2020) and 4.0 (Dec 2021).
  %\item Served on the Review Committee for the Cosyne conference (2016, 2020,
    %2021)~\see{confsvc}.
  %\item Provided scientific peer-review for top computational neuroscience,
    %biology, and artificial intelligence journals including Neuron, eLife, PLOS
    %Computational Biology, Nature Machine Intelligence, and Neural Computation
    %(with the honorific of `Communicator', reserved for highly impactful peer
    %reviewers)~\see{service}.
  %\item Invited to provide inputs to guide programmatic development of DARPA/MTO
    %MEC (Jan 2022).
  %\item Invited and served as a reviewer on multiple NSF panels and as an ad-hoc
    %reviewer to evaluate funding applications according to scientific merit and
    %programmatic criteria (2021--2022)~\see{programsvc}.
  %\item Provided feedback to the NSF to help establish criteria and standards
    %for logistical discussion plans for panels with difficult multidisciplinary,
    %multiday workloads (April 2022).
  %\item Invited and participated in the NSF Workshop on Present and Future
    %Frameworks for Theoretical Neuroscience (Feb 2019)~\see{natltalks} and
    %contributed to a working-group preprint on the role of theory in neuroscience
    %(Levenstein, et al. 2020)~\see{preprints} that is currently under review at
    %the Journal of Neuroscience~\see{pubs}.
  %\item Proposed and co-organized national scientific and educational symposia,
    %e.g., at the NIH BRAIN Investigators Meeting (June 2020)~\see{natltalks} and
    %as part of the JHUAPL STEM Academy Program funded by my collaborative NSF
    %project (Jan 2021)~\see{eduprogram}.
  %\item Across the foregoing efforts, I have sought to convey overarching
    %guidance on emerging directions and caveats regarding methods, approaches, and
    %questions for neuroscience and related subfields.
%\end{itemize}


%%% HSA Self-Assessment Questionnaire --- Supporting Details}

\pagebreak
\newsection{Self-Assessment Questionnaire --- Supporting Details}{questionnaire}
\setlist{itemsep=1pt}

\begin{enumerate}
  \item I qualify for this position because I have completed all requirements
    for and successfully obtained a Ph.D. in Neurobiology \& Behavior from Columbia
    University in 2009.
  \item I confirm that I meet the basic experience requirements for all grades.
  \item Qualifying experience is described above under \cfonly{basicreqs}.
  \item I obtained a Ph.D. in Neurobiology \& Behavior from Columbia University
    in 2009.
  \item My qualifying experience occurred in an Academic Environment,
    as described above under \cfonly{basicreqs} and detailed below under
    \cfonly{workexp} with respect to my roles as a Research Associate,
    Postdoctoral Fellow, and Graduate Research Assistant at R1 research
    universities including Brandeis, Columbia, and Johns Hopkins.
  \item \emph{ibid.}
  \item \emph{ibid.}
  \item Job duties, related skills, and responsibilities for the listed
    qualifying positions are listed in detail within the corresponding
    position-related sections starting under \cfonly{workexp}.
  \item Listed areas of scientific expertise span my training, publications, and
    awarded grants as described above under \cfonly{basicreqs} and detailed below
    under \cfonly{workexp} and \cfonly{pubs}.
  \item My Ph.D. was granted by the Department of Neurobiology \& Behavior at
    Columbia University, where I conducted my doctoral studies as a Graduate
    Research Assistant in the Center for Theoretical Neuroscience as described
    under \cfonly{education}.
\end{enumerate}

For assessment items 11--37, supporting details for each skill or capability are
provided in the notes below the corresponding item.

\begin{enumerate}
  \setcounter{enumi}{10}
  \item Analyze the scientific merit of current and projected research programs. 
    \itemnote{Extramural contributor to development of funding programs at IARPA
    and the NSF~\see{programsvc}.}
  \item Analyze trends and new scientific fields to assess the adequacy of
    research being conducted and proposed in a given area.
    \itemnote{My education was grounded in mathematics and
      philosophy~\see{education}, which has allowed me to build methods that I use
      in my research from scratch and to assess their inferential utility, e.g.,
      with respect to statistical validity, etc. As I developed my independent
      research program~\see{res}, I gained substantial experience
      evaluating current trends in methods, approaches, and scientific questions
    in current and emerging fields.}
  \item Identify and recommend the program direction of new research.
    \itemnote{In nearly 23 years of experience in neuroscience
      research~\see{education}~\see{pubs}~\see{res}, I have developed deep
      intuitions and agility in recognizing and evaluating new lines of research.
      This interdisciplinary research experience has trained me to recognize
      critical distinctions between approaches that support useful recommendations
    on the potential strengths of new research.}
  \item Identify gaps in scientific research.
    \itemnote{As described for the previous two items, my interdisciplinary
      research and long history in neuroscience~\see{education}~\see{pubs} and
      related fields has informed my ability to assess the current state of the
      science. In particular, my theoretical interests inform how I survey for
    epistemic gaps based on processes of inductive reasoning.}
  \item Manage the development of organizational or research group policies in
    response to changes in levels of funding or legislative/program changes.
    \itemnote{Based on my extended history of interactions with program officers
      as described above in \cfonly{duties} and \cfonly{basicreqs}, I have extensive
      familiarity with organizational policies and requirements at funding agencies in
    response to funding level changes due to legislation or other factors.}
  \item Monitor scientific progress of research or clinical programs to assure
    that objectives are met.
    \itemnote{Given my experience, particularly as JHU/SOM Research
      Associate~\see{job1}, I have had to continually monitor scientific
      progress in internally and externally funded research projects at varied
      levels of budget and scope, e.g., for small team projects like my JHU/SLI
    award~\see{sligrant} and larger projects like my NSF award~\see{nsfgrant}.}
  \item Provide guidance or interpret regulations and funding policies regarding
    biomedical, behavioral health, or health-related research.
    \itemnote{As described above under \cfonly{duties}, my extensive
      interactions with program officers and federal funding agencies has deeply
      informed my ability to assess and interpret policy effects for health-releated
    research.}
  \item To verify your level of experience for the previous seven (7) questions,
    please list the name, email address, and phone number of someone who we can
    contact to get more information.
    \itemnote{Prof. Knierim was my postdoctoral supervisor from
      2009--2014~\see{job3}, and he is familiar and up-to-date with my current
    levels of experience and expertise.}
  \item Administer unsolicited grant/contract proposals and provide support
    regarding funding options and areas of program emphasis.
    \itemnote{As described above under \cfonly{duties} and below under
      \cfonly{programsvc}, my extensive interactions with program officers and my
      service on merit-review panels has given me substantial knowledge about the
      processes involved in unsolicited proposals and has informed my ability to
    support funding options aligned with program priorities.}
  \item Appoint and manage grants, contracts, or cooperative agreements review
    panels.
    \itemnote{My participation on multiple federal merit-review
      panels~\see{programsvc} and intensive interaction with BRAIN program staff
      over the last two years in my current role as a Scientific Program Manager
      for the NIH BRAIN Initiative~\see{jobobd} has given me substantial training
      in grants administration processes related to the logistics of appointing
      panels for these purposes.}
    \item Evaluate the technical and scientific merit of funding applications and
      proposals being submitted for approval.
      \itemnote{My extensive background in health-related research and
        scientific review based in the highly multidisciplinary field of
        neuroscience~\see{education}~\see{res}~\see{service} has provided me
        world-class preparation for evaluating the technical and scientific merit of
      funding applications.}
    \item Review applications for funding created by others to ensure
      they have been prepared properly. \itemnote{Reviewing NIH grant
        applications for correctness has been a core responsibility of my
        subject matter expert (SME) duties in my current role~\see{jobobd}
        over the last two years. Before joining the NIH, I prepared numerous
        funding applications and grant proposals intended for various funding
        agencies~\see{funding}~\see{res}, which has given me substantial
        real-world experience and knowledge regarding the proper preparation of
      funding applications.} 
    \item Recommend and distribute research funding for new research
      initiatives/efforts.
      \itemnote{My knowledge and familiarity with the processes of recommending
        and distributing research funding derive from my experience in my current
        role~\see{jobobd} interacting with program officers and BRAIN Teams as
      funding payplans are developed through the Council cycle.~\see{funding}.}
    \item Manage research fund distribution amongst competing interests within an
      organization.
      \itemnote{\emph{ibid.}}
    \item Obtain independent grant funding for research from either the private or
      public sector.
      \itemnote{I have successfully obtained independent research funding from
        public sector agencies~\see{funding} and have submitted several applications to
    private foundations.}
  \item Write or co-write research grants.
    \itemnote{I have extensive experience writing varied forms of grant
    proposals and funding applications. \cf{res}.}
  \item Have you ever served as a Program/Project Officer responsible for
    developing, coordinating, and administering grants, cooperative agreements and
    contracts?
    \itemnote{I have not served as a Program Officer, but I have worked closely
      with Program Officers across the 10 BRAIN-participating ICs in the last two
      years in my current role~\see{jobobd}. At the NIH, I have taken core curriculum
    training courses for Program Officers.}
  \item Deliver presentations to senior medical/health practitioners,
    professional societies, and academia on pertinent matters relating to
    biomedical, behavioral health, or clinical health-related research.
    \itemnote{As described above under \cfonly{duties} and \cfonly{basicreqs},
      I have extensive experience in scientific communication in many settings to
      varied audiences including professional, academic, and educational groups.
    \cfcf{talks}{presentations}.}
  \item Prepare major components of position papers or reviews on policy or
    critical scientific issues related to biomedical, behavioral health, or clinical
    health-related research.
    \itemnote{I have written original research articles across the extent of my
      career, and I have more recently taken to writing higher-level perspective-style
      review papers that give broader views of critical scientific problems and
      questions. My extensive writing experience will translate to preparation of
      position papers, policy reviews, etc., related to scientific or health-related
    issues. \cfcf{pubs}{preprints}.}
  \item Present briefings on research to high level officials providing multiple
    alternatives and advice on the best course of action.
    \itemnote{See notes for items \#17, \#19, \#28, and \#29 above.}
  \item Write reports, evaluations, summary statements, or correspondences as a
    part of a review process.
    \itemnote{As described in the notes for items \#21, \#22, and \#29 and
      my experience writing summary statements on review panels~\see{programsvc},
      I have extensive knowledge, experience, and skill in evaluating proposals
      within a review process and in writing clearly about scientific, technical, and
      health-related topics. These skills will translate to the writing of reports and
    summary statements.}
  \item Coach and mentor staff engaged in biomedical, behavioral health, or
    clinical health-related research.
    \itemnote{\cf{mentoring}.}
  \item Create project plans, including project scope, goals, tasks, resources,
    schedules, costs, contingencies, and communications.
    \itemnote{\cf{res}.}
  \item Manage and direct multiple research or project teams concurrently.
    \itemnote{In the narrative summaries provided above under \cfonly{duties}
      and \cfonly{basicreqs}, the activities that I described occurred
    concurrently or overlapping in time.}
  \item Manage research or project execution to ensure adherence to budget,
    schedule, and scope.
    \itemnote{I have extensive experience managing and co-managing projects of
      varied scale, from small teams to large cross-divisional collaborations,
    including scheduling, budget, and scope of work. \cf{res}.}
  \item Monitor the performance of research or project team members, providing,
    and documenting technical feedback.
    \itemnote{As noted for item \#35, my experience in research project
      management also extends to interacting usefully with team members by listening
      to them, monitoring their progress, and providing constructive and helpful
    feedback at appropriate times.}
  \item Resolve conflicts, differences, or problems among colleagues,
    subordinates, or team members.
    \itemnote{I have worked to resolve conflicts and differences in approach
      between team members in extramural program environments and in research
    projects of various size and scope. \cfcf{res}{mentoring}.}
\end{enumerate}

 
%%% Work Experience %%%

\pagebreak
%!TEX root = cv.tex

\vbox{%
\newsection{Work Experience}{workexp}

\newsubsection{NIH/NINDS Scientific Program Manager [C] (2023--present)}{jobobd}

\begin{tabular*}{6.3in}{l@{\extracolsep{\fill}}r}
  \textbf{Scientific Program Manager} & NSC Bldg \\
  \textbf{Office of the BRAIN Director (OBD)} & 6001 Executive Blvd \\
  NIH National Institute of Neurological Disorders and Stroke (NINDS) &  Rockville, MD\\
  \emph{Employed under contract by Kelly Government Services} &  \\
\end{tabular*}
\\[.1in]
\textbf{1/2023 -- present \\ Full-Time Equivalent, 40--80 Hours/Week} \\
}

\subsection*{Job Duties, Related Skills, and Responsibilities}

\begin{itemize}[noitemsep]
\item Coordinated, developed, and implemented NIH BRAIN Initiative data management and sharing policies
\item Advised BRAIN Initiative Director on strategic priorities for neuroscience research programs
\item Organized a major scientific workshop bringing together neuroscience and AI research communities
\item Evaluated research proposals and provided expert review for multiple BRAIN teams
\item Coordinated cross-agency interactions and programs between NIH, NSF, DOE, ARPA-H, and DOD
\item Developed standard operating procedures to improve program efficiency and effectiveness
\item Presented scientific findings at international conferences and research institutions
\item Managed implementation of data sharing requirements across multiple research consortia
\item Created and led working groups to advance interdisciplinary priorities for AI and neuroscience
\item Provided guidance to investigators on grant applications and program requirements
\item Established new collaborative networks between academic, commercial, and government partners
\item Analyzed research portfolios to identify gaps and strategic opportunities
\item Organized symposia and panels featuring leading experts in neuroscience, AI, and robotics
\item Developed strategic vision for emerging research priorities in theoretical neuroscience
\item Coordinated scientific meetings between program staff and funded investigators
\end{itemize}


\vbox{%
\newsubsection{JHU/SOM Research Associate (2019--2022)}{job1}

\begin{tabular*}{6.3in}{l@{\extracolsep{\fill}}r}
  \textbf{Research Associate (Faculty Rank)} & 733 North Broadway \\
  \textbf{Johns Hopkins University School of Medicine} & Edward D. Miller Research Building \\
  Department of Biomedical Engineering & Baltimore, MD \\
\end{tabular*}
\\[.1in]
\textbf{7/2019 -- 6/2022 \\ Full-Time Equivalent, 40--80 Hours/Week} \\
}

\subsection*{Job Duties, Related Skills, and Responsibilities}

\begin{itemize}[noitemsep]
  \item Initiated research collaborations and continued grant application efforts
  \item Developed computational models of autonomous neural control
  \item Presented research findings in conference presentations, invited talks, and published articles
  \item Delivered invited talks at NIH BRAIN Investigators Meeting symposium and the Air Force Research Lab
  \item Interpreted research about artificial intelligence, swarm cognition, and neuroethology
  \item Supervised high-school and masters students in computational methods
  \item Directed research conduct, budget, and administration as co-PI of NSF-awarded project
  \item Primary and senior authorships on peer-reviewed papers on neural control systems
  \item Responsible for most operational aspects of my sponsor lab (3 years)
  \item Responsible for multiple scientific projects and grant application efforts
  \item Served as reviewer for top journals, conferences, and funding panels
  \item Demonstrated extensive scientific expertise and leadership in funding panels, invited talks, and grants 
  \item Coordinated teaching activities to support STEM component of NSF project
\end{itemize}

\vbox{%
\newsubsection{JHU/SOM Postdoctoral Fellow (2014--2019)}{job2}

\begin{tabular*}{6.3in}{l@{\extracolsep{\fill}}r}
  \textbf{Postdoctoral Fellow} & 733 North Broadway \\
  \textbf{Johns Hopkins University School of Medicine} & Edward D. Miller Research Building \\
  Department of Biomedical Engineering & Baltimore, MD \\
\end{tabular*}
\\[.1in]
\textbf{8/2014 -- 6/2019 \\ Full-Time Equivalent, 40--80 Hours/Week} \\
}

\subsection*{Job Duties, Related Skills, and Responsibilities}

\begin{itemize}[noitemsep]
  \item Independently developed research collaborations and organized grant efforts 
  \item Developed analysis protocols and conducted computational modeling studies 
  \item Presented scientific research findings at conferences and in published papers 
  \item Supervised an exchange student in masters program
  \item Principal investigator for my internal JHU project
  \item First and second authorships on peer-reviewed research articles
  \item Responsible for operating aspects of my sponsor lab by conducting multiple projects and grant efforts 
  \item Demonstrated research-driven leadership to coordinate collaborations
  \item Served as peer reviewer for research journals
  \item Directed research activities to coordinate preliminary work for proposals
\end{itemize}


\vbox{%
\newsubsection{JHU/MBI Postdoctoral Fellow (2009--2014)}{job3}

\begin{tabular*}{6.3in}{l@{\extracolsep{\fill}}r}
  \textbf{Postdoctoral Fellow} & 3400 N. Charles Street \\
  \textbf{Johns Hopkins University} & Krieger Hall \\
  Zanvyl Krieger Mind/Brain Institute & Baltimore, MD \\
\end{tabular*}
\\[.1in]
\textbf{7/2009 -- 7/2014 \\ Full-Time Equivalent, 40--80 Hours/Week} \\
}

\subsection*{Job Duties, Related Skills, and Responsibilities}

\begin{itemize}[noitemsep]
  \item Initiated computational neuroscience research projects to investigate the oscillatory interference theory of temporal coding for path integration
  \item Developed protocols and analysis pipelines for quantifying a particular investigatory behavior in rats during spatial navigation tasks
  \item Conducted research studies using neuroinformatics and neurobehavioral data analysis to discover a behavioral basis of memory formation
  \item Documented analysis and modeling findings in my lab notebooks, lab meeting presentations, scientific conferences, and published research articles
  \item Interpreted published research results in the fields of neural coding, pulse-coupled networks, behavioral ethology, and place cell physiology
  \item Independently developed distinct research projects based on detailed quantification of behavior in experiments and abstract theoretical models of neural coding
  \item Primary authorship of two peer-reviewed journal articles
  \item Responsible for conducting nonoverlapping modeling and analysis projects over the same timeframe
  \item Supervised a graduate student research assistant who learned to use my analysis protocols and pipelines for their thesis work
  \item Presented results from my thesis work at a scientific conference
  \item Served as a peer reviewer for several research journals
  \item Directed the research activities of my student mentee for their thesis work
\end{itemize}


\vbox{%
\newsubsection{Columbia Graduate Research Assistant (2005--2009)}{job4}
%
\begin{tabular*}{6.3in}{l@{\extracolsep{\fill}}r}
  \textbf{Graduate Research Assistant} & 3227 Broadway \\
  \textbf{Columbia University} & Jerome L. Greene Science Center \\
  Center for Theoretical Neuroscience & New York, NY \\
\end{tabular*}
\\[.1in]
\textbf{8/2005 -- 6/2009 \\ Full-Time Equivalent, 40--80 Hours/Week} \\
}

\subsection*{Job Duties, Related Skills, and Responsibilities}

\begin{itemize}[noitemsep]
  \item Initiated multiple interrelated subprojects contributing to my doctoral thesis
  \item Developed a series of protocols for quantifying hippocampal remapping in random foraging experiments to support and strengthen computational modeling results
  \item Conducted computational modeling studies of spatial navigation and the neural coding of space in the hippocampus and entorhinal cortex of rodents
  \item Documented modeling and data analysis findings in lab notebooks, presentations, my doctoral thesis, and two peer-reviewed publications
  \item Interpreted research literatures of experimental, theoretical, and computational approaches to investigating spatial memory and hippocampal function
  \item Independently developed theoretical and computational modeling research projects toward the completion of my doctoral studies
  \item Primary authorship of an original research article describing the main results of my thesis work that was published in a peer-reviewed journal
\end{itemize}


\vbox{%
\newsubsection{Brandeis Graduate Research Assistant (2003--2005)}{job5}

\begin{tabular*}{6.3in}{l@{\extracolsep{\fill}}r}
  \textbf{Graduate Research Assistant} & 415 South Street \\
  \textbf{Brandeis University} & Volen National Center for Complex Systems \\
  Department of Biology & Waltham, MA \\
\end{tabular*}
\\[.1in]
\textbf{8/2003 -- 7/2005 \\ Full-Time Equivalent, 40--80 Hours/Week} \\
}

\subsection*{Job Duties, Related Skills, and Responsibilities}

\begin{itemize}[noitemsep]
  \item Initiated rotation research projects in three different labs based on distinct questions and methodological (computational) approaches
  \item Developed protocols for a rotation lab to facilitate data analysis with the statistical inference programs that I developed for their data
  \item Conducted modeling and data analysis studies in neuroscience and cognitive psychology
  \item Documented my findings in lab notebooks, lab meeting presentations, and similar venues throughout my rotations
  \item Interpreted published research literature in many subfields of neuroscience, computational methods, and statistical and machine learning
  \item Independently developed projects including a doctoral qualifying proposal, several rotation projects, and modeling projects in my graduate lab
  \item Primary authorship of a peer-reviewed research paper based on modeling results from a rotation project
  \item Responsible for multiple projects including rotation projects, qualifying proposal, and new seedling project after joining my graduate lab
  \item Coordinated teaching activities, including extra review sessions, for an introductory neuroscience course and a biology laboratory course with professors and other teaching assistants
\end{itemize}


\vbox{%
\newsubsection{U.Va. Undergraduate Research Assistant (2000--2003)}{job6}

\begin{tabular*}{6.3in}{l@{\extracolsep{\fill}}r}
  \textbf{Undergraduate Research Assistant} & 2028 Cobb Hall \\
  \textbf{University of Virginia} & Laboratory of Computational Neurodynamics \\
  Department of Neurosurgery & Charlottesville, VA \\
\end{tabular*}
\\[.1in]
\textbf{6/2000 -- 7/2003 \\ Part-Time, 20 Hours/Week} \\
}

\subsection*{Job Duties, Related Skills, and Responsibilities}

\begin{itemize}[noitemsep]
  \item Initiated computational modeling projects based on the lab’s existing simulation software to investigate new questions about hippocampal sequence learning
  \item Conducted modeling studies examining limitations of goal-directed sequence learning and recall in a detailed spiking network model of CA3 hippocampus
  \item Documented extensive findings in code, lab meeting presentations, and my research notebooks
  \item Interpreted published literature in the fields of hippocampal physiology and anatomy, spatial navigation and place cells, and episodic memory
  \item Presented the findings of my modeling study as a conference poster and paper at a major international conference for neural networks
\end{itemize}


\vbox{%
\newsubsection{NIH Research Intern (Federal Position, 1996 \& 1997)}{job7}

\begin{tabular*}{6.3in}{l@{\extracolsep{\fill}}r}
  \textbf{High-School Research Intern} & 6555 Rock Spring Drive \\
  \textbf{NIH/CIT} & Building 12A \\
  Center for Molecular Modeling & Bethesda, MD \\
\end{tabular*}
\\[.1in]
\textbf{Summers 1996 -- 1997 \\ Part-Time, 20 Hours/Week \\ Grade: GS-1} \\
}

\subsection*{Job Duties, Related Skills, and Responsibilities}

\begin{itemize}[noitemsep]
  \item Initiated two computational research projects (under supervision) in biochemistry, biophysics, and molecular dynamics
  \item Conducted simulation studies using beowulf high-performance clusters for molecular dynamics models and ligand binding quantification
  \item Documented findings for my projects in lab notebooks and presentation materials
  \item Interpreted published results about deoxyhypusine synthase and hyperthermophilic proteins to guide my computational modeling approaches
  \item Presented results from modeling studies at NIH Poster Day
\end{itemize}




%%% Professional and Academic References --- Contact Info %%%

\pagebreak
\newsection{Professional \& Academic References}{references}
\input{../../jobsearch/references/references_nih}


%%% CURRICULUM VITAE --- SUPPORTING INFORMATION %%%

\pagebreak
\newsection{Curriculum Vitae --- Supporting Information}{cv}
%! TEX program = xelatex
{\fontspec{Verdana}\small
\begin{tabular*}{6.675in}{c@{\extracolsep{\fill}}rl}
  \hline\\[0.02in]
  \textbf{\Large Joseph Daniel Monaco, Ph.D.}          & \textsc{Email}          & \href{mailto:jmonaco@jhu.edu}{\texttt{jmonaco@jhu.edu}} \\
  \multirow{2}{*}{\large }                             & \textsc{Web}            & \href{http://jdmonaco.com/}{\texttt{jdmonaco.com}} \\
  {\small Johns Hopkins University School of Medicine} & \textsc{ORCID}          & \href{http://jdmonaco.com/orcid}{\texttt{0000-0003-0792-8322}} \\
  {\small 720 Rutland Avenue, 407 Traylor}             & \textsc{GitHub}         & \href{https://github.com/jdmonaco?tab=repositories}{\texttt{github.com/jdmonaco}} \\
  {\small Baltimore, MD, 21205, USA}                   & \textsc{Google Scholar} & \href{http://jdmonaco.com/google-scholar}{\texttt{gceOLZEAAAAJ}} \\[0.1in]
  \hline
\end{tabular*}
}\\[0.1in]



%%% Educational History %%%

%!TEX root = cv.tex

\newsection{Education}{education}

\begin{itemize}[itemsep=6pt]
  \item
    \begin{tabular*}{6.3in}{l@{\extracolsep{\fill}}r}
      \textbf{Columbia University} & New York, NY \\
      Center for Theoretical Neuroscience & 2005--2009 \\
      Degrees: Ph.D. (2009); M.Phil. (2008); M.A. (2006) \\
      Advisor: Larry~Abbott\\[0.04in]
      \textbf{Brandeis University} & Waltham, MA \\
      Neuroscience Graduate Program & 2003--2005 \\
    \end{tabular*}
  %\item
    %\begin{tabular*}{6.3in}{l@{\extracolsep{\fill}}r}
      %\textbf{Brandeis University} & Waltham, MA \\
      %Department of Biology & 2003--2005\\
      %Graduate Program in Neuroscience, \textit{\ul{Continued at Columbia University}} \\
      %Advisor: Larry~Abbott\\
    %\end{tabular*}
  \item
    \begin{tabular*}{6.3in}{l@{\extracolsep{\fill}}r}
      \textbf{University of Virginia} & Charlottesville, VA \\
      Laboratory of Computational Neurodynamics & 1999--2003\\
      Degrees: B.A.~Mathematics; B.A.~Cognitive Science; Minor, Philosophy \\
      Advisor: W. B. `Chip' Levy\\
      Echols Scholar \\
    \end{tabular*}
\end{itemize}




%%% Publications %%%

\renewcommand{\itemnote}[1]{}
%
%!TEX root = cv-ninds-hsa.tex
%\subsection*{Pending}
%\label{sec:pendingpubs}

%\begin{description}
%\end{description}

\subsection*{Journal Articles}
\label{sec:journalpubs}

\begin{description}
  \item \href{https://dx.doi.org/10.1007/s12559-022-10081-9}
    {\joehl{Monaco JD} and Hwang GM. (2024). \itemtitle{Neurodynamical computing
      at the information boundaries of intelligent systems}. \emph{Cognitive
    Computation}, 16, 1--13. \doi{10.1007/s12559-022-10081-9}}
  \item \href{https://doi.org/10.1186/s12984-024-01308-x}
    {Hwang GM, Kulwatno J, Cruz TH, Chen D, Ajisafe T, \joehl{Monaco
      JD}, Nitkin R, George SM, Lucas C, Zehnder SM, and Zhang L.
      (2024). \itemtitle{NSF DARE --- Transforming Modeling in
      Neurorehabilitation: Perspectives and Opportunities from US Funding Agencies}.
    \emph{Journal of NeuroEngineering and Rehabilitiation}, 21(17).}
  \item \href{https://doi.org/10.1523/JNEUROSCI.1179-22.2022}
    {Levenstein D, Alvarez VA, Amarasingham A, Azab H, Zhe S. Chen, Gerkin
      RC, Hasenstaub A, Iyer R, Jolivet RB, Marzen~S, \joehl{Monaco JD}, Prinz
      AA, Quraishi SA, Santamaria F, Shivkumar S, Singh MF, Traub R, Rotstein
      HG, Nadim F, and Redish AD. (2023). \itemtitle{On the role of theory and
      modeling in neuroscience}. \emph{Journal of Neuroscience}, 43(7), 1074--88.
    \doi{10.1523/JNEUROSCI.1179-22.2022}} [\arxivlink{2003.13825}]
  \item \href{https://doi.org/10.1016/j.array.2022.100218}{Hadzic
      A, Hwang GM, Zhang K, Schultz KM, and \joehl{Monaco JD}. (2022).
      \itemtitle{Bayesian optimization of distributed neurodynamical
      controller models for spatial navigation}. \emph{Array}, 15, 100218.
    \doi{10.1016/j.array.2022.100218}}
    [\arxivlink{2111.00599}]\label{sec:hadzicpub}
  \item \href{https://dx.doi.org/10.1007/s00422-020-00823-z}
    {\joehl{Monaco JD}, Hwang GM, Schultz KM, and Zhang K. (2020).
      \itemtitle{Cognitive swarming in complex environments with attractor
      dynamics and oscillatory computing}. \emph{Biological Cybernetics}, 114,
    269--284. \doi{10.1007/s00422-020-00823-z}} [\arxivlink{1909.06711}]
  \item \href{https://dx.doi.org/10.1016/j.cub.2020.01.083} 
    {Wang CH, \joehl{Monaco JD}, and Knierim JJ. (2020). \itemtitle{Hippocampal
      place cells encode local surface texture boundaries}. \emph{Current Biology},
    30, 1--13. \doi{10.1016/j.cub.2020.01.083}} [\biorxivlink{10.1101/764282}]
    \itemnote{I mentored the first author in data analysis of rat behavior and
      single-unit recordings, developed the software toolchain used to conduct the
    analyses, and provided intellectual guidance.}\label{sec:wangpub}
  \item \href{https://dx.doi.org/10.1371/journal.pcbi.1006741}
    {\joehl{Monaco JD}, De Guzman RM, Blair HT, and Zhang K. (2019).
      \itemtitle{Spatial synchronization codes from coupled rate-phase
      neurons}. \emph{PLOS Computational Biology}, 15(1), e1006741.
    \doi{10.1371/journal.pcbi.1006741}} [\biorxivlink{10.1101/211458}]
  \item \href{https://www.cell.com/cell/fulltext/S0092-8674(18)31228-5}
    {Tabuchi M, \joehl{Monaco JD}, Duan G, Bell BJ, Liu S, Zhang K, and
      Wu MN. (2018). \itemtitle{Clock-generated temporal codes determine
      synaptic plasticity to control sleep}. \emph{Cell}, 175(5), 1213--27.
    \doi{10.1016/j.cell.2018.09.016}}
    \itemnote{I developed two modeling strategies for the Wu lab’s circadian clock
      neuron experiments in \emph{Drosophila}. My generative statistical model was
      integrated into stimulation protocols as a timing control for behavioral
      results, and my mechanistic molecular/neuronal model explained observed trends
      and made predictions corroborated by the data. My results or contributions are
    featured in 3/7 main figures and 3/6 supplementary figures.}
  \item \href{https://dx.doi.org/10.1038/nn.3687}
    {\joehl{Monaco JD}, Rao G, Roth ED, and Knierim JJ. (2014).
      \itemtitle{Attentive scanning behavior drives one-trial potentiation of
      hippocampal place fields}. \emph{Nature Neuroscience}, 17(5), 725--731.
    \doi{10.1038/nn.3687}}
    [\href{https://jdmonaco.com/files/monaco-2014-author-manuscript.pdf}{\unpubtitle{pdf}}]
    [\href{https://jdmonaco.com/files/monaco-2014-suppinfo.pdf}{\unpubtitle{supp}}]
  \item \href{https://dx.doi.org/10.3389/fncom.2011.00039}
    {\joehl{Monaco JD}, Knierim JJ, and Zhang K. (2011). \itemtitle{Sensory
        feedback, error correction, and remapping in a multiple oscillator model of
      place cell activity}. \emph{Frontiers in Computational Neuroscience}, 5:39.
    \doi{10.3389/fncom.2011.00039}}
  \item \href{https://dx.doi.org/10.1523/JNEUROSCI.1433-11.2011}
    {\joehl{Monaco JD} and Abbott LF. (2011). \itemtitle{Modular
        realignment of entorhinal grid cell activity as a basis for hippocampal
      remapping}. \emph{Journal of Neuroscience}, 31(25), 9414--25.
    \doi{10.1523/jneurosci.1433-11.2011}}
  \item \href{https://dx.doi.org/10.1371/journal.pbio.1000140}
    {Muzzio IA, Levita L, Kulkarni J, \joehl{Monaco J}, Kentros CG, Stead
      M, Abbott LF, and Kandel ER. (2009). \itemtitle{Attention enhances the
        retrieval and stability of visuospatial and olfactory representations
      in the dorsal hippocampus}. \emph{PLOS Biology}, 7(6), e1000140.
    \doi{10.1371/journal.pbio.1000140}}
    \itemnote{I contributed oscillatory power analyses and group-level statistical
      analyses of spiking and bursting for odor vs.\ visuospatial tasks in
    single-unit hippocampal recordings from freely-moving mice.}
  \item \href{https://dx.doi.org/10.1101/lm.363207}
    {\joehl{Monaco JD}, Abbott LF, and Kahana MJ. (2007).
      \itemtitle{Lexico-semantic structure and the recognition
      word-frequency effect}. \emph{Learning \& Memory}, 14(3), 204--213.
    \doi{10.1101/lm.363207}}
\end{description}

% TODO Remove this temporary re-enabling of itemnote display for HSA application
\renewcommand{\itemnote}[1]{
  \begin{description}
    \item[$\rightarrow$] \hspace{.09in}{\color{darkgray}\it #1}
  \end{description}
}

\subsection*{Conference Papers}
\label{sec:confpapers}

\begin{description}
  \item \href{https://jdmonaco.com/files/buckley-2022-ISEC-author.pdf} {Buckley
      E, \joehl{Monaco JD}, Schultz KM, Chalmers R, Hadzic A, Zhang K, Hwang GM,
      and Carr MD. (2022). \itemtitle{An interdisciplinary approach to high school
      curriculum development: Swarming Powered by Neuroscience}. \emph{Proceedings
    of 2022 IEEE Integrated STEM Education Conference (ISEC'22)}.}
    [\arxivlink{2109.05545}]
    \itemnote{\href{https://jdmonaco.com/files/buckley-2022-ISEC-bestpaper.pdf}{
    This paper received First Place in the ISEC Best Paper Award competition.}}
  \item \href{https://www.jhuapl.edu/Content/techdigest/pdf/V35-N04/35-04-Hwang.pdf}
    {Hwang GM, Schultz KM, \joehl{Monaco JD}, and Zhang K. (2021).
      \itemtitle{Neuro-Inspired Dynamic Replanning in Swarms—Theoretical
      Neuroscience Extends Swarming in Complex Environments}. \emph{Johns Hopkins
    APL Technical Digest}, 35, 443--447.}
  \item \href{https://dx.doi.org/10.1117/12.2518966}
    {\joehl{Monaco JD}, Hwang GM, Schultz KM, and Zhang K. (2019).
      \itemtitle{Cognitive swarming: An approach from the theoretical neuroscience
      of hippocampal function}. \emph{Proceedings of SPIE (International society
      for optics and photonics) Defense \& Commercial Sensing}. Micro- and
      Nanotechnology Sensors, Systems, and Applications XI, 109822D, 1--10.
    \doi{10.1117/12.2518966}}
    [\href{https://jdmonaco.com/files/monaco-SPIE2019-cognitive-swarming.pdf}{\unpubtitle{pdf}}]
  \item \href{https://dx.doi.org/10.1109/IJCNN.2003.1223655}
      {\joehl{Monaco JD} and Levy WB. (2003). \itemtitle{T-maze training of a
          recurrent CA3 model reveals the necessity of novelty-based modulation
        of LTP in hippocampal region CA3}. \emph{Proceedings of 2003 IEEE/INNS
        International Joint Conference on Neural Networks (IJCNN'03)}, 1655--1660.
      \doi{10.1109/IJCNN.2003.1223655}}
      [\href{https://jdmonaco.com/files/monaco-2003-tmaze.pdf}{\unpubtitle{pdf}}]
    \itemnote{This paper received First Place in the IJCNN Best Student Poster competition.}
\end{description}

% TODO and this...
\renewcommand{\itemnote}[1]{}

\subsection*{Preprints}
\label{sec:preprints}

\begin{description}
  \item \href{https://arxiv.org/abs/2105.07284}
    {\joehl{Monaco JD}, Rajan K, and Hwang GM. (2021). \itemtitle{A brain
      basis of dynamical intelligence for AI and computational neuroscience}.
    \arxiv{2105.07284}}
\end{description}

\subsection*{Thesis}
\label{sec:thesis}

\begin{description}
  \item \href{https://jdmonaco.com/files/monaco-phdthesis-2009.pdf}
    {\joehl{Monaco JD}. (2009). \itemtitle{Models and mechanisms for integrating
    cortical feature spaces}. Doctoral Dissertation, Columbia University, New York.}
  \href{https://search.proquest.com/docview/304862872/abstract}
  {\unpubtitle{ProQuest Publication No. AAT 3393609}}
  \href{https://jdmonaco.com/files/monaco-phdthesis-2009.pdf}{[\unpubtitle{fullcolor}]}
\end{description}




%%% Funding Award History %%%

\smallskip
%!TEX root = cv.tex

\newsection{Funding Award History}{funding}

\begin{itemize}
  \item \href{https://www.nsf.gov/awardsearch/showAward?AWD_ID=1835279&HistoricalAwards=false}
    {\itemtitle{NCS-FO: Spatial intelligence for swarms based on hippocampal
    dynamics}}\hspace{\stretch{1}}2018--2021
    \begin{itemize}
      \item NSF\slash NCS FOUNDATIONS (BRAIN Initiative) Award No.~1835279: \$862K/\$997K (Direct/Total)
      \item \textbf{Lead PI:} Kechen Zhang
      \item \textbf{Co-PIs, JHUAPL}: Grace Hwang, Robert W. Chalmers, Kevin
        Schultz, and M. Dwight Carr
      \item \textbf{Research Associate (FY19)/Co-PI (FY20--FY21): \joehl{Joseph D. Monaco}}
    \end{itemize}
  \itemnote{I co-developed this project and co-wrote the proposal
      with a JHUAPL colleague (see \emph{\nameref{sec:nsfgrant}} on
      p.\pageref{sec:nsfgrant}). As a Research Associate faculty at JHU as of
    FY20, my project role was promoted to co-PI.}
\end{itemize}

\begin{itemize}
  \item \href{https://projectreporter.nih.gov/project_info_description.cfm?aid=9652210&icde=42555668&ddparam=&ddvalue=&ddsub=&cr=2&csb=default&cs=ASC&pball=}
    {\itemtitle{Spiking network models of sharp-wave ripple sequences with\\
    gamma-locked attractor dynamics}}\hspace{\stretch{1}}2018--2020
    \begin{itemize}
      \item NIH/NINDS R03 Award No.~NS109923: \$50K/\$82K (Direct/Total)
      \item \textbf{PI:} Kechen Zhang
      \item \textbf{Research Associate:} \joehl{Joseph D. Monaco}
    \end{itemize}
  \itemnote{I conceived this project, generated preliminary data, and wrote the
      proposal (see \emph{\nameref{sec:nihgrant}} on p.\pageref{sec:nihgrant}).
    As a Postdoctoral Fellow, JHU policy precluded a PI role.}
\end{itemize}

\begin{itemize}
  \item \href{https://scienceoflearning.jhu.edu/research/learning-to-explore-paths-through-space/}
    {\itemtitle{Learning to explore paths through space}}\hspace{\stretch{1}}2016--2018
    \begin{itemize}
      \item JHU/Science of Learning Institute (SLI) Award: \$150K
      \item \textbf{PI:} Kechen Zhang
      \item \textbf{Co-PI:} David J.~Foster (now at UC Berkeley)
      \item \textbf{Research Associate:} \joehl{Joseph D.~Monaco}
    \end{itemize}
  \itemnote{I conceived this project, initiated the collaboration
      between the Zhang and Foster labs, and wrote the proposal (see
      \emph{\nameref{sec:sligrant}} on p.\pageref{sec:sligrant}). As a
    Postdoctoral Fellow, JHU policy precluded a PI role.}
\end{itemize}


%
\renewcommand{\itemnote}[1]{
  \begin{description}
    \item[$\rightarrow$] \hspace{.09in}{\color{darkgray}\it #1}
  \end{description}
}


%%% Professional Service --- Peer Review & Programmatic Contributions %%%

\smallskip
%!TEX root = cv.tex

\newsection{Professional Service --- Scientific Peer Review}{service}

\subsection*{Journals}
\vspace{-0.1in}

\lefttabline{0.8in}{2023}{Nature Communications}
\lefttabline{0.8in}{2021}{PLOS Computational Biology}
\lefttabline{0.8in}{2021}{Nature Machine Intelligence}
\lefttabline{0.8in}{2020}{Neuroscience and Biobehavioral Reviews}
\lefttabline{0.8in}{2020}{Scientific Reports}
\lefttabline{0.8in}{2019}{eLife}
\lefttabline{0.8in}{2019}{Hippocampus}
\lefttabline{0.8in}{2018--2019}{Neuron}
\lefttabline{0.8in}{2018}{Neural Computation (including as `Communicator')}
\lefttabline{0.8in}{2018}{PLOS ONE}
\lefttabline{0.8in}{2017}{PeerJ}
\lefttabline{0.8in}{2015}{IEEE Transactions in Biomedical Engineering}
\lefttabline{0.8in}{2012--2020}{IEEE Neural Networks}
\lefttabline{0.8in}{2012}{Biological Cybernetics}
\lefttabline{0.8in}{2012}{Neurocomputing}
\lefttabline{0.8in}{2012}{Neuroscience}

\subsection*{U.S. Funding Agencies}
\label{sec:programsvc}
\vspace{-0.1in}

\lefttabline{0.8in}{2024}{AFOSR (Air Force Ofc of Sponsored Research), Ad-Hoc Reviewer}
\lefttabline{0.8in}{2023--pres.}{NIH BRAIN Initiative, Extramural SME \& Programmatic Review}
\lefttabline{0.8in}{2022}{NSF CAREER Ad-Hoc Reviewer}
\lefttabline{0.8in}{2022}{NSF EFRI Preliminary Review Panel}
\lefttabline{0.8in}{2022}{NSF EFRI Final Review Panel}
\lefttabline{0.8in}{2020--2022}{NSF EFRI Program Development, Extramural Contributor}
\lefttabline{0.8in}{2014}{IARPA Program Development, Extramural Contributor}

\vbox{%
\subsection*{Conferences}
\label{sec:confsvc}
\vspace{-0.1in}

\lefttabline{0.8in}{2024}{NICE (Neuro-Inspired Computational Elements) Conference, Ad-Hoc Reviewer}
\lefttabline{0.8in}{2020--2021}{Cosyne, Review committee member}
\lefttabline{0.8in}{2016}{Cosyne, Review committee member}
}




%%% Professional and Scientific Presentations %%% 

\smallskip
%!TEX root = cv.tex

\newsection{Professional \& Scientific Presentations}{talks}

\subsection*{International}
\vspace{-0.1in}
\label{sec:intltalks}

\begin{longtable}{@{\hspace{0.0in}}l>{\raggedright\arraybackslash}p{.82\textwidth}}
  12/14/2024 & \href{https://neurips.cc/virtual/2024/workshop/84721}
  {``\itemtitle{Looking forward from the BRAIN workshop to a transformative
    future for NeuroAI}.'' \emph{Invited Speaker}. NeuroAI Workshop, NeurIPS (Neural
  Information Processing Systems) 2024 Conference, Vancouver, BC, Canada}
  [\href{https://neurips.cc/virtual/2024/84817}{\unpubtitle{SlidesLive}}] \\
  \tabularnewline
  11/12--13/2024 \hspace{0.1in} & \href{https://braininitiative.nih.gov/news-events/events/brain-neuroai-workshop}
  {``\itemtitle{Introduction to the Workshop: Gaps, Questions, and
    Opportunities}.'' \emph{Organizer, Speaker, Moderator}. BRAIN NeuroAI Workshop,
  NIH Main Campus, Bethesda, MD (Hybrid).}
  [\href{https://videocast.nih.gov/watch=55160}{\unpubtitle{VideoCast Day 1}} \&
  \href{https://videocast.nih.gov/watch=55262}{\unpubtitle{Day 2}}] 
  [\href{https://braininitiative.nih.gov/sites/default/files/documents/BRAIN%20NeuroAI%20Workshop%20Summary_2024_508c.pdf}{\unpubtitle{Meeting Summary}}] \\
  \tabularnewline
  10/5/2024 & \href{https://www.sfn.org/meetings/neuroscience-2024/sessions-and-events/professional-development-workshops#Saturday,-October-05}
  {``\itemtitle{The NIH BRAIN Initiative: Working with AI in Neuroscience}.''
    \emph{Invited Speaker, Panelist}. Society for Neuroscience 2024, Professional
    Development Workshop on ``\unpubtitle{Working With and Working For AI}'',
  Chicago, IL} \\
  \tabularnewline
  9/25/2024 & \href{https://neuroinformatics.incf.org/2024/sessions}{``\itemtitle{The
    present and future of the BRAIN data ecosystem for neuroscience and beyond}.''
    \emph{Co-Organizer, Speaker, Panelist}. INCF (International Neuroinformatics
    Coordinating Facility) Assembly. {Panel on \unpubtitle{\emph{BRAIN
          Initiative Informatics Perspectives: Funding and Building a Sustainable Data
  Ecosystem}}}, Austin, TX} \\
  \tabularnewline
  7/31/2024 & \href{https://iconsneuromorphic.cc/schedule/}
  {\unpubtitle{Discussion Panel on Neuromorphic Systems and U.S. Government}.
  \emph{Invited Speaker, Panelist}. IEEE/ACM ICONS (International Conference on 
Neuromorphic Systems) 2024, George Mason University, Arlington, VA } \\
  \tabularnewline
  7/12/2024 &
  \href{https://sites.google.com/view/telluride-2024/home}{``\itemtitle{Coordination 
      dynamics of behavior and cognitive computation: Rethinking emergent control}.'' 
      \emph{Invited Talk}. Telluride Neuromorphic Workshop 2024, Telluride, CO}.
    [\href{https://youtu.be/7b69GXB2jBc?si=xuzXOZDqXWl6I_sO}{\unpubtitle{YouTube}}] \\
  \tabularnewline
  10/13/2023 & \href{https://jdmonaco.com/files/monaco-IAPCT-2023-slides.pdf}
    {``\itemtitle{Cognitive-narrative dynamics of self-perspective control
    across the lifespan}.''} \emph{Invited Talk}. 33rd Annual International
    Association for Perceptual Control Theory (IAPCT) Conference, 
  \href{https://www.iapct.org/uncategorized/utc-4-boston-time-zone/}
  {Session 7 on \unpubtitle{\emph{Consciousness and the Self}}}, Virtual
  [\href{https://jdmonaco.com/files/monaco-IAPCT-2023-slides.pdf}{\unpubtitle{pdf}}] \\
  \tabularnewline
  10/12/2023 & \href{https://odin.mit.edu/schedule.html}
    {``\itemtitle{Beyond ‘FAIR’: What does sustainable protocolization of
    open data in neuroscience look like?}'' \emph{Invited Panelist, Keynote
    Speaker}. Open Data in Neuroscience (ODIN) Symposium, Massachusetts Institute
  of Technology, Boston, MA} \\
  \tabularnewline
  3/8/2023 & \href{https://meetings.aps.org/Meeting/MAR23/Session/M01.13}
  {``\itemtitle{Neurodynamical computing at the information boundaries of
    intelligent systems}.'' \emph{Contributed Talk}. American Physical Society (APS)
  March Meeting, Las Vegas, NV}
  [\href{https://jdmonaco.com/files/monaco-APS-March-Meeting-2023-slides.pdf}{\unpubtitle{pdf}}] \\
  \tabularnewline
  2/1/2022 & ``\unpubtitle{Theory-Driven Data Science to
  Understand the Neural Dynamics of Memory and Behavior}.'' \emph{Invited Talk}.
  Department of Cell \& Systems Biology, University of Toronto, Canada, Virtual \\
  \tabularnewline
  %12/1/2021 & ``\unpubtitle{Learning as swarming: Cognitive
  %flexibility from the neural dynamics of phase-coupled attractor maps}.''
  %\emph{Contributed Talk}. Neuromatch 4.0 Conference, Virtual \\
  %\tabularnewline
  12/1/2021 & \href{https://youtu.be/3mKkLksOyfk}{``\itemtitle{Learning as
  swarming: Cognitive flexibility from the neural dynamics of phase-coupled
  attractor maps}.'' \emph{Contributed Talk}. Neuromatch 4.0 Conference,
  Virtual \itemtitle{[YouTube]}} \\
  \tabularnewline
  %10/29/2020 & ``\unpubtitle{Spatial theta-phase coding in
  %the lateral septum: A theory of allocentric feedback during navigation}.''
  %\emph{Contributed Talk}. Neuromatch 3.0 Conference, Virtual \\
  %\tabularnewline
  10/29/2020 & \href{https://www.youtube.com/watch?v=WwYDMpD7j4Q}{``\itemtitle{Spatial
  theta-phase coding in the lateral septum: A theory of allocentric feedback
  during navigation}.'' \emph{Contributed Talk}. Neuromatch 3.0 Conference,
  Virtual \itemtitle{[YouTube]}} \\
  \tabularnewline
  10/7/2020 & ``\unpubtitle{Computing path integration with
  oscillatory phase codes in biological and artificial systems}.'' \emph{Data
  Blitz}. iNAV Symposium 2020, Virtual \\
  \tabularnewline
  7/1/2010 & ``\unpubtitle{Medial versus lateral modes for
  reconfiguring hippocampal representations}.'' \emph{Invited Talk}. Grid
  Cell Meeting, Gatsby Computational Neuroscience Unit, UCL, UK \\
\end{longtable}


\subsection*{National}
\label{sec:natltalks}
\vspace{-0.1in}

\begin{longtable}{@{\hspace{0.1in}}l>{\raggedright\arraybackslash}p{.82\textwidth}}
  8/28/2024 & \href{https://www.aim-ahead.net/webinar-series/ai-cares/}
  {``\itemtitle{The NIH BRAIN Initiative: AI in Neuroscience}.'' \emph{Invited
  Talk}. AI-CARES (AI Career Advancement and Resources) Webinar Series,
  AIM-AHEAD, Virtual}
  %Webinar Series, AIM-AHEAD (Artificial Intelligence/Machine Learning Consortium
  %to Advance Health Equity and Researcher Diversity), Virtual}
  [\href{https://www.aim-ahead.net/media/u3bmfrsa/ai-cares-dr-monaco.pdf}{\unpubtitle{pdf}}] \\
  \tabularnewline
  8/21/2024 & \href{https://braininitiative.nih.gov/sites/default/files/documents/BRAIN%20NEWG_Agenda_Aug%202024%20v8_508C.pdf}
    {``\itemtitle{Introduction and Overview of AI, Neuroscience, and Ethics}.''
      \emph{Co-Organizer, Speaker}. BRAIN Neuroethics Working Group (NEWG)
    Workshop, NIH/NINDS, Virtual} 
    [\href{https://videocast.nih.gov/watch=54989}{\unpubtitle{VideoCast}}] \\
  \tabularnewline
  6/18/2024 & \href{https://brainmeeting.swoogo.com/2024/agenda#Wednesday}
  {``\itemtitle{Specialty Session: BRAIN and the Future of Computing: Emerging
    Perspectives on Embodied NeuroAI Research}.'' \emph{Co-Organizer, Moderator}.
  10th Annual BRAIN Initiative Conference, Rockville, MD} 
  [\href{https://youtu.be/3W0o0Mmc60o?si=ejmzy70NFG0jAOp_}{\unpubtitle{YouTube}}] \\
  \tabularnewline
  9/26--28/2023 & BRAIN Initiative Cell Atlas Network (BICAN)
  Knowledge Base Workshop. \emph{BRAIN Liaison \& Invited Participant}. Allen
  Institute for Brain Sciences, Seattle, WA \\
  \tabularnewline
  7/17--18/2023 \hspace{0.1in} &
  \href{https://event.roseliassociates.com/brain-newg-ws-july-2023/}
  {\itemtitle{Workshop on Ethics of Sharing Individual Level Human Brain
    Data Collected in Biomedical Research}. \emph{Co-Organizer, Breakout
    Moderator/Reporter}. BRAIN Initiative Neuroethics Working Group (NEWG), NIH,
  Bethesda, MD (Hybrid)} \\
  \tabularnewline
  5/9/2023 & 
  \href{https://jdmonaco.com/files/monaco_TheoryOfTheory_slides.pdf}
  {``\itemtitle{Theory of theory: On the role of
  theory and modeling in neuroscience}.'' \emph{Invited Extramural Seminar}.
  NIH, Virtual [\unpubtitle{pdf}]} \\
  \tabularnewline
  4/28/2023 & 
  \href{https://jdmonaco.com/files/monaco-2023-afrl-quest-slides.pdf}
  {``\itemtitle{Neurodynamical Articulation: Decoupling Intelligence from the
  Experiencing Self}.'' \emph{Invited Public Seminar}. QuEST, Air Force Research
  Lab/Autonomous Capabilities Team 3 (AFRL/ACT3), Virtual [\unpubtitle{pdf}]} \\
  \tabularnewline
  12/21/2022 & ``\unpubtitle{Finding Causal Paths Across Scales:
  Embodied Control, Ethological Interaction, and Theory-Driven Neural Data
  Science}.'' \emph{Invited Talk}. Division of Neuroscience and Behavior,
  NIH/NIDA, Virtual \\
  \tabularnewline
  11/17/2022 & ``\unpubtitle{Finding Causal Paths Across Scales:
  Embodied Control, Ethological Interaction, and Theory-Driven Neural Data
  Science}.'' \emph{Invited Talk}. Division of Neuroscience and Basic Behavioral
  Science, NIH/NIMH, Virtual \\
  \tabularnewline
  8/26/2022 &
  \href{https://jdmonaco.com/files/monaco-2022-afrl-quest-slides.pdf}
  {``\itemtitle{Brain oscillations: From cortical computing to the existential
  nonduality of conscious agents}.'' \emph{Invited Public Seminar}. Qualia
  Exploitation for Sensor Technology (QuEST), Air Force Research Lab/Autonomous
  Capabilities Team 3 (AFRL/ACT3), Virtual [\unpubtitle{pdf}]}\\
  \tabularnewline
  6/1/2020 & \href{https://youtu.be/2jy1ENYHRAw?t=902}
  {``\itemtitle{Can Transitory Neurodynamics Unify Learning Theories for Brains
  and Machines?}'' \emph{Invited Talk \& Panel Discussion}.}
  6th Annual BRAIN Initiative Investigators Meeting,
  \href{https://www.labroots.com/webinar/symposium-1-dynamical-systems-neuroscience-reciprocally-advance-machine-learning} 
  {Symposium 1 on \unpubtitle{\emph{How Can Dynamical Systems Neuroscience
  Reciprocally Advance Machine Learning?}}}, NIH, Virtual 
  [\href{https://youtu.be/2jy1ENYHRAw?t=902}{{\itemtitle{YouTube}}}] \\
  \label{sec:symposium}
  \tabularnewline
  5/18/2020 & ``\unpubtitle{Computational Approaches to the
  Neural Dynamics of Time, Memory, and Behavior}.'' \emph{Invited Talk}.
  Department of Neuroscience, Medical Discovery Team for Optical Imaging,
  University of Minnesota, Virtual \\
  \tabularnewline
  2/24/2020 & ``\unpubtitle{Computational Mechanisms of Memory:
  Linking Behavior, Space, \& Time}.'' \emph{Invited Talk}. Department of
  Psychology, University of Nevada, Las Vegas, NV \\
  \tabularnewline
  1/31/2020 & ``\unpubtitle{Attractors, memory, and oscillations:
  Computational motifs of spatial learning}.'' \emph{Invited Talk}.
  Department of Biological Sciences, University of Texas at El Paso, El Paso, TX \\
  \tabularnewline
  4/17/2019 & ``\unpubtitle{Emergent dynamics of hippocampal
  circuitry as a basis for robust self-organized planning in mobile swarms}.''
  \emph{Invited Talk}. International Society for Optics and Photonics (SPIE)
  Defense \& Commercial Sensing 2019 Conference, Baltimore, MD \\
  \tabularnewline
  4/10/2019 & NSF/Neural \& Cognitive Systems (NCS) PI
  Workshop. \emph{Invited Participant}. Marriott Wardman Park Hotel, Washington, D.C. \\
  \tabularnewline
  2/3--7/2019 \hspace{0.1in} & NSF/BRAIN Initiative Workshop: Present and Future Frameworks
  of Theoretical Neuroscience. \emph{Invited Participant}. University of Texas,
  San Antonio, TX \\
  \tabularnewline
  1/3/2014 & \href{https://jdmonaco.com/files/ScanningSlide.pdf}
  {``\itemtitle{Head scans drive the formation and potentiation of place
  fields during exploration}.'' \emph{Data Blitz}. 38th Winter Conference on
  Neurobiology of Learning \& Memory, Park City, UT} \\
  \tabularnewline
  4/10/2009 & ``\unpubtitle{Rapid spatial map formation and remapping by
  competing over grid cell inputs}.'' \emph{Thesis Seminar}. Department of
  Neurobiology \& Behavior, Columbia University, New York, NY 
  [\href{https://jdmonaco.com/files/monaco-2009-thesis-seminar-Keynote.mp4}
  {\unpubtitle{Keynote Movie Export (mp4)}}] \\
\end{longtable}

\subsection*{Regional}
\vspace{-0.1in}

\begin{longtable}{@{\hspace{0.2in}}l>{\raggedright\arraybackslash}p{.82\textwidth}}
  10/2/2019 \hspace{0.1in} & ``\unpubtitle{Oscillations, attractors, and
  sequences: Extending hippocampal computations to artificial systems}.''
  \emph{Invited Talk}. Kavli Neuroscience Discovery Institute, Johns Hopkins
  University, Baltimore, MD\\
  \tabularnewline
  1/22/2016 & ``\unpubtitle{Hippocampal circuits for space, memory, and
  navigation: From minimal models to biologically inferred networks}.''
  \emph{Invited Talk}. Department of Pharmacology, University of Maryland,
  Baltimore, MD\\
  \tabularnewline
  9/6/2014 & ``\unpubtitle{Stopping to look: How attentive scanning behavior
  reveals the formation of new memories}.'' \emph{Department Retreat Seminar}.
  Department of Neuroscience, Johns Hopkins University, Baltimore, MD\\
  \tabularnewline
  4/21/2014 & ``\unpubtitle{Landmark influence: How attention to sensory cues
  stabilizes and updates the hippocampal cognitive representation of space}.''
  \emph{Advanced Researcher Seminar}. Zanvyl Krieger Mind/Brain Institute, Johns
  Hopkins University, Baltimore, MD\\
  \tabularnewline
  4/1/2014 & ``\unpubtitle{Hippocampus and declarative memory:
  Head scanning}.'' \emph{Department `Lab Lunch' Seminar}. Department of
  Neuroscience, Johns Hopkins University, Baltimore, MD\\
\end{longtable}


\smallskip
\subsection*{Research Poster Presentations}
\label{sec:presentations}
\vspace{-0.1in}

\begin{description}[itemsep=8pt]
  \item[\quad]
    \href{https://www.cvent.com/events/6th-annual-brain-initiative-investigators-meeting/custom-116-4e2aadaa6cd549a3a4b53113cd172ad2.aspx}
    {\joehl{Monaco JD}, Hwang GM, Schultz K, Zhang K. (2020).
    \itemtitle{Cognitive swarming in complex environments with attractor
        dynamics and oscillatory computing}. \emph{6th Annual BRAIN Initiative
    Investigators Meeting}. Online, with audio narration. June~2020.}
  \item[\quad]
    \href{https://www.fens.org/Meetings/The-Brain-Conferences/Dynamics-of-the-brain/}
    {\joehl{Monaco JD}, Hwang GM, De Guzman RM, Blair HT, Zhang K. (2019).
    \itemtitle{Spatial rate-phase coding in lateral septal ‘phaser cells’:
        single-unit data and theta-bursting models}. \emph{FENS (Federation of
        European Neuroscience Societies) Dynamics of the brain: Temporal aspects of
    computation}. North Copenhagen, Denmark. June~2019.}
  \item[\quad]
    \href{https://www.cvent.com/events/5th-annual-brain-initiative-investigators-meeting/event-summary-de9c0d8f934b46eb8d80b55bcfbfe96a.aspx}
    {\joehl{Monaco JD}, Hwang GM, Schultz K, Zhang K. (2019).
    \itemtitle{Self-organized swarm control using neural principles of spatial
      phase coding}. \emph{5th Annual BRAIN Initiative Investigators Meeting}.
    Washington, D.C. April~2019.} 
  \item[\quad]
    \href{https://abstractsonline.com/pp8/#!/4649/presentation/10884}
    {Hwang GM, Schultz K, \joehl{Monaco JD}, Chalmers RW, Lau SW, Yeh BY,
      Zhang K. (2018). \itemtitle{Self-organized swarm control using neural
      principles of spatial phase coding}. \emph{Society for Neuroscience}.
    San Diego, CA. November~2018.}
  \item[\quad]
    \href{https://www.abstractsonline.com/pp8/#!/4376/presentation/6085}
    {\joehl{Monaco J}, Blair HT, Zhang K. (2017). \itemtitle{Decoding
        septohippocampal theta cells during exploration reveals unbiased
      environmental cues in firing phase}. \emph{Society for Neuroscience}.
    Washington, D.C. November~2017.}
  \item[\quad]
    \href{https://jdmonaco.com/files/monaco-paper-cosyne15.pdf}
    {\joehl{Monaco JD}, Blair HT, Zhang K. (2015). \itemtitle{Spatial
        rate/phase correlations in theta cells can stabilize randomly drifting path
    integrators}. \emph{Cosyne}. Salt Lake City, UT. March~2015.}
  \item[\quad]
    \href{https://www.abstractsonline.com/Plan/ViewAbstract.aspx?sKey=973d2662-ba7a-4ad2-aff9-fe0d4b77c262&cKey=9917ffaf-9e31-4213-acb9-4aab498ab4cd&mKey=54c85d94-6d69-4b09-afaa-502c0e680ca7}
    {\joehl{Monaco J}, Blair HT, Zhang K. (2014). \itemtitle{Spatial rate/phase
        codes provide landmark-based error correction in a temporal model of theta
    cells}. \emph{Society for Neuroscience}. Washington, D.C.  November~2014.}
  \item[\quad]
    \href{https://www.abstractsonline.com/Plan/ViewAbstract.aspx?sKey=bfb59866-8deb-44a6-9515-a7aab630507b&cKey=d201b3aa-7725-452e-b0dd-c41d204b5b54&mKey=54c85d94-6d69-4b09-afaa-502c0e680ca7}
    {Wang CH, Rao G, \joehl{Monaco JD}, Deshmukh SS, Knierim JJ. (2014).
    \itemtitle{Potentiation of place fields along the CA1 transverse axis by
      investigatory head-scanning behavior}. \emph{Society for Neuroscience}. 
    Washington, D.C. November~2014.}
  \item[\quad]
    \href{https://www.abstractsonline.com/Plan/ViewAbstract.aspx?sKey=32eccac1-4e1d-4e81-bf5c-f39bcb605757&cKey=4710dece-cc8e-4b48-8764-49ea174b91ef&mKey=8d2a5bec-4825-4cd6-9439-b42bb151d1cf}
    {\joehl{Monaco J}, Rao G, Knierim JJ. (2013). \itemtitle{Scanning behavior
        in novel environments promotes \emph{de novo} formation of hippocampal place
    fields in rats}. \emph{Society for Neuroscience}. San Diego, CA. November~2013.}
  \item[\quad]
    \href{https://www.abstractsonline.com/Plan/ViewAbstract.aspx?sKey=f5b9fa94-7d15-48c7-9d67-b89cd2883025&cKey=a53349ca-41b1-4664-b022-85d0d1fe59b8&mKey=70007181-01C9-4DE9-A0A2-EEBFA14CD9F1}
    {\joehl{Monaco J}, Rao G, Knierim JJ. (2012). \itemtitle{Hippocampal LFP
      during rodent head-scanning behavior: Theta and sharp-wave ripples}.
    \emph{Society for Neuroscience}. New Orleans, LA. October~2012.}
  \item[\quad]
    \href{https://www.abstractsonline.com/Plan/ViewAbstract.aspx?sKey=c48e9f5f-1274-4486-85bf-38ee591629e1&cKey=190bd951-c183-428d-a4c5-01eb61556d79&mKey=8334BE29-8911-4991-8C31-32B32DD5E6C8}
    {\joehl{Monaco J}, Rao G, Knierim JJ. (2011). \itemtitle{Hippocampal place
        cell firing during head-scanning movements is associated with the formation
    of new place fields}. \emph{Society for Neuroscience}. Washington, D.C. November~2011.}
  \item[\quad]
    \href{https://www.abstractsonline.com/Plan/ViewAbstract.aspx?sKey=c48e9f5f-1274-4486-85bf-38ee591629e1&cKey=3ec26e6f-8c59-4be2-bad3-e1572d75e07e&mKey=8334BE29-8911-4991-8C31-32B32DD5E6C8}
    {Rao G, \joehl{Monaco J}, Knierim JJ. (2011). \itemtitle{Environmental
        novelty promotes rodent head-scanning behavior linked to enhanced entorhinal
      activity}. \emph{Society for Neuroscience}. Washington,
    D.C. November~2011.}
  \item[\quad]
    \href{https://www.frontiersin.org/10.3389/conf.fnins.2010.03.00192/event_abstract}
    {\joehl{Monaco JD}, Zhang K, Blair HT, Knierim JJ. (2010).
    \itemtitle{Cue-based feedback enables remapping in a multiple oscillator
      model of place cell activity}. \emph{Cosyne}. Salt Lake City, UT.
    February~2010.}
  \item[\quad] \joehl{Monaco JD}, Abbott LF. (2009). \unpubtitle{Dynamic
      hippocampal remapping using recurrent inhibition on realigning grid cell
    inputs}. \emph{Cosyne}. Salt Lake City, UT. February~2009.
  \item[\quad] \joehl{Monaco JD}, Muzzio IA, Levita L, Abbott LF. (2006).
    \unpubtitle{Entorhinal input and global remapping of hippocampal place
    fields}. \emph{CNS}. Edinburgh, UK. July~2006.
  \item[\quad] \joehl{Monaco JD}, Abbott LF. (2006). \unpubtitle{Entorhinal
    input and the remapping of hippocampal place fields}. \emph{Cosyne}. Salt Lake
    City, UT. March~2006.
  \item[\quad] \joehl{Monaco JD}, Levy WB. (2003). \unpubtitle{T-maze training
      of a recurrent CA3 model reveals the necessity of novelty-based modulation of
    LTP in hippocampal region CA3}. \emph{IJCNN}. Portland, OR. July~2003.
  \item[\quad] \joehl{Monaco JD}, Perlstein RP. (1997). \unpubtitle{Monte-Carlo
      analysis of deoxyhypusine synthase inhibitor ligand conformations}. \emph{NIH
    Poster Day}. Bethesda, MD. August~1997.
\end{description}




%%% Recognition and Coverage of My Work %%%

\smallskip
%!TEX root = cv.tex

\newsection{Recognition \& Coverage of My Work}{recognition}

\subsection*{Awards \& Honors}
\label{sec:awards}
\vspace{-0.1in}

\lefttabline{0.8in}{2024}{Kelly Government Services, Distinguished Achievement Award}
\lefttabline{0.8in}{2022}{IEEE/ISEC Best Paper Award, First Place}
\lefttabline{0.8in}{2003}{IEEE/IJCNN Student Paper Award, First Place}
\lefttabline{0.8in}{2002}{U.Va. John A. Harrison III Undergraduate Research Award}
\lefttabline{0.8in}{1999--2003}{U.Va. Echols Scholar}
\lefttabline{0.8in}{1999}{State of Maryland Merit Scholastic Award}
\lefttabline{0.8in}{1999}{AP Scholar with Distinction}
\lefttabline{0.8in}{1999}{National Merit Scholarship Commended Student}
\lefttabline{0.8in}{1999}{Johns Hopkins Mathematics Competition (2nd Place, Individual Calculus)}
\lefttabline{0.8in}{1999}{Maryland Distinguished Scholar}

\subsection*{News \& Views}
\vspace{-0.1in}

\begin{itemize}[itemsep=6pt]
  \item \href{https://dx.doi.org/10.1016/j.cub.2020.02.085}
    {Place R, Nitz DA. (2020). \itemtitle{Cognitive Maps: Distortions of the Hippocampal 
      Space Map Define Neighborhoods}. \emph{Current Biology}, 30(8): R340--R342.}
  \item \href{https://dx.doi.org/10.1016/j.cell.2018.10.047}
    {Colwell CS, Donlea J. (2018). \itemtitle{Temporal coding of sleep}. \emph{Cell}, 175(5): 1177--9.}
  \item \href{https://dx.doi.org/10.1038/nn.3700}
    {Dupret D, Csicsvari J. (2014). \itemtitle{Turning heads to remember
    places}. \emph{Nature Neuroscience}, 17(5): 643--44.}
\end{itemize}

\subsection*{Post-Publication Reviews}
\vspace{-0.1in}

\begin{itemize}[itemsep=6pt]
  \item \href{https://facultyopinions.com/prime/718333676#eval793494783}
    {Moser E, Rowland D. (May 12, 2014). ``\itemtitle{This exciting study finds
        an unexpected relationship between exploratory head scanning behavior
      and the development of new place fields in the rat hippocampus...}”
    \emph{F1000/Faculty Opinions}.}
  \item \href{https://facultyopinions.com/prime/718333676#eval793493493}
    {Maler L. (April 10, 2014). ``\itemtitle{This elegant and original study has
        demonstrated a strong link between the neural activity of hippocampal pyramidal
        neurons (PNs) during head scanning behavior and their subsequent acquisition of
    a new place field...}'' \emph{F1000/Faculty Opinions}.}
  \item \href{https://facultyopinions.com/prime/11553956}
    {Giocomo L, Moser E. (June 29, 2011) ``\itemtitle{This paper presents an
        interesting computational model which utilizes grid-cell modularity to generate
    robust remapping...}'' \emph{F1000/Faculty Opinions}.}
\end{itemize}

\subsection*{Other Press}
\label{sec:press}
\vspace{-0.1in}

\begin{itemize}[itemsep=6pt]
  \item \href{https://www.fastcompany.com/90529833/best-workplaces-for-innovators-2020 -johns-hopkins-university-apl} 
    {``\itemtitle{Johns Hopkins University APL is one of Fast Company’s Best Workplaces
      for Innovators}.'' (July 29, 2020). \emph{Fast Company}.
      \aurl{https://www.fastcompany.com/90529833/best-workplaces-for-innovators- 2020-johns-hopkins-university-apl}}
      \itemnote{My NSF project (see p.\pageref{sec:funding}) was the basis for
      \#3 ranking of JHUAPL.}
  \item \href{https://blogs.plos.org/biologue/2019/03/20/better-use-of-mouse-models-skin-infection-dynamics-and-phaser-cells-in-navigation/}
    {``\itemtitle{Better Use of Mouse Models, Skin Infection
      Dynamics, and Phaser Cells in Navigation}.'' (March 20,
      2019). \emph{PLOS Computational Biology: Biologue}.
      \aurl{https://blogs.plos.org/biologue/2019/03/20/ 
    better-use-of-mouse-models-skin-infection-dynamics-and-phaser-cells-in-navigation/}}
  \itemnote{Editor-in-Chief's selection of papers.}
  %\item \href{https://nationalsciencefoundation.tumblr.com/post/183448836933/brain-awareness-week-2019-rats-and-robots}
    %{``\itemtitle{Brain Awareness Week 2019—Rats and Robots: NSF-funded researchers
        %take a lesson from rat navigation instincts to improve algorithm[s] for
      %robots}.'' (March 14, 2019). \emph{National Science Foundation/Tumblr}.
    %\aurl{https://nationalsciencefoundation.tumblr.com/post/ 183448836933/brain-awareness-week-2019-rats-and-robots}}
  \item \href{https://www.medicaldaily.com/cognitive-map-can-show-real-time-when-memories-form-thanks-place-cells-brain-276790}
    {``\itemtitle{Cognitive Map Can Show In Real-Time When Memories Form, Thanks
      To Place Cells In The Brain}.'' (April 15, 2014). Chris Weller/Medical Daily.
    \aurl{https://www.medicaldaily.com/cognitive-map-can- show-real-time-when-memories-form-thanks-place-cells-brain-276790}}
\end{itemize}




%%% Communications & Media %%%

\smallskip
%!TEX root = cv.tex

\vbox{%
\newsection{Communications \& Media}{comms}

\subsection*{Websites}
\label{sec:web}
\vspace{-0.1in}
}

\begin{description}[itemsep=6pt]
  \item \href{https://www.ninds.nih.gov/about-ninds/who-we-are/staff-directory/joseph-monaco}
    {``\itemtitle{Joseph Monaco, Ph.D. -- Scientific
      Program Manager, NIH BRAIN Initiative}.'' Website.
    \aurl{https://www.ninds.nih.gov/about-ninds/who-we-are/staff-directory/joseph-monaco}}
  \item \href{https://jdmonaco.com/}
    {``\itemtitle{Briefly Balanced: Complexity through neural dynamics and physical constraints}.'' Website. \aurl{https://jdmonaco.com/}}
  \item \href{https://jdmonaco.com/google-scholar}
    {\itemtitle{Google Scholar}. Website. \aurl{https://scholar.google.com/citations?hl=en\& user=gceOLZEAAAAJ\&view\_op=list\_works\&sortby=pubdate}}
  \item \href{https://orcid.org/0000-0003-0792-8322}
    {\itemtitle{ORCID Profile}. Website. \aurl{https://orcid.org/0000-0003-0792-8322}}
  \item \href{https://www.linkedin.com/in/jdmonaco/}
    {\itemtitle{LinkedIn Profile}. Website. \aurl{https://www.linkedin.com/in/jdmonaco/}}
  \item \href{https://www.ncbi.nlm.nih.gov/pubmed/?term=monaco_jd+OR+(monaco_j+AND+muzzio_ia)}
    {\itemtitle{PubMed Listing}. Website.
    \aurl{https://www.ncbi.nlm.nih.gov/pubmed/?term=monaco\_jd}}
  \item \href{https://github.com/jdmonaco}
    {\itemtitle{GitHub Overview}. Website. \aurl{https://github.com/jdmonaco}}
\end{description}

\vbox{%
\subsection*{Media \& Press Releases}
\label{sec:media}
\vspace{-0.1in}
}

\begin{description}[itemsep=6pt]
  \item \href{https://www.thetransmitter.org/brain-inspired/grace-hwang-and-joe-monaco-discuss-the-future-of-neuroai/} 
    {``\itemtitle{Brain Inspired 200: Grace Hwang and Joe Monaco discuss
      the future of NeuroAI}.'' Brain-Inspired Podcast (The Transmitter).
      December 4, 2024. \aurl{https://www.thetransmitter.org/brain-inspired/
    grace-hwang-and-joe-monaco-discuss-the-future-of-neuroai/}}
  \item \href{https://www.jhuapl.edu/NewsStory/220919-stem-teaching-tool-recognized-ieee-isec-2022}{
      ``\itemtitle{Novel Teaching Tool Earns Hopkins Collaborators
      International Conference Honors}.'' JHUAPL Press Office. Sept 19, 2022.
      \aurl{https://www.jhuapl.edu/NewsStory/220919-stem-teaching-tool-
    recognized-ieee-isec-2022}}
  \item \href{https://kavlijhu.org/news/32} {``\itemtitle{Can
        robotic swarms navigate using learning rules devised for brain
      dynamics?}'' JHU/Kavli Neuroscience Discovery Insitute. May 3, 2020.
    \aurl{https://kavlijhu.org/news/32}}
  \item \href{https://www.youtube.com/watch?v=ic4zEgVMSsA}
    {``\itemtitle{Swarmalators}.'' JHUAPL Press Office. May 9, 2019.
    \aurl{https://www.youtube.com/watch?v=ic4zEgVMSsA}}
  \item \href{https://hub.jhu.edu/2018/10/02/brain-robot-swarms-study/}
    {``\itemtitle{What do animal brains have in common with swarms of robots?
      Maybe more than you think}.'' Geoff Brown/JHU Office of Communications. Oct 2,
    2018. \aurl{https://hub.jhu.edu/2018/10/02/
  brain-robot-swarms-study/}}
  \item \href{https://www.jhuapl.edu/PressRelease/181001}
    {``\itemtitle{Do Robot Swarms Work Like Brains?}'' JHUAPL Press Office. October 1, 2018.
    \aurl{https://www.jhuapl.edu/PressRelease/181001}}
  \item \href{https://hub.jhu.edu/2014/04/14/memory-brain-place-cells/}
    {``\itemtitle{Where does a memory begin? Johns Hopkins neuroscientists think they
      know}.'' Latarsha Gatlin/JHU Office of Communications. April 14, 2014.
    \aurl{https://hub.jhu.edu/2014/04/14/memory-brain-place-cells/}}
  \item \href{https://www.youtube.com/watch?v=Jm8OiLJqKJQ}
    {``\itemtitle{Johns Hopkins Researchers Probe Mysteries of
      the Brain}.'' JHU Office of Communications. April 14, 2014.
    \aurl{https://www.youtube.com/watch?v=Jm8OiLJqKJQ}}
\end{description}




%%% Research Program Development %%%

\medskip
%!TEX root = cv.tex

\newsection{Research Program Development}{research}

\smallskip
\subsection*{Patents \& Tech Development}
\label{sec:patents}

\lefttabline{0.8in}{7/5/2022}{Inventor, 
  \href{https://www.freepatentsonline.com/11378975.html}{\itemtitle{Autonomous
Navigation Technology, US patent issued, 11,378,975}}}
\lefttabline{0.8in}{1/3/2020}{Inventor, Autonomous Navigation Technology, US
patent application, 16,734,294}
\lefttabline{0.8in}{5/10/2019}{Inventor, Neuroinspired Algorithms for Swarming
Applications, provisional patent, 62/845,957}
\lefttabline{0.8in}{1/3/2019}{Inventor, Neuroinspired Algorithms for Swarming
Applications, provisional patent, 62/787,891}


\smallskip
\subsection*{Team Leadership \& Funding Development}
\label{sec:res}

\researchactivity
{April 2010/2011}
{Fellowship Proposal (NIH/NINDS F32 NRSA)}
{Behavioral Coordination of Entorhinal-Hippocampal Activity for Real-Time
Sensory Updating of Spatial Memory}
{In collaboration with my postdoctoral sponsor Jim Knierim, I conceived and
  developed a postdoctoral fellowship training proposal as a NIH F32 NRSA
  application. The proposal integrated computational modeling with spatial
  navigation experiments based on behavioral data from position-tracking sensors
  and neural data from multiregional hippocampal--entorhinal single-unit ensemble
  recordings. The application received a 21st percentile rank; I followed up the
  2010 application with a 2011 resubmission following discussions with NINDS PO
Jim Gnadt.}
\label{sec:nrsa}

\researchactivity
{Mar. 2016--2018}
{Grant Award (JHU/SLI)}
{Learning to explore paths through space}
{This internal JHU award (2016--2018; see \cfonly{funding})
  resulted from a collaboration with David J. Foster (now at UC Berkeley) that
  I initiated to conduct modeling studies informed by his lab’s hippocampal
  reactivation data. By integrating Prof.~Zhang’s mathematical theories of
  spatial cognitive maps, I wrote and submitted a proposal for a \$200K/2-year
  project to the JHU Science of Learning Institute. The proposal was awarded at
  the \$150K level and research outcomes included (1) novel theories of temporal
  synchronization coding that inspired the 2017 NSF proposal effort, and (2)
  preliminary dynamical models of sharp-wave reactivation that provided the
foundation for the 2018 NIH R03 award.}
\label{sec:sligrant}

\researchactivity
{April--June 2016}
{Grant Proposal Selectively Funded (DARPA/BTO)}
{Noninvasive Gastrovagal Stimulation for Enhanced Neuroplasticity of Cortical
and Hippocampal Networks during Cognitive Training (GEN-C)}
{In response to DARPA announcement BAA-16-24 of the “Targeted Neuroplasticity
  Training (TNT)” program, I worked with colleagues from JHUAPL and JHU/SoM
  Center for Neurogastroenterology to develop a collaborative program involving 3
  PIs and 5 co-Is (8 labs) across divisions, departments, and fields. I recruited
  experimental labs from JHU/MBI and coordinated proposed contributions to
  maximize scientific impact with a budget of \$9.8M/5 years. I coordinated the
  40-page research narrative, including writing, editing, and/or integrating
  each lab’s contributions and worked with ORA to submit the proposal. While
  not funded in total, DARPA/BTO PM Doug Weber funded select components, leading
  to JHUAPL Work Agreement No.~145563 “BCI (Brain Computer Interface)
Technologies” in 2018.}
%Technologies” in 2018 for \$24,604 to the lab of Prof.~Pasricha.}

\researchactivity
{Nov. 2017--2021}
{Grant Award (NSF/NCS)}
{NCS-FO: Spatial intelligence for swarms based on hippocampal dynamics}
{This NSF-awarded project (2018--2021; see \cfonly{funding}) was the result of
  6 months of collaboration, brain-storming, and team-building between the Zhang
  lab at JHU/SOM and a group of JHUAPL engineers, mathematicians, and scientists.
  The project was initially inspired by results that I presented at my Society
  for Neuroscience 2017 meeting poster. I wrote Aim 1 and integrated the full
  research narrative with inputs from our collaborators for the proposal of
  this \$997K/2-year project to develop those initial ideas into technological
  applications (e.g., robotics, autonomous control, AI) that reciprocally inform
  neuroscience. The project led to multiple publications, a patent, and a NIH
BRAIN Investigators Meeting symposium talk.}
\label{sec:nsfgrant}

\researchactivity
{Jan. 2018--2020}
{Grant Award (NIH/NINDS)}
{Spiking network models of sharp-wave ripple sequences with gamma-locked
attractor dynamics}
{To continue with the collaboration that I initiated with David J. Foster (UC
  Berkeley) on the basis of the internal SLI award (see above), I wrote a small
  modeling proposal that integrated preliminary results from the SLI project and
  recent research developments in the memory reactivation field. This proposal
  was awarded (2018--2020; see \cfonly{funding}) through the NIH/NINDS R03
  mechanism and I am currently utilizing this support to build a foundation for
future efforts along this research track.}
\label{sec:nihgrant}

\whitepaper
{Feb.--Mar. 2018}
{Schultz K, Zhang K, and \joehl{Monaco J}}
{BrainSWARRMM: Brain-like Sharp-Waves for Autonomous Replanning \&
Reconnaissance on Matrix Manifolds}
{In response to the Office of Naval Research (ONR) Special Notice
  N00014-18-R-SN05, Topic 3, I helped organize a series of collaborative meetings
  to design a \$2M/4-year project between JHUAPL and JHU/SoM. I co-authored the
resulting white paper that was submitted for consideration to ONR.}

\whitepaper
{May--June 2018}
{Zhang K, \joehl{Monaco JD}, Hwang GM, Schultz KM, Kobilarov M, Foster DJ,
Jacobs J, and Itti L}
{An Integrative Theoretical Framework of the Neural Self-Organization of Active
Perception for Autonomous Spatial Navigation}
{In response to ONR MURI Announcement N00014-18-S-F006 and with the help of
  JHUAPL, I coordinated a series of meetings with 5 PIs across 4 universities
  (Columbia, UC Berkeley, USC, JHU) to design an innovative research program that
  targeted reciprocal advances in experimental \& theoretical neuroscience and
  robotics \& AI across species and scales. The resulting \$7.5M/5-year project
  that I outlined in the white paper was not invited for a full submission. We
  debriefed with the sponsor, ONR PM Marc Steinberg, who revealed that ONR was
  impressed with the project but that they were seeking a different balance of
elements with respect to neuroscience and AI.}

\whitepaper
{August 2019}
{\joehl{Monaco J}, Zhang K, and Schultz K.}
{SW2Mem: Graph Spectral Decoding of Hippocampal-Cortical Loops for Artificial
Consolidation and Dreaming}
{In response to ONR Special Notice N00014-19-S-SN08, Topic 5.1 I conceived this
  project, created the preliminary model and datasets, guided the preliminary
  analyses with JHUAPL collaborators, and wrote \& submitted the white paper to
  ONR outlining a potential \$1.05M/3-year project. ONR declined to invite us to
submit a full proposal.}

%\whitepaper
%{August 2019}
%{Schultz K, Agarwala S, Zhang K, and \joehl{Monaco J}}
%{Brain-like Distributed Surveillance using Heterogeneous Agents for integRated
%Perception, and Planning (BD-SHARPP)}
%{In response to ONR Special Notice N00014-18-R-SN05, Topic 3, we submitted a
  %revised version of the March 2018 white paper that was specifically invited by
%ONR PM Tom McKenna.}

%\researchactivity
%{Sept. 11, 2019}
%{NSF Project Review}
%{Annual advisory board review symposium}
%{I delivered a seminar on Aim 1 progress at a JHUAPL-hosted symposium for our
  %project’s yearly review, attended by DARPA/I2O PM Hava Siegelmann and other
%outside experts.}

\researchactivity
{Feb. 26, 2020}
{Grant Proposal (NSF/NCS) }
{NCS-FO: Neuroeconomics as a biomimetic control theory for mobile robotic
decision making}
{This FY21 proposal was submitted to the NSF/NCS FOUNDATIONS program; while
  it was discussed and received high scores, the application was declined. I
  co-developed this project in collaboration with colleagues at the University of
  Pittsburgh Medical Center (UPMC), JHU Whiting School of Engineering (JHU/WSE),
  and JHUAPL. Our interdisciplinary project brought together multiscale human
  electrophysiological recordings (UPMC), latent state-space models (JHU/WSE),
  control- and game-theoretic analysis (JHUAPL), and mechanistic neural models
  (JHU/BME, for which I would have been co-PI). We proposed to investigate and
  characterize the neural bases of metacognitive brain states that influence
  decision-making during social \& economic games. As a high-risk/high-reward
  element, we proposed to algorithmicize our results to advance human-robot
interaction.}

\researchactivity
{Jan. 14, 2022}
{Grant Proposal (JHU/Discovery Award) }
{Algorithms of flexible navigation in mice and robots}
{This intramural FY23 proposal for a JHU Discovery Award resulted from a new
  collaboration with Patricia Janak (PI; JHU/PBS) and Céline Drieu (postdoctoral
  fellow), in which we seek to integrate advanced large-scale neural recording
  technologies with my theoretical modeling of neural systems as a distributed
  control problem. Fundamental questions of neural systems communication will
  be addressed using convergent data-driven and theory-driven approaches
  to understanding the cognitive dynamics that enable mice to perform spatial
goal-directed memory tasks.}





%%% Educational Activities %%%

\medskip
\pagebreak
%!TEX root = cv.tex

\vbox{%
\newsection{Education Activities}{teaching}

\smallskip
\subsection*{Educational Program Development}
\vspace{-0.1in}
\label{sec:eduprogram}
}

\begin{tabular}{@{\hspace{0.2in}}l>{\raggedright\arraybackslash}p{.82\textwidth}}
  2018--2021 \hspace{0.1in} & My NSF project with JHUAPL (see \cfonly{funding}
  and \cfonly{nsfgrant}) was successfully funded with a substantial STEM
  component for high-school students involving the development of both a
  12-week course and an intense 2-day seminar called ``Swarming Powered by
  Neuroscience.'' I worked with our STEM education collaborators at JHUAPL to
  develop computational resources required for the two curricula. Additionally,
  I participated in and delivered two virtual lectures about our research
  for the 4-day STEM workshop (developed due to Covid requirements) with 40+
  students that was held Jan 2021.
\end{tabular}

\smallskip
\subsection*{Mentoring \& Supervision}
\label{sec:mentoring}
\vspace{-0.1in}

\begin{longtable}{@{\hspace{0.2in}}l>{\raggedright\arraybackslash}p{.82\textwidth}}
  Spring 2021 \hspace{0.05in} & Darius Carr, STEM high school student; I mentored
  Darius as part of a local high school program that facilitates research
  internships for underrepresented students. I developed a computational
  research project with him that deepened his current interests in neuroscience,
  python programming, and scientific research. \\
  \tabularnewline
  2020--2022 \hspace{0.1in} & Armin Hadzic, junior machine learning engineer
  at JHUAPL; I supervised Armin in translating computational neuroscience
  models into the domain of reinforcement learning and Bayesian optimization
  to investigate autonomous swarming with neural control. Our project led to
  a first author peer-reviewed research publication for Armin in Array (see
  \cfonly{hadzicpub}). In 2022, I provided letters of recommendation in
  support of Armin's applications to Ph.D. programs in computer science. \\
  \tabularnewline
  2019--2023 & Sreelakshmi Rajendrakumar, master's student in JHU/Biomedical
  Engineering (BME); I mentored Sreelakshmi in hippocampal physiology and
  single-unit data analysis. In 2023, I provided letters of recommendation in
  support of Sree's applications to Ph.D. programs in operations research and
  causal inference. \\
  \tabularnewline
  2014 & Manning Zhang, M.S., graduate student in JHU/BME; I mentored Manning
  through an exchange program with Shanghai Jiao Tong University and submitted
  a letter of recommendation supporting her admission to the JHU/BME master's
  program. \\
  \tabularnewline
  2013--2015 & Chia-Hsuan Wang, Ph.D., graduate student at the JHU/MBI; I worked
  extensively with Chia-Hsuan to take over my previous studies of behavior
  and place cells in the Knierim lab, leading to a Society for Neuroscience
  conference poster in 2014. I supported her subsequent thesis research based on
  my analytics and informatics software, resulting in a paper in Current Biology
  (see \cfonly{wangpub}). \\
\end{longtable}


\subsection*{Classroom Instruction}
\vspace{-0.1in}

\begin{tabular}{@{\hspace{0.2in}}l>{\raggedright\arraybackslash}p{.82\textwidth}}
  Fall 2004 & Teaching Assistant for undergraduate ``Introduction to
  Neuroscience'' course, Brandeis University; I assisted Prof. Eve Marder by
  supervising classes, grading examinations, and giving review lectures.\\
  \tabularnewline
  Spring 2005 \hspace{.1in} & Teaching Assistant for undergraduate ``Biology
  Laboratory'' course, Brandeis University \\
\end{tabular}




\end{document}
