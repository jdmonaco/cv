% JMonacoCV.tex
%
% (c) 2002 Matthew M. Boedicker <mboedick@mboedick.org> http://mboedick.org
% $Id: resume.tex,v 1.7 2004/02/02 14:21:06 mboedick Exp $
%

\documentclass[10pt]{article}
\usepackage{fullpage}
\usepackage{palatino}
\usepackage{hyperref}
\usepackage{multirow}
\usepackage{tabu}
\textheight=9.0in
\pagestyle{empty}
\raggedbottom
\raggedright
\setlength{\tabcolsep}{0in}
\addtolength{\voffset}{-.4in}
\addtolength{\textheight}{.7in}
\addtolength{\hoffset}{-.25in}
\addtolength{\textwidth}{.2in}

\begin{document}


\begin{tabular*}{6.5in}{c@{\extracolsep{\fill}}rl}
\hline\\[0.02in]
\textsc{\textbf{\Large Joseph Daniel Monaco, Ph.D.}}      & Phone          & \href{tel:16172725668}{\texttt{617-272-5668}} \\
\multirow{2}{*}{\large Postdoctoral Fellow}               & Email          & \href{mailto:jmonaco@jhu.edu}{\texttt{jmonaco@jhu.edu}} \\
                                                          & Web            & \href{http://jdmonaco.com/}{\texttt{jdmonaco.com}} \\
Biomedical Engineering Department                         & ORCID          & \href{http://jdmonaco.com/orcid}{\texttt{0000-0003-0792-8322}} \\
Johns Hopkins University School of Medicine               & Google Scholar & \href{http://jdmonaco.com/google-scholar}{\texttt{gceOLZEAAAAJ}} \\[0.1in]
\hline
\end{tabular*}
\vspace{0.1in}

{\large \textbf{Research Bio}} 
\vspace{0.1in}

I study the hippocampus and its roles in navigation, spatial cognition, and episodic memory. My research began when
William ``Chip'' Levy recruited me as a first-year undergraduate at the University of Virginia to study the recurrent
network dynamics of the CA3 subregion of hippocampus. Our paper on goal-directed spatial navigation received an IEEE
Best Paper award. As a graduate student at Brandeis University, I collaborated with Michael Kahana on a study of
recognition memory and joined Larry Abbott's theory group. After moving with Larry to the Center for Theoretical
Neuroscience at Columbia University, I collaborated with Isabel Muzzio and Eric R. Kandel on a study of spatial map
stability in mice and, following the landmark discovery of grid cells, developed a novel model of hippocampal
processing that made key predictions about the functional organization of grid cells. I currently hold a postdoctoral
position at Johns Hopkins University working with Kechen Zhang and James Knierim to study how experience constructs the
cognitive map of space using computational models and state-of-the-art statistical analysis. My most recent work
demonstrated, for the first time, a behavioral link to memory formation in real time.

\vspace{0.1in}

{\large \textbf{Recent Publications}}

\begin{description}
\item \textbf{Monaco JD}, Rao G, Roth ED, and Knierim JJ. (2014). \emph{Attentive scanning behavior drives one-trial potentiation of hippocampal place fields}. Nature Neuroscience, 17(5), 725--731. doi:~\href{http://dx.doi.org/10.1038/nn.3687}{10.1038/nn.3687}.
\begin{itemize}
  \item \underline{\bf News \& Views}: Dupret D and Csicsvari J. (2014). \emph{Turning heads to remember places}. Nature Neuroscience, 17(5), 643--644. doi:~\href{http://dx.doi.org/10.1038/nn.3700}{10.1038/nn.3700}.
  \item \underline{\bf F1000Prime}: Moser E and Rowland D: ``This exciting study finds an unexpected relationship between exploratory head scanning behavior and the development of new place fields in the rat hippocampus...'' Faculty of 1000, May 12, 2014. \href{http://f1000prime.com/718333676#eval793494783}{f1000.com/718333676}
  \item \underline{\bf F1000Prime}: Maler L: ``This elegant and original study has demonstrated a strong link between the neural activity of hippocampal pyramidal neurons (PNs) during head scanning behavior and their subsequent acquisition of a new place field...'' Faculty of 1000, April 10, 2014. \href{http://f1000prime.com/718333676#eval793493493}{f1000.com/718333676}
\end{itemize}
\item \textbf{Monaco JD}, Knierim JJ, and Zhang K. (2011). \emph{Sensory feedback, error correction, and remapping in a multiple oscillator model of place cell activity}. Frontiers in Computational Neuroscience, 5:39. doi:~\href{http://dx.doi.org/10.3389/fncom.2011.00039}{10.3389/fncom.2011.00039}.
\item \textbf{Monaco JD} and Abbott LF. (2011). \emph{Modular realignment of entorhinal grid cell activity as a basis for hippocampal remapping}. Journal of Neuroscience, 31(25), 9414--25. \href{http://dx.doi.org/10.1523/JNEUROSCI.1433-11.2011}{doi:~10.1523/jneurosci.1433-11.2011}
\begin{itemize}
    \item \underline{\bf F1000Prime}: Giocomo L and Moser E: ``This paper presents an interesting computational model which utilizes grid-cell modularity to generate robust remapping...'' Faculty of 1000, June 29, 2011. \href{http://f1000.com/11553956}{f1000.com/11553956}
\end{itemize}
\end{description}

{\large \textbf{Education}}
\begin{itemize}
  \item 
    \begin{tabular*}{6.3in}{l@{\extracolsep{\fill}}r}
    	\textbf{Columbia University} & New York, NY \\
    	Ph.D., Neurobiology \& Behavior & 2005--9 \\
    	Degrees: M.A. (2006), M.Phil. (2008), Ph.D. (2009) & \\
    	\textit{Transferred from Brandeis University} & \\
    \end{tabular*}
  \item 
    \begin{tabular*}{6.3in}{l@{\extracolsep{\fill}}r}
    	\textbf{Brandeis University} & Waltham, MA \\
    	Ph.D., Neuroscience & 2003--5 \\
    	\textit{Continued at Columbia University} & \\
    \end{tabular*}
  \item
    \begin{tabular*}{6.3in}{l@{\extracolsep{\fill}}r}
    	\textbf{University of Virginia} & Charlottesville, VA \\
    	Degrees: B.A. Cognitive Science; B.A. Mathematics & 1999--2003 \\
    	Echols Scholar & \\
    \end{tabular*}
\end{itemize}

\pagebreak

{\large \textbf{Positions}}
\begin{itemize}

\item
	\begin{tabular*}{6in}{l@{\extracolsep{\fill}}r}
		\textbf{Postdoctoral Fellow} & 2009--present\\
        Zanvyl Krieger Mind/Brain Institute, Johns Hopkins University, Baltimore, MD\\
        Biodmedical Engineering Department, Johns Hopkins School of Medicine, Baltimore, MD\\
        PIs: James J. Knierim, Ph.D., Kechen Zhang, Ph.D. \\
        Collaborators: H. Tad Blair, Ph.D. \\
	\end{tabular*}

    % \begin{itemize}
    %   \item data-driven modeling efforts to provide explanatory and predictive insights into cortico-hippocampal responses in double cue-rotation experiments; and
    %   \item quantitative data analysis of electrophysiological recordings in rat hippocampus for insight into region-specific dynamics of spatial code formation in altered environments.
    % \end{itemize}

\item
	\begin{tabular*}{6in}{l@{\extracolsep{\fill}}r}
		\textbf{Graduate Research Assistant} & 2005--9\\
		Center for Theoretical Neuroscience, Columbia University, New York, NY\\
        Advisor: L.~F. Abbott, Ph.D.\\
        Collaborators: Isabel Muzzio, Ph.D., Eric R. Kandel, Ph.D. (2006--8) \\
	\end{tabular*}

    % \begin{itemize}
    %   \item developed Python-based framework for analyzing an extensive dataset of experimental mouse brain recordings, including both spike-timing-based and frequency-domain statistical analyses such as theta-band phaselocking; and
    %   \item added functionality to a previous C codebase for acquisition and analysis to allow flexible cross-platform access to the dataset.
    % \end{itemize}

\item
	\begin{tabular*}{6in}{l@{\extracolsep{\fill}}r}
		\textbf{Graduate Research Assistant} & 2003--5\\
		Volen Center for Complex Systems, Brandeis University, Waltham, MA \\
        Advisor: L.~F. Abbott, Ph.D.\\
        Collaborator: Michael J. Kahana, Ph.D. (2004--6) \\
	\end{tabular*}

\item
	\begin{tabular*}{6in}{l@{\extracolsep{\fill}}r}
		\textbf{Undergraduate Research Assistant} & 2000--3\\
		Lab. of Computational Neurodynamics, University of Virginia, Charlottesville, VA \\
		Advisor: W.~B Levy, Ph.D.
	\end{tabular*}

    % \begin{itemize}
    %   \item maintained and updated C/C++ codebase for highly extensible model of rat CA3 hippocampus;
    %   \item employed CA3 model to research a variety of context-dependent sequence learning paradigms: cued completion, sequence disambiguation, goal finding, and T-maze decision making; and
    %   \item performed Linux system administration for a laboratory network of workstations. 
    % \end{itemize}

% \item 
  % \begin{tabular*}{6in}{l@{\extracolsep{\fill}}r}
  %   \textbf{Technical Assistant} & 2001--2\\
  %   Dept. of Economics, University of Virginia, Charlottesville, VA \\
  %   Supervisor: Charles Holt, Ph.D.
  % \end{tabular*}
	
    % \begin{itemize}
    %   \item maintained and provided basic administration for a database-driven web server used in experimental game theory courses; and
    %   \item setup and maintained a large wireless network of handheld computers for use in classroom-based game theory experiments.
    % \end{itemize}

\item
	\begin{tabular*}{6in}{l@{\extracolsep{\fill}}r}
		\textbf{Research Intern} & Summers 1997/8 \\
		Center for Molecular Modeling, LSB/CIT, NIH, Bethesda, MD\\
		PI: R.~Perlstein, Ph.D., P.~Steinbeck, Ph.D.
	\end{tabular*}

    % \begin{itemize}
    %   \item performed inhibitor ligand analysis for deoxyhypusine synthase using clustering and Monte-Carlo techniques to search for functional conformations ('97); and
    %   \item used LoBoS beowulf cluster to perform molecular dynamics simulations of various hyperthermophilic proteins to discern temperature-dependent structural differences from their mesothermic counterparts ('98).
    %   \end{itemize}
\end{itemize}

{\large \textbf{Publications}}\nopagebreak

\begin{description}
\item \textbf{Monaco JD}, Rao G, Roth ED, and Knierim JJ. (2014). \emph{Attentive scanning behavior drives one-trial potentiation of hippocampal place fields}. Nature Neuroscience, 17(5), 725--731. doi:~\href{http://dx.doi.org/10.1038/nn.3687}{10.1038/nn.3687}.
\item \textbf{Monaco JD}, Knierim JJ, and Zhang K. (2011). \emph{Sensory feedback, error correction, and remapping in a multiple oscillator model of place cell activity}. Frontiers in Computational Neuroscience, 5:39. doi:~\href{http://dx.doi.org/10.3389/fncom.2011.00039}{10.3389/fncom.2011.00039}.
\item \textbf{Monaco JD} and Abbott LF. (2011). \emph{Modular realignment of entorhinal grid cell activity as a basis for hippocampal remapping}. Journal of Neuroscience, 31(25), 9414--25. \href{http://dx.doi.org/10.1523/JNEUROSCI.1433-11.2011}{doi:~10.1523/jneurosci.1433-11.2011}
\item \textbf{Monaco JD}. (2009). \emph{Models and mechanisms for integrating cortical feature spaces}. Ph.D. Dissertation, Columbia University. \href{http://gradworks.umi.com/33/93/3393609.html}{ProQuest Publication No. AAT 3393609}.
\item Muzzio IA, Levita L, Kulkarni J, \textbf{Monaco J}, Kentros CG, Stead M, Abbott LF, and Kandel ER. (2009). \emph{Attention enhances the retrieval and stability of visuospatial and olfactory representations in the dorsal hippocampus}. PLoS Biology, 7(6), e1000140.
\item \textbf{Monaco JD}, Abbott LF, and Kahana MJ. (2007). \emph{Lexico-semantic structure and the recognition word-frequency effect}. Learning \& Memory, 14(3), 204--213. 
\item \textbf{Monaco JD} and Levy WB. (2003). \emph{T-maze training of a recurrent CA3 model reveals the necessity of novelty-based modulation of LTP in hippocampal region CA3}. Proceedings International Joint Conference on Neural Networks, 1655--1660.\nopagebreak
% \begin{itemize}
%     \item IJCNN 2003 Best Student Paper, First Place
% \end{itemize}
\end{description}

{\large \textbf{Presentations}}\nopagebreak

\begin{description}
\item \textbf{Talks}\nopagebreak
\item[\quad] \textbf{Joseph Monaco}, ``Landmark influence: How attention to sensory cues stabilizes and updates the hippocampal cognitive representation of space.'' \emph{Advanced Researcher Seminar}. Krieger Mind/Brain Institute, Johns Hopkins University, Baltimore, MD. April 21, 2014.
\item[\quad] \textbf{Joseph Monaco}, ``Head scans drive the formation and potentiation of place fields during exploration.'' \emph{Data Blitz}. Winter Conference on Learning \& Memory, Park City, UT. January 3, 2014.
\item[\quad] \textbf{Joseph Monaco}, ``Medial versus lateral modes for reconfiguring hippocampal representations.'' \emph{Grid Cells: Formation and Function}. Gatsby Computational Neuroscience Unit, UCL, UK. July 1, 2010.
\item[\quad] \textbf{Joseph Monaco}, ``Rapid spatial map formation and remapping by competing over grid cell inputs.'' \emph{Thesis Seminar}. Columbia University Medical Center, New York, NY. April 10, 2009.
\end{description}

\begin{description}
\item \textbf{Posters}\nopagebreak
\item[\quad] \textbf{Monaco J}, Rao G, and Knierim JJ. (2013). Scanning behavior in novel environments promotes \emph{de novo} formation of hippocampal place fields in rats. Neuroscience 2013, 670.07/JJJ44. San Diego, CA. November 2013.
\item[\quad] \textbf{Monaco J}, Rao G, and Knierim JJ. (2012). Hippocampal LFP during rodent head-scanning behavior: Theta and sharp-wave ripples. Neuroscience 2012, 812.14/FFF24. New Orleans, LA. October 2012.
\item[\quad] \textbf{Monaco J}, Rao G, and Knierim JJ. (2011). Hippocampal place cell firing during head-scanning movements is associated with the formation of new place fields. Neuroscience 2011, 97.13/WW28. Washington, D.C. November 2011.
\item[\quad] Rao G, \textbf{Monaco J}, and Knierim JJ. (2011). Environmental novelty promotes rodent head-scanning behavior linked to enhanced entorhinal activity. Neuroscience 2011, 97.12/WW27. Washington, D.C. November 2011.
\item[\quad] \textbf{Monaco JD}, Zhang K, Blair HT and Knierim JJ. (2010). Cue-based feedback enables remapping in a multiple oscillator model of place cell activity. COSYNE 2010. Salt Lake City, UT. February 2010.
\item[\quad] \textbf{Monaco JD} and Abbott LF. (2009). Dynamic hippocampal remapping using recurrent inhibition on realigning grid cell inputs. COSYNE 2009. Salt Lake City, UT. February 2009.
\item[\quad] \textbf{Monaco JD}, Muzzio IA, Levita L and Abbott LF. (2006). Entorhinal input and global remapping of hippocampal place fields. CNS*2006. Edinburgh, UK. July 2006.
\item[\quad] \textbf{Monaco JD} and Abbott LF. (2006). Entorhinal input and the remapping of hippocampal place fields. COSYNE 2006. Salt Lake City, UT. March 2006.
\item[\quad] \textbf{Monaco JD} and Levy WB. (2003). T-maze training of a recurrent CA3 model reveals the necessity of novelty-based modulation of LTP in hippocampal region CA3. IJCNN 2003. Portland, OR. July 2003.
\item[\quad] \textbf{Monaco JD} and Perlstein RP. (1997). Monte-Carlo analysis of deoxyhypusine synthase inhibitor ligand conformations. NIH Poster Day. Bethesda, MD. August 1997. 
\end{description}

% \item[] \textbf{Attended:}
% \begin{itemize}
% \item COSYNE (Computational Systems Neuroscience): 2009 and 2005--2007 Salt Lake City, UT; 2009 Workshops, Snowbird, UT; 2007 and 2006 Workshops, Park City, UT
% \item CNS (Computational Neurosciences): 2006, Edinburgh, UK
% \item SfN (Society for Neuroscience): 2008 and 2005, Washington, D.C.; 2004, San Diego, CA
% \item IJCNN (International Joint Conference on Neural Networks): 2003, Portland, OR
% \item NIH Poster Day: 1997, Bethesda, MD
% \end{itemize}
% \end{description}

{\large \textbf{Professional}}\nopagebreak
\begin{itemize}
  % \item \emph{Ad-hoc Reviewer:} Neuroscience, IEEE/INNS Neural Networks, Neurocomputing, Biological Cybernetics
    \item Neuroscience, \emph{Ad-hoc reviewer}
    \item IEEE/INNS Neural Networks, \emph{Ad-hoc reviewer}
    \item Neurocomputing, \emph{Ad-hoc reviewer}
    \item Biological Cybernetics, \emph{Ad-hoc reviewer}
    \item Society for Neuroscience, \emph{Postdoc Member (2011--present)}
\end{itemize}
    
    
{\large \textbf{Awards}}\nopagebreak

\begin{itemize}
    \item IJCNN Student Paper Competition, First Place (2003)
    \item John A. Harrison III Undergraduate Research Award (U.Va., 2002)                          
    \item Echols Scholar at the University of Virginia (1999--2003)                              
    \item Pre-university: Maryland Distinguished Scholar (1999), Johns Hopkins Mathematics Competition (2nd Place Individual Calculus, 1999), National Merit Scholarship Commended Student, AP Scholar with Distinction, State of Maryland Merit Scholastic Award 
\end{itemize}

% {\large \textbf{Technical Proficiency}}
% 
% \begin{description}
%     \item[\quad Platforms] OSX, GNU/Linux, Windows, Matlab
%     \item[\quad Languages] Python, C, C++, \LaTeX, Perl
% \end{description}
\end{document}

