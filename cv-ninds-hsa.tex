%! TEX program = xelatex
%
% Curriculum vitae for Joseph D. Monaco
%
\documentclass[10pt]{article}

\usepackage{fullpage}
\usepackage[usenames,dvipsnames]{color}
\usepackage[hidelinks,xetex]{hyperref}
\usepackage{multirow}
\usepackage{sectsty}
\usepackage{tabu}
\usepackage{longtable}
\usepackage{soul}
\usepackage{fontspec}

% Page geometry formatting
\raggedbottom
\raggedright
\textheight=9in
\setlength{\tabcolsep}{0in}
\addtolength{\footskip}{.3in}
\addtolength{\voffset}{-.4in}
\addtolength{\textheight}{.4in}
\addtolength{\hoffset}{-.25in}
\addtolength{\textwidth}{.2in}

% Define colors
\definecolor{hopkinsblue}{RGB}{0,48,130}
\definecolor{lightblue}{rgb}{.88,.92,.97}
\definecolor{lightgold}{rgb}{1,.80,0}
\definecolor{lightred}{rgb}{1,.3,.2}
\definecolor{lightgray}{rgb}{.80,.80,.80}
\definecolor{dimgray}{rgb}{.50,.50,.50}
\definecolor{darkgray}{rgb}{.30,.30,.30}

% Set up highlighting and underlining
\sethlcolor{lightblue}
\setul{0.13ex}{}

% Choose font package
\setmainfont[Ligatures=TeX]{Helvetica Neue LT Std} 

% Section section* formatting
\sectionfont{\large\color{hopkinsblue}\bfseries}
\subsectionfont{\vspace{-1ex}\color{hopkinsblue}\normalsize\bfseries}

% Formatting macros
\newcommand{\itemtitle}[1]{{\color{hopkinsblue}\ul{#1}}}
\newcommand{\unpubtitle}[1]{{\color{hopkinsblue} #1}}
\newcommand{\itemnote}[1]{
  \begin{description}
    \item[$\rightarrow$] \hspace{.09in}{\color{darkgray}\it #1}
  \end{description}
}
\newcommand{\joehl}[1]{\hl{\textbf{#1}}}
\newcommand{\doi}[1]{{\color{darkgray}doi:}~{\color{dimgray}\texttt{#1}}}
\newcommand{\arxiv}[1]{\emph{ArXiv Preprint}.
  {\color{darkgray}arxiv:}{\color{dimgray}\texttt{#1}}}
\newcommand{\aurl}[1]{{\color{dimgray}\texttt{#1}}}
\newcommand{\researchnote}[1]{
  \begin{description}
    \item[] {\hspace{2.2ex}\color{darkgray} #1}
  \end{description}
}
\newcommand{\researchactivity}[4]{
  \begin{minipage}[t]{\textwidth}
    \begin{tabular}{@{\hspace{2ex}}l>{\raggedright\arraybackslash}p{.8\textwidth}}
      \makebox[1.2in][l]{#1} & \textbf{#2:}
      ``\unpubtitle{#3}'' 
    \end{tabular}
  \researchnote{\hspace{1ex} #4}
  \end{minipage}
  \medbreak
}
\newcommand{\whitepaper}[4]{
  \begin{minipage}[t]{\textwidth}
    \begin{tabular}{@{\hspace{2ex}}l>{\raggedright\arraybackslash}p{.8\textwidth}}
      \makebox[1.2in][l]{#1} & \textbf{White Paper:} #2.
      ``\unpubtitle{#3}'' 
    \end{tabular}
  \researchnote{\hspace{1ex} #4}
  \end{minipage}
  \medbreak
}
\newcommand{\lefttabline}[3]{\hspace{2ex}\makebox[#1][l]{#2} #3\\}

% PDF document info and setup
\hypersetup{
  baseurl=https://jdmonaco.com/cv-monaco.pdf,
  pdftitle=Curriculum Vitae for Joseph D. Monaco,
  pdfauthor=Joseph D. Monaco,
  pdfdisplaydoctitle=true,
  pdfpagemode=UseThumbs,
  pdfstartview=FitV,
  pdfpagelayout=TwoColumnLeft,
  pdftoolbar=false,
  pdfwindowui=false,
  pdfcenterwindow=true,
}

% "See section ..." macro
\newcommand{\see}[1]{[\textcolor{hopkinsblue}{p.\pageref{sec:#1}}]}
\newcommand{\cfonly}[1]{\textcolor{hopkinsblue}{\emph{\nameref{sec:#1}} on p.\pageref{sec:#1}}}
\newcommand{\cf}[1]{\textcolor{hopkinsblue}{See \emph{\nameref{sec:#1}} on p.\pageref{sec:#1}}}
\newcommand{\cfcf}[2]{\textcolor{hopkinsblue}{See \emph{\nameref{sec:#1}} on
  p.\pageref{sec:#1} and \emph{\nameref{sec:#2}} on p.\pageref{sec:#2}}}

% Typesetting
\setlength{\parskip}{0.75em}


\begin{document}

\begin{center}
  \textbf{\LARGE\color{hopkinsblue} Joseph D. Monaco, Ph.D.} \\[0.1in]
  1028 Pier Pointe Ldng \\
  Baltimore, Maryland 21230 United States \\
  \textsc{Mobile:} \href{tel:16172725668}{\color{hopkinsblue}\texttt{617.272.5668}} \\
  \textsc{Email:} \href{mailto:joe@selfmotion.net}{\color{hopkinsblue}\texttt{joe@selfmotion.net}}
  \vspace{.1in}
\end{center}
%\thispagestyle{empty}

\vspace{-.2in}
\section*{Duties \& Requirements}
\vspace{-.1in}
\hrule
\vspace{.3in}

\subsection*{Duties --- HSA/Program Officer}
\label{sec:duties}

\begin{itemize}
  \item[-] \emph{Perform scientific and administrative reviews and analyses of
    applications/proposals from a programmatic viewpoint.}
\end{itemize}

In 2021, I contributed to the programmatic development of a competitive NSF EFRI
program topic. In 2022, I served on the preliminary and final review panels
for that EFRI topic and as an ad-hoc reviewer for the NSF CAREER program. From
2016--2021, I served three times on the Review Committee for the Computational
Systems Neuroscience (Cosyne) conference. \cf{service}.

\begin{itemize}
  \item[-] \emph{Consult with and advise grantees/contractors during preparation
    of applications/proposals and provide guidance on program issues.}
\end{itemize}

From 2018--2022, I met with various program directors from the Navy (ONR) and
the NSF to receive advice on the preparation of applications and discussed in
detail how to maximize program relevance. Based on this experience, I wrote or
co-wrote and submitted a number of whitepapers and grant applications at various
levels of project size and budget. Through these discussions and the subsequent
development and programmatic review of my submitted applications, I gained
substantial knowledge of the role of program officers in guiding investigators
as they develop applications so as to align with mission criteria and funding
priorities of the agency. \cf{resprogram}.

\begin{itemize}
  \item[-] \emph{Develop, coordinate and administer grants, cooperative
      agreements, and contracts established to fulfill the mission of the Branch and
    Division.}
\end{itemize}

In 2014, I worked with program directors and technical staff at IARPA as an
outside consultant and contributed intellectually to the development of RFIs and
solicitations at that agency. \cf{service}

\begin{itemize}
  \item[-] \emph{Visit universities, research institutions, commercial
      organizations, other government agencies, entrepreneur/innovation centers, and
      public and private organizations to promote and explain the objectives of the
    program.}
\end{itemize}

Since 2009, I have delivered numerous seminars and invited lectures at various
top research universities---e.g., the University of Minnesota in 2020 and
the University of Toronto in 2022---and public organizations---e.g., the
Neuromatch Conference series from 2020--2021 and the US Air Force Research
Laboratory (AFRL) in 2022---at the national and international levels to promote
and explain the objectives and findings of my research program. \cf{talks}.

Furthermore, I have visited extensively with private commercial innovation
centers including the Johns Hopkins University Applied Physics Laboratory
(JHUAPL)~\see{press} for collaborative efforts to develop federal funding
applications and manage the resulting cross-divisional research, outreach, and
educational projects. \cfcf{resprogram}{eduprogram}. My extensive experience
in persuasive presentation of scientific concepts, objectives, and findings
to varied academic, commercial, and learner groups will directly translate to
communication and outreach of extramural program objectives.

\subsection*{Basic Experience Requirements}
\label{sec:basicreqs}

\begin{itemize}
  \item[-] \emph{Demonstrated independence in planning, organizing, and
conducting biomedical behavioral health-related research.}
  \item[-] \emph{Served effectively, in research program administration and
acquired an understanding of the history, interests, internal dynamics, and
relationships of organizations in which health research is conducted.}
\end{itemize}

%This experience may be gained as a principal investigator of a grant or
%contract, or may otherwise be gained through active involvement in initiating
%research projects, developing protocols, conducting studies, documenting
%findings, interpreting results in a published report (journal), supervising
%staff, and managing the budget.

From 2019--2022, as Research Associate faculty in the Johns Hopkins University
School of Medicine (JHU/SOM)~\see{job1}, I served as principal investigator
(or co-PI) of multiple federal grant-awarded projects that I initiated
independently, including a NIH/NINDS R03 award and a NSF/NCS FOUNDATIONS
award, which is part of the US BRAIN Initiative. These projects investigated,
respectively, physiological mechanisms underlying cognitive mapping and episodic
memory formation, and theoretical advances in the understanding of neural
oscillations as a basis of self-organizing spatiotemporal activity patterns
in brains. \cf{funding}. During that time, I also independently initiated new
collaborations and planned and developed several other research projects,
leading to a cross-disciplinary NSF application in 2020 and an internal JHU
application in 2022. \cf{resprogram}.

To manage these projects, I developed protocols including reference
computational models, conducted data analyses, supervised and mentored graduate
students and junior staff~\see{mentoring}, and administered the grants by
coordinating with JHU administrators, integrating and writing progress reports
for the agencies, applying for no-cost extensions, and depositing published
papers in agency-hosted archives such as the NSF Public Access Repository. The
following peer-reviewed publications resulted from these projects:
%
\begin{quote}
  • Hadzic, Hwang, Zhang, Schultz, Monaco (2022) \emph{Array}. \\
  • Buckley, Monaco, Schultz, Chalmers, Hadzic, Zhang (2022) In: IEEE
  Integrated STEM Education Conference (ISEC'22). \\
  • Schultz, Villafañe-Delgado, Reilly, Saksena, Hwang (2022) In: Emergent
  Behavior in System of Systems Engineering. \\
  • Hwang, Schultz, Monaco, Zhang (2021) \emph{JHUAPL Technical Digest}. \\ 
  • Monaco, Hwang, Schultz, Zhang (2020) \emph{Biological Cybernetics}. \\ 
  • Monaco, Hwang, Schultz, Zhang (2019) In: International Society for
  Photonics (SPIE'19). \\
\end{quote}
%
\cf{pubs} for complete publication details, including two additional
pending publications in review or revision stages. The first author
(Hadzic) of the first paper above was a JHUAPL junior staff member whom I
mentored~\see{mentoring}.

From 2016--2018, as a JHU/SOM Postdoctoral Fellow~\see{job2}, I independently
initiated collaborations and research projects resulting in an internal
JHU Science of Learning Institute award~\see{previnternalsupport} and the
development and submission of two external (NIH and NSF) applications that
were awarded starting FY19. \cfcf{funding}{resprogram}. In that time, I
additionally conceived and coordinated a large collaborative project proposal
among 8 labs across JHU that was selectively funded by DARPA/BTO and targeted
noninvasive stimulation of neuroplasticity for health-related interventions and
augmentation~\see{genc}.

From 2014--2016, I conducted a computational modeling and neural data analysis
study in which I developed neural coding protocols, documented my findings, and
interpreted the results in a substantial original research article integrating
multiple convergent methodological approaches that I carried from preprint to
high-impact publication:

\begin{quote}
  • Monaco, De Guzman, Blair, Zhang (2019) \emph{PLOS Computational Biology}. \\ 
  • Monaco, Blair, Zhang (2017) \emph{bioRxiv}. \\ 
\end{quote}
%
\cf{pubs} for complete publication details. This 2019 publication laid the
theoretical and experimental groundwork for establishing my collaboration with
JHUAPL, the resultant NSF-awarded project~\see{funding}, my co-organization
of and participation in a well-attended NIH BRAIN Investigators symposium in
2020~\see{symposium}, and my interactions with NSF program directors that led
to my extramural contributions to the programmatic development and review of a
competitively funded NSF EFRI topic from 2020--2022~\see{programsvc}.

From 2009--2014, as a JHU Mind/Brain Institute (JHU/MBI) Postdoctoral
Fellow~\see{job3}, I conducted a statistical modeling study of neurobehavioral
datasets and developed protocols for behavioral quantification of rats
performing spatial navigation experiments. My study provided the first evidence
for mammalian long-term memory formation tied to individual behaviors (i.e.,
lateral head-scanning movements). In collaboration with my postdoctoral lab, I
interpreted and published my findings in high-impact journals:

\begin{quote}
  • Monaco, Rao, Roth, Knierim (2014) \emph{Nature Neuroscience}. \\ 
  • Wang, Monaco, Knierim (2020) \emph{Current Biology}. \\ 
\end{quote}
%
\cf{pubs} for complete publication details. The first author (Wang) of the
second paper above was a graduate student whom I mentored~\see{mentoring}.
My head-scanning study provided the basis for my 2010 application and 2011
resubmission of a NIH NRSA Postdoctoral Fellowship (F32)~\see{nrsa}, as guided
by discussions that I initiated with NINDS PO Jim Gnadt.

In summary, from 2010 to the present, I have developed, executed, and
administered an independent, growing program in health-related preclinical
neuroscience and neural data science research. This programmatic evolution
relied on manifold extended interactions with program officers at various
federal funding agencies. Through this experience, I have gained a substantial
understanding of the history, interests, internal dynamics, and relationships
within and between funding organizations, and of the public service performed by
program officers.


\subsection*{Requirements for the GS-12 Level}

\begin{itemize}
  \item[-] \emph{One year of independent health research beyond the doctoral
degree and one year of health research program administration.}
\end{itemize}
 
As described above under \textcolor{hopkinsblue}{\emph{\nameref{sec:basicreqs}}}
and below under \textcolor{hopkinsblue}{\emph{\nameref{sec:resprogram}}},
I have three years of Research Associate faculty experience developing
and administering an independent cross-divisional and interdisciplinary
research program~\see{job1}, five years of postdoctoral experience performing
preclinical health-related neuroscience research~\see{job2}, and five years of
postdoctoral experience in statistical neurobehavioral modeling and neural data
science~\see{job3}. 

In particular, I have$\ldots$
%
\begin{itemize}
  \item[•] Served as principal investigator or equivalent on competitively
    awarded grants from 2016--2021: \cf{funding};
  \item[•] Independently developed research projects, and demonstrated
    primary authorship of five original peer-reviewed research publications
    from 2011--present, including in top journals like Nature Neuroscience:
    \cfcf{resprogram}{pubs}; and
  \item[•] Held a faculty research position in Biomedical Engineering from
    2019--2022: \cf{job1}.
\end{itemize}


\subsection*{Requirements for the GS-13 Level}

\begin{itemize}
  \item[-] \emph{Managed significant independent research projects}.
    \itemnote{\cfcf{basicreqs}{resprogram}.}
  \item[-] \emph{Supervised graduate researchers and junior technical staff}.
    \itemnote{\cf{mentoring}.}
  \item[-] \emph{Published in refereed journals}. \itemnote{\cf{pubs}.}
  \item[-] \emph{Presented published work to scientific organizations}.
    \itemnote{\cfcf{talks}{presentations}.}
  \item[-] \emph{Served as a reviewer on peer-review panels and journals}.
    \itemnote{\cfcf{service}{programsvc}.}
  \item[-] \emph{Held the position of Assistant or Associate Professor or
      equivalent}. \itemnote{\cf{cv} and the above-linked sections to assess the
        depth and breadth of my academic credentials. Note that, as of 2019, the
        JHU/SOM Provost for Postdoctoral Affairs evaluated my C.V. as being equivalent
        to a mid-tenure-track Assistant Professor, i.e., approximately 3 years from
        tenure and promotion to Associate Professor rank. Given my experience accrued
        in the three years since that evaluation, I would estimate that my position as
        of 2022 should be fairly judged as equivalent to a recently tenured Associate
      Professor.}
  \end{itemize}

My professional experience in the academic environment of the JHU/SOM Department
of Biomedical Engineering has involved each of the listed requirements
above, particularly with respect to the Research Asssociate faculty position
that I held from 2019--2022~\see{job1}. The summaries above starting with
\cfonly{basicreqs} provide additional context, and the linked sections indicated
next to the arrows under each requirement provide further supporting details.


\subsection*{Requirements for the GS-14 Level}

\begin{itemize}
  \item[-] \emph{At least one year of professional experience that demonstrates
      extensive scientific expertise incorporating research experience with
      varied responsibilities for providing leadership in a scientific area, and
      functioning as a leader for a variety of efforts, such as directing research
      and coordinating committee and teaching activities, and organizing and
    chairing sessions at national scientific meetings.}
  \item[-] \emph{Served as an appointed member of a scientific peer-review panel
    or editorial board.} \itemnote{\cf{programsvc}.}
  \item[-] \emph{Held the position of Associate Professor, Professor or
      equivalent.} \itemnote{\cf{cv} and also my note under the last listed
      requirement for the GS-13 level above.}
\end{itemize}

I have three years of professional experience as a Research Associate~\see{job1}
and five years of experience as a Postdoctoral Fellow~\see{job2} in which
I employed my extensive research experience-based scientific expertise to
undertake responsibilities for my independently developed portfolio of
leadership efforts in computational and theoretical neuroscience and related
subfields of control theory, autonomous systems, embodied cognition, and
artificial intelligence. While these leadership efforts have been
described above in response to other duties and requirements, they are also
detailed in \cfonly{resprogram}, \cfonly{service}, \cfonly{presentations}, and
\cfonly{eduprogram}. Recognition of some of these efforts---including \emph{Fast
Company} magazine ranking JHUAPL as the \#3 ``Most Innovative Workplace''
based on my NSF-funded project---is detailed under \cfonly{recognition}.

Selected examples of my research-based scientific leadership efforts include:
%
\begin{itemize}
  \item Directed research conducted by my student mentees~\see{mentoring}
    and key personnel or co-investigators on my funded research
    projects~\see{resprogram}.
  \item Delivered invited research-based talks at national and international
    scientific meetings and organizations~\see{natltalks}, e.g., AFRL (Sept 2022),
    University of Toronto (Feb 2022), NIH BRAIN Investigators Meeting (June 2020),
    JHU Kavli Neuroscience Discovery Institute (Oct 2019).
  \item Contributed talks presenting novel theoretical integration of scientific
    findings to international scientific/educational meetings~\see{intltalks}, e.g.,
    Neuromatch Conference 3.0 (Oct 2020) and 4.0 (Dec 2021).
  \item Served on the Review Committee for the Cosyne conference (2016, 2020,
    2021)~\see{confsvc}.
  \item Provided scientific peer-review for top computational neuroscience,
    biology, and artificial intelligence journals including Neuron, eLife, PLOS
    Computational Biology, Nature Machine Intelligence, and Neural Computation
    (with the honorific of `Communicator', reserved for highly impactful peer
    reviewers)~\see{service}.
  \item Invited to provide inputs to guide programmatic development of DARPA/MTO
    MEC (Jan 2022).
  \item Invited and served as a reviewer on multiple NSF panels and as an ad-hoc
    reviewer to evaluate funding applications according to scientific merit and
    programmatic criteria (2021--2022)~\see{programsvc}.
  \item Provided feedback to the NSF to help establish criteria and standards
    for logistical discussion plans for panels with difficult multidisciplinary,
    multiday workloads (April 2022).
  \item Invited and participated in the NSF Workshop on Present and Future
    Frameworks for Theoretical Neuroscience (Feb 2019)~\see{natltalks} and
    contributed to a working-group preprint on the role of theory in neuroscience
    (Levenstein, et al. 2020)~\see{preprints} that is currently under review at
    the Journal of Neuroscience~\see{pubs}.
  \item Proposed and co-organized national scientific and educational symposia,
    e.g., at the NIH BRAIN Investigators Meeting (June 2020)~\see{natltalks}
    and as part of the JHUAPL STEM Academy Program funded by my NSF-funded
    collaborative project (Jan 2021)~\see{eduprogram}.
  \item Across the foregoing efforts, I have sought to convey overarching
    guidance on emerging directions and caveats regarding methods, approaches, and
    questions for neuroscience and related subfields.
\end{itemize}


%\pagebreak
\section*{Self-Assessment Questionnaire}
\vspace{-.1in}
\hrule
\vspace{.3in}

\begin{enumerate}
  \item I qualify for this position because I have completed all requirements
    for and successfully obtained a Ph.D. in Neurobiology \& Behavior from Columbia
    University in 2009.
  \item I confirm that I meet the basic experience requirements for all grades.
  \item Qualifying experience is described above under \cfonly{basicreqs}.
  \item I obtained a Ph.D. in Neurobiology \& Behavior from Columbia University
    in 2009.
  \item My qualifying experience occurred in an Academic Environment,
    as described above under \cfonly{basicreqs} and detailed below under
    \cfonly{workexp} with respect to my roles as a Research Associate,
    Postdoctoral Fellow, and Graduate Research Assistant at R1 research
    universities including Brandeis, Columbia, and Johns Hopkins.
  \item \emph{ibid.}
  \item \emph{ibid.}
  \item Job duties, related skills, and responsibilities for the listed
    qualifying positions are listed in detail within the corresponding
    position-related sections starting under \cfonly{workexp}.
  \item Listed areas of scientific expertise span my training, publications, and
    awarded grants as described above under \cfonly{basicreqs} and detailed below
    under \cfonly{workexp} and \cfonly{pubs}.
  \item My Ph.D. was granted by the Department of Neurobiology \& Behavior at
    Columbia University, where I conducted my doctoral studies as a Graduate
    Research Assistant in the Center for Theoretical Neuroscience as described
    under \cfonly{education}.
\end{enumerate}

For assessment items 11--37, supporting details for each skill or capability are
provided in the notes below the corresponding item.

\begin{enumerate}
  \setcounter{enumi}{10}
  \item Analyze the scientific merit of current and projected research programs. 
    \itemnote{Extramural contributor to development of funding programs at IARPA
    and the NSF~\see{programsvc}.}
  \item Analyze trends and new scientific fields to assess the adequacy of
    research being conducted and proposed in a given area.
    \itemnote{My education was grounded in mathematics and
      philosophy~\see{education}, which has allowed me to build methods that I use
      in my research from scratch and to assess their inferential utility, e.g.,
      with respect to statistical validity, etc. As I developed my independent
      research program~\see{resprogram}, I gained substantial experience
      evaluating current trends in methods, approaches, and scientific questions
    in current and emerging fields.}
  \item Identify and recommend the program direction of new research.
    \itemnote{In nearly 23 years of experience in neuroscience
      research~\see{education}~\see{pubs}~\see{resprogram}, I have developed deep
      intuitions and agility in recognizing and evaluating new lines of research.
      My interdisciplinary projects have allowed me to make distinctions between
      approaches make good-faith recommendations on the potential strengths of new
    research.}
  \item Identify gaps in scientific research.
    \itemnote{As described for the previous two items, my interdisciplinary
      research and long history in neuroscience~\see{education}~\see{pubs} and
      related fields has informed my ability to assess the current state of the
      science. In particular, my theoretical interests inform how I survey for
    epistemic gaps based on processes of inductive reasoning.}
  \item Manage the development of organizational or research group policies in
    response to changes in levels of funding or legislative/program changes.
    \itemnote{Based on my extended history of interactions with program officers
      as described above in \cfonly{duties} and \cfonly{basicreqs}, I have extensive
      familiarity with organizational policies and requirements at funding agencies in
    response to funding level changes due to legislation or other factors.}
  \item Monitor scientific progress of research or clinical programs to assure
    that objectives are met.
    \itemnote{Given my experience, particularly as JHU/SOM Research
      Associate~\see{job1}, I have had to continually monitor scientific progress in
      internally and externally funded research projects at varied levels of budget
      and scope, e.g., for small team projects like my JHU/SLI award~\see{sli} and larger
    projects like my NSF award~\see{nsf}.}
  \item Provide guidance or interpret regulations and funding policies regarding
    biomedical, behavioral health, or health-related research.
    \itemnote{As described above under \cfonly{duties}, my extensive
      interactions with program officers and federal funding agencies has deeply
      informed my ability to assess and interpret policy effects for health-releated
    research.}
  \item To verify your level of experience for the previous seven (7) questions,
    please list the name, email address, and phone number of someone who we can
    contact to get more information.
    \itemnote{Prof. Knierim was my postdoctoral supervisor from
      2009--2014~\see{job3}, and he is familiar and up-to-date with my current
    levels of experience and expertise.}
  \item Administer unsolicited grant/contract proposals and provide support
    regarding funding options and areas of program emphasis.
    \itemnote{As described above under \cfonly{duties} and below under
      \cfonly{programsvc}, my extensive interactions with program officers and my
      service on merit-review panels has given me substantial knowledge about the
      processes involved in unsolicited proposals and has informed my ability to
    support funding options aligned with program priorities.}
  \item Appoint and manage grants, contracts, or cooperative agreements review
    panels.
    \itemnote{My participation on multiple federal merit-review
      panels~\see{programsvc}---in addition to observing the processes that my spouse
      (see note for item \#27 below) goes through to do these tasks as a federal
      funding agency program officer---has given me substantive exposure to grants
      administration processes related to the logistics of appointing panels for
      these purposes. This exposure has provided me knowledge that will
    translate to performance of these tasks.}
  \item Evaluate the technical and scientific merit of funding applications and
    proposals being submitted for approval.
    \itemnote{My extensive background in health-related research and scientific
      review based in the highly contentious and multidisciplinary field of
      neuroscience~\see{education}~\see{resprogram}~\see{service} has provided me
      world-class preparation for evaluating the technical and scientific merit of
    funding applications.}
  \item Review applications for funding created by others to ensure they have
    been prepared properly.
    \itemnote{I have prepared numerous funding applications and grant proposals
      intended for various funding agencies~\see{funding}~\see{resprogram}, which has
      given me substantial real-world experience and knowledge regarding the proper
    preparation of funding applications.}
  \item Recommend and distribute research funding for new research
    initiatives/efforts.
    \itemnote{My knowledge and familiarity with the processes of recommending
      and distributing research funding derive from my interactions with program
      officers as recommendations are being finalized~\see{funding} and from
      observing my spouse (see note for item \#27 below) perform this role at the
    level of federal funding agencies.}
  \item Manage research fund distribution amongst competing interests within an
    organization.
    \itemnote{\emph{ibid.}}
  \item Obtain independent grant funding for research from either the private or
    public sector.
    \itemnote{I have successfully obtained independent research funding from
      public sector agencies~\see{funding} and have submitted several applications to
    private foundations.}
  \item Write or co-write research grants.
    \itemnote{I have extensive experience writing varied forms of grant
    proposals and funding applications. \cf{resprogram}.}
  \item Have you ever served as a Program/Project Officer responsible for
    developing, coordinating, and administering grants, cooperative agreements and
    contracts?
    \itemnote{I have only served in academic environments. Though I should note
      that my spouse is a program officer at another agency, so some of my familiarity
    with the role comes through that relationship.}
  \item Deliver presentations to senior medical/health practitioners,
    professional societies, and academia on pertinent matters relating to
    biomedical, behavioral health, or clinical health-related research.
    \itemnote{As described above under \cfonly{duties} and \cfonly{basicreqs},
      I have extensive experience in scientific communication in many settings to
      varied audiences including professional, academic, and educational groups.
    \cfcf{talks}{presentations}.}
  \item Prepare major components of position papers or reviews on policy or
    critical scientific issues related to biomedical, behavioral health, or clinical
    health-related research.
    \itemnote{I have written original research articles across the extent of my
      career, and I have more recently taken to writing higher-level perspective-style
      review papers that give broader views of critical scientific problems and
      questions. My extensive writing experience will translate to preparation of
      position papers, policy reviews, etc., related to scientific or health-related
    issues. \cfcf{pubs}{preprints}.}
  \item Present briefings on research to high level officials providing multiple
    alternatives and advice on the best course of action.
    \itemnote{See notes for items \#17, \#19, \#28, and \#29 above.}
  \item Write reports, evaluations, summary statements, or correspondences as a
    part of a review process.
    \itemnote{As described in the notes for items \#21, \#22, and \#29 and
      my experience writing summary statements on review panels~\see{programsvc},
      I have extensive knowledge, experience, and skill in evaluating proposals
      within a review process and in writing clearly about scientific, technical, and
      health-related topics. These skills will translate to the writing of reports and
    summary statements.}
  \item Coach and mentor staff engaged in biomedical, behavioral health, or
    clinical health-related research.
    \itemnote{\cf{mentoring}.}
  \item Create project plans, including project scope, goals, tasks, resources,
    schedules, costs, contingencies, and communications.
    \itemnote{\cf{resprogram}.}
  \item Manage and direct multiple research or project teams concurrently.
    \itemnote{In the narrative summaries provided above under \cfonly{duties}
      and \cfonly{basicreqs}, the activities that I described occurred
    concurrently or overlapping in time.}
  \item Manage research or project execution to ensure adherence to budget,
    schedule, and scope.
    \itemnote{I have extensive experience managing and co-managing projects of
      varied scale, from small teams to large cross-divisional collaborations,
    including scheduling, budget, and scope of work. \cf{resprogram}.}
  \item Monitor the performance of research or project team members, providing,
    and documenting technical feedback.
    \itemnote{As noted for item \#35, my experience in research project
      management also extends to interacting usefully with team members by listening
      to them, monitoring their progress, and providing constructive and helpful
    feedback at appropriate times.}
  \item Resolve conflicts, differences, or problems among colleagues,
    subordinates, or team members.
    \itemnote{I have worked to resolve conflicts and differences in approach
      between team members in research projects of various size and scope.
    \cfcf{resprogram}{mentoring}.}
\end{enumerate}

 
\pagebreak
\section*{Work Experience}
\label{sec:workexp}
\vspace{-.1in}
\hrule
\vspace{.3in}

\section{JHU/SOM Research Associate }
\label{sec:job1}

\begin{tabular*}{6.3in}{l@{\extracolsep{\fill}}r}
  \textbf{Research Associate (Faculty)} & 733 North Broadway \\
  \textbf{Johns Hopkins University School of Medicine} & Edward D. Miller Research Building \\
  Department of Biomedical Engineering & Baltimore, MD \\
\end{tabular*}
\\[.1in]
\textbf{7/2019 -- 6/2022 \\ Full-Time Equivalent, 40--80 Hours/Week} \\


\subsection*{Job Duties, Related Skills, and Responsibilities}

\begin{itemize}
  \item[-] Initiated research collaborations and continued grant application efforts
  \item[-] Developed computational models of autonomous neural control
  \item[-] Presented research findings in conference presentations, invited talks, and published articles
  \item[-] Delivered invited talks at NIH BRAIN Investigators Meeting symposium and the Air Force Research Lab
  \item[-] Interpreted research about artificial intelligence, swarm cognition, and neuroethology
  \item[-] Supervised high-school and masters students in computational methods
  \item[-] Directed research conduct, budget, and administration as co-PI of NSF-awarded project
  \item[-] Primary and senior authorships on peer-reviewed papers on neural control systems
  \item[-] Responsible for most operational aspects of my sponsor lab (3 years)
  \item[-] Responsible for multiple scientific projects and grant application efforts
  \item[-] Served as reviewer for top journals, conferences, and funding panels
  \item[-] Demonstrated extensive scientific expertise and leadership in funding panels, invited talks, and grants 
  \item[-] Coordinated teaching activities to support STEM component of NSF project
\end{itemize}



\vspace{.2in}
\hrule
\section{JHU/SOM Postdoctoral Fellow}
\label{sec:job2}

\begin{tabular*}{6.3in}{l@{\extracolsep{\fill}}r}
  \textbf{Postdoctoral Fellow} & 733 North Broadway \\
  \textbf{Johns Hopkins University School of Medicine} & Edward D. Miller Research Building \\
  Department of Biomedical Engineering & Baltimore, MD \\
\end{tabular*}
\\[.1in]
\textbf{8/2014 -- 6/2019 \\ Full-Time Equivalent, 40--80 Hours/Week} \\


\subsection*{Job Duties, Related Skills, and Responsibilities}

\begin{itemize}
  \item[-] Independently developed research collaborations and organized grant efforts 
  \item[-] Developed analysis protocols and conducted computational modeling studies 
  \item[-] Presented scientific research findings at conferences and in published papers 
  \item[-] Supervised an exchange student in masters program
  \item[-] Principal investigator for my internal JHU project
  \item[-] First and second authorships on peer-reviewed research articles
  \item[-] Responsible for operating aspects of my sponsor lab by conducting multiple projects and grant efforts 
  \item[-] Demonstrated research-driven leadership to coordinate collaborations
  \item[-] Served as peer reviewer for research journals
  \item[-] Directed research activities to coordinate preliminary work for proposals
\end{itemize}



\vspace{.2in}
\hrule
\section{JHU/MBI Postdoctoral Fellow}
\label{sec:job3}

\begin{tabular*}{6.3in}{l@{\extracolsep{\fill}}r}
  \textbf{Postdoctoral Fellow} & 3400 N. Charles Street \\
  \textbf{Johns Hopkins University} & Krieger Hall \\
  Zanvyl Krieger Mind/Brain Institute & Baltimore, MD \\
\end{tabular*}
\\[.1in]
\textbf{7/2009 -- 7/2014 \\ Full-Time Equivalent, 40--80 Hours/Week} \\

\subsection*{Job Duties, Related Skills, and Responsibilities}

\begin{itemize}
  \item[-] Initiated computational neuroscience research projects to investigate the oscillatory interference theory of temporal coding for path integration
  \item[-] Developed protocols and analysis pipelines for quantifying a particular investigatory behavior in rats during spatial navigation tasks
  \item[-] Conducted research studies using neuroinformatics and neurobehavioral data analysis to discover a behavioral basis of memory formation
  \item[-] Documented analysis and modeling findings in my lab notebooks, lab meeting presentations, scientific conferences, and published research articles
  \item[-] Interpreted published research results in the fields of neural coding, pulse-coupled networks, behavioral ethology, and place cell physiology
  \item[-] Independently developed distinct research projects based on detailed quantification of behavior in experiments and abstract theoretical models of neural coding
  \item[-] Primary authorship of two peer-reviewed journal articles
  \item[-] Responsible for conducting nonoverlapping modeling and analysis projects over the same timeframe
  \item[-] Supervised a graduate student research assistant who learned to use my analysis protocols and pipelines for their thesis work
  \item[-] Presented results from my thesis work at a scientific conference
  \item[-] Served as a peer reviewer for several research journals
  \item[-] Directed the research activities of my student mentee for their thesis work
\end{itemize}



\pagebreak
%\vspace{.2in}
\hrule
\section{Columbia Graduate Research Assistant}
\label{sec:job4}

\begin{tabular*}{6.3in}{l@{\extracolsep{\fill}}r}
  \textbf{Graduate Research Assistant} & 3227 Broadway \\
  \textbf{Columbia University} & Jerome L. Greene Science Center \\
  Center for Theoretical Neuroscience & New York, NY \\
\end{tabular*}
\\[.1in]
\textbf{8/2005 -- 6/2009 \\ Full-Time Equivalent, 40--80 Hours/Week} \\


\subsection*{Job Duties, Related Skills, and Responsibilities}

\begin{itemize}
  \item[-] Initiated multiple interrelated subprojects contributing to my doctoral thesis
  \item[-] Developed a series of protocols for quantifying hippocampal remapping in random foraging experiments to support and strengthen computational modeling results
  \item[-] Conducted computational modeling studies of spatial navigation and the neural coding of space in the hippocampus and entorhinal cortex of rodents
  \item[-] Documented modeling and data analysis findings in lab notebooks, presentations, my doctoral thesis, and two peer-reviewed publications
  \item[-] Interpreted research literatures of experimental, theoretical, and computational approaches to investigating spatial memory and hippocampal function
  \item[-] Independently developed theoretical and computational modeling research projects toward the completion of my doctoral studies
  \item[-] Primary authorship of an original research article describing the main results of my thesis work that was published in a peer-reviewed journal
\end{itemize}



\vspace{.2in}
\hrule
\section{Brandeis Graduate Research Assistant}
\label{sec:job5}

\begin{tabular*}{6.3in}{l@{\extracolsep{\fill}}r}
  \textbf{Graduate Research Assistant} & 415 South Street \\
  \textbf{Brandeis University} & Volen National Center for Complex Systems \\
  Department of Biology & Waltham, MA \\
\end{tabular*}
\\[.1in]
\textbf{8/2003 -- 7/2005 \\ Full-Time Equivalent, 40--80 Hours/Week} \\


\subsection*{Job Duties, Related Skills, and Responsibilities}

\begin{itemize}
  \item[-] Initiated rotation research projects in three different labs based on distinct questions and methodological (computational) approaches
  \item[-] Developed protocols for a rotation lab to facilitate data analysis with the statistical inference programs that I developed for their data
  \item[-] Conducted modeling and data analysis studies in neuroscience and cognitive psychology
  \item[-] Documented my findings in lab notebooks, lab meeting presentations, and similar venues throughout my rotations
  \item[-] Interpreted published research literature in many subfields of neuroscience, computational methods, and statistical and machine learning
  \item[-] Independently developed projects including a doctoral qualifying proposal, several rotation projects, and modeling projects in my graduate lab
  \item[-] Primary authorship of a peer-reviewed research paper based on modeling results from a rotation project
  \item[-] Responsible for multiple projects including rotation projects, qualifying proposal, and new seedling project after joining my graduate lab
  \item[-] Coordinated teaching activities, including extra review sessions, for an introductory neuroscience course and a biology laboratory course with professors and other teaching assistants
\end{itemize}


\vspace{.2in}
\hrule
\section{U.Va. Undergraduate Research Assistant}
\label{sec:job6}

\begin{tabular*}{6.3in}{l@{\extracolsep{\fill}}r}
  \textbf{Undergraduate Research Assistant} & 2028 Cobb Hall \\
  \textbf{University of Virginia} & Laboratory of Computational Neurodynamics \\
  Department of Neurosurgery & Charlottesville, VA \\
\end{tabular*}
\\[.1in]
\textbf{6/2000 -- 7/2003 \\ Part-Time, 20 Hours/Week} \\


\subsection*{Job Duties, Related Skills, and Responsibilities}

\begin{itemize}
  \item[-] Initiated computational modeling projects based on the lab’s existing simulation software to investigate new questions about hippocampal sequence learning
  \item[-] Conducted modeling studies examining limitations of goal-directed sequence learning and recall in a detailed spiking network model of CA3 hippocampus
  \item[-] Documented extensive findings in code, lab meeting presentations, and my research notebooks
  \item[-] Interpreted published literature in the fields of hippocampal physiology and anatomy, spatial navigation and place cells, and episodic memory
  \item[-] Presented the findings of my modeling study as a conference poster and paper at a major international conference for neural networks
\end{itemize}


\vspace{.2in}
\hrule
\section{NIH Research Intern (This is a federal job)}
\label{sec:job7}

\begin{tabular*}{6.3in}{l@{\extracolsep{\fill}}r}
  \textbf{High-School Research Intern} & 6555 Rock Spring Drive \\
  \textbf{NIH/CIT} & Building 12A \\
  Center for Molecular Modeling & Bethesda, MD \\
\end{tabular*}
\\[.1in]
\textbf{Summers 1996 -- 1997 \\ Part-Time, 20 Hours/Week \\ Grade: GS-1} \\


\subsection*{Job Duties, Related Skills, and Responsibilities}

\begin{itemize}
  \item[-] Initiated two computational research projects (under supervision) in biochemistry, biophysics, and molecular dynamics
  \item[-] Conducted simulation studies using beowulf high-performance clusters for molecular dynamics models and ligand binding quantification
  \item[-] Documented findings for my projects in lab notebooks and presentation materials
  \item[-] Interpreted published results about deoxyhypusine synthase and hyperthermophilic proteins to guide my computational modeling approaches
  \item[-] Presented results from modeling studies at NIH Poster Day
\end{itemize}



\section*{Education}
\vspace{-.1in}
\hrule
\vspace{.3in}
\label{sec:education}

\begin{itemize}
  \item
    \begin{tabular*}{6.3in}{l@{\extracolsep{\fill}}r}
      \textbf{Columbia University} & New York, NY \\
      Department of Neurobiology \& Behavior & 2005--2009 \\
      Center for Theoretical Neuroscience\\
      Degrees: Ph.D. (2009); M.Phil. (2008); M.A. (2006) \\
      Advisor: Larry~Abbott\\
    \end{tabular*}
  \item
    \begin{tabular*}{6.3in}{l@{\extracolsep{\fill}}r}
      \textbf{Brandeis University} & Waltham, MA \\
      Department of Biology & 2003--2005\\
      Volen Center for Complex Systems\\
      Graduate Program in Neuroscience, \textit{\ul{Continued at Columbia University}} \\
      Advisor: Larry~Abbott\\
    \end{tabular*}
  \item
    \begin{tabular*}{6.3in}{l@{\extracolsep{\fill}}r}
      \textbf{University of Virginia} & Charlottesville, VA \\
      Laboratory of Computational Neurodynamics & 1999--2003\\
      Degrees: B.A.~Mathematics; B.A.~Cognitive Science; Minor in Philosophy \\
      Advisor: William (Chip) Levy\\
      Echols Scholar \\
    \end{tabular*}
\end{itemize}



\section*{Professional Publications}
\label{sec:pubs}
\vspace{-.1in}
\hrule
\vspace{.3in}

\renewcommand{\itemnote}[1]{}

% Insert the text for all the publication lists:
\section*{Journal Publications}

\begin{description}
  \item Buckley E, \joehl{Monaco JD}, Schultz KM, Chalmers R, Hadzic A,
    Zhang K, Hwang GM, and Carr MD. (\emph{\color{lightred}Under review}). \itemtitle{An
      interdisciplinary approach to high school curriculum development: Swarming
    Powered by Neuroscience}. \emph{Frontiers in Education}.
  \item \joehl{Monaco JD}, Rajan K, and Hwang GM. (\emph{\color{lightred}In revision}).
      \itemtitle{A brain basis of dynamical intelligence for AI and computational
    neuroscience}. \emph{Nature Machine Intelligence}.
  \item \href{https://doi.org/10.1007/s00422-020-00823-z}
    {\joehl{Monaco JD}, Hwang GM, Schultz KM, and Zhang K. (2020).
    \itemtitle{Cognitive swarming in complex environments with attractor
      dynamics and oscillatory computing}. \emph{Biological Cybernetics}, 114,
    269--284. \doi{10.1007/s00422-020-00823-z}}.
  \item \begin{samepage}\href{https://doi.org/10.1016/j.cub.2020.01.083} 
      {Wang CH, \joehl{Monaco JD}, and Knierim JJ. (2020). \itemtitle{Hippocampal
        place cells encode local surface texture boundaries}. \emph{Current Biology},
      30, 1--13. \doi{10.1016/j.cub.2020.01.083}}.
  \itemnote{I mentored the first author in data analysis of rat behavior and
      single-unit recordings, developed the software toolchain used to conduct the
    analyses, and provided intellectual guidance.}\end{samepage}
  \item \href{https://doi.org/10.1371/journal.pcbi.1006741}
    {\joehl{Monaco JD}, De Guzman RM, Blair HT, and Zhang K. (2019).
    \itemtitle{Spatial synchronization codes from coupled rate-phase
      neurons}. \emph{PLOS Computational Biology}, 15(1), e1006741.
    \doi{10.1371/journal.pcbi.1006741}}.
  \item \href{https://www.cell.com/cell/fulltext/S0092-8674(18)31228-5}
    {Tabuchi M, \joehl{Monaco JD}, Duan G, Bell BJ, Liu S, Zhang K, and
      Wu MN. (2018). \itemtitle{Clock-generated temporal codes determine
      synaptic plasticity to control sleep}. \emph{Cell}, 175(5), 1213--27.
    \doi{10.1016/j.cell.2018.09.016}}.
  \itemnote{I developed two modeling strategies for the Wu lab’s circadian clock
      neuron experiments in \emph{Drosophila}. My generative statistical model was
      integrated into stimulation protocols as a timing control for behavioral
      results, and my mechanistic molecular/neuronal model explained observed trends
      and made predictions corroborated by the data. My results or contributions are
    featured in 3/7 main figures and 3/6 supplementary figures.}
  \item \href{http://doi.org/10.1038/nn.3687}
    {\joehl{Monaco JD}, Rao G, Roth ED, and Knierim JJ. (2014).
    \itemtitle{Attentive scanning behavior drives one-trial potentiation of
      hippocampal place fields}. \emph{Nature Neuroscience}, 17(5), 725--731.
    \doi{10.1038/nn.3687}}.
  \item \href{http://doi.org/10.3389/fncom.2011.00039}
    {\joehl{Monaco JD}, Knierim JJ, and Zhang K. (2011). \itemtitle{Sensory
        feedback, error correction, and remapping in a multiple oscillator model of
      place cell activity}. \emph{Frontiers in Computational Neuroscience}, 5:39.
    \doi{10.3389/fncom.2011.00039}}.
  \item \href{http://doi.org/10.1523/JNEUROSCI.1433-11.2011}
    {\joehl{Monaco JD} and Abbott LF. (2011). \itemtitle{Modular
        realignment of entorhinal grid cell activity as a basis for hippocampal
      remapping}. \emph{Journal of Neuroscience}, 31(25), 9414--25.
    \doi{10.1523/jneurosci.1433-11.2011}}.
  \item \href{http://doi.org/10.1371/journal.pbio.1000140}
    {Muzzio IA, Levita L, Kulkarni J, \joehl{Monaco J}, Kentros CG, Stead
      M, Abbott LF, and Kandel ER. (2009). \itemtitle{Attention enhances the
        retrieval and stability of visuospatial and olfactory representations
      in the dorsal hippocampus}. \emph{PLOS Biology}, 7(6), e1000140.
    \doi{10.1371/journal.pbio.1000140}}.
  \itemnote{I contributed oscillatory power analyses and group-level statistical
      analyses of spiking and bursting for odor vs.\ visuospatial tasks in
    single-unit hippocampal recordings from freely-moving mice.}
  \item \href{http://doi.org/10.1101/lm.363207}
    {\joehl{Monaco JD}, Abbott LF, and Kahana MJ. (2007).
    \itemtitle{Lexico-semantic structure and the recognition
      word-frequency effect}. \emph{Learning \& Memory}, 14(3), 204--213.
    \doi{10.1101/lm.363207}}.
\end{description}

\section*{Conference Papers}

\begin{description}
  \item \href{https://www.jhuapl.edu/Content/techdigest/pdf/V35-N04/35-04-Hwang.pdf}
    {Hwang GM, Schultz KM, \joehl{Monaco JD}, and Zhang K. (2021).
    \itemtitle{Neuro-Inspired Dynamic Replanning in Swarms—Theoretical
        Neuroscience Extends Swarming in Complex Environments}. \emph{Johns Hopkins
    APL Technical Digest}, 35, 443--447.}
  \item \href{https://doi.org/10.1117/12.2518966}
    {\joehl{Monaco JD}, Hwang GM, Schultz KM, and Zhang K. (2019).
    \itemtitle{Cognitive swarming: An approach from the theoretical neuroscience
        of hippocampal function}. \emph{Proceedings of SPIE (International society
      for optics and photonics) Defense \& Commercial Sensing}. Micro- and
      Nanotechnology Sensors, Systems, and Applications XI, 109822D, 1--10.
    \doi{10.1117/12.2518966}}.
  \item \href{http://doi.org/10.1109/IJCNN.2003.1223655}
    {\joehl{Monaco JD} and Levy WB. (2003). \itemtitle{T-maze training
        of a recurrent CA3 model reveals the necessity of novelty-based
        modulation of LTP in hippocampal region CA3}. \emph{Proceedings of
      International Joint Conference on Neural Networks}, 1655--1660.
    \doi{10.1109/IJCNN.2003.1223655}}.
  \itemnote{This paper received First Place in the IJCNN Student Paper
    Competition.}
\end{description}

\section*{Preprints}

\begin{description}
  \item \href{https://arxiv.org/abs/2109.05545}{Buckley E, \joehl{Monaco JD},
      Schultz KM, Chalmers R, Hadzic A, Zhang K, Hwang GM, and Carr MD. (2021).
    \itemtitle{An interdisciplinary approach to high school curriculum development:
    Swarming Powered by Neuroscience}. \arxiv{2109.05545}}.
  \item \href{https://arxiv.org/abs/2105.07284}
    {\joehl{Monaco JD}, Rajan K, and Hwang GM. (2021). \itemtitle{A brain
      basis of dynamical intelligence for AI and computational neuroscience}.
    \arxiv{2105.07284}}.
  \item \href{https://arxiv.org/abs/2003.13825}
    {Levenstein D, Alvarez VA, Amarasingham A, Azab H, Gerkin RC, Hasenstaub
      A, Iyer R, Jolivet RB, Marzen~S, \joehl{Monaco JD}, Prinz AA, Quraishi
      S, Santamaria F, Shivkumar S, Singh MF, Stockton DB, Traub R, Rotstein
      HG, Nadim F, and Redish AD. (2020). \itemtitle{On the role of theory and
    modeling in neuroscience}. \arxiv{2003.13825}}.
  \item \href{https://arxiv.org/abs/1909.06711}
    {\joehl{Monaco JD}, Hwang GM, Schultz KM, and Zhang K. (2019).
    \itemtitle{Cognitive swarming in complex environments with attractor
    dynamics and oscillatory computing}. \arxiv{1909.06711}}.
  \item \href{http://doi.org/10.1101/764282}
    {Wang CH, \joehl{Monaco JD}, and Knierim JJ. (2019). \itemtitle{Hippocampal
      place cells encode local surface texture boundaries}. \emph{bioRxiv}.
    \doi{10.1101/764282}}.
  \item \href{http://dx.doi.org/10.1101/211458}
    {\joehl{Monaco JD}, Blair HT, and Zhang K. (2017). \itemtitle{Spatial theta
        cells in competitive burst synchronization networks: Reference frames from
    phase codes}. \emph{bioRxiv}. \doi{10.1101/211458}}.
\end{description}

\section*{Thesis}

\begin{description}
  \item \href{http://search.proquest.com/docview/304862872/abstract}
    {\joehl{Monaco JD}. (2009). \itemtitle{Models and mechanisms for integrating
      cortical feature spaces}. Doctoral Dissertation, Columbia University, New
    York. \emph{ProQuest Publication No. AAT 3393609}}.
  \itemnote{\href{https://jdmonaco.com/files/monaco-phdthesis-2009.pdf}{Click
    here for the original version with high-quality color figures.}}
\end{description}



\renewcommand{\itemnote}[1]{
  \begin{description}
    \item[$\rightarrow$] \hspace{.09in}{\color{darkgray}\it #1}
  \end{description}
}


\section*{References}
\label{sec:references}
\vspace{-.1in}
\hrule
\vspace{.3in}

% Insert the text for all the references:
\input{../../jobsearch/package/refcontacts/references_nih}


\pagebreak

\section*{Additional Information (Curriculum Vitae)} 
\label{sec:cv}
\vspace{-.1in}
\hrule
\vspace{.3in}

\section*{Joseph D. Monaco, Ph.D.}
{
  \fontspec{Verdana}\small
  \begin{tabular*}{3.0in}{c@{\extracolsep{\fill}}rlc@{\extracolsep{\fill}}}
    %\hline\\
    & \textsc{Email} & \href{mailto:joe@selfmotion.net}{\color{hopkinsblue}\texttt{joe@selfmotion.net}} & \\
    & \textsc{Web}   & \href{https://jdmonaco.com/}{\color{hopkinsblue}\texttt{jdmonaco.com}} & \\
    & \textsc{ORCID} & \href{https://jdmonaco.com/orcid}{\color{hopkinsblue}\texttt{0000-0003-0792-8322}} & \\
    & \textsc{GitHub} & \href{https://jdmonaco.com/github}{\color{hopkinsblue}\texttt{github.com/jdmonaco}} & \\
    & \textsc{Google Scholar} & \href{https://jdmonaco.com/google-scholar}{\color{hopkinsblue}\texttt{gceOLZEAAAAJ}} & \\
  \end{tabular*}
  \vspace{.15in}
  %\hrule
}

\section*{Funding Awards} 
\label{sec:funding}

\begin{itemize}
  \item \href{https://www.nsf.gov/awardsearch/showAward?AWD_ID=1835279&HistoricalAwards=false}
    {\itemtitle{NCS-FO: Spatial intelligence for swarms based on hippocampal
    dynamics}}\hspace{\stretch{1}}2018--2021
    \begin{itemize}
      \item NSF\slash NCS FOUNDATIONS (BRAIN Initiative) Award No.~1835279: \$862K/\$997K (Direct/Total)
      \item \textbf{Lead PI:} Kechen Zhang
      \item \textbf{Co-PIs, JHUAPL}: Grace Hwang, Robert W. Chalmers, Kevin
        Schultz, and M. Dwight Carr
      \item \textbf{Research Associate (FY19)/Co-PI (FY20--FY21): \joehl{Joseph D. Monaco}}
    \end{itemize}
  \itemnote{I co-developed this project and co-wrote the proposal
      with a JHUAPL colleague (see \emph{\nameref{sec:nsf}} on
      p.\pageref{sec:nsf}). As a Research Associate faculty at JHU as of
    FY20, my project role was promoted to co-PI.}
\end{itemize}

\begin{itemize}
  \item \href{https://projectreporter.nih.gov/project_info_description.cfm?aid=9652210&icde=42555668&ddparam=&ddvalue=&ddsub=&cr=2&csb=default&cs=ASC&pball=}
    {\itemtitle{Spiking network models of sharp-wave ripple sequences with\\
    gamma-locked attractor dynamics}}\hspace{\stretch{1}}2018--2020
    \begin{itemize}
      \item NIH/NINDS R03 Award No.~NS109923: \$50K/\$82K (Direct/Total)
      \item \textbf{PI:} Kechen Zhang
      \item \textbf{Research Associate:} \joehl{Joseph D. Monaco}
    \end{itemize}
  \itemnote{I conceived this project, generated preliminary data, and wrote the
      proposal (see \emph{\nameref{sec:nih}} on p.\pageref{sec:nih}).
    As a Postdoctoral Fellow, JHU policy precluded a PI role.}
\end{itemize}

%\subsection*{Previous Internal Support} 

\begin{itemize}
  \item \href{https://scienceoflearning.jhu.edu/research/learning-to-explore-paths-through-space/}
    {\itemtitle{Learning to explore paths through space}}\hspace{\stretch{1}}2016--2018
    \begin{itemize}
      \item JHU/Science of Learning Institute (SLI) Award: \$150K
      \item \textbf{PI:} Kechen Zhang
      \item \textbf{Co-PI:} David J.~Foster (now at UC Berkeley)
      \item \textbf{Research Associate:} \joehl{Joseph D.~Monaco}
    \end{itemize}
  \itemnote{I conceived this project, initiated the collaboration
      between the Zhang and Foster labs, and wrote the proposal (see
      \emph{\nameref{sec:sli}} on p.\pageref{sec:sli}). As a
    Postdoctoral Fellow, JHU policy precluded a PI role.}
  \label{sec:previnternalsupport}
\end{itemize}

\section*{Research Program --- Development \& Administration}

\subsection*{Inventions \& Patents}
\label{sec:patents}

\lefttabline{0.8in}{7/5/2022}{Inventor, 
  \href{https://www.freepatentsonline.com/11378975.html}{\itemtitle{Autonomous
  Navigation Technology, US patent issued, 11,378,975}}}
\lefttabline{0.8in}{1/3/2020}{Inventor, Autonomous Navigation Technology, US
  patent application, 16,734,294}
\lefttabline{0.8in}{5/10/2019}{Inventor, Neuroinspired Algorithms for Swarming
  Applications, provisional patent, 62/845,957}
\lefttabline{0.8in}{1/3/2019}{Inventor, Neuroinspired Algorithms for Swarming
  Applications, provisional patent, 62/787,891}

\subsection*{Research Program Building \& Leadership}
\label{sec:resprogram}

\researchactivity
{April 2010/2011}
{Fellowship Proposal (NIH/NINDS F32 NRSA)}
{Behavioral Coordination of Entorhinal-Hippocampal Activity for Real-Time
Sensory Updating of Spatial Memory}
{In collaboration with my posdoctoral sponsor Jim Knierim, I conceived and
  developed a postdoctoral fellowship training proposal as a NIH F32 NRSA
  application. The proposal integrated computational modeling with spatial
  navigation experiments based on behavioral data from position-tracking sensors
  and neural data from multiregional hippocampal--entorhinal single-unit ensemble
  recordings. The application received a 21st percentile rank; I followed up the
  2010 application with a 2011 resubmission following discussions with NINDS PO
Jim Gnadt.}
\label{sec:nrsa}

\researchactivity
{Mar. 2016--2018}
{Grant Award (JHU/SLI)}
{Learning to explore paths through space}
{This internal JHU award (2016--2018; see
  \emph{\nameref{sec:previnternalsupport}} on p.\pageref{sec:previnternalsupport})
  resulted from a collaboration with David J. Foster (now at UC Berkeley) that
  I initiated to conduct modeling studies informed by his lab’s hippocampal
  reactivation data. By integrating Prof.~Zhang’s mathematical theories of
  spatial cognitive maps, I wrote and submitted a proposal for a \$200K/2-year
  project to the JHU Science of Learning Institute. The proposal was awarded at
  the \$150K level and research outcomes included (1) novel theories of temporal
  synchronization coding that inspired the 2017 NSF proposal effort, and (2)
  preliminary dynamical models of sharp-wave reactivation that provided the
foundation for the 2018 NIH R03 award.}
\label{sec:sli}

\researchactivity
{April--June 2016}
{Grant Proposal (DARPA/BTO)}
{Noninvasive Gastrovagal Stimulation for Enhanced Neuroplasticity of Cortical
and Hippocampal Networks during Cognitive Training (GEN-C)}
{In response to DARPA announcement BAA-16-24 of the “Targeted Neuroplasticity
  Training (TNT)” program, I worked with colleagues from JHUAPL and JHU/SOM
  Center for Neurogastroenterology to develop a collaborative program involving 3
  PIs and 5 co-Is (8 labs) across divisions, departments, and fields. I recruited
  experimental labs from JHU/MBI and coordinated proposed contributions to
  maximize scientific impact with a budget of \$9.8M/5 years. I coordinated the
  40-page research narrative, including writing, editing, and/or integrating
  each lab’s contributions and worked with ORA to submit the proposal. While
  not funded in total, DARPA/BTO PM Doug Weber funded select components, leading
  to JHUAPL Work Agreement No.~145563 “BCI (Brain Computer Interface)
Technologies” in 2018.}
\label{sec:genc}
%Technologies” in 2018 for \$24,604 to the lab of Prof.~Pasricha.}

\researchactivity
{Nov. 2017--2021\label{sec:nsf}}
{Grant Award (NSF/NCS)}
{NCS-FO: Spatial intelligence for swarms based on hippocampal dynamics}
{This NSF-awarded project (2018--2021; see \emph{\nameref{sec:funding}}
  on p.\pageref{sec:funding}) was the result of 6 months of collaboration,
  brain-storming, and team-building between the Zhang lab at JHU/SOM and a group
  of JHUAPL engineers, mathematicians, and scientists. The project was initially
  inspired by results that I presented at my Society for Neuroscience 2017 meeting
  poster. I wrote Aim 1 and integrated the full research narrative with inputs
  from our collaborators for the proposal of this \$997K/2-year project to develop
  those initial ideas into technological applications (e.g., robotics, autonomous
  control, AI) that reciprocally inform neuroscience. The project has so far
  produced three posters, a conference talk \& proceedings publication, three
  patent applications, a preprint, a research article in Biological Cybernetics, a
  NIH BRAIN Investigators Meeting symposium talk, and a substantial STEM program.
  We received a no-cost extension through FY21 to complete the final phase of the
project.}

\researchactivity
{Jan. 2018--2020}
{Grant Award (NIH/NINDS)}
{Spiking network models of sharp-wave ripple sequences with gamma-locked
attractor dynamics}
{To continue with the collaboration that I initiated with David J. Foster
  (UC Berkeley) on the basis of the internal SLI award (see above), I wrote
  a small modeling proposal that integrated preliminary results from the SLI
  project and recent research developments in the memory reactivation field.
  This proposal was awarded (2018--2020; see \emph{\nameref{sec:funding}} on
  p.\pageref{sec:funding}) through the NIH/NINDS R03 mechanism, supporting
  theoretical and modeling efforts that provided a foundation for my subsequent
research directions.}
\label{sec:nih}

\whitepaper
{Feb.--Mar. 2018}
{Schultz K, Zhang K, and \joehl{Monaco J}}
{BrainSWARRMM: Brain-like Sharp-Waves for Autonomous Replanning \&
Reconnaissance on Matrix Manifolds}
{In response to the Office of Naval Research (ONR) Special Notice
  N00014-18-R-SN05, Topic 3, I helped organize a series of collaborative meetings
  to design a \$2M/4-year project between JHUAPL and JHU/SOM. I co-authored the
resulting white paper that was submitted for consideration to ONR.}

\whitepaper
{May--June 2018}
{Zhang K, \joehl{Monaco JD}, Hwang GM, Schultz KM, Kobilarov M, Foster DJ,
Jacobs J, and Itti L}
{An Integrative Theoretical Framework of the Neural Self-Organization of Active
Perception for Autonomous Spatial Navigation}
{In response to ONR MURI Announcement N00014-18-S-F006 and with the help of
  JHUAPL, I coordinated a series of meetings with 5 PIs across 4 universities
  (Columbia, UC Berkeley, USC, JHU) to design an innovative research program that
  targeted reciprocal advances in experimental \& theoretical neuroscience and
  robotics \& AI across species and scales. The resulting \$7.5M/5-year project
  that I outlined in the white paper was not invited for a full submission. We
  debriefed with the sponsor, ONR PM Marc Steinberg, who revealed that ONR was
  impressed with the project but that they were seeking a different balance of
elements with respect to neuroscience and AI.}

\whitepaper
{August 2019}
{\joehl{Monaco J}, Zhang K, and Schultz K.}
{SW2Mem: Graph Spectral Decoding of Hippocampal-Cortical Loops for Artificial
Consolidation and Dreaming}
{In response to ONR Special Notice N00014-19-S-SN08, Topic 5.1 I conceived this
  project, created the preliminary model and datasets, guided the preliminary
  analyses with JHUAPL collaborators, and wrote \& submitted the white paper to
  ONR outlining a potential \$1.05M/3-year project. ONR declined to invite us to
submit a full proposal.}

\whitepaper
{August 2019}
{Schultz K, Agarwala S, Zhang K, and \joehl{Monaco J}}
{Brain-like Distributed Surveillance using Heterogeneous Agents for integRated
Perception, and Planning (BD-SHARPP)}
{In response to ONR Special Notice N00014-18-R-SN05, Topic 3, we submitted a
  revised version of the March 2018 white paper that was specifically invited by
ONR PM Tom McKenna.}

\researchactivity
{Sept. 11, 2019}
{NSF Project Review}
{Annual advisory board review symposium}
{I delivered a seminar on Aim 1 progress at a JHUAPL-hosted symposium for our
  project’s yearly review, attended by DARPA/I2O PM Hava Siegelmann and other
outside experts.}

\researchactivity
{Feb. 26, 2020}
{Grant Proposal (NSF/NCS) }
{NCS-FO: Neuroeconomics as a biomimetic control theory for mobile robotic
decision making}
{This FY21 proposal was submitted to the NSF/NCS FOUNDATIONS program; while
  it was discussed and received high scores, the application was declined. I
  co-developed this project in collaboration with colleagues at the University of
  Pittsburgh Medical Center (UPMC), JHU Whiting School of Engineering (JHU/WSE),
  and JHUAPL. Our interdisciplinary project brought together multiscale human
  electrophysiological recordings (UPMC), latent state-space models (JHU/WSE),
  control- and game-theoretic analysis (JHUAPL), and mechanistic neural models
  (JHU/BME, for which I would have been co-PI). We proposed to investigate and
  characterize the neural bases of metacognitive brain states that influence
  decision-making during social \& economic games. As a high-risk/high-reward
  element, we proposed to algorithmicize our results to advance human-robot
interaction.}

\researchactivity
{Jan. 14, 2022}
{Grant Proposal (JHU/Discovery Award) }
{Algorithms of flexible navigation in mice and robots}
{This intramural FY23 application for a JHU Discovery Award resulted from a new
  collaboration with Patricia Janak (PI; JHU/PBS) and Céline Drieu (postdoctoral
  fellow), in which we proposed to integrate advanced large-scale neural recording
  technologies with my theoretical modeling of neural systems as a distributed
  control problem. Fundamental questions of neural systems communication were
  to be addressed using convergent data-driven and theory-driven approaches
  to understanding the cognitive dynamics that enable mice to perform spatial
goal-directed memory tasks.}

\section*{Professional Service --- Scientific Peer Review}
\label{sec:service}

\subsection*{Journals}

\lefttabline{0.8in}{2021}{PLOS Computational Biology}
\lefttabline{0.8in}{2021}{Nature Machine Intelligence}
\lefttabline{0.8in}{2020}{Neuroscience and Biobehavioral Reviews}
\lefttabline{0.8in}{2020}{Scientific Reports}
\lefttabline{0.8in}{2019}{eLife}
\lefttabline{0.8in}{2019}{Hippocampus}
\lefttabline{0.8in}{2018--2019}{Neuron}
\lefttabline{0.8in}{2018}{Neural Computation (including as `Communicator')}
\lefttabline{0.8in}{2018}{PLOS ONE}
\lefttabline{0.8in}{2017}{PeerJ}
\lefttabline{0.8in}{2015}{IEEE Transactions in Biomedical Engineering}
\lefttabline{0.8in}{2012--2020}{IEEE Neural Networks}
\lefttabline{0.8in}{2012}{Biological Cybernetics}
\lefttabline{0.8in}{2012}{Neurocomputing}
\lefttabline{0.8in}{2012}{Neuroscience}

\subsection*{Funding Agencies}
\label{sec:programsvc}

\lefttabline{0.8in}{2022}{NSF CAREER Ad-Hoc Reviewer}
\lefttabline{0.8in}{2022}{NSF EFRI Preliminary Review Panel}
\lefttabline{0.8in}{2022}{NSF EFRI Final Review Panel}
\lefttabline{0.8in}{2020--2022}{NSF EFRI Program Development, Extramural Contributor}
\lefttabline{0.8in}{2014}{IARPA Program Development, Extramural Contributor}

\vbox{%
\subsection*{Conferences}
\label{sec:confsvc}

\lefttabline{0.8in}{2020--2021}{Cosyne, Review committee member}
\lefttabline{0.8in}{2016}{Cosyne, Review committee member}
}

\section*{Scientific Presentations}
\label{sec:talks}

\subsection*{Workshops, Seminars, and Invited Talks --- Regional}

\begin{longtable}{@{\hspace{0.2in}}l>{\raggedright\arraybackslash}p{.82\textwidth}}
  10/2/2019 \hspace{0.1in} & ``\unpubtitle{Oscillations, attractors, and
  sequences: Extending hippocampal computations to artificial systems}.''
  \emph{Invited Lecture}. Kavli Neuroscience Discovery Institute, Johns Hopkins
  University, Baltimore, MD\\
  \tabularnewline
  9/25/2019 & ``\unpubtitle{Decoding septohippocampal theta cells during
  exploration reveals unbiased environmental cues in firing phase}.''
  \emph{Poster Session}. Kavli Neuroscience Discovery Institute, Johns Hopkins
  University, Baltimore, MD\\
  \tabularnewline
  12/7/2016 & ``\unpubtitle{Spatial rate/phase correlations in theta cells
  can stabilize randomly drifting path integrators}.'' \emph{Poster Session}.
  Greater Baltimore SfN Meeting, Baltimore, MD\\
  \tabularnewline
  1/22/2016 & ``\unpubtitle{Hippocampal circuits for space, memory, and
  navigation: From minimal models to biologically inferred networks}.''
  \emph{Invited Lecture}. Department of Pharmacology, University of Maryland,
  Baltimore, MD\\
  \tabularnewline
  9/6/2014 & ``\unpubtitle{Stopping to look: How attentive scanning behavior
  reveals the formation of new memories}.'' \emph{Department Retreat Seminar}.
  Department of Neuroscience, Johns Hopkins University, Baltimore, MD\\
  \tabularnewline
  4/21/2014 & ``\unpubtitle{Landmark influence: How attention to sensory cues
  stabilizes and updates the hippocampal cognitive representation of space}.''
  \emph{Advanced Researcher Seminar}. Zanvyl Krieger Mind/Brain Institute, Johns
  Hopkins University, Baltimore, MD\\
  \tabularnewline
  4/1/2014 & ``\unpubtitle{Hippocampus and declarative memory:
  Head scanning}.'' \emph{Department `Lab Lunch' Seminar}. Department of
  Neuroscience, Johns Hopkins University, Baltimore, MD\\
\end{longtable}

\subsection*{Workshops, Seminars, and Invited Talks --- National}
\label{sec:natltalks}

\begin{longtable}{@{\hspace{0.2in}}l>{\raggedright\arraybackslash}p{.82\textwidth}}
  8/26/2022 \hspace{0.3in} &
  \href{https://jdmonaco.com/files/monaco-2022-afrl-quest-slides.pdf}
  {``\unpubtitle{Brain oscillations: From cortical computing to the existential
    nonduality of conscious agents}.'' \emph{Invited Public Seminar}. Qualia
    Exploitation for Sensor Technology (QuEST), Air Force Research Lab/Autonomous
  Capabilities Team 3 (AFRL/ACT3), Online. \itemtitle{[PDF]}}\\
  \tabularnewline
  6/1/2020 \hspace{0.3in} & \label{sec:symposium}
  \href{https://youtu.be/2jy1ENYHRAw?t=902}{``\unpubtitle{Can Transitory
    Neurodynamics Unify Learning Theories for Brains and Machines?}''
    \emph{Invited Lecture \& Panel Discussion}. Symposium on ``How Can Dynamical
    Systems Neuroscience Reciprocally Advance Machine Learning?'', 6th Annual
  BRAIN Initiative Investigators Meeting, NIH, Online. \itemtitle{[YouTube]}}\\
  \tabularnewline
  5/18/2020 \hspace{0.3in} & ``\unpubtitle{Computational Approaches to the
  Neural Dynamics of Time, Memory, and Behavior}.'' \emph{Invited Lecture}.
  Department of Neuroscience, Medical Discovery Team for Optical Imaging,
  University of Minnesota, Online\\
  \tabularnewline
  2/24/2020 \hspace{0.3in} & ``\unpubtitle{Computational Mechanisms of Memory:
  Linking Behavior, Space, \& Time}.'' \emph{Invited Lecture}. Department of
  Psychology, University of Nevada, Las Vegas, NV\\
  \tabularnewline
  1/31/2020 \hspace{0.3in} & ``\unpubtitle{Attractors, memory, and oscillations:
  Computational motifs of spatial learning}.'' \emph{Invited Lecture}.
  Department of Biological Sciences, University of Texas at El Paso, El Paso, TX\\
  \tabularnewline
  4/17/2019 \hspace{0.3in} & ``\unpubtitle{Emergent dynamics of hippocampal
  circuitry as a basis for robust self-organized planning in mobile swarms}.''
  \emph{Invited Lecture}. SPIE (International society for optics and photonics)
  Defense \& Commercial Sensing 2019 conference, Baltimore, MD\\
  \tabularnewline
  4/10/2019 & NSF/Neural \& Cognitive Systems (NCS) PI
  Workshop. \emph{Participant}. Marriott Wardman Park Hotel, Washington, D.C.\\
  \tabularnewline
  2/3--2/7/2019 & NSF/BRAIN Initiative Workshop: Present and Future Frameworks
  of Theoretical Neuroscience. \emph{Invited Participant}. University of Texas,
  San Antonio, TX\\
  \tabularnewline
  1/3/2014 & ``\unpubtitle{Head scans drive the formation and potentiation
  of place fields during exploration}.'' \emph{Data Blitz}. 38th Annual Winter
  Conference on Neurobiology of Learning \& Memory, Park City, UT\\
  \tabularnewline
  4/10/2009 & ``\unpubtitle{Rapid spatial map formation and remapping by
  competing over grid cell inputs}.'' \emph{Thesis Seminar}. Department of
  Neurobiology \& Behavior, Columbia University Medical Center, New York, NY\\
\end{longtable}

\subsection*{Workshops, Seminars, and Invited Talks --- International}
\label{sec:intltalks}

\begin{tabular}{@{\hspace{0.2in}}l>{\raggedright\arraybackslash}p{.82\textwidth}}
  2/01/2022 \hspace{0.2in} & ``\unpubtitle{Theory-Driven Data Science to
    Understand the Neural Dynamics of Memory and Behavior}.'' \emph{Invited Talk}.
    Department of Cell \& Systems Biology, University of Toronto, Canada, Online \\
  \tabularnewline
  12/01/2021 \hspace{0.2in} &
  \href{https://youtu.be/3mKkLksOyfk}{``\unpubtitle{Learning as swarming:
      Cognitive flexibility from the neural dynamics of phase-coupled attractor
    maps}.'' \emph{Contributed Talk}. Neuromatch 4.0 Conference, Online.
  \itemtitle{[YouTube]}}\\
  \tabularnewline
  10/29/2020 \hspace{0.2in} &
  \href{https://www.youtube.com/watch?v=WwYDMpD7j4Q}{``\unpubtitle{Spatial
      theta-phase coding in the lateral septum: a theory of allocentric feedback
    during navigation}.'' \emph{Contributed Talk}. Neuromatch 3.0 Conference,
  Online. \itemtitle{[YouTube]}}\\
  \tabularnewline
  10/7/2020 \hspace{0.2in} & ``\unpubtitle{Computing path integration with
  oscillatory phase codes in biological and artificial systems}.'' \emph{Data
  Blitz}. iNAV Symposium 2020, Online\\
  \tabularnewline
  7/1/2010 \hspace{0.2in} & ``\unpubtitle{Medial versus lateral modes for
  reconfiguring hippocampal representations}.'' \emph{Invited Lecture}. Grid
  Cell Meeting, Gatsby Computational Neuroscience Unit, University College
  London, UK\\
\end{tabular}

\subsection*{Conference Presentations}
\label{sec:presentations}

\begin{description}
  \item[\quad]
    \href{https://www.cvent.com/events/6th-annual-brain-initiative-investigators-meeting/custom-116-4e2aadaa6cd549a3a4b53113cd172ad2.aspx}
    {\joehl{Monaco JD}, Hwang GM, Schultz K, Zhang K. (2020).
    \itemtitle{Cognitive swarming in complex environments with attractor
        dynamics and oscillatory computing}. \emph{6th Annual BRAIN Initiative
    Investigators Meeting}. Online, with audio narration. June~2020.}
  \item[\quad]
    \href{https://www.fens.org/Meetings/The-Brain-Conferences/Dynamics-of-the-brain/}
    {\joehl{Monaco JD}, Hwang GM, De Guzman RM, Blair HT, Zhang K. (2019).
    \itemtitle{Spatial rate-phase coding in lateral septal ‘phaser cells’:
        single-unit data and theta-bursting models}. \emph{FENS (Federation of
        European Neuroscience Societies) Dynamics of the brain: Temporal aspects of
    computation}. North Copenhagen, Denmark. June~2019.}
  \item[\quad]
    \href{https://www.cvent.com/events/5th-annual-brain-initiative-investigators-meeting/event-summary-de9c0d8f934b46eb8d80b55bcfbfe96a.aspx}
    {\joehl{Monaco JD}, Hwang GM, Schultz K, Zhang K. (2019).
    \itemtitle{Self-organized swarm control using neural principles of spatial
      phase coding}. \emph{5th Annual BRAIN Initiative Investigators Meeting}.
    Washington, D.C. April~2019.} 
  \item[\quad]
    \href{https://abstractsonline.com/pp8/#!/4649/presentation/10884}
    {Hwang GM, Schultz K, \joehl{Monaco JD}, Chalmers RW, Lau SW, Yeh BY,
      Zhang K. (2018). \itemtitle{Self-organized swarm control using neural
      principles of spatial phase coding}. \emph{Society for Neuroscience}.
    San Diego, CA. November~2018.}
  \item[\quad]
    \href{https://www.abstractsonline.com/pp8/#!/4376/presentation/6085}
    {\joehl{Monaco J}, Blair HT, Zhang K. (2017). \itemtitle{Decoding
        septohippocampal theta cells during exploration reveals unbiased
      environmental cues in firing phase}. \emph{Society for Neuroscience}.
    Washington, D.C. November~2017.}
  \item[\quad]
    \href{https://jdmonaco.com/files/monaco-paper-cosyne15.pdf}
    {\joehl{Monaco JD}, Blair HT, Zhang K. (2015). \itemtitle{Spatial
        rate/phase correlations in theta cells can stabilize randomly drifting path
    integrators}. \emph{Cosyne}. Salt Lake City, UT. March~2015.}
  \item[\quad]
    \href{https://www.abstractsonline.com/Plan/ViewAbstract.aspx?sKey=973d2662-ba7a-4ad2-aff9-fe0d4b77c262&cKey=9917ffaf-9e31-4213-acb9-4aab498ab4cd&mKey=54c85d94-6d69-4b09-afaa-502c0e680ca7}
    {\joehl{Monaco J}, Blair HT, Zhang K. (2014). \itemtitle{Spatial rate/phase
        codes provide landmark-based error correction in a temporal model of theta
    cells}. \emph{Society for Neuroscience}. Washington, D.C.  November~2014.}
  \item[\quad]
    \href{https://www.abstractsonline.com/Plan/ViewAbstract.aspx?sKey=bfb59866-8deb-44a6-9515-a7aab630507b&cKey=d201b3aa-7725-452e-b0dd-c41d204b5b54&mKey=54c85d94-6d69-4b09-afaa-502c0e680ca7}
    {Wang CH, Rao G, \joehl{Monaco JD}, Deshmukh SS, Knierim JJ. (2014).
    \itemtitle{Potentiation of place fields along the CA1 transverse axis by
      investigatory head-scanning behavior}. \emph{Society for Neuroscience}. 
    Washington, D.C. November~2014.}
  \item[\quad]
    \href{https://www.abstractsonline.com/Plan/ViewAbstract.aspx?sKey=32eccac1-4e1d-4e81-bf5c-f39bcb605757&cKey=4710dece-cc8e-4b48-8764-49ea174b91ef&mKey=8d2a5bec-4825-4cd6-9439-b42bb151d1cf}
    {\joehl{Monaco J}, Rao G, Knierim JJ. (2013). \itemtitle{Scanning behavior
        in novel environments promotes \emph{de novo} formation of hippocampal place
    fields in rats}. \emph{Society for Neuroscience}. San Diego, CA. November~2013.}
  \item[\quad]
    \href{https://www.abstractsonline.com/Plan/ViewAbstract.aspx?sKey=f5b9fa94-7d15-48c7-9d67-b89cd2883025&cKey=a53349ca-41b1-4664-b022-85d0d1fe59b8&mKey=70007181-01C9-4DE9-A0A2-EEBFA14CD9F1}
    {\joehl{Monaco J}, Rao G, Knierim JJ. (2012). \itemtitle{Hippocampal LFP
      during rodent head-scanning behavior: Theta and sharp-wave ripples}.
    \emph{Society for Neuroscience}. New Orleans, LA. October~2012.}
  \item[\quad]
    \href{https://www.abstractsonline.com/Plan/ViewAbstract.aspx?sKey=c48e9f5f-1274-4486-85bf-38ee591629e1&cKey=190bd951-c183-428d-a4c5-01eb61556d79&mKey=8334BE29-8911-4991-8C31-32B32DD5E6C8}
    {\joehl{Monaco J}, Rao G, Knierim JJ. (2011). \itemtitle{Hippocampal place
        cell firing during head-scanning movements is associated with the formation
    of new place fields}. \emph{Society for Neuroscience}. Washington, D.C. November~2011.}
  \item[\quad]
    \href{https://www.abstractsonline.com/Plan/ViewAbstract.aspx?sKey=c48e9f5f-1274-4486-85bf-38ee591629e1&cKey=3ec26e6f-8c59-4be2-bad3-e1572d75e07e&mKey=8334BE29-8911-4991-8C31-32B32DD5E6C8}
    {Rao G, \joehl{Monaco J}, Knierim JJ. (2011). \itemtitle{Environmental
        novelty promotes rodent head-scanning behavior linked to enhanced entorhinal
      activity}. \emph{Society for Neuroscience}. Washington,
    D.C. November~2011.}
  \item[\quad]
    \href{https://www.frontiersin.org/10.3389/conf.fnins.2010.03.00192/event_abstract}
    {\joehl{Monaco JD}, Zhang K, Blair HT, Knierim JJ. (2010).
    \itemtitle{Cue-based feedback enables remapping in a multiple oscillator
      model of place cell activity}. \emph{Cosyne}. Salt Lake City, UT.
    February~2010.}
  \item[\quad] \joehl{Monaco JD}, Abbott LF. (2009). \unpubtitle{Dynamic
      hippocampal remapping using recurrent inhibition on realigning grid cell
    inputs}. \emph{Cosyne}. Salt Lake City, UT. February~2009.
  \item[\quad] \joehl{Monaco JD}, Muzzio IA, Levita L, Abbott LF. (2006).
    \unpubtitle{Entorhinal input and global remapping of hippocampal place
    fields}. \emph{CNS}. Edinburgh, UK. July~2006.
  \item[\quad] \joehl{Monaco JD}, Abbott LF. (2006). \unpubtitle{Entorhinal
    input and the remapping of hippocampal place fields}. \emph{Cosyne}. Salt Lake
    City, UT. March~2006.
  \item[\quad] \joehl{Monaco JD}, Levy WB. (2003). \unpubtitle{T-maze training
      of a recurrent CA3 model reveals the necessity of novelty-based modulation of
    LTP in hippocampal region CA3}. \emph{IJCNN}. Portland, OR. July~2003.
  \item[\quad] \joehl{Monaco JD}, Perlstein RP. (1997). \unpubtitle{Monte-Carlo
      analysis of deoxyhypusine synthase inhibitor ligand conformations}. \emph{NIH
    Poster Day}. Bethesda, MD. August~1997.
\end{description}

\section*{Educational Activities}
%\medskip
%\smallskip
\subsection*{Educational Program Building}
\label{sec:eduprogram}

\begin{tabular}{@{\hspace{0.2in}}l>{\raggedright\arraybackslash}p{.82\textwidth}}
  2018--2021 \hspace{0.1in} & The NSF project (see
  \emph{\nameref{sec:funding}} on p.\pageref{sec:funding}) was
  successfully funded with a substantial STEM component for high-school students
  involving the development of both a 12-week course and an intense 2-day
  seminar called ``Swarming Powered by Neuroscience.'' I worked with our STEM
  education collaborators at JHUAPL to develop computational resources required
  for the curricula. Additionally, I participated in and delivered two zoom
  lectures about our research for the virtual 4-day STEM workshop (developed due
  to Covid requirements) with 40+ students that was held in January 2021.
\end{tabular}

\subsection*{Mentoring \& Supervision}
\label{sec:mentoring}

\begin{tabular}{@{\hspace{0.2in}}l>{\raggedright\arraybackslash}p{.82\textwidth}}
  Spring 2021 & Darius Carr, STEM high-school student; I mentored
  Darius as part of a local high school program that facilitates research
  internships for underrepresented students. I developed a computational
  research project with him that deepened his current interests in neuroscience,
  python programming, and scientific research. \\
  \tabularnewline
  2020--2022 \hspace{0.1in} & Armin Hadzic, machine learning engineer at JHUAPL;
  I supervised Armin in translating computational neuroscience models into the
  domain of reinforcement learning and Bayesian optimization to investigate
  autonomous swarming with neural control. Our project led to a first author
  peer-reviewed research publication for Armin in Array. \\
  \tabularnewline
  2019--2020 & Sreelakshmi Rajendrakumar, masters student in
  JHU/Biomedical Engineering (BME); I mentored Sreelakshmi in hippocampal
  physiology and single-unit data analysis. \\
  \tabularnewline
  2014 & Manning Zhang, M.S., graduate student in JHU/BME; I mentored Manning
  through an exchange program with Shanghai Jiao Tong University and submitted
  a letter of recommendation supporting her admission to the JHU/BME masters
  program. \\
  \tabularnewline
  2013--2015 & Chia-Hsuan Wang, Ph.D., graduate student at the JHU/MBI; I worked
  extensively with Chia-Hsuan to take over my previous studies of behavior
  and place cells in the Knierim lab, leading to a Society for Neuroscience
  conference poster in 2014. I supported her subsequent thesis research based
  on my analytics and informatics software, resulting in a paper in Current
  Biology. \\
\end{tabular}

\subsection*{Classroom Instruction}
\label{sec:classroom}

\begin{tabular}{@{\hspace{0.2in}}l>{\raggedright\arraybackslash}p{.82\textwidth}}
  Fall 2004 & Teaching Assistant for undergraduate ``Introduction to
  Neuroscience'' course, Brandeis University; I assisted Prof. Eve Marder by
  supervising classes, grading examinations, and giving review lectures.\\
  \tabularnewline
  Spring 2005 \hspace{.1in} & Teaching Assistant for undergraduate ``Biology
  Laboratory'' course, Brandeis University \\
\end{tabular}

\section*{Websites}

\begin{description}
  \item \href{https://jdmonaco.com/}
    {``\itemtitle{Briefly Balanced: Theoretical neuroscience of behavior in
    space and time}.'' Website. \aurl{https://jdmonaco.com/}}
  \item \href{https://www.ncbi.nlm.nih.gov/pubmed/?term=monaco_jd+OR+(monaco_j+AND+muzzio_ia)}
    {\itemtitle{PubMed Listing}. Website.
      \aurl{https://www.ncbi.nlm.nih.gov/pubmed/?term=monaco\_jd+OR+(monaco\_j+AND+muzzio\_ia)}}
  \item \href{https://jdmonaco.com/google-scholar}
    {\itemtitle{Google Scholar}. Website. \aurl{https://scholar.google.com/citations?hl=en\& user=gceOLZEAAAAJ\&view\_op=list\_works\&sortby=pubdate}}
  \item \href{https://github.com/jdmonaco}
    {\itemtitle{GitHub Overview}. Website. \aurl{https://github.com/jdmonaco}}
  \item \href{https://twitter.com/j_d_monaco}
    {\itemtitle{Twitter Feed}. Social Media. \aurl{https://twitter.com/j\_d\_monaco}}
\end{description}

\section*{Media Releases}

\begin{description}
  \item \href{https://kavlijhu.org/news/32} {``\itemtitle{Can
        robotic swarms navigate using learning rules devised for brain
      dynamics?}'' JHU/Kavli Neuroscience Discovery Insitute. May 3, 2020.
    \aurl{https://kavlijhu.org/news/32}}
  \item \href{https://www.youtube.com/watch?v=ic4zEgVMSsA}
    {``\itemtitle{Swarmalators}.'' JHUAPL Press Office. May 9, 2019.
    \aurl{https://www.youtube.com/watch?v=ic4zEgVMSsA}}
  \item \href{https://hub.jhu.edu/2018/10/02/brain-robot-swarms-study/}
    {``\itemtitle{What do animal brains have in common with swarms of robots?
      Maybe more than you think}.'' Geoff Brown/JHU Office of Communications. Oct 2,
    2018. \aurl{https://hub.jhu.edu/2018/10/02/brain-robot-swarms-study/}}
  \item \href{https://www.jhuapl.edu/PressRelease/181001}
    {``\itemtitle{Do Robot Swarms Work Like Brains?}'' JHUAPL Press Office. October 1, 2018.
    \aurl{https://www.jhuapl.edu/PressRelease/181001}}
  \item \href{https://hub.jhu.edu/2014/04/14/memory-brain-place-cells/}
    {``\itemtitle{Where does a memory begin? Johns Hopkins neuroscientists think they
      know}.'' Latarsha Gatlin/JHU Office of Communications. April 14, 2014.
    \aurl{https://hub.jhu.edu/2014/04/14/memory-brain-place-cells/}}
  \item \href{https://www.youtube.com/watch?v=Jm8OiLJqKJQ}
    {``\itemtitle{Johns Hopkins Researchers Probe Mysteries of
      the Brain}.'' JHU Office of Communications. April 14, 2014.
    \aurl{https://www.youtube.com/watch?v=Jm8OiLJqKJQ}}
\end{description}

\section*{Recognition \& Coverage of My Work}
\label{sec:recognition}

\subsection*{Awards \& Honors}

\lefttabline{0.8in}{2022}{IEEE/ISEC Best Paper Award, First Place}
\lefttabline{0.8in}{2003}{IEEE/IJCNN Student Paper Award, First Place}
\lefttabline{0.8in}{2002}{U.Va. John A. Harrison III Undergraduate Research Award}
\lefttabline{0.8in}{1999--2003}{U.Va. Echols Scholar}
\lefttabline{0.8in}{1999}{State of Maryland Merit Scholastic Award}
\lefttabline{0.8in}{1999}{AP Scholar with Distinction}
\lefttabline{0.8in}{1999}{National Merit Scholarship Commended Student}
\lefttabline{0.8in}{1999}{Johns Hopkins Mathematics Competition (2nd Place, Individual Calculus)}
\lefttabline{0.8in}{1999}{Maryland Distinguished Scholar}

\subsection*{News \& Views}

\begin{itemize}
  \item \href{https://dx.doi.org/10.1016/j.cub.2020.02.085}
    {Place R, Nitz DA. (2020). \itemtitle{Cognitive Maps: Distortions of the Hippocampal 
      Space Map Define Neighborhoods}. \emph{Current Biology}, 30(8): R340--R342.}
  \item \href{https://dx.doi.org/10.1016/j.cell.2018.10.047}
    {Colwell CS, Donlea J. (2018). \itemtitle{Temporal coding of sleep}. \emph{Cell}, 175(5): 1177--9.}
  \item \href{https://dx.doi.org/10.1038/nn.3700}
    {Dupret D, Csicsvari J. (2014). \itemtitle{Turning heads to remember
    places}. \emph{Nature Neuroscience}, 17(5): 643--44.}
\end{itemize}

\subsection*{Post-Publication Reviews}

\begin{itemize}
  \item \href{https://facultyopinions.com/prime/718333676#eval793494783}
    {Moser E, Rowland D. (May 12, 2014). ``\itemtitle{This exciting study finds
        an unexpected relationship between exploratory head scanning behavior
      and the development of new place fields in the rat hippocampus...}”
    \emph{F1000/Faculty Opinions}.}
  \item \href{https://facultyopinions.com/prime/718333676#eval793493493}
    {Maler L. (April 10, 2014). ``\itemtitle{This elegant and original study has
        demonstrated a strong link between the neural activity of hippocampal pyramidal
        neurons (PNs) during head scanning behavior and their subsequent acquisition of
    a new place field...}'' \emph{F1000/Faculty Opinions}.}
  \item \href{https://facultyopinions.com/prime/11553956}
    {Giocomo L, Moser E. (June 29, 2011) ``\itemtitle{This paper presents an
        interesting computational model which utilizes grid-cell modularity to generate
    robust remapping...}'' \emph{F1000/Faculty Opinions}.}
\end{itemize}

\subsection*{Other Press}
\label{sec:press}

\begin{itemize}
  \item \href{https://www.fastcompany.com/90529833/best-workplaces-for-innovators-2020 -johns-hopkins-university-apl} 
    {``\itemtitle{Johns Hopkins University APL is one of Fast Company’s Best Workplaces
      for Innovators}.'' (July 29, 2020). \emph{Fast Company}.
      \aurl{https://www.fastcompany.com/90529833/best-workplaces-for-innovators- 2020-johns-hopkins-university-apl}}
      \itemnote{My NSF project (see p.\pageref{sec:funding}) was the basis for
      \#3 ranking of JHUAPL.}
  \item \href{https://blogs.plos.org/biologue/2019/03/20/better-use-of-mouse-models-skin-infection-dynamics-and-phaser-cells-in-navigation/}
    {``\itemtitle{Better Use of Mouse Models, Skin Infection
      Dynamics, and Phaser Cells in Navigation}.'' (March 20,
      2019). \emph{PLOS Computational Biology: Biologue}.
      \aurl{https://blogs.plos.org/biologue/2019/03/20/ 
    better-use-of-mouse-models-skin-infection-dynamics-and-phaser-cells-in-navigation/}}
  \itemnote{Editor-in-Chief's selection of papers.}
  %\item \href{https://nationalsciencefoundation.tumblr.com/post/183448836933/brain-awareness-week-2019-rats-and-robots}
    %{``\itemtitle{Brain Awareness Week 2019—Rats and Robots: NSF-funded researchers
        %take a lesson from rat navigation instincts to improve algorithm[s] for
      %robots}.'' (March 14, 2019). \emph{National Science Foundation/Tumblr}.
    %\aurl{https://nationalsciencefoundation.tumblr.com/post/ 183448836933/brain-awareness-week-2019-rats-and-robots}}
  \item \href{https://www.medicaldaily.com/cognitive-map-can-show-real-time-when-memories-form-thanks-place-cells-brain-276790}
    {``\itemtitle{Cognitive Map Can Show In Real-Time When Memories Form, Thanks
      To Place Cells In The Brain}.'' (April 15, 2014). Chris Weller/Medical Daily.
    \aurl{https://www.medicaldaily.com/cognitive-map-can- show-real-time-when-memories-form-thanks-place-cells-brain-276790}}
\end{itemize}

\end{document}
