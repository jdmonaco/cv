%!TEX root = cv.tex

\newsection{Research Program Development}{research}

\smallskip
\subsection*{Patents \& Tech Development}
\label{sec:patents}

\lefttabline{0.8in}{7/5/2022}{Inventor, 
  \href{https://www.freepatentsonline.com/11378975.html}{\itemtitle{Autonomous
Navigation Technology, US patent issued, 11,378,975}}}
\lefttabline{0.8in}{1/3/2020}{Inventor, Autonomous Navigation Technology, US
patent application, 16,734,294}
\lefttabline{0.8in}{5/10/2019}{Inventor, Neuroinspired Algorithms for Swarming
Applications, provisional patent, 62/845,957}
\lefttabline{0.8in}{1/3/2019}{Inventor, Neuroinspired Algorithms for Swarming
Applications, provisional patent, 62/787,891}


\smallskip
\subsection*{Team Leadership \& Funding Development}
\label{sec:res}

\researchactivity
{April 2010/2011}
{Fellowship Proposal (NIH/NINDS F32 NRSA)}
{Behavioral Coordination of Entorhinal-Hippocampal Activity for Real-Time
Sensory Updating of Spatial Memory}
{In collaboration with my postdoctoral sponsor Jim Knierim, I conceived and
  developed a postdoctoral fellowship training proposal as a NIH F32 NRSA
  application. The proposal integrated computational modeling with spatial
  navigation experiments based on behavioral data from position-tracking sensors
  and neural data from multiregional hippocampal--entorhinal single-unit ensemble
  recordings. The application received a 21st percentile rank; I followed up the
  2010 application with a 2011 resubmission following discussions with NINDS PO
Jim Gnadt.}
\label{sec:nrsa}

\researchactivity
{Mar. 2016--2018}
{Grant Award (JHU/SLI)}
{Learning to explore paths through space}
{This internal JHU award (2016--2018; see \cfonly{funding})
  resulted from a collaboration with David J. Foster (now at UC Berkeley) that
  I initiated to conduct modeling studies informed by his lab’s hippocampal
  reactivation data. By integrating Prof.~Zhang’s mathematical theories of
  spatial cognitive maps, I wrote and submitted a proposal for a \$200K/2-year
  project to the JHU Science of Learning Institute. The proposal was awarded at
  the \$150K level and research outcomes included (1) novel theories of temporal
  synchronization coding that inspired the 2017 NSF proposal effort, and (2)
  preliminary dynamical models of sharp-wave reactivation that provided the
foundation for the 2018 NIH R03 award.}
\label{sec:sligrant}

\researchactivity
{April--June 2016}
{Grant Proposal Selectively Funded (DARPA/BTO)}
{Noninvasive Gastrovagal Stimulation for Enhanced Neuroplasticity of Cortical
and Hippocampal Networks during Cognitive Training (GEN-C)}
{In response to DARPA announcement BAA-16-24 of the “Targeted Neuroplasticity
  Training (TNT)” program, I worked with colleagues from JHUAPL and JHU/SoM
  Center for Neurogastroenterology to develop a collaborative program involving 3
  PIs and 5 co-Is (8 labs) across divisions, departments, and fields. I recruited
  experimental labs from JHU/MBI and coordinated proposed contributions to
  maximize scientific impact with a budget of \$9.8M/5 years. I coordinated the
  40-page research narrative, including writing, editing, and/or integrating
  each lab’s contributions and worked with ORA to submit the proposal. While
  not funded in total, DARPA/BTO PM Doug Weber funded select components, leading
  to JHUAPL Work Agreement No.~145563 “BCI (Brain Computer Interface)
Technologies” in 2018.}
%Technologies” in 2018 for \$24,604 to the lab of Prof.~Pasricha.}

\researchactivity
{Nov. 2017--2021}
{Grant Award (NSF/NCS)}
{NCS-FO: Spatial intelligence for swarms based on hippocampal dynamics}
{This NSF-awarded project (2018--2021; see \cfonly{funding}) was the result of
  6 months of collaboration, brain-storming, and team-building between the Zhang
  lab at JHU/SOM and a group of JHUAPL engineers, mathematicians, and scientists.
  The project was initially inspired by results that I presented at my Society
  for Neuroscience 2017 meeting poster. I wrote Aim 1 and integrated the full
  research narrative with inputs from our collaborators for the proposal of
  this \$997K/2-year project to develop those initial ideas into technological
  applications (e.g., robotics, autonomous control, AI) that reciprocally inform
  neuroscience. The project led to multiple publications, a patent, and a NIH
BRAIN Investigators Meeting symposium talk.}
\label{sec:nsfgrant}

\researchactivity
{Jan. 2018--2020}
{Grant Award (NIH/NINDS)}
{Spiking network models of sharp-wave ripple sequences with gamma-locked
attractor dynamics}
{To continue with the collaboration that I initiated with David J. Foster (UC
  Berkeley) on the basis of the internal SLI award (see above), I wrote a small
  modeling proposal that integrated preliminary results from the SLI project and
  recent research developments in the memory reactivation field. This proposal
  was awarded (2018--2020; see \cfonly{funding}) through the NIH/NINDS R03
  mechanism and I am currently utilizing this support to build a foundation for
future efforts along this research track.}
\label{sec:nihgrant}

\whitepaper
{Feb.--Mar. 2018}
{Schultz K, Zhang K, and \joehl{Monaco J}}
{BrainSWARRMM: Brain-like Sharp-Waves for Autonomous Replanning \&
Reconnaissance on Matrix Manifolds}
{In response to the Office of Naval Research (ONR) Special Notice
  N00014-18-R-SN05, Topic 3, I helped organize a series of collaborative meetings
  to design a \$2M/4-year project between JHUAPL and JHU/SoM. I co-authored the
resulting white paper that was submitted for consideration to ONR.}

\whitepaper
{May--June 2018}
{Zhang K, \joehl{Monaco JD}, Hwang GM, Schultz KM, Kobilarov M, Foster DJ,
Jacobs J, and Itti L}
{An Integrative Theoretical Framework of the Neural Self-Organization of Active
Perception for Autonomous Spatial Navigation}
{In response to ONR MURI Announcement N00014-18-S-F006 and with the help of
  JHUAPL, I coordinated a series of meetings with 5 PIs across 4 universities
  (Columbia, UC Berkeley, USC, JHU) to design an innovative research program that
  targeted reciprocal advances in experimental \& theoretical neuroscience and
  robotics \& AI across species and scales. The resulting \$7.5M/5-year project
  that I outlined in the white paper was not invited for a full submission. We
  debriefed with the sponsor, ONR PM Marc Steinberg, who revealed that ONR was
  impressed with the project but that they were seeking a different balance of
elements with respect to neuroscience and AI.}

\whitepaper
{August 2019}
{\joehl{Monaco J}, Zhang K, and Schultz K.}
{SW2Mem: Graph Spectral Decoding of Hippocampal-Cortical Loops for Artificial
Consolidation and Dreaming}
{In response to ONR Special Notice N00014-19-S-SN08, Topic 5.1 I conceived this
  project, created the preliminary model and datasets, guided the preliminary
  analyses with JHUAPL collaborators, and wrote \& submitted the white paper to
  ONR outlining a potential \$1.05M/3-year project. ONR declined to invite us to
submit a full proposal.}

%\whitepaper
%{August 2019}
%{Schultz K, Agarwala S, Zhang K, and \joehl{Monaco J}}
%{Brain-like Distributed Surveillance using Heterogeneous Agents for integRated
%Perception, and Planning (BD-SHARPP)}
%{In response to ONR Special Notice N00014-18-R-SN05, Topic 3, we submitted a
  %revised version of the March 2018 white paper that was specifically invited by
%ONR PM Tom McKenna.}

%\researchactivity
%{Sept. 11, 2019}
%{NSF Project Review}
%{Annual advisory board review symposium}
%{I delivered a seminar on Aim 1 progress at a JHUAPL-hosted symposium for our
  %project’s yearly review, attended by DARPA/I2O PM Hava Siegelmann and other
%outside experts.}

\researchactivity
{Feb. 26, 2020}
{Grant Proposal (NSF/NCS) }
{NCS-FO: Neuroeconomics as a biomimetic control theory for mobile robotic
decision making}
{This FY21 proposal was submitted to the NSF/NCS FOUNDATIONS program; while
  it was discussed and received high scores, the application was declined. I
  co-developed this project in collaboration with colleagues at the University of
  Pittsburgh Medical Center (UPMC), JHU Whiting School of Engineering (JHU/WSE),
  and JHUAPL. Our interdisciplinary project brought together multiscale human
  electrophysiological recordings (UPMC), latent state-space models (JHU/WSE),
  control- and game-theoretic analysis (JHUAPL), and mechanistic neural models
  (JHU/BME, for which I would have been co-PI). We proposed to investigate and
  characterize the neural bases of metacognitive brain states that influence
  decision-making during social \& economic games. As a high-risk/high-reward
  element, we proposed to algorithmicize our results to advance human-robot
interaction.}

\researchactivity
{Jan. 14, 2022}
{Grant Proposal (JHU/Discovery Award) }
{Algorithms of flexible navigation in mice and robots}
{This intramural FY23 proposal for a JHU Discovery Award resulted from a new
  collaboration with Patricia Janak (PI; JHU/PBS) and Céline Drieu (postdoctoral
  fellow), in which we seek to integrate advanced large-scale neural recording
  technologies with my theoretical modeling of neural systems as a distributed
  control problem. Fundamental questions of neural systems communication will
  be addressed using convergent data-driven and theory-driven approaches
  to understanding the cognitive dynamics that enable mice to perform spatial
goal-directed memory tasks.}


