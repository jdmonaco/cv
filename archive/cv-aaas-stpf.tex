%! TEX program = xelatex
%
% Curriculum vitae for Joseph D. Monaco
%
\documentclass[10pt]{article}

\usepackage{fullpage}
\usepackage[usenames,dvipsnames]{color}
\usepackage[hidelinks,xetex]{hyperref}
\usepackage{multirow}
\usepackage{sectsty}
\usepackage{tabu}
\usepackage{longtable}
\usepackage{soul}
\usepackage{fontspec}

% Page geometry formatting
\raggedbottom
\raggedright
\textheight=9in
\setlength{\tabcolsep}{0in}
\addtolength{\footskip}{-.2in}
\addtolength{\voffset}{-.6in}
\addtolength{\textheight}{1in}
\addtolength{\hoffset}{-.25in}
\addtolength{\textwidth}{.25in}

% Define colors
\definecolor{hopkinsblue}{RGB}{0,48,130}
\definecolor{lightblue}{rgb}{.88,.92,.97}
\definecolor{lightgold}{rgb}{1,.80,0}
\definecolor{lightred}{rgb}{1,.3,.2}
\definecolor{lightgray}{rgb}{.80,.80,.80}
\definecolor{dimgray}{rgb}{.50,.50,.50}
\definecolor{darkgray}{rgb}{.30,.30,.30}

% Set up highlighting and underlining
\sethlcolor{lightblue}
\setul{0.13ex}{}

% Choose font package
\setmainfont[Ligatures=TeX]{Helvetica Neue LT Std} 

% Section section* formatting
\sectionfont{\large\color{hopkinsblue}\bfseries}
\subsectionfont{\vspace{-1.5ex}\color{hopkinsblue}\normalsize\bfseries}

% Formatting macros
\newcommand{\itemtitle}[1]{{\color{hopkinsblue}\ul{#1}}}
\newcommand{\unpubtitle}[1]{{\color{hopkinsblue} #1}}
\newcommand{\itemnote}[1]{
  \begin{description}
    \item[$\rightarrow$] \hspace{.09in}{\color{darkgray}\it #1}
  \end{description}
}
\newcommand{\joehl}[1]{\hl{\textbf{#1}}}
\newcommand{\doi}[1]{{\color{darkgray}doi:}~{\color{dimgray}\texttt{#1}}}
\newcommand{\arxiv}[1]{\emph{ArXiv Preprint}.
  {\color{darkgray}arxiv:}{\color{dimgray}\texttt{#1}}}
\newcommand{\aurl}[1]{{\color{dimgray}\texttt{#1}}}
\newcommand{\researchnote}[1]{
  \begin{description}
    \item[] {\hspace{2.2ex}\color{darkgray} #1}
  \end{description}
}
\newcommand{\researchactivity}[4]{
  \begin{minipage}[t]{\textwidth}
    \begin{tabular}{@{\hspace{2ex}}l>{\raggedright\arraybackslash}p{.8\textwidth}}
      \makebox[1.2in][l]{#1} & \textbf{#2:}
      ``\unpubtitle{#3}'' 
    \end{tabular}
  \researchnote{\hspace{1ex} #4}
  \end{minipage}
  \medbreak
}
\newcommand{\whitepaper}[4]{
  \begin{minipage}[t]{\textwidth}
    \begin{tabular}{@{\hspace{2ex}}l>{\raggedright\arraybackslash}p{.8\textwidth}}
      \makebox[1.2in][l]{#1} & \textbf{White Paper:} #2.
      ``\unpubtitle{#3}'' 
    \end{tabular}
  \researchnote{\hspace{1ex} #4}
  \end{minipage}
  \medbreak
}
\newcommand{\lefttabline}[3]{\hspace{2ex}\makebox[#1][l]{#2} #3\\}

% PDF document info and setup
\hypersetup{
  baseurl=https://jdmonaco.com/cv-monaco.pdf,
  pdftitle=Curriculum Vitae for Joseph D. Monaco,
  pdfauthor=Joseph D. Monaco,
  pdfdisplaydoctitle=true,
  pdfpagemode=UseThumbs,
  pdfstartview=FitV,
  pdfpagelayout=TwoColumnLeft,
  pdftoolbar=false,
  pdfwindowui=false,
  pdfcenterwindow=true,
}

% "See section ..." macro
\newcommand{\see}[1]{[\textcolor{hopkinsblue}{p.\pageref{sec:#1}}]}
\newcommand{\cfonly}[1]{\textcolor{hopkinsblue}{\emph{\nameref{sec:#1}} on p.\pageref{sec:#1}}}
\newcommand{\cf}[1]{\textcolor{hopkinsblue}{See \emph{\nameref{sec:#1}} on p.\pageref{sec:#1}}}
\newcommand{\cfcf}[2]{\textcolor{hopkinsblue}{See \emph{\nameref{sec:#1}} on
  p.\pageref{sec:#1} and \emph{\nameref{sec:#2}} on p.\pageref{sec:#2}}}

% Typesetting
\setlength{\parskip}{0em}
\newcommand{\newsection}[1]{%
  \section*{#1}
  \vspace{-.125in}
  \hrule
  \vspace{.25in}
  \label{sec:#1}
}

\begin{document}

\begin{center}
  \textbf{\LARGE\color{hopkinsblue} Joseph D. Monaco, Ph.D.} \\[0.1in]
  1028 Pier Pointe Ldng \\
  Baltimore, Maryland 21230 United States \\[2mm]
  \fontspec{Verdana}\small
  \textsc{Mobile:} \href{tel:16172725668}{\color{hopkinsblue}\texttt{617.272.5668}} •
  \textsc{Email:} \href{mailto:joe@selfmotion.net}{\color{hopkinsblue}\texttt{joe@selfmotion.net}} • 
  \textsc{Web:} \href{https://jdmonaco.com/}{\color{hopkinsblue}\texttt{jdmonaco.com}} \\  
  \textsc{ORCID:} \href{https://jdmonaco.com/orcid}{\color{hopkinsblue}\texttt{0000-0003-0792-8322}} • 
  \textsc{GitHub:} \href{https://jdmonaco.com/github}{\color{hopkinsblue}\texttt{github.com/jdmonaco}} • 
  \textsc{Google Scholar:} \href{https://jdmonaco.com/google-scholar}{\color{hopkinsblue}\texttt{gceOLZEAAAAJ}} 
\end{center}

\newsection{Education}

\begin{itemize}
  \item
    \begin{tabular*}{6.3in}{l@{\extracolsep{\fill}}r}
      \textbf{Columbia University} & New York, NY \\
      Department of Neurobiology \& Behavior & 2005--2009 \\
      Center for Theoretical Neuroscience\\
      Degrees: Ph.D. (2009); M.Phil. (2008); M.A. (2006) \\
      Advisor: Larry~Abbott \\
    \end{tabular*}
  \item
    \begin{tabular*}{6.3in}{l@{\extracolsep{\fill}}r}
      \textbf{Brandeis University} & Waltham, MA \\
      Department of Biology & 2003--2005\\
      Volen Center for Complex Systems\\
      Graduate Program in Neuroscience, \textit{\ul{Continued at Columbia University}} \\
      Advisors: Michael Kahana; Larry~Abbott \\
    \end{tabular*}
  \item
    \begin{tabular*}{6.3in}{l@{\extracolsep{\fill}}r}
      \textbf{University of Virginia} & Charlottesville, VA \\
      Laboratory of Computational Neurodynamics & 1999--2003\\
      Degrees: B.A.~Mathematics; B.A.~Cognitive Science; Minor in Philosophy \\
      Advisor: William (Chip) Levy\\
      Echols Scholar \\
    \end{tabular*}
\end{itemize}


\newsection{Work Experience}

\section{JHU/SOM Research Associate }
\label{sec:job1}

\begin{tabular*}{6.3in}{l@{\extracolsep{\fill}}r}
  \textbf{Research Associate (Faculty)} & 733 North Broadway \\
  \textbf{Johns Hopkins University School of Medicine} & Edward D. Miller Research Building \\
  Department of Biomedical Engineering & Baltimore, MD \\
\end{tabular*}
\\[.1in]
\textbf{7/2019 -- 6/2022 \\ Full-Time Equivalent, 40--80 Hours/Week} \\


\subsection*{Job Duties, Related Skills, and Responsibilities}

\quad •~Initiated research collaborations and continued grant application efforts
•~Developed computational models of autonomous neural control
•~Presented research findings in conference presentations, invited talks, and published articles
•~Delivered invited talks at NIH BRAIN Investigators Meeting symposium and the Air Force Research Lab
•~Interpreted research about artificial intelligence, swarm cognition, and neuroethology
•~Supervised high-school and masters students in computational methods
•~Directed research conduct, budget, and administration as co-PI of NSF-awarded project
•~Primary and senior authorships on peer-reviewed papers on neural control systems
•~Responsible for most operational aspects of my sponsor lab (3 years)
•~Responsible for multiple scientific projects and grant application efforts
•~Served as reviewer for top journals, conferences, and funding panels
•~Demonstrated extensive scientific expertise and leadership in funding panels, invited talks, and grants 
•~Coordinated teaching activities to support STEM component of NSF project

%\vspace{.2in}
%\hrule
\section{JHU/SOM Postdoctoral Fellow}
\label{sec:job2}

\begin{tabular*}{6.3in}{l@{\extracolsep{\fill}}r}
  \textbf{Postdoctoral Fellow} & 733 North Broadway \\
  \textbf{Johns Hopkins University School of Medicine} & Edward D. Miller Research Building \\
  Department of Biomedical Engineering & Baltimore, MD \\
\end{tabular*}
\\[.1in]
\textbf{8/2014 -- 6/2019 \\ Full-Time Equivalent, 40--80 Hours/Week} \\


\subsection*{Job Duties, Related Skills, and Responsibilities}

\quad •~Independently developed research collaborations and organized grant efforts 
•~Developed analysis protocols and conducted computational modeling studies 
•~Presented scientific research findings at conferences and in published papers 
•~Supervised an exchange student in masters program
•~Principal investigator for my internal JHU project
•~First and second authorships on peer-reviewed research articles
•~Responsible for operating aspects of my sponsor lab by conducting multiple projects and grant efforts 
•~Demonstrated research-driven leadership to coordinate collaborations
•~Served as peer reviewer for research journals
•~Directed research activities to coordinate preliminary work for proposals


%\vspace{.2in}
%\hrule
\section{JHU/MBI Postdoctoral Fellow}
\label{sec:job3}

\begin{tabular*}{6.3in}{l@{\extracolsep{\fill}}r}
  \textbf{Postdoctoral Fellow} & 3400 N. Charles Street \\
  \textbf{Johns Hopkins University} & Krieger Hall \\
  Zanvyl Krieger Mind/Brain Institute & Baltimore, MD \\
\end{tabular*}
\\[.1in]
\textbf{7/2009 -- 7/2014 \\ Full-Time Equivalent, 40--80 Hours/Week} \\

\subsection*{Job Duties, Related Skills, and Responsibilities}

\quad •~Initiated computational neuroscience research projects to investigate the oscillatory interference theory of temporal coding for path integration
•~Developed protocols and analysis pipelines for quantifying a particular investigatory behavior in rats during spatial navigation tasks
•~Conducted research studies using neuroinformatics and neurobehavioral data analysis to discover a behavioral basis of memory formation
•~Documented analysis and modeling findings in my lab notebooks, lab meeting presentations, scientific conferences, and published research articles
•~Interpreted published research results in the fields of neural coding, pulse-coupled networks, behavioral ethology, and place cell physiology
•~Independently developed distinct research projects based on detailed quantification of behavior in experiments and abstract theoretical models of neural coding
•~Primary authorship of two peer-reviewed journal articles
•~Responsible for conducting nonoverlapping modeling and analysis projects over the same timeframe
•~Supervised a graduate student research assistant who learned to use my analysis protocols and pipelines for their thesis work
•~Presented results from my thesis work at a scientific conference
•~Served as a peer reviewer for several research journals
•~Directed the research activities of my student mentee for their thesis work


%\vspace{.2in}
%\hrule
\section{Columbia Graduate Research Assistant}
\label{sec:job4}

\begin{tabular*}{6.3in}{l@{\extracolsep{\fill}}r}
  \textbf{Graduate Research Assistant} & 3227 Broadway \\
  \textbf{Columbia University} & Jerome L. Greene Science Center \\
  Center for Theoretical Neuroscience & New York, NY \\
\end{tabular*}
\\[.1in]
\textbf{8/2005 -- 6/2009 \\ Full-Time Equivalent, 40--80 Hours/Week} \\


\subsection*{Job Duties, Related Skills, and Responsibilities}

\quad •~Initiated multiple interrelated subprojects contributing to my doctoral thesis
•~Developed a series of protocols for quantifying hippocampal remapping in random foraging experiments to support and strengthen computational modeling results
•~Conducted computational modeling studies of spatial navigation and the neural coding of space in the hippocampus and entorhinal cortex of rodents
•~Documented modeling and data analysis findings in lab notebooks, presentations, my doctoral thesis, and two peer-reviewed publications
•~Interpreted research literatures of experimental, theoretical, and computational approaches to investigating spatial memory and hippocampal function
•~Independently developed theoretical and computational modeling research projects toward the completion of my doctoral studies
•~Primary authorship of an original research article describing the main results of my thesis work that was published in a peer-reviewed journal


\pagebreak
\newsection{Publications}

\renewcommand{\itemnote}[1]{}

% Insert the text for all the publication lists:
\section*{Journal Publications}

\begin{description}
  \item Buckley E, \joehl{Monaco JD}, Schultz KM, Chalmers R, Hadzic A,
    Zhang K, Hwang GM, and Carr MD. (\emph{\color{lightred}Under review}). \itemtitle{An
      interdisciplinary approach to high school curriculum development: Swarming
    Powered by Neuroscience}. \emph{Frontiers in Education}.
  \item \joehl{Monaco JD}, Rajan K, and Hwang GM. (\emph{\color{lightred}In revision}).
      \itemtitle{A brain basis of dynamical intelligence for AI and computational
    neuroscience}. \emph{Nature Machine Intelligence}.
  \item \href{https://doi.org/10.1007/s00422-020-00823-z}
    {\joehl{Monaco JD}, Hwang GM, Schultz KM, and Zhang K. (2020).
    \itemtitle{Cognitive swarming in complex environments with attractor
      dynamics and oscillatory computing}. \emph{Biological Cybernetics}, 114,
    269--284. \doi{10.1007/s00422-020-00823-z}}.
  \item \begin{samepage}\href{https://doi.org/10.1016/j.cub.2020.01.083} 
      {Wang CH, \joehl{Monaco JD}, and Knierim JJ. (2020). \itemtitle{Hippocampal
        place cells encode local surface texture boundaries}. \emph{Current Biology},
      30, 1--13. \doi{10.1016/j.cub.2020.01.083}}.
  \itemnote{I mentored the first author in data analysis of rat behavior and
      single-unit recordings, developed the software toolchain used to conduct the
    analyses, and provided intellectual guidance.}\end{samepage}
  \item \href{https://doi.org/10.1371/journal.pcbi.1006741}
    {\joehl{Monaco JD}, De Guzman RM, Blair HT, and Zhang K. (2019).
    \itemtitle{Spatial synchronization codes from coupled rate-phase
      neurons}. \emph{PLOS Computational Biology}, 15(1), e1006741.
    \doi{10.1371/journal.pcbi.1006741}}.
  \item \href{https://www.cell.com/cell/fulltext/S0092-8674(18)31228-5}
    {Tabuchi M, \joehl{Monaco JD}, Duan G, Bell BJ, Liu S, Zhang K, and
      Wu MN. (2018). \itemtitle{Clock-generated temporal codes determine
      synaptic plasticity to control sleep}. \emph{Cell}, 175(5), 1213--27.
    \doi{10.1016/j.cell.2018.09.016}}.
  \itemnote{I developed two modeling strategies for the Wu lab’s circadian clock
      neuron experiments in \emph{Drosophila}. My generative statistical model was
      integrated into stimulation protocols as a timing control for behavioral
      results, and my mechanistic molecular/neuronal model explained observed trends
      and made predictions corroborated by the data. My results or contributions are
    featured in 3/7 main figures and 3/6 supplementary figures.}
  \item \href{http://doi.org/10.1038/nn.3687}
    {\joehl{Monaco JD}, Rao G, Roth ED, and Knierim JJ. (2014).
    \itemtitle{Attentive scanning behavior drives one-trial potentiation of
      hippocampal place fields}. \emph{Nature Neuroscience}, 17(5), 725--731.
    \doi{10.1038/nn.3687}}.
  \item \href{http://doi.org/10.3389/fncom.2011.00039}
    {\joehl{Monaco JD}, Knierim JJ, and Zhang K. (2011). \itemtitle{Sensory
        feedback, error correction, and remapping in a multiple oscillator model of
      place cell activity}. \emph{Frontiers in Computational Neuroscience}, 5:39.
    \doi{10.3389/fncom.2011.00039}}.
  \item \href{http://doi.org/10.1523/JNEUROSCI.1433-11.2011}
    {\joehl{Monaco JD} and Abbott LF. (2011). \itemtitle{Modular
        realignment of entorhinal grid cell activity as a basis for hippocampal
      remapping}. \emph{Journal of Neuroscience}, 31(25), 9414--25.
    \doi{10.1523/jneurosci.1433-11.2011}}.
  \item \href{http://doi.org/10.1371/journal.pbio.1000140}
    {Muzzio IA, Levita L, Kulkarni J, \joehl{Monaco J}, Kentros CG, Stead
      M, Abbott LF, and Kandel ER. (2009). \itemtitle{Attention enhances the
        retrieval and stability of visuospatial and olfactory representations
      in the dorsal hippocampus}. \emph{PLOS Biology}, 7(6), e1000140.
    \doi{10.1371/journal.pbio.1000140}}.
  \itemnote{I contributed oscillatory power analyses and group-level statistical
      analyses of spiking and bursting for odor vs.\ visuospatial tasks in
    single-unit hippocampal recordings from freely-moving mice.}
  \item \href{http://doi.org/10.1101/lm.363207}
    {\joehl{Monaco JD}, Abbott LF, and Kahana MJ. (2007).
    \itemtitle{Lexico-semantic structure and the recognition
      word-frequency effect}. \emph{Learning \& Memory}, 14(3), 204--213.
    \doi{10.1101/lm.363207}}.
\end{description}

\section*{Conference Papers}

\begin{description}
  \item \href{https://www.jhuapl.edu/Content/techdigest/pdf/V35-N04/35-04-Hwang.pdf}
    {Hwang GM, Schultz KM, \joehl{Monaco JD}, and Zhang K. (2021).
    \itemtitle{Neuro-Inspired Dynamic Replanning in Swarms—Theoretical
        Neuroscience Extends Swarming in Complex Environments}. \emph{Johns Hopkins
    APL Technical Digest}, 35, 443--447.}
  \item \href{https://doi.org/10.1117/12.2518966}
    {\joehl{Monaco JD}, Hwang GM, Schultz KM, and Zhang K. (2019).
    \itemtitle{Cognitive swarming: An approach from the theoretical neuroscience
        of hippocampal function}. \emph{Proceedings of SPIE (International society
      for optics and photonics) Defense \& Commercial Sensing}. Micro- and
      Nanotechnology Sensors, Systems, and Applications XI, 109822D, 1--10.
    \doi{10.1117/12.2518966}}.
  \item \href{http://doi.org/10.1109/IJCNN.2003.1223655}
    {\joehl{Monaco JD} and Levy WB. (2003). \itemtitle{T-maze training
        of a recurrent CA3 model reveals the necessity of novelty-based
        modulation of LTP in hippocampal region CA3}. \emph{Proceedings of
      International Joint Conference on Neural Networks}, 1655--1660.
    \doi{10.1109/IJCNN.2003.1223655}}.
  \itemnote{This paper received First Place in the IJCNN Student Paper
    Competition.}
\end{description}

\section*{Preprints}

\begin{description}
  \item \href{https://arxiv.org/abs/2109.05545}{Buckley E, \joehl{Monaco JD},
      Schultz KM, Chalmers R, Hadzic A, Zhang K, Hwang GM, and Carr MD. (2021).
    \itemtitle{An interdisciplinary approach to high school curriculum development:
    Swarming Powered by Neuroscience}. \arxiv{2109.05545}}.
  \item \href{https://arxiv.org/abs/2105.07284}
    {\joehl{Monaco JD}, Rajan K, and Hwang GM. (2021). \itemtitle{A brain
      basis of dynamical intelligence for AI and computational neuroscience}.
    \arxiv{2105.07284}}.
  \item \href{https://arxiv.org/abs/2003.13825}
    {Levenstein D, Alvarez VA, Amarasingham A, Azab H, Gerkin RC, Hasenstaub
      A, Iyer R, Jolivet RB, Marzen~S, \joehl{Monaco JD}, Prinz AA, Quraishi
      S, Santamaria F, Shivkumar S, Singh MF, Stockton DB, Traub R, Rotstein
      HG, Nadim F, and Redish AD. (2020). \itemtitle{On the role of theory and
    modeling in neuroscience}. \arxiv{2003.13825}}.
  \item \href{https://arxiv.org/abs/1909.06711}
    {\joehl{Monaco JD}, Hwang GM, Schultz KM, and Zhang K. (2019).
    \itemtitle{Cognitive swarming in complex environments with attractor
    dynamics and oscillatory computing}. \arxiv{1909.06711}}.
  \item \href{http://doi.org/10.1101/764282}
    {Wang CH, \joehl{Monaco JD}, and Knierim JJ. (2019). \itemtitle{Hippocampal
      place cells encode local surface texture boundaries}. \emph{bioRxiv}.
    \doi{10.1101/764282}}.
  \item \href{http://dx.doi.org/10.1101/211458}
    {\joehl{Monaco JD}, Blair HT, and Zhang K. (2017). \itemtitle{Spatial theta
        cells in competitive burst synchronization networks: Reference frames from
    phase codes}. \emph{bioRxiv}. \doi{10.1101/211458}}.
\end{description}

\section*{Thesis}

\begin{description}
  \item \href{http://search.proquest.com/docview/304862872/abstract}
    {\joehl{Monaco JD}. (2009). \itemtitle{Models and mechanisms for integrating
      cortical feature spaces}. Doctoral Dissertation, Columbia University, New
    York. \emph{ProQuest Publication No. AAT 3393609}}.
  \itemnote{\href{https://jdmonaco.com/files/monaco-phdthesis-2009.pdf}{Click
    here for the original version with high-quality color figures.}}
\end{description}



\renewcommand{\itemnote}[1]{
  \begin{description}
    \item[$\rightarrow$] \hspace{.09in}{\color{darkgray}\it #1}
  \end{description}
}


\newsection{Recent Talks}

\vspace{-.13in}
\begin{longtable}{@{\hspace{0.2in}}l>{\raggedright\arraybackslash}p{.82\textwidth}}
  8/26/2022 \hspace{0.3in} &
  \href{https://jdmonaco.com/files/monaco-2022-afrl-quest-slides.pdf}
  {``\unpubtitle{Brain oscillations: From cortical computing to the existential
    nonduality of conscious agents}.'' \emph{Invited Public Seminar}. Qualia
    Exploitation for Sensor Technology (QuEST), Air Force Research Lab/Autonomous
  Capabilities Team 3 (AFRL/ACT3), Online. \itemtitle{[PDF]}}\\
  2/01/2022 \hspace{0.2in} & ``\unpubtitle{Theory-Driven Data Science to
    Understand the Neural Dynamics of Memory and Behavior}.'' \emph{Invited Talk}.
    Department of Cell \& Systems Biology, University of Toronto, Canada, Online \\
  12/01/2021 \hspace{0.2in} &
  \href{https://youtu.be/3mKkLksOyfk}{``\unpubtitle{Learning as swarming:
      Cognitive flexibility from the neural dynamics of phase-coupled attractor
    maps}.'' \emph{Contributed Talk}. Neuromatch 4.0 Conference, Online.
  \itemtitle{[YouTube]}}\\
  10/29/2020 \hspace{0.2in} &
  \href{https://www.youtube.com/watch?v=WwYDMpD7j4Q}{``\unpubtitle{Spatial
      theta-phase coding in the lateral septum: a theory of allocentric feedback
    during navigation}.'' \emph{Contributed Talk}. Neuromatch 3.0 Conference,
  Online. \itemtitle{[YouTube]}}\\
  10/7/2020 \hspace{0.2in} & ``\unpubtitle{Computing path integration with
  oscillatory phase codes in biological and artificial systems}.'' \emph{Data
  Blitz}. iNAV Symposium 2020, Online\\
  6/1/2020 \hspace{0.3in} & \label{sec:symposium}
  \href{https://youtu.be/2jy1ENYHRAw?t=902}{``\unpubtitle{Can Transitory
    Neurodynamics Unify Learning Theories for Brains and Machines?}''
    \emph{Invited Lecture \& Panel Discussion}. Symposium on ``How Can Dynamical
    Systems Neuroscience Reciprocally Advance Machine Learning?'', 6th Annual
  BRAIN Initiative Investigators Meeting, NIH, Online. \itemtitle{[YouTube]}}\\
  5/18/2020 \hspace{0.3in} & ``\unpubtitle{Computational Approaches to the
  Neural Dynamics of Time, Memory, and Behavior}.'' \emph{Invited Lecture}.
  Department of Neuroscience, Medical Discovery Team for Optical Imaging,
  University of Minnesota, Online\\
\end{longtable}


%\section*{References}
%\label{sec:references}
%\vspace{-.1in}
%\hrule
%\vspace{.3in}

% Insert the text for all the references:
%\input{../../jobsearch/package/refcontacts/references_nih}


\newsection{Funding Awards} 

\begin{itemize}
  \item \href{https://www.nsf.gov/awardsearch/showAward?AWD_ID=1835279&HistoricalAwards=false}
    {\itemtitle{NCS-FO: Spatial intelligence for swarms based on hippocampal
    dynamics}}\hspace{\stretch{1}}2018--2021
    \begin{itemize}
      \item NSF\slash NCS FOUNDATIONS (BRAIN Initiative) Award No.~1835279: \$862K/\$997K (Direct/Total)
        %\item \textbf{Lead PI:} Kechen Zhang
        %\item \textbf{Co-PIs, JHUAPL}: Grace Hwang, Robert W. Chalmers, Kevin
        %Schultz, and M. Dwight Carr
        %\item \textbf{Research Associate (FY19)/Co-PI (FY20--FY21): \joehl{Joseph D. Monaco}}
    \end{itemize}
    \itemnote{I co-developed this project and co-wrote the proposal with a JHUAPL
      colleague. As a Research Associate faculty at JHU as of FY20, my project role
    was promoted to co-PI.}
\end{itemize}

\begin{itemize}
  \item \href{https://projectreporter.nih.gov/project_info_description.cfm?aid=9652210&icde=42555668&ddparam=&ddvalue=&ddsub=&cr=2&csb=default&cs=ASC&pball=}
    {\itemtitle{Spiking network models of sharp-wave ripple sequences with\\
    gamma-locked attractor dynamics}}\hspace{\stretch{1}}2018--2020
    \begin{itemize}
      \item NIH/NINDS R03 Award No.~NS109923: \$50K/\$82K (Direct/Total)
        %\item \textbf{PI:} Kechen Zhang
        %\item \textbf{Research Associate:} \joehl{Joseph D. Monaco}
    \end{itemize}
    \itemnote{I conceived this project, generated preliminary data, and wrote the
    proposal. As a Postdoctoral Fellow, JHU policy precluded a PI role.}
\end{itemize}

\begin{itemize}
  \item \href{https://scienceoflearning.jhu.edu/research/learning-to-explore-paths-through-space/}
    {\itemtitle{Learning to explore paths through space}}\hspace{\stretch{1}}2016--2018
    \begin{itemize}
      \item JHU/Science of Learning Institute (SLI) Award: \$150K
      %\item \textbf{PI:} Kechen Zhang
      %\item \textbf{Co-PI:} David J.~Foster (now at UC Berkeley)
        %\item \textbf{Research Associate:} \joehl{Joseph D.~Monaco}
    \end{itemize}
    \itemnote{I conceived this project, initiated the collaboration between the
      Zhang and Foster labs, and wrote the proposal. As a Postdoctoral Fellow, JHU
    policy precluded a PI role.}
    \label{sec:previnternalsupport}
\end{itemize}

%\section*{Research Program --- Development \& Administration}

%\subsection*{Research Program Building \& Leadership}
%\label{sec:resprogram}

%\researchactivity
%{April 2010/2011}
%{Fellowship Proposal (NIH/NINDS F32 NRSA)}
%{Behavioral Coordination of Entorhinal-Hippocampal Activity for Real-Time
%Sensory Updating of Spatial Memory}
%{In collaboration with my posdoctoral sponsor Jim Knierim, I conceived and
  %developed a postdoctoral fellowship training proposal as a NIH F32 NRSA
  %application. The proposal integrated computational modeling with spatial
  %navigation experiments based on behavioral data from position-tracking sensors
  %and neural data from multiregional hippocampal--entorhinal single-unit ensemble
  %recordings. The application received a 21st percentile rank; I followed up the
  %2010 application with a 2011 resubmission following discussions with NINDS PO
%Jim Gnadt.}
%\label{sec:nrsa}

%\researchactivity
%{Mar. 2016--2018}
%{Grant Award (JHU/SLI)}
%{Learning to explore paths through space}
%{This internal JHU award (2016--2018; see
  %\emph{\nameref{sec:previnternalsupport}} on p.\pageref{sec:previnternalsupport})
  %resulted from a collaboration with David J. Foster (now at UC Berkeley) that
  %I initiated to conduct modeling studies informed by his lab’s hippocampal
  %reactivation data. By integrating Prof.~Zhang’s mathematical theories of
  %spatial cognitive maps, I wrote and submitted a proposal for a \$200K/2-year
  %project to the JHU Science of Learning Institute. The proposal was awarded at
  %the \$150K level and research outcomes included (1) novel theories of temporal
  %synchronization coding that inspired the 2017 NSF proposal effort, and (2)
  %preliminary dynamical models of sharp-wave reactivation that provided the
%foundation for the 2018 NIH R03 award.}
%\label{sec:sli}

%\researchactivity
%{April--June 2016}
%{Grant Proposal (DARPA/BTO)}
%{Noninvasive Gastrovagal Stimulation for Enhanced Neuroplasticity of Cortical
%and Hippocampal Networks during Cognitive Training (GEN-C)}
%{In response to DARPA announcement BAA-16-24 of the “Targeted Neuroplasticity
  %Training (TNT)” program, I worked with colleagues from JHUAPL and JHU/SOM
  %Center for Neurogastroenterology to develop a collaborative program involving 3
  %PIs and 5 co-Is (8 labs) across divisions, departments, and fields. I recruited
  %experimental labs from JHU/MBI and coordinated proposed contributions to
  %maximize scientific impact with a budget of \$9.8M/5 years. I coordinated the
  %40-page research narrative, including writing, editing, and/or integrating
  %each lab’s contributions and worked with ORA to submit the proposal. While
  %not funded in total, DARPA/BTO PM Doug Weber funded select components, leading
  %to JHUAPL Work Agreement No.~145563 “BCI (Brain Computer Interface)
%Technologies” in 2018.}
%\label{sec:genc}
%%Technologies” in 2018 for \$24,604 to the lab of Prof.~Pasricha.}

%\researchactivity
%{Nov. 2017--2021\label{sec:nsf}}
%{Grant Award (NSF/NCS)}
%{NCS-FO: Spatial intelligence for swarms based on hippocampal dynamics}
%{This NSF-awarded project (2018--2021; see \emph{\nameref{sec:funding}}
  %on p.\pageref{sec:funding}) was the result of 6 months of collaboration,
  %brain-storming, and team-building between the Zhang lab at JHU/SOM and a group
  %of JHUAPL engineers, mathematicians, and scientists. The project was initially
  %inspired by results that I presented at my Society for Neuroscience 2017 meeting
  %poster. I wrote Aim 1 and integrated the full research narrative with inputs
  %from our collaborators for the proposal of this \$997K/2-year project to develop
  %those initial ideas into technological applications (e.g., robotics, autonomous
  %control, AI) that reciprocally inform neuroscience. The project has so far
  %produced three posters, a conference talk \& proceedings publication, three
  %patent applications, a preprint, a research article in Biological Cybernetics, a
  %NIH BRAIN Investigators Meeting symposium talk, and a substantial STEM program.
  %We received a no-cost extension through FY21 to complete the final phase of the
%project.}

%\researchactivity
%{Jan. 2018--2020}
%{Grant Award (NIH/NINDS)}
%{Spiking network models of sharp-wave ripple sequences with gamma-locked
%attractor dynamics}
%{To continue with the collaboration that I initiated with David J. Foster
  %(UC Berkeley) on the basis of the internal SLI award (see above), I wrote
  %a small modeling proposal that integrated preliminary results from the SLI
  %project and recent research developments in the memory reactivation field.
  %This proposal was awarded (2018--2020; see \emph{\nameref{sec:funding}} on
  %p.\pageref{sec:funding}) through the NIH/NINDS R03 mechanism, supporting
  %theoretical and modeling efforts that provided a foundation for my subsequent
%research directions.}
%\label{sec:nih}

%\whitepaper
%{Feb.--Mar. 2018}
%{Schultz K, Zhang K, and \joehl{Monaco J}}
%{BrainSWARRMM: Brain-like Sharp-Waves for Autonomous Replanning \&
%Reconnaissance on Matrix Manifolds}
%{In response to the Office of Naval Research (ONR) Special Notice
  %N00014-18-R-SN05, Topic 3, I helped organize a series of collaborative meetings
  %to design a \$2M/4-year project between JHUAPL and JHU/SOM. I co-authored the
%resulting white paper that was submitted for consideration to ONR.}

%\whitepaper
%{May--June 2018}
%{Zhang K, \joehl{Monaco JD}, Hwang GM, Schultz KM, Kobilarov M, Foster DJ,
%Jacobs J, and Itti L}
%{An Integrative Theoretical Framework of the Neural Self-Organization of Active
%Perception for Autonomous Spatial Navigation}
%{In response to ONR MURI Announcement N00014-18-S-F006 and with the help of
  %JHUAPL, I coordinated a series of meetings with 5 PIs across 4 universities
  %(Columbia, UC Berkeley, USC, JHU) to design an innovative research program that
  %targeted reciprocal advances in experimental \& theoretical neuroscience and
  %robotics \& AI across species and scales. The resulting \$7.5M/5-year project
  %that I outlined in the white paper was not invited for a full submission. We
  %debriefed with the sponsor, ONR PM Marc Steinberg, who revealed that ONR was
  %impressed with the project but that they were seeking a different balance of
%elements with respect to neuroscience and AI.}

%\whitepaper
%{August 2019}
%{\joehl{Monaco J}, Zhang K, and Schultz K.}
%{SW2Mem: Graph Spectral Decoding of Hippocampal-Cortical Loops for Artificial
%Consolidation and Dreaming}
%{In response to ONR Special Notice N00014-19-S-SN08, Topic 5.1 I conceived this
  %project, created the preliminary model and datasets, guided the preliminary
  %analyses with JHUAPL collaborators, and wrote \& submitted the white paper to
  %ONR outlining a potential \$1.05M/3-year project. ONR declined to invite us to
%submit a full proposal.}

%\whitepaper
%{August 2019}
%{Schultz K, Agarwala S, Zhang K, and \joehl{Monaco J}}
%{Brain-like Distributed Surveillance using Heterogeneous Agents for integRated
%Perception, and Planning (BD-SHARPP)}
%{In response to ONR Special Notice N00014-18-R-SN05, Topic 3, we submitted a
  %revised version of the March 2018 white paper that was specifically invited by
%ONR PM Tom McKenna.}

%\researchactivity
%{Sept. 11, 2019}
%{NSF Project Review}
%{Annual advisory board review symposium}
%{I delivered a seminar on Aim 1 progress at a JHUAPL-hosted symposium for our
  %project’s yearly review, attended by DARPA/I2O PM Hava Siegelmann and other
%outside experts.}

%\researchactivity
%{Feb. 26, 2020}
%{Grant Proposal (NSF/NCS) }
%{NCS-FO: Neuroeconomics as a biomimetic control theory for mobile robotic
%decision making}
%{This FY21 proposal was submitted to the NSF/NCS FOUNDATIONS program; while
  %it was discussed and received high scores, the application was declined. I
  %co-developed this project in collaboration with colleagues at the University of
  %Pittsburgh Medical Center (UPMC), JHU Whiting School of Engineering (JHU/WSE),
  %and JHUAPL. Our interdisciplinary project brought together multiscale human
  %electrophysiological recordings (UPMC), latent state-space models (JHU/WSE),
  %control- and game-theoretic analysis (JHUAPL), and mechanistic neural models
  %(JHU/BME, for which I would have been co-PI). We proposed to investigate and
  %characterize the neural bases of metacognitive brain states that influence
  %decision-making during social \& economic games. As a high-risk/high-reward
  %element, we proposed to algorithmicize our results to advance human-robot
%interaction.}

%\researchactivity
%{Jan. 14, 2022}
%{Grant Proposal (JHU/Discovery Award) }
%{Algorithms of flexible navigation in mice and robots}
%{This intramural FY23 application for a JHU Discovery Award resulted from a new
  %collaboration with Patricia Janak (PI; JHU/PBS) and Céline Drieu (postdoctoral
  %fellow), in which we proposed to integrate advanced large-scale neural recording
  %technologies with my theoretical modeling of neural systems as a distributed
  %control problem. Fundamental questions of neural systems communication were
  %to be addressed using convergent data-driven and theory-driven approaches
  %to understanding the cognitive dynamics that enable mice to perform spatial
%goal-directed memory tasks.}

\newsection{Scientific Peer Review}

\subsection*{Journals}

\lefttabline{0.8in}{2021}{PLOS Computational Biology}
\lefttabline{0.8in}{2021}{Nature Machine Intelligence}
\lefttabline{0.8in}{2020}{Neuroscience and Biobehavioral Reviews}
\lefttabline{0.8in}{2020}{Scientific Reports}
\lefttabline{0.8in}{2019}{eLife}
\lefttabline{0.8in}{2019}{Hippocampus}
\lefttabline{0.8in}{2018--2019}{Neuron}
\lefttabline{0.8in}{2018}{Neural Computation (including as `Communicator')}
\lefttabline{0.8in}{2018}{PLOS ONE}
\lefttabline{0.8in}{2017}{PeerJ}
\lefttabline{0.8in}{2015}{IEEE Transactions in Biomedical Engineering}
\lefttabline{0.8in}{2012--2020}{IEEE Neural Networks}
\lefttabline{0.8in}{2012}{Biological Cybernetics}
\lefttabline{0.8in}{2012}{Neurocomputing}
\lefttabline{0.8in}{2012}{Neuroscience}

\subsection*{Funding Agencies}
\label{sec:programsvc}

\lefttabline{0.8in}{2022}{NSF CAREER Ad-Hoc Reviewer}
\lefttabline{0.8in}{2022}{NSF EFRI Preliminary Review Panel}
\lefttabline{0.8in}{2022}{NSF EFRI Final Review Panel}
\lefttabline{0.8in}{2020--2022}{NSF EFRI Program Development, Extramural Contributor}
\lefttabline{0.8in}{2014}{IARPA Program Development, Extramural Contributor}

\subsection*{Conferences}
\label{sec:confsvc}

\lefttabline{0.8in}{2020--2021}{Cosyne, Review committee member}
\lefttabline{0.8in}{2016}{Cosyne, Review committee member}

\subsection*{Mentoring \& Supervision}
\label{sec:mentoring}

\lefttabline{0.8in}{Spring 2021}{Darius Carr, STEM high-school student}
\lefttabline{0.8in}{2020--2022}{Armin Hadzic, machine learning engineer at JHUAPL}
\lefttabline{0.8in}{2019--2020}{Sreelakshmi Rajendrakumar, masters student in JHU/Biomedical Engineering (BME)}
\lefttabline{0.8in}{2014}{Manning Zhang, M.S., graduate student in JHU/BME}
\lefttabline{0.8in}{2013--2015}{Chia-Hsuan Wang, Ph.D., graduate student at the JHU/MBI}

\newsection{Recognition}

%\subsection*{Awards \& Honors}

\vspace{-.13in}
\lefttabline{0.8in}{7/5/2022}{Inventor, 
  \href{https://www.freepatentsonline.com/11378975.html}{\itemtitle{Autonomous
  Navigation Technology, US patent issued, 11,378,975}}}
\lefttabline{0.8in}{2022}{IEEE/ISEC Best Paper Award, First Place}
\lefttabline{0.8in}{2003}{IEEE/IJCNN Student Paper Award, First Place}
\lefttabline{0.8in}{2002}{U.Va. John A. Harrison III Undergraduate Research Award}
\lefttabline{0.8in}{1999--2003}{U.Va. Echols Scholar}
\lefttabline{0.8in}{1999}{State of Maryland Merit Scholastic Award}
\lefttabline{0.8in}{1999}{AP Scholar with Distinction}
\lefttabline{0.8in}{1999}{National Merit Scholarship Commended Student}
\lefttabline{0.8in}{1999}{Johns Hopkins Mathematics Competition (2nd Place, Individual Calculus)}
\lefttabline{0.8in}{1999}{Maryland Distinguished Scholar}

\subsection*{News \& Views}

\begin{itemize}
  \item \href{https://dx.doi.org/10.1016/j.cub.2020.02.085}
    {Place R, Nitz DA. (2020). \itemtitle{Cognitive Maps: Distortions of the Hippocampal 
      Space Map Define Neighborhoods}. \emph{Current Biology}, 30(8): R340--R342.}
  \item \href{https://dx.doi.org/10.1016/j.cell.2018.10.047}
    {Colwell CS, Donlea J. (2018). \itemtitle{Temporal coding of sleep}. \emph{Cell}, 175(5): 1177--9.}
  \item \href{https://dx.doi.org/10.1038/nn.3700}
    {Dupret D, Csicsvari J. (2014). \itemtitle{Turning heads to remember
    places}. \emph{Nature Neuroscience}, 17(5): 643--44.}
\end{itemize}

\end{document}
